\section{Conclusions}\label{chapter-05:section:conclusion}

In this chapter, we conducted an experimental evaluation of the \ac{rmp} framework, 
which encompasses both exact and approximate Bayesian inference by
minimizing a \ac{cbfe} functional.
The \ac{rmp} framework supports hybrid algorithms with analytical message update rules, including
\ac{bp}, \ac{ep}, \ac{em}, \ac{vmp}, and provides flexibility in terms of variational constraints.

The implementation of \ac{rmp} in the form of RxInfer, developed in the Julia programming language,
showed to be an efficient and scalable realization of an automated \ac{cbfe} minimization process that can be used
as a fast approximate Bayesian inference process.
RxInfer has successfully handled large-scale models with hundreds of thousands of unknowns,
demonstrating its potential for real-world applications.
The RxInfer framework offers a high level of customization through the creation of custom
nodes, message update rules, and approximation methods.

The experimental results on various standard signal processing models revealed substantial
performance improvements of RxInfer compared to alternative Julia-based frameworks for automated Bayesian inference.
The proposed implementation runs smoothly on regular office computers, avoiding the need for
expensive supercomputers or \ac{gpu} support, making it cost-effective for Bayesian inference on
large datasets with hundreds of thousands of observations.

Sampling-based methods, such as \ac{nuts}, suffer from
scalability challenges, making them unsuitable for real-time Bayesian inference applications.
These methods rely on generating samples from the posterior distribution to approximate the
desired results.
However, as the complexity and size of the probabilistic model increase, the number of
required samples also increases.
Consequently, the computational burden becomes prohibitively high, making sampling-based
methods inefficient for large-scale problems with a substantial number of unknowns.
Moreover, sampling-based algorithms often exhibit slow convergence, especially in high-dimensional
spaces, further exacerbating their scalability issues.
As real-time applications demand quick and efficient inference, the time and resources
required for sampling-based methods make them impractical for such scenarios, prompting the
need for alternative approaches, such as \ac{rmp}, which can handle large models
efficiently and provide results in a timely manner.

The benchmark results indicated that the overhead associated with managing the reactive nature of
the architecture is minimal and that \ac{rmp} consistently outperforms the reference message
passing-based implementation.
The lack of explicit scheduling in the proposed architecture provides practical advantages by
avoiding the need to traverse the entire factor graph, thereby saving computational resources.

We believe that the introduction of a reactive programming approach to message passing-based
inference opens up new avenues for further research, bringing real-time Bayesian inference
closer to real-world applications.
The RxInfer framework's ability to handle large models and simplify the model exploration
process in signal processing applications makes it a promising tool for a wide range of
probabilistic modeling tasks.
The framework lays the foundation for advancing the state-of-the-art in Bayesian inference and
promotes the adoption of \ac{rmp} in various fields, encouraging more researchers and practitioners
to explore the benefits of reactive message passing.
As the field of Bayesian inference continues to evolve, we anticipate that the RxInfer framework
will be further refined and extended, contributing to ongoing progress in simulating large-scale Bayesian inference processes.

