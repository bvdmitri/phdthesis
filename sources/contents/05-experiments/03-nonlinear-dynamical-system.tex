\section{Non-linear dynamical system}\label{chapter-05:section:nonlinear-dynamical-system}

In this section, we delve into a more complex model and inference setting by considering a \ac{nlds}.
The real world is often "nonlinear" and many physical processes are better described by
nonlinear differential equations \citep{roubicek_nonlinear_2013}.
Inference in \acp{nlds} finds extensive applications in various industries beyond
signal processing \citep{Revach_kalmannet_2022}, including fields such as robotics
\citep{cernousko_control_2008}, astronomy \citep{Contopoulos_astronomy_nlds}, biology
\citep{Janson_bilogy_nlds}, economics \citep{hsieh_chaos_1991}, climate modeling
\citep{mukhin_principal_2015}, and many many more (\hyperlink{experiments:utility}{\emph{Utility}}).
As in the previous example, an \ac{nlds} is a state-space model that evolves over time
$t$, where the subsequent state of the system depends solely on the preceding state.
However, unlike the previous example, the relationship between subsequent states in this model is
nonlinear, introducing additional challenges in the inference process.

In its general form, an \ac{nlds} model can be expressed as follows
\begin{equation}
  \label{eq:sim:nlds}
  \begin{split}
    s_t &= f(s_{t - 1}) + v_{t}, \\% ~~\sigma_{t} \sim \mathcal{N}(0, \Sigma)\\ 
    y_t &= g(s_t) + w_{t}, % ~~\omega_{t} \sim \mathcal{N}(0, \Omega)
  \end{split}
\end{equation}
where $s_t$ denotes the state of the system at time $t$, and $f$
represents an arbitrary nonlinear state-transition function.
The observation at time $t$ is denoted by $y_t$, and $g$ represents an arbitrary (potentially
also nonlinear) observational function.
The terms $v_{t}$ and $w_{t}$ represent process and measurement noise signals and are often assumed to be Gaussian distributed with zero mean and
covariance matrices $\Sigma$ and $\Omega$, respectively:
\begin{equation}
    \label{eq:sim:nlds-stochastic-gaussian}
    \begin{split}
        v_{t} &\sim \mathcal{N}(0, \Sigma) \\
        w_{t} &\sim \mathcal{N}(0, \Omega) \\
    \end{split}
\end{equation}
By expressing the model~\eqref{eq:sim:nlds} with the assumption~\eqref{eq:sim:nlds-stochastic-gaussian} 
in terms of probability densities, we obtain the model specification
\begin{equation}
  \label{eq:sim:nlds_probabilities}
  \begin{aligned}
    p(s_t\vert s_{t-1}, \Sigma)
     & = \mathcal{N}(s_t \vert f(s_{t-1}), \Sigma) \\ p(y_t\vert s_{t}, \Omega) & = \mathcal{N}(y_t
       \vert g(s_{t}), \Omega)
  \end{aligned}
\end{equation}
where the first equation, $p(s_t\vert
  s_{t-1}, \Sigma)$, denotes the conditional probability distribution of the state $s_t$ given the
previous state $s_{t-1}$ and the covariance matrix $\Sigma$.
Similarly, the second equation, $p(y_t\vert s_{t}, \Omega)$, represents the conditional probability
distribution of the observation $y_t$ given the current state $s_t$ and the covariance
matrix $\Omega$.

The complete probabilistic model can be expressed as
\begin{equation}
  \label{eq:sim:nlds_model} p(\bm{y}, \bm{s}, \Sigma, \Omega) =
  \underbrace{p(\Sigma)p(\Omega)p(s_1)}_{\mathrm{prior}}\underbrace{\prod_{t = 1}^{T} p(y_t\vert
    s_t, \Omega)}_{\mathrm{likelihood}}\underbrace{\prod_{t = 2}^{T}p(s_t\vert s_{t - 1},
    \Sigma)}_{\mathrm{state~transitions}}
\end{equation}
where $p(\Sigma)$, $p(\Omega)$, and
$p(s_1)$ denote the priors for $\Sigma$, $\Omega$, and $s_1$, respectively.
Prior terms contribute to the overall prior probability of the model, incorporating prior
beliefs or knowledge about the covariance matrices and the initial state.
This formulation provides a comprehensive representation of the probabilistic model that
includes priors, likelihoods, and state transitions.
Figure~\eqref{fig:sim:lds_model_graph} provides a visual representation of the probabilistic
model~\eqref{eq:sim:lds_model} in the form of \ac{tffg}.
The \ac{tffg} visualizes the dependencies and flow of information within the model and illustrates
the interconnections between the different components of the \ac{nlds} model.

The probabilistic model~\eqref{eq:sim:nlds_model} is similar to the one discussed in
Section~\ref{chapter-05:section:linear-dynamical-system}, but with the notable difference that
both the transition and the observational functions are nonlinear.
The introduction of nonlinearities adds complexity to the inference procedure, as analytical
closed-form solutions for the exact Bayesian inference with arbitrary nonlinearities are generally not
available.
Therefore, we need to employ approximate inference techniques to estimate the posterior
distributions in this setting.

\begin{figure}
  \centering
  \resizebox{\textwidth}{!}{\begin{tikzpicture}
  \node[] (splink) {$\cdots$};
  \node[box, node distance=3mm] (spN) [right=of splink] {$\mathcal{N}$};
  \node[smallbox] (speq) [right=of spN] {\small $=$};
  \node[box, node distance=5mm] (spf) [right=of speq] {$f$};
  \node[box, node distance=5mm] (spfN) [right=of spf] {$\mathcal{N}$};
  \node[box, node distance=3mm] (spg) [below=of speq] {$g$};
  \node[box, node distance=3mm] (spgN) [below=of spg] {$\mathcal{N}$};
  \node[clamped, node distance=5mm] (yp) [below=of spgN] {};

  \path[line] (splink) edge[-] (spN);
  \path[line] (spN) edge[-] node[pos=0.5, anchor=south]{$s_{t - 1}$} (speq);
  \path[line] (speq) edge[-] (spf);
  \path[line] (speq) edge[-] (spg);
  \path[line] (spf) edge[-] (spfN);
  \path[line] (spg) edge[-] node[pos=0.5, anchor=east](spgleft){} node[pos=0.5, anchor=west](spgright){} (spgN);
  \path[line] (spgN) edge[-] node[pos=0.5, anchor=west]{$y_{t - 1}$} (yp);

  \node[smallbox] (seq) [right=of spfN] {\small $=$};
  \node[box, node distance=5mm] (sf) [right=of seq] {$f$};
  \node[box, node distance=5mm] (sfN) [right=of sf] {$\mathcal{N}$};
  \node[box, node distance=3mm] (sg) [below=of seq] {$g$};
  \node[box, node distance=3mm] (sgN) [below=of sg] {$\mathcal{N}$};
  \node[clamped, node distance=5mm] (y) [below=of sgN] {};

  \path[line] (spfN) edge[-] node[pos=0.5, anchor=south]{$s_{t}$} (seq);
  \path[line] (seq) edge[-] (sf);
  \path[line] (seq) edge[-] (sg);
  \path[line] (sf) edge[-] (sfN);
  \path[line] (sg) edge[-] node[pos=0.5, anchor=east](sgleft){} node[pos=0.5, anchor=west](sgright){} (sgN);
  \path[line] (sgN) edge[-] node[pos=0.5, anchor=west]{$y_{t}$} (y);

  \node[] (snlink) [right=of sfN] {$\cdots$};
  \node[box, node distance=5mm] (sTN) [right=of snlink] {$\mathcal{N}$};
  \node[smallbox, draw=white, node distance=5mm] (sTeq) [right=of sTN] {};
  \node[box, node distance=3mm] (sTg) [below=of sTeq] {$g$};
  \node[box, node distance=3mm] (sTgN) [below=of sTg] {$\mathcal{N}$};
  \node[clamped, node distance=5mm] (yT) [below=of sTgN] {};

  \path[line] (sfN) edge[-] node[pos=0.5, anchor=south]{$s_{t + 1}$} (snlink);
  \path[line] (snlink) edge[-] (sTN);
  \path[line] (sTN) edge[-] node[pos=0.5, anchor=south]{$s_T$} (sTeq.center);
  \draw[] (sTeq.center) -- (sTg);
  \path[line] (sTg) edge[-] node[pos=0.5, anchor=east](sTgleft){} node[pos=0.5, anchor=west](sTgright){} (sTgN);
  \path[line] (sTgN) edge[-] node[pos=0.5, anchor=west]{$y_{T}$} (yT);

  \node[box] (s0) [left=of splink] {$p(s_1)$};
  \path[line] (s0) edge[-] node[pos=0.5, anchor=south]{$s_1$} (splink);

  \node[smallbox] (pPsp) [above=of spN] {\small $=$};
  \node[node distance=4mm] (pPlink) [left=of pPsp] {$\cdots$};
  \node[box] (pP) [left=of pPlink] {$p(\Sigma)$};
  \node[smallbox] (pPs) [above=of spfN] {\small $=$};
  \node[smallbox] (pPsn) [above=of sfN] {\small $=$};
  \node[smallbox, draw=white] (pPsTeq) [right=of pPsn] {$\cdots$};

  \path[line] (pP) edge[-] node[pos=0.5, anchor=south]{$\Sigma$} (pPlink);
  \path[line] (pPlink) edge[-] (pPsp);
  \path[line] (pPsp) edge[-] (pPs);
  \path[line] (pPs) edge[-] (pPsn);
  \path[line] (pPsn) edge[-] (pPsTeq);
  \path[line] (pPsp) edge[-] (spN);
  \path[line] (pPs) edge[-] (spfN);
  \path[line] (pPsn) edge[-] (sfN);
  \draw[] (pPsTeq) -| (sTN);

  \node[smallbox] (pQsp) [left=of spgleft] {\small $=$};
  \node[smallbox] (pQs) [left=of sgleft] {\small $=$};
  \node[smallbox, draw=white, node distance=17mm] (pQsT) [left=of sTgleft] {$\cdots$};
  \node[] (pQsTextra) [right=of pQsT] {};
  \node[node distance=5mm] (pQlink) [left=of pQsp] {$\cdots$};
  \node[box] (pQ) [left=of pQlink] {$p(\Omega)$};

  \path[line] (pQ) edge[-] node[pos=0.5, anchor=south]{$\Omega$} (pQlink);
  \path[line] (pQlink) edge[-] (pQsp);
  \draw[] (pQsp) -- (spgleft.center);
  %\draw[path, densely dotted] (spgleft.center) -- (spgright.center);
  \draw[] (spgright.center) -- (pQs);
  \draw[] (pQs) -- (sgleft.center);
  %\draw[path, densely dotted] (sgleft.center) -- (sgright.center);
  \draw[] (sgright.center) -- (pQsT);
  \draw[] (pQsT) -- (pQsTextra.center);
  \draw[] (pQsTextra.center) |- (sTgN);
  \draw[] (pQsp) |- (spgN);
  \draw[] (pQs) |- (sgN);

\end{tikzpicture}

}
  \caption{
    A \ac{tffg} representation of the probabilistic model~\eqref{eq:sim:nlds_model} for the \ac{nlds}~\eqref{eq:sim:nlds}-\eqref{eq:sim:nlds-stochastic-gaussian}.
    The $s_t$ represent the hidden states, while $y_t$ corresponds to the
    observations.
    The $\Sigma$ and $\Omega$ are covariance matrices of the Gaussian noise components for states
    and observations, respectively.
    The state transition function $f$ and observational function $g$ are the nonlinear components of the model.
    Factor nodes $f$ and $g$ indicate a nonlinear function.
    The $\cdots$ symbol denotes the repetitive structure in the corresponding graph.
  }
  \label{fig:sim:nlds_model_graph}
\end{figure}

\subsection{Example of a nonlinear dynamical system}

As a particular simplistic example of an \ac{nlds} system, we choose the dynamics of a double
pendulum physical system.
The double pendulum system consists of two pendulums (rods) connected to each other.
Despite its simple appearance, the double pendulum exhibits complex and chaotic behavior,
making it an interesting case study \citep{levien_double_1993}.
Figure~\ref{fig:sim:double_pendulum_notation} provides an illustration of the double
pendulum system, depicting the rods and their movement restricted to two dimensions in the
vertical plane.
The dynamics of the double pendulum is described by a set of coupled ordinary differential
equations, which capture the relationship between the state of pendulums and their motion.
Due to its chaotic nature, the behavior of the double pendulum is highly sensitive to the
initial conditions, resulting in unpredictable and complex motion patterns.

% The motion equations of the double pendulum system can be discretized in time and solved
% numerically using methods such as the Runge-Kutta method
% \citep[Chapter~8]{hasselblatt_handbook_2002}.
Assuming that $s_t$ is the state of the system at time $t$, the evolution of the state of the double pendulum system can be
rewritten as \ac{nlds}~\eqref{eq:sim:nlds} (see Appendix~\ref{appendix:proofs:double_pendulum_dynamics}). 
Similar to the previous example in Section~\ref{chapter-05:section:linear-dynamical-system}, the number of latent states in the system increases linearly with the number of available observations.
The primary challenge lies in accurately estimating and tracking the evolution of the latent
states of the system given noisy measurements at specific points in time. 
More formally, we are interested in estimating the following Bayesian posteriors:
\begin{equation}
    \label{eq:sim:nlds-problem-statement}
    p(s_t\vert\hat{\bm{y}}_{1:T}) = \int p(\bm{y}, \bm{s}, \Sigma, \Omega)\prod_{i = 1}^{T}\delta(y_i - \hat{y}_i)\mathrm{d}\Sigma\mathrm{d}\Omega\mathrm{d}s_{\setminus t}\mathrm{d}\bm{y}~~\forall t \in 1:T.
\end{equation}

\begin{figure}
  \centering
  \resizebox{0.75\textwidth}{!}{\begin{tikzpicture}

  \node[fill=black, circle, inner sep=0.25mm, label={above left:{\tiny $(0, 0)$}}] (origin) {};
  \node[node distance=30mm] (xaxis) [right=of origin] {};
  \node[node distance=15mm] (yaxis) [below=of origin] {};

  \path[] ([xshift=-4mm]origin.center) edge[-stealth] node[pos=1.0, anchor=south]{\tiny $x$} (xaxis.center);
  \path[] ([yshift=4mm]origin.center) edge[stealth-] node[pos=0.0, anchor=south]{\tiny $y$} (yaxis.center);

  \node[circle, fill=black, inner sep=0.5mm, label={right:{\tiny $m_1$}}] (m1) [xshift=13mm, yshift=-6mm] {};
  \node[circle, fill=black, inner sep=0.5mm, label={right:{\tiny $m_2$}}] (m2) [xshift=20mm, yshift=-15mm] {};

  \path[-, densely dotted] ([xshift=13mm]origin.center) edge[-] node[pos=0, anchor=south]{\tiny $x_1$} (m1.center);
  \path[-, densely dotted] ([yshift=-6mm]origin.center) edge[-] node[pos=0, anchor=east]{\tiny $y_1$} (m1.center);
  \path[line] (origin) edge[-] node[pos=0.7, anchor=south]{\tiny $l_1$} (m1);

  \path[-, densely dotted] ([xshift=20mm]origin.center) edge[-] node[pos=0, anchor=south]{\tiny $x_2$} (m2.center);
  \path[-, densely dotted] ([yshift=-15mm]origin.center) edge[-] node[pos=0, anchor=east]{\tiny $y_2$} (m2.center);
  \path[line] (m1) edge[-] node[pos=0.7, anchor=south]{\tiny $l_2$} (m2);

  \node[node distance=5mm] (belowm1) [below=of m1] {};
  \path[-, densely dotted] (m1) edge[-] (belowm1);

  \pic [draw, -, radius=1mm] {angle = yaxis--origin--m1};
  \pic [draw, -, radius=1mm] {angle = belowm1--m1--m2};

  \node[] (t1) [xshift=5.0mm, yshift=-4.5mm]{\tiny $\theta_1$};
  \node[] (t1) [xshift=15.5mm, yshift=-12mm]{\tiny $\theta_2$};

  \node[] (gstart) [xshift=27.5mm, yshift=-3mm] {};
  \node[] (gend) [xshift=27.5mm, yshift=-10mm] {};
  \path[] (gstart) edge[-stealth] node[pos=0.5, anchor=west]{\tiny $g$} (gend);
\end{tikzpicture}
}
  \caption{
    An illustration of the double pendulum system.
    The system consists of two rods with lengths $l_i$ and two bobs with masses $m_i$.
    The rods are connected to each other.
    The state of the system at time $t$ is fully described by the state vector $(\theta_1,
      \theta_2, \dot{\theta}_1, \dot{\theta}_2)_t$, where $\theta_1$ and $\theta_2$ represent the 
      relative angles and $\dot{\theta}_1$ and $\dot{\theta}_2$ represent the angular velocities,
    respectively.
    The vector $g$ represents the gravitational force.
  }
  \label{fig:sim:double_pendulum_notation}
\end{figure}

\subsubsection{Simulated measurements}

Several variants of the double pendulum system may be considered: the two rods may be of equal
or unequal lengths and masses, they may be simple pendulums or compound pendulums (also called
complex pendulums), and the motion may be in three dimensions or restricted to the vertical
plane.

In the experiments, we consider a specific variant of the double pendulum system.
The two rods are assumed to be identical simple pendulums of unit length $l_1 = l_2 = l = 1$.
The masses of the bobs are assumed to be different and are denoted as $m_1$ and $m_2$
respectively.
The motion of the system is restricted to two dimensions in the vertical plane.
The states of the system, denoted as $s_t$, are 4-dimensional vectors $(\theta_1, \theta_2,
  \dot{\theta}_1, \dot{\theta}_2)_t$, representing the relative angles and angular velocities.
We also assume that the time difference (elapsed time) between two observations is fixed and known.

To make the inference procedure more challenging, we assume that the observation function is
given by $g(s_t) = \mathrm{dot}(s_t, \left[ 0, 1, 0, 0 \right]) = \theta_2$, which means that only the second component of the state vector $s_t$ is directly observable.
The other components of the state vector cannot be observed directly.
Additionally, the variance $\Omega$ of the noise component $w_t$ in~\eqref{eq:sim:nlds}
is assumed to be unknown, and the covariance $\Sigma$ of the noise component $v_t$ is assumed to be
small.

Figure~\ref{fig:sim:pendulum_example_states} shows the evolution of the double pendulum system
over the first 250 time steps, together with the corresponding observations.
The simulation is carried out using the Runge-Kutta (RK4) method, with an initial state of $s_1
  = (1.2, 0.2, 0.0, 0.0)$.

\begin{figure}
  \centering
  \begin{subfigure}[t]{0.475\textwidth}
    \centering
    \resizebox{\textwidth}{!}{
        % % Recommended preamble:
% \usetikzlibrary{arrows.meta}
% \usetikzlibrary{backgrounds}
% \usepgfplotslibrary{patchplots}
% \usepgfplotslibrary{fillbetween}
% \pgfplotsset{%
%     layers/standard/.define layer set={%
%         background,axis background,axis grid,axis ticks,axis lines,axis tick labels,pre main,main,axis descriptions,axis foreground%
%     }{
%         grid style={/pgfplots/on layer=axis grid},%
%         tick style={/pgfplots/on layer=axis ticks},%
%         axis line style={/pgfplots/on layer=axis lines},%
%         label style={/pgfplots/on layer=axis descriptions},%
%         legend style={/pgfplots/on layer=axis descriptions},%
%         title style={/pgfplots/on layer=axis descriptions},%
%         colorbar style={/pgfplots/on layer=axis descriptions},%
%         ticklabel style={/pgfplots/on layer=axis tick labels},%
%         axis background@ style={/pgfplots/on layer=axis background},%
%         3d box foreground style={/pgfplots/on layer=axis foreground},%
%     },
% }

\begin{tikzpicture}[/tikz/background rectangle/.style={fill={rgb,1:red,1.0;green,1.0;blue,1.0}, fill opacity={1.0}, draw opacity={1.0}}, show background rectangle]
\begin{axis}[point meta max={nan}, point meta min={nan}, legend cell align={left}, legend columns={1}, title={}, title style={at={{(0.5,1)}}, anchor={south}, font={{\fontsize{18 pt}{23.400000000000002 pt}\selectfont}}, color={rgb,1:red,0.0;green,0.0;blue,0.0}, draw opacity={1.0}, rotate={0.0}, align={center}}, legend style={color={rgb,1:red,0.0;green,0.0;blue,0.0}, draw opacity={1.0}, line width={1}, solid, fill={rgb,1:red,1.0;green,1.0;blue,1.0}, fill opacity={1.0}, text opacity={1.0}, font={{\fontsize{14 pt}{18.2 pt}\selectfont}}, text={rgb,1:red,0.0;green,0.0;blue,0.0}, cells={anchor={center}}, at={(0.02, 0.02)}, anchor={south west}}, axis background/.style={fill={rgb,1:red,1.0;green,1.0;blue,1.0}, opacity={1.0}}, anchor={north west}, xshift={1.0mm}, yshift={-1.0mm}, width={99.6mm}, height={74.2mm}, scaled x ticks={false}, xlabel={Time step index}, x tick style={color={rgb,1:red,0.0;green,0.0;blue,0.0}, opacity={1.0}}, x tick label style={color={rgb,1:red,0.0;green,0.0;blue,0.0}, opacity={1.0}, rotate={0}}, xlabel style={at={(ticklabel cs:0.5)}, anchor=near ticklabel, at={{(ticklabel cs:0.5)}}, anchor={near ticklabel}, font={{\fontsize{16 pt}{20.8 pt}\selectfont}}, color={rgb,1:red,0.0;green,0.0;blue,0.0}, draw opacity={1.0}, rotate={0.0}}, xmajorgrids={true}, xmin={-6.469999999999999}, xmax={257.47}, xticklabels={{$0$,$50$,$100$,$150$,$200$,$250$}}, xtick={{0.0,50.0,100.0,150.0,200.0,250.0}}, xtick align={inside}, xticklabel style={font={{\fontsize{14 pt}{18.2 pt}\selectfont}}, color={rgb,1:red,0.0;green,0.0;blue,0.0}, draw opacity={1.0}, rotate={0.0}}, x grid style={color={rgb,1:red,0.0;green,0.0;blue,0.0}, draw opacity={0.1}, line width={0.5}, solid}, axis x line*={left}, x axis line style={color={rgb,1:red,0.0;green,0.0;blue,0.0}, draw opacity={1.0}, line width={1}, solid}, scaled y ticks={false}, ylabel={Angle (radians)}, y tick style={color={rgb,1:red,0.0;green,0.0;blue,0.0}, opacity={1.0}}, y tick label style={color={rgb,1:red,0.0;green,0.0;blue,0.0}, opacity={1.0}, rotate={0}}, ylabel style={at={(ticklabel cs:0.5)}, anchor=near ticklabel, at={{(ticklabel cs:0.5)}}, anchor={near ticklabel}, font={{\fontsize{16 pt}{20.8 pt}\selectfont}}, color={rgb,1:red,0.0;green,0.0;blue,0.0}, draw opacity={1.0}, rotate={0.0}}, ymajorgrids={true}, ymin={-4.064075544819777}, ymax={2.3669820967456285}, yticklabels={{$-4$,$-3$,$-2$,$-1$,$0$,$1$,$2$}}, ytick={{-4.0,-3.0,-2.0,-1.0,0.0,1.0,2.0}}, ytick align={inside}, yticklabel style={font={{\fontsize{14 pt}{18.2 pt}\selectfont}}, color={rgb,1:red,0.0;green,0.0;blue,0.0}, draw opacity={1.0}, rotate={0.0}}, y grid style={color={rgb,1:red,0.0;green,0.0;blue,0.0}, draw opacity={0.1}, line width={0.5}, solid}, axis y line*={left}, y axis line style={color={rgb,1:red,0.0;green,0.0;blue,0.0}, draw opacity={1.0}, line width={1}, solid}, colorbar={false}]
    \addplot[color={rgb,1:red,0.3059;green,0.4745;blue,0.6549}, name path={baae5c62-2824-46fd-858f-798d4b087ecf}, draw opacity={1.0}, line width={2}, solid]
        table[row sep={\\}]
        {
            \\
            1.0  1.2  \\
            2.0  1.1975151335062217  \\
            3.0  1.1970915600789533  \\
            4.0  1.19534634473405  \\
            5.0  1.1915766509805357  \\
            6.0  1.1848493575796306  \\
            7.0  1.1770800678424318  \\
            8.0  1.171864165361489  \\
            9.0  1.1621801659387943  \\
            10.0  1.1535629165712973  \\
            11.0  1.141411216245086  \\
            12.0  1.1287961854949984  \\
            13.0  1.115313529253476  \\
            14.0  1.1005697733298465  \\
            15.0  1.0847299364321075  \\
            16.0  1.0682234391516674  \\
            17.0  1.0499067813055099  \\
            18.0  1.0303343823088553  \\
            19.0  1.0073565051419662  \\
            20.0  0.9825366453297035  \\
            21.0  0.9576164478756791  \\
            22.0  0.9294183232794121  \\
            23.0  0.9010540594657842  \\
            24.0  0.871177584792846  \\
            25.0  0.8387534517016493  \\
            26.0  0.8057038102328733  \\
            27.0  0.7685330020538078  \\
            28.0  0.7292676576998472  \\
            29.0  0.6897696092836253  \\
            30.0  0.6510980021951602  \\
            31.0  0.612423474596998  \\
            32.0  0.5716349931446794  \\
            33.0  0.5312005456391712  \\
            34.0  0.49082456757771825  \\
            35.0  0.4510744962823734  \\
            36.0  0.41196698302997964  \\
            37.0  0.3741257502378474  \\
            38.0  0.33682834653492205  \\
            39.0  0.29906848098745864  \\
            40.0  0.2632068497472198  \\
            41.0  0.228512722781819  \\
            42.0  0.19392671449094387  \\
            43.0  0.1597211144056951  \\
            44.0  0.1262304434907114  \\
            45.0  0.0924920952566912  \\
            46.0  0.06290372087302695  \\
            47.0  0.030759781377157607  \\
            48.0  -0.00019681684207249295  \\
            49.0  -0.03048593511435672  \\
            50.0  -0.059864055765992644  \\
            51.0  -0.08846075775352978  \\
            52.0  -0.11801535576010934  \\
            53.0  -0.14606175876149094  \\
            54.0  -0.1728002036841634  \\
            55.0  -0.19811453282468772  \\
            56.0  -0.22321336513983644  \\
            57.0  -0.24676826356595563  \\
            58.0  -0.2690219043706597  \\
            59.0  -0.28879821866571687  \\
            60.0  -0.30827741896017696  \\
            61.0  -0.3258030996845972  \\
            62.0  -0.34399531620454155  \\
            63.0  -0.35874384274678023  \\
            64.0  -0.37165970519829494  \\
            65.0  -0.38383748327804124  \\
            66.0  -0.39509490023155414  \\
            67.0  -0.40329931377336636  \\
            68.0  -0.40857216473046554  \\
            69.0  -0.4113500548127559  \\
            70.0  -0.41034053709838464  \\
            71.0  -0.4068516442375843  \\
            72.0  -0.40361054944639163  \\
            73.0  -0.39695295224771066  \\
            74.0  -0.38882945373319316  \\
            75.0  -0.3782519152019781  \\
            76.0  -0.3679323730802571  \\
            77.0  -0.35794380182176166  \\
            78.0  -0.34539970103964324  \\
            79.0  -0.3355173266761075  \\
            80.0  -0.3280702268312168  \\
            81.0  -0.32082298318352864  \\
            82.0  -0.3145960820564577  \\
            83.0  -0.3095664231771796  \\
            84.0  -0.30610422223427197  \\
            85.0  -0.30360272750690004  \\
            86.0  -0.3036187943229742  \\
            87.0  -0.30626703747376666  \\
            88.0  -0.3087172643195013  \\
            89.0  -0.31180449343857075  \\
            90.0  -0.3150305159677781  \\
            91.0  -0.3207579598205615  \\
            92.0  -0.32807003198091367  \\
            93.0  -0.33593202804874084  \\
            94.0  -0.3429605640923118  \\
            95.0  -0.35197137548101587  \\
            96.0  -0.3608155036416177  \\
            97.0  -0.3711962350839144  \\
            98.0  -0.38022804039756897  \\
            99.0  -0.38996544739969935  \\
            100.0  -0.399223926489659  \\
            101.0  -0.410014479203897  \\
            102.0  -0.4207589595400376  \\
            103.0  -0.4298825773447677  \\
            104.0  -0.43863619739852805  \\
            105.0  -0.44892265326297776  \\
            106.0  -0.4570540785038736  \\
            107.0  -0.4658356094534848  \\
            108.0  -0.47472761287526904  \\
            109.0  -0.48320172886313545  \\
            110.0  -0.4908051366828858  \\
            111.0  -0.49667123555427767  \\
            112.0  -0.5033095749392856  \\
            113.0  -0.5121771516371678  \\
            114.0  -0.5181976331508031  \\
            115.0  -0.523015155874087  \\
            116.0  -0.528859929132881  \\
            117.0  -0.5340502551787951  \\
            118.0  -0.53954838141458  \\
            119.0  -0.5423968975831579  \\
            120.0  -0.5436024976987137  \\
            121.0  -0.5468403142975105  \\
            122.0  -0.5482744013999638  \\
            123.0  -0.55013538025475  \\
            124.0  -0.5510837075435782  \\
            125.0  -0.553417724267389  \\
            126.0  -0.5533321642799661  \\
            127.0  -0.5528945058438742  \\
            128.0  -0.5514719283944148  \\
            129.0  -0.5488966930627701  \\
            130.0  -0.5461860421463555  \\
            131.0  -0.5440932976565519  \\
            132.0  -0.5392222044124598  \\
            133.0  -0.5343142081793605  \\
            134.0  -0.5306700169185303  \\
            135.0  -0.525843515465054  \\
            136.0  -0.5219267630618546  \\
            137.0  -0.5155884501298923  \\
            138.0  -0.5100456345018718  \\
            139.0  -0.5056072946383906  \\
            140.0  -0.4990695777308651  \\
            141.0  -0.49360968509416325  \\
            142.0  -0.48615619142021466  \\
            143.0  -0.47990071453724625  \\
            144.0  -0.4744550326720059  \\
            145.0  -0.46673524036556174  \\
            146.0  -0.4602419992410883  \\
            147.0  -0.4516059844059045  \\
            148.0  -0.4444233154634801  \\
            149.0  -0.4388371444137217  \\
            150.0  -0.4338432428599476  \\
            151.0  -0.4268186710022263  \\
            152.0  -0.42014954014142314  \\
            153.0  -0.41526093790691315  \\
            154.0  -0.4116755896080328  \\
            155.0  -0.40784835083540943  \\
            156.0  -0.40398383906468993  \\
            157.0  -0.4030891539728474  \\
            158.0  -0.40058540807119236  \\
            159.0  -0.40018734408082496  \\
            160.0  -0.4002452702260852  \\
            161.0  -0.4010309351600541  \\
            162.0  -0.40396888226884936  \\
            163.0  -0.40545692218719254  \\
            164.0  -0.40948236016498035  \\
            165.0  -0.41563059451908363  \\
            166.0  -0.42379810682412267  \\
            167.0  -0.4319152742285834  \\
            168.0  -0.44038192248955943  \\
            169.0  -0.45013963064179674  \\
            170.0  -0.4579857936211864  \\
            171.0  -0.4664674266139653  \\
            172.0  -0.47164864520534816  \\
            173.0  -0.4758253168111328  \\
            174.0  -0.4770860226799992  \\
            175.0  -0.47686592673089057  \\
            176.0  -0.4747980752011135  \\
            177.0  -0.4705216624384824  \\
            178.0  -0.4642965721751808  \\
            179.0  -0.45455622902970894  \\
            180.0  -0.4449589425799307  \\
            181.0  -0.4327109363087674  \\
            182.0  -0.4176722130844967  \\
            183.0  -0.4022921555397081  \\
            184.0  -0.3845046845167431  \\
            185.0  -0.3650606021349645  \\
            186.0  -0.3453750561269453  \\
            187.0  -0.32212769817558845  \\
            188.0  -0.2988035149071209  \\
            189.0  -0.27436868212581755  \\
            190.0  -0.25017233635191016  \\
            191.0  -0.22200383919331307  \\
            192.0  -0.19374017883719744  \\
            193.0  -0.16571344954797784  \\
            194.0  -0.13661098102940178  \\
            195.0  -0.10529876448674946  \\
            196.0  -0.0734953542951545  \\
            197.0  -0.0410537980389718  \\
            198.0  -0.008238733484454912  \\
            199.0  0.024791282082862708  \\
            200.0  0.06017033974980184  \\
            201.0  0.09793684459262833  \\
            202.0  0.13394133858774057  \\
            203.0  0.17176996466755437  \\
            204.0  0.21039892071287483  \\
            205.0  0.2510154143778975  \\
            206.0  0.2898169784372966  \\
            207.0  0.3319145967300665  \\
            208.0  0.37414428096452823  \\
            209.0  0.4183206931283684  \\
            210.0  0.462891318183566  \\
            211.0  0.5071840476177338  \\
            212.0  0.5487511253636029  \\
            213.0  0.5899835137553773  \\
            214.0  0.6299724567803545  \\
            215.0  0.6690026388252036  \\
            216.0  0.705351438212168  \\
            217.0  0.7405442912925966  \\
            218.0  0.7758875285510587  \\
            219.0  0.8076618348447357  \\
            220.0  0.8384725883894171  \\
            221.0  0.8691619888023103  \\
            222.0  0.8974837790254814  \\
            223.0  0.9225428268700723  \\
            224.0  0.9450031709106218  \\
            225.0  0.9685303427123059  \\
            226.0  0.9905253372222244  \\
            227.0  1.0114990165398179  \\
            228.0  1.028479007046821  \\
            229.0  1.0458429273393943  \\
            230.0  1.061442756914741  \\
            231.0  1.0777222154155037  \\
            232.0  1.0914119009102001  \\
            233.0  1.1052414921812326  \\
            234.0  1.116962920234714  \\
            235.0  1.1265067072366524  \\
            236.0  1.135099849358075  \\
            237.0  1.1431523497406946  \\
            238.0  1.1497604532512318  \\
            239.0  1.1555509991269148  \\
            240.0  1.1602030937033547  \\
            241.0  1.1621115986257473  \\
            242.0  1.162251490184189  \\
            243.0  1.1618570748091008  \\
            244.0  1.161508175347564  \\
            245.0  1.1596898332611592  \\
            246.0  1.1546940205367722  \\
            247.0  1.150852301705998  \\
            248.0  1.1457446674304295  \\
            249.0  1.1384683215921845  \\
            250.0  1.129549151674926  \\
        }
        ;
    \addlegendentry {$\theta_1$}
    \addplot[color={rgb,1:red,0.949;green,0.5569;blue,0.1686}, name path={dedab142-d332-43e7-ac2f-4b039b57a752}, draw opacity={1.0}, line width={2}, dashed]
        table[row sep={\\}]
        {
            \\
            1.0  0.2  \\
            2.0  0.20014795282996192  \\
            3.0  0.2015892702791613  \\
            4.0  0.2045304988955912  \\
            5.0  0.20629802635064515  \\
            6.0  0.2099446717416698  \\
            7.0  0.2138006248954691  \\
            8.0  0.2186281467489561  \\
            9.0  0.22459386508769225  \\
            10.0  0.22939330864455088  \\
            11.0  0.23713790355167816  \\
            12.0  0.2439054505923744  \\
            13.0  0.2514137618449681  \\
            14.0  0.2629415136011364  \\
            15.0  0.27354101306426004  \\
            16.0  0.28543310267026956  \\
            17.0  0.298792504365207  \\
            18.0  0.3155287916474878  \\
            19.0  0.33356701919655113  \\
            20.0  0.3527934275601199  \\
            21.0  0.37578007440250727  \\
            22.0  0.4002368756250385  \\
            23.0  0.4262635892360739  \\
            24.0  0.4540431688387428  \\
            25.0  0.4838727309174653  \\
            26.0  0.5167826492437508  \\
            27.0  0.5519668617808965  \\
            28.0  0.588053994698271  \\
            29.0  0.6249524241731168  \\
            30.0  0.663541982949194  \\
            31.0  0.7018194162753675  \\
            32.0  0.7404503599503693  \\
            33.0  0.7778837908371161  \\
            34.0  0.8127863212507864  \\
            35.0  0.8466768314361579  \\
            36.0  0.8748698650163873  \\
            37.0  0.9027697552624105  \\
            38.0  0.9264315000129855  \\
            39.0  0.945924374025345  \\
            40.0  0.9629028681602438  \\
            41.0  0.9739748814949962  \\
            42.0  0.9848964706159017  \\
            43.0  0.9937184392096085  \\
            44.0  0.9980606527356798  \\
            45.0  0.9990878224065965  \\
            46.0  0.9953825026774266  \\
            47.0  0.9915038132737892  \\
            48.0  0.9842015097910931  \\
            49.0  0.9735136458423738  \\
            50.0  0.9586968409858955  \\
            51.0  0.9432410948716828  \\
            52.0  0.922808901312495  \\
            53.0  0.9017747709206567  \\
            54.0  0.8760247114692843  \\
            55.0  0.8478089251027414  \\
            56.0  0.818004858422634  \\
            57.0  0.7851674696206382  \\
            58.0  0.7487242197785741  \\
            59.0  0.7113490307997925  \\
            60.0  0.6697121759588444  \\
            61.0  0.626877981756683  \\
            62.0  0.5801265660403458  \\
            63.0  0.5319217376821287  \\
            64.0  0.4805633684660777  \\
            65.0  0.4263328306632688  \\
            66.0  0.3698727974055543  \\
            67.0  0.3095117840687658  \\
            68.0  0.24628455720822065  \\
            69.0  0.18136033185672965  \\
            70.0  0.11207893216634322  \\
            71.0  0.03881554879661123  \\
            72.0  -0.037402651242302375  \\
            73.0  -0.11586515882907368  \\
            74.0  -0.1995395433915428  \\
            75.0  -0.2839508407160594  \\
            76.0  -0.3682575035224065  \\
            77.0  -0.45420936321701577  \\
            78.0  -0.5382768135862344  \\
            79.0  -0.6210585715568504  \\
            80.0  -0.7004763025692795  \\
            81.0  -0.776712001227956  \\
            82.0  -0.8515343081096088  \\
            83.0  -0.9214892520443937  \\
            84.0  -0.9900464643373615  \\
            85.0  -1.0534256056697218  \\
            86.0  -1.1154071289844663  \\
            87.0  -1.171992416944711  \\
            88.0  -1.2273660586138282  \\
            89.0  -1.2807507727602123  \\
            90.0  -1.3328100171253912  \\
            91.0  -1.3814856543603447  \\
            92.0  -1.4287493892500025  \\
            93.0  -1.4727743384898777  \\
            94.0  -1.5148899902493198  \\
            95.0  -1.5559741920222443  \\
            96.0  -1.5940677229470062  \\
            97.0  -1.6312634329015123  \\
            98.0  -1.666807152491991  \\
            99.0  -1.701145488786691  \\
            100.0  -1.733080684550524  \\
            101.0  -1.7645363950731643  \\
            102.0  -1.7956372071487925  \\
            103.0  -1.8236141039029647  \\
            104.0  -1.851552191312872  \\
            105.0  -1.8751132928999443  \\
            106.0  -1.8965620921958326  \\
            107.0  -1.9193907800329624  \\
            108.0  -1.9393352645690394  \\
            109.0  -1.9590921544678312  \\
            110.0  -1.9761907268477568  \\
            111.0  -1.9915158082313058  \\
            112.0  -2.0044096664982916  \\
            113.0  -2.0147177742712237  \\
            114.0  -2.026204760941373  \\
            115.0  -2.0373534075240882  \\
            116.0  -2.044262048517907  \\
            117.0  -2.0537575315997136  \\
            118.0  -2.05965125045974  \\
            119.0  -2.065353670736444  \\
            120.0  -2.0687601566861575  \\
            121.0  -2.0693349636101055  \\
            122.0  -2.072066163568076  \\
            123.0  -2.070398015185167  \\
            124.0  -2.071838015087258  \\
            125.0  -2.069400313356204  \\
            126.0  -2.0652774520802  \\
            127.0  -2.0589310748191303  \\
            128.0  -2.0520338243308127  \\
            129.0  -2.043922617311372  \\
            130.0  -2.0359038379552086  \\
            131.0  -2.0263998778555314  \\
            132.0  -2.016561004709457  \\
            133.0  -2.0048760850481853  \\
            134.0  -1.992756234532112  \\
            135.0  -1.9794420052131425  \\
            136.0  -1.9633727499578038  \\
            137.0  -1.945651290554721  \\
            138.0  -1.926665741531999  \\
            139.0  -1.9055981126466544  \\
            140.0  -1.8819257878939737  \\
            141.0  -1.8591495850116222  \\
            142.0  -1.8345104783747905  \\
            143.0  -1.8092600504878071  \\
            144.0  -1.7816587778763533  \\
            145.0  -1.7524444303691145  \\
            146.0  -1.7228664385200716  \\
            147.0  -1.688684929778569  \\
            148.0  -1.6562895526996102  \\
            149.0  -1.6217506812387135  \\
            150.0  -1.5848973436753144  \\
            151.0  -1.5459660004189906  \\
            152.0  -1.5039164953461193  \\
            153.0  -1.4614902409015877  \\
            154.0  -1.4171333971580777  \\
            155.0  -1.3710010490386064  \\
            156.0  -1.3233369237635857  \\
            157.0  -1.2736828141239902  \\
            158.0  -1.2214386760793725  \\
            159.0  -1.167260818532911  \\
            160.0  -1.1111516944497613  \\
            161.0  -1.0506373508988947  \\
            162.0  -0.988160795962799  \\
            163.0  -0.9225268317059587  \\
            164.0  -0.8546415861941944  \\
            165.0  -0.7843151467382317  \\
            166.0  -0.7104904355204124  \\
            167.0  -0.6339901661737442  \\
            168.0  -0.5565930772259681  \\
            169.0  -0.4773143742598667  \\
            170.0  -0.3980910653622422  \\
            171.0  -0.3178105308863873  \\
            172.0  -0.24060114448414854  \\
            173.0  -0.1654682557226259  \\
            174.0  -0.09414335347680591  \\
            175.0  -0.023848473812658373  \\
            176.0  0.04067352660202277  \\
            177.0  0.10114936009143856  \\
            178.0  0.1600438260256045  \\
            179.0  0.21590888885467965  \\
            180.0  0.27051381001312214  \\
            181.0  0.32024601454184626  \\
            182.0  0.3688655171413613  \\
            183.0  0.4166246633613116  \\
            184.0  0.4594926774239902  \\
            185.0  0.5028304074982022  \\
            186.0  0.540569940491248  \\
            187.0  0.577560331603903  \\
            188.0  0.6105478870991597  \\
            189.0  0.6421585888022179  \\
            190.0  0.6706159041419734  \\
            191.0  0.6970500799111491  \\
            192.0  0.7197435094036865  \\
            193.0  0.7378551980813102  \\
            194.0  0.7542124843678307  \\
            195.0  0.7683047546682596  \\
            196.0  0.7789386775231583  \\
            197.0  0.7868957504121493  \\
            198.0  0.7931073460366627  \\
            199.0  0.7949246770230469  \\
            200.0  0.7936005050218228  \\
            201.0  0.7876767104576587  \\
            202.0  0.779403956411658  \\
            203.0  0.7677783320264244  \\
            204.0  0.7512274884741433  \\
            205.0  0.7330989643798353  \\
            206.0  0.7113177420516267  \\
            207.0  0.6869604421044824  \\
            208.0  0.6584226637486279  \\
            209.0  0.6282798814320983  \\
            210.0  0.597040090763739  \\
            211.0  0.563408595273468  \\
            212.0  0.5308882639127307  \\
            213.0  0.4979971109363854  \\
            214.0  0.46745131373259935  \\
            215.0  0.4366339640803057  \\
            216.0  0.4090669652652999  \\
            217.0  0.3819664646070613  \\
            218.0  0.3561757980325981  \\
            219.0  0.33644602672112117  \\
            220.0  0.3190420517392564  \\
            221.0  0.3023093379197373  \\
            222.0  0.2901469208248329  \\
            223.0  0.27831479583644775  \\
            224.0  0.2695358598517463  \\
            225.0  0.2625652018648371  \\
            226.0  0.2558374592161741  \\
            227.0  0.24875346147758706  \\
            228.0  0.24456479842304105  \\
            229.0  0.2424772766150704  \\
            230.0  0.24126154788867626  \\
            231.0  0.24009929568431823  \\
            232.0  0.24093024019387352  \\
            233.0  0.24460884483487844  \\
            234.0  0.2479604518592741  \\
            235.0  0.2529483639549797  \\
            236.0  0.2592313065893545  \\
            237.0  0.2660505277266568  \\
            238.0  0.2753968809006906  \\
            239.0  0.2826702483163156  \\
            240.0  0.29083210440505375  \\
            241.0  0.30121957178765885  \\
            242.0  0.31062819768866534  \\
            243.0  0.322404289926968  \\
            244.0  0.3351003414524155  \\
            245.0  0.34836689517394465  \\
            246.0  0.3628691781222756  \\
            247.0  0.37882112822957353  \\
            248.0  0.39517863432842987  \\
            249.0  0.41133787688550866  \\
            250.0  0.4269921433178552  \\
        }
        ;
    \addlegendentry {$\theta_2$}
    \addplot[color={rgb,1:red,0.349;green,0.6314;blue,0.3098}, name path={f469b151-d3fa-4a7d-8fad-4fb5970f9e40}, only marks, draw opacity={0.5}, line width={0}, solid, mark={*}, mark size={1.5 pt}, mark repeat={1}, mark options={color={rgb,1:red,0.0;green,0.0;blue,0.0}, draw opacity={0.5}, fill={rgb,1:red,0.349;green,0.6314;blue,0.3098}, fill opacity={0.5}, line width={0.0}, rotate={0}, solid}]
        table[row sep={\\}]
        {
            \\
            1.0  -0.10463079207969089  \\
            2.0  0.9709541278698491  \\
            3.0  0.3826891498147659  \\
            4.0  0.41641585082867305  \\
            5.0  0.659856718876079  \\
            6.0  0.37309108060081797  \\
            7.0  1.277659227214526  \\
            8.0  -0.19303706078989558  \\
            9.0  0.35580802260263017  \\
            10.0  0.4154877776578587  \\
            11.0  0.08475922338824993  \\
            12.0  -0.8452966573500975  \\
            13.0  0.3417902753703519  \\
            14.0  0.6069223705843  \\
            15.0  0.9297344880935607  \\
            16.0  0.05025408188589675  \\
            17.0  1.0287920904818406  \\
            18.0  1.0302846133497814  \\
            19.0  0.06568379158028509  \\
            20.0  0.8363837594455024  \\
            21.0  0.10036847257368126  \\
            22.0  0.4902930378946067  \\
            23.0  0.11581875588086227  \\
            24.0  1.2054264474448024  \\
            25.0  0.3269833139162768  \\
            26.0  -0.3934508922246426  \\
            27.0  1.2118855803455573  \\
            28.0  1.3977119591295202  \\
            29.0  0.805660614027555  \\
            30.0  1.0722720007441147  \\
            31.0  1.5340629219074977  \\
            32.0  -0.1529852672033296  \\
            33.0  0.4166049780884919  \\
            34.0  1.0565797344620997  \\
            35.0  1.0988549858273748  \\
            36.0  1.077992652514692  \\
            37.0  1.447692159945075  \\
            38.0  0.8868147842381108  \\
            39.0  0.38251603360326836  \\
            40.0  2.1849710314183057  \\
            41.0  1.4986435686209565  \\
            42.0  0.7461763966081623  \\
            43.0  1.7120526583371927  \\
            44.0  2.0358056971170706  \\
            45.0  0.3546316352120128  \\
            46.0  0.6201515860348237  \\
            47.0  0.9060009540870649  \\
            48.0  0.6330473980926208  \\
            49.0  0.6509003809063695  \\
            50.0  1.3056413524415742  \\
            51.0  0.7380761998265967  \\
            52.0  0.9115803251671858  \\
            53.0  0.7776977937893889  \\
            54.0  0.49400826563649414  \\
            55.0  0.8280253506114502  \\
            56.0  0.9714094042840613  \\
            57.0  1.183765627779868  \\
            58.0  0.07592202000607662  \\
            59.0  0.45914001558124645  \\
            60.0  0.9607804316607527  \\
            61.0  1.2336600316546493  \\
            62.0  1.1673807166198131  \\
            63.0  -0.9745249497644445  \\
            64.0  0.7214391034396188  \\
            65.0  -0.001432125820121699  \\
            66.0  -0.3036598110254172  \\
            67.0  0.5642033723340689  \\
            68.0  0.727415173156427  \\
            69.0  -0.6721386463273321  \\
            70.0  -0.5751161680881756  \\
            71.0  0.21352473008639106  \\
            72.0  0.20441010136123186  \\
            73.0  -0.9613505515301143  \\
            74.0  -0.9739260380969228  \\
            75.0  -0.2572256202840269  \\
            76.0  -1.1397447369682545  \\
            77.0  0.18301729638287145  \\
            78.0  0.8650257617696033  \\
            79.0  -0.5799973020685869  \\
            80.0  -0.3971568241477475  \\
            81.0  -1.6814343213445573  \\
            82.0  -1.6732638995861895  \\
            83.0  -2.5418885217799705  \\
            84.0  -0.27636187996809414  \\
            85.0  -0.9231920405064601  \\
            86.0  -1.0591269235261427  \\
            87.0  -0.7771359229045115  \\
            88.0  -0.5346802424307127  \\
            89.0  -0.6866123267619486  \\
            90.0  -2.0885354989893004  \\
            91.0  -1.461152671011556  \\
            92.0  -1.520009175098537  \\
            93.0  -2.215458310401248  \\
            94.0  -1.2366537650821887  \\
            95.0  -1.1562225264394663  \\
            96.0  -1.2065511655000467  \\
            97.0  -1.389958566399675  \\
            98.0  -2.177687870929832  \\
            99.0  -2.926971894410806  \\
            100.0  -2.371137091309751  \\
            101.0  -1.3013116850635476  \\
            102.0  -2.0431616700834043  \\
            103.0  -1.740967559636707  \\
            104.0  -1.969208196003724  \\
            105.0  -2.41270639445722  \\
            106.0  -1.119984053064603  \\
            107.0  -2.264662487848284  \\
            108.0  -1.8155388740812481  \\
            109.0  -1.9542334298197457  \\
            110.0  -2.668756471899233  \\
            111.0  -1.2461301485985095  \\
            112.0  -1.5590886485275022  \\
            113.0  -1.9473757719467382  \\
            114.0  -2.496220551341682  \\
            115.0  -2.400648370731073  \\
            116.0  -2.876140856309562  \\
            117.0  -1.8080737089351717  \\
            118.0  -2.4813346519059043  \\
            119.0  -1.76610190688205  \\
            120.0  -2.9816347420156153  \\
            121.0  -2.233223410114454  \\
            122.0  -2.0530288103186023  \\
            123.0  -2.058620473390166  \\
            124.0  -2.5703252305408393  \\
            125.0  -1.5986356245803166  \\
            126.0  -2.0799102465591672  \\
            127.0  -1.2661824824090342  \\
            128.0  -3.8820644794924544  \\
            129.0  -0.8546056103667701  \\
            130.0  -2.0753790412416167  \\
            131.0  -1.2568377704510545  \\
            132.0  -2.348198772850747  \\
            133.0  -2.407369200176149  \\
            134.0  -2.0633093642665985  \\
            135.0  -1.9166002561120732  \\
            136.0  -2.9100093244455425  \\
            137.0  -2.053455833714235  \\
            138.0  -2.117548361727147  \\
            139.0  -1.9029797235214092  \\
            140.0  -1.536492888762661  \\
            141.0  -2.898355421550715  \\
            142.0  -2.214935322062658  \\
            143.0  -3.349044932408097  \\
            144.0  -1.4775331187598062  \\
            145.0  -2.1755224474791066  \\
            146.0  -2.3125359128953202  \\
            147.0  -0.8362768272825517  \\
            148.0  -1.9011186253621393  \\
            149.0  -2.4106296759824364  \\
            150.0  -1.4477499239119813  \\
            151.0  -1.6219099226531968  \\
            152.0  -1.1457375473677036  \\
            153.0  -1.8212811814703276  \\
            154.0  -1.4832783831944556  \\
            155.0  -1.055653924478813  \\
            156.0  -1.017627623061574  \\
            157.0  -0.680321860896667  \\
            158.0  -1.8741281396154394  \\
            159.0  -0.959974588389475  \\
            160.0  -0.8943162137694862  \\
            161.0  0.4744610650331067  \\
            162.0  -1.2372107490081337  \\
            163.0  -0.822216892536155  \\
            164.0  0.08394014227663427  \\
            165.0  -0.8997289730400873  \\
            166.0  -0.323576080571593  \\
            167.0  -1.1195279781672451  \\
            168.0  -0.9291308289479432  \\
            169.0  -0.7534991634579427  \\
            170.0  -0.1915876143961938  \\
            171.0  -0.7889922531747685  \\
            172.0  0.05624533461853887  \\
            173.0  -0.21366226029138208  \\
            174.0  -0.709217945675594  \\
            175.0  -0.32474821943429893  \\
            176.0  0.11549861280817  \\
            177.0  0.15324606369064422  \\
            178.0  -0.5838885445064906  \\
            179.0  -0.28702774061416314  \\
            180.0  0.6306369298733958  \\
            181.0  1.0911798032965283  \\
            182.0  1.0038739911907122  \\
            183.0  0.7249651502556405  \\
            184.0  -1.2056104668213805  \\
            185.0  0.6713789385525382  \\
            186.0  0.2038464317135404  \\
            187.0  1.1009422161169056  \\
            188.0  1.0244883117754235  \\
            189.0  1.0050564588618194  \\
            190.0  0.9600955551163666  \\
            191.0  1.4151152616000602  \\
            192.0  0.798267677108591  \\
            193.0  0.023530418089968586  \\
            194.0  0.768019165947492  \\
            195.0  0.012259833547316301  \\
            196.0  1.307046099642534  \\
            197.0  0.769818918405917  \\
            198.0  1.9583321872685426  \\
            199.0  0.9799215451099285  \\
            200.0  0.5454496843225098  \\
            201.0  1.3376672541521293  \\
            202.0  -0.06211304160335973  \\
            203.0  -0.15118658328258594  \\
            204.0  0.581362995283576  \\
            205.0  0.8988238078265551  \\
            206.0  -0.09180328054864095  \\
            207.0  0.7800519163536174  \\
            208.0  0.206726114429426  \\
            209.0  1.672690567205215  \\
            210.0  -0.005868794724597559  \\
            211.0  0.8211341545860765  \\
            212.0  0.08252195686467967  \\
            213.0  0.3123176837795772  \\
            214.0  0.7645364139490456  \\
            215.0  0.6110715264079557  \\
            216.0  -0.022368096496587386  \\
            217.0  -0.17690173949807964  \\
            218.0  -0.8896881097911429  \\
            219.0  0.45571695558439634  \\
            220.0  0.8220430928282134  \\
            221.0  -0.6222498497117714  \\
            222.0  -0.0977104706261217  \\
            223.0  0.7057951763840716  \\
            224.0  -0.0779363561885717  \\
            225.0  2.120245049593542  \\
            226.0  0.683973437771664  \\
            227.0  0.23929765220699353  \\
            228.0  1.4820770392265583  \\
            229.0  0.2451633723494544  \\
            230.0  -0.24854178917955855  \\
            231.0  0.9691593285067869  \\
            232.0  -0.9425277370733105  \\
            233.0  0.3384207654328142  \\
            234.0  0.9901574941768617  \\
            235.0  0.6455508571123019  \\
            236.0  0.76087010639677  \\
            237.0  -0.3708762578591095  \\
            238.0  0.07408114749492126  \\
            239.0  0.4362943209412486  \\
            240.0  0.5846691681922602  \\
            241.0  -0.02201894994977066  \\
            242.0  0.5258171736388496  \\
            243.0  -0.07316167779978161  \\
            244.0  -0.16833891147331403  \\
            245.0  0.9017184566057852  \\
            246.0  -0.1381641450105242  \\
            247.0  1.2946107805691316  \\
            248.0  0.5608361674494352  \\
            249.0  -0.27132892612125253  \\
            250.0  -0.3063152021371821  \\
        }
        ;
    \addlegendentry {$y$}
\end{axis}
\end{tikzpicture}

        \includegraphics{contents/05-experiments/plots/nlds/03-pendulum_example_angles.pdf}
    }
    \caption{Simulated evolution of the angles $\theta_1$ and $\theta_2$ and corresponding measurements $y = \theta_2 + \omega$ at each time step index $t$.
    }
    \label{fig:sim:pendulum_example_angles}
  \end{subfigure}
  \hfill
  \begin{subfigure}[t]{0.475\textwidth}
    \centering
    \resizebox{\textwidth}{!}{
        % % Recommended preamble:
% \usetikzlibrary{arrows.meta}
% \usetikzlibrary{backgrounds}
% \usepgfplotslibrary{patchplots}
% \usepgfplotslibrary{fillbetween}
% \pgfplotsset{%
%     layers/standard/.define layer set={%
%         background,axis background,axis grid,axis ticks,axis lines,axis tick labels,pre main,main,axis descriptions,axis foreground%
%     }{
%         grid style={/pgfplots/on layer=axis grid},%
%         tick style={/pgfplots/on layer=axis ticks},%
%         axis line style={/pgfplots/on layer=axis lines},%
%         label style={/pgfplots/on layer=axis descriptions},%
%         legend style={/pgfplots/on layer=axis descriptions},%
%         title style={/pgfplots/on layer=axis descriptions},%
%         colorbar style={/pgfplots/on layer=axis descriptions},%
%         ticklabel style={/pgfplots/on layer=axis tick labels},%
%         axis background@ style={/pgfplots/on layer=axis background},%
%         3d box foreground style={/pgfplots/on layer=axis foreground},%
%     },
% }

\begin{tikzpicture}[/tikz/background rectangle/.style={fill={rgb,1:red,1.0;green,1.0;blue,1.0}, fill opacity={1.0}, draw opacity={1.0}}, show background rectangle]
\begin{axis}[point meta max={nan}, point meta min={nan}, legend cell align={left}, legend columns={1}, title={}, title style={at={{(0.5,1)}}, anchor={south}, font={{\fontsize{18 pt}{23.400000000000002 pt}\selectfont}}, color={rgb,1:red,0.0;green,0.0;blue,0.0}, draw opacity={1.0}, rotate={0.0}, align={center}}, legend style={color={rgb,1:red,0.0;green,0.0;blue,0.0}, draw opacity={1.0}, line width={1}, solid, fill={rgb,1:red,1.0;green,1.0;blue,1.0}, fill opacity={1.0}, text opacity={1.0}, font={{\fontsize{14 pt}{18.2 pt}\selectfont}}, text={rgb,1:red,0.0;green,0.0;blue,0.0}, cells={anchor={center}}, at={(0.02, 0.02)}, anchor={south west}}, axis background/.style={fill={rgb,1:red,1.0;green,1.0;blue,1.0}, opacity={1.0}}, anchor={north west}, xshift={1.0mm}, yshift={-1.0mm}, width={99.6mm}, height={74.2mm}, scaled x ticks={false}, xlabel={Time step index}, x tick style={color={rgb,1:red,0.0;green,0.0;blue,0.0}, opacity={1.0}}, x tick label style={color={rgb,1:red,0.0;green,0.0;blue,0.0}, opacity={1.0}, rotate={0}}, xlabel style={at={(ticklabel cs:0.5)}, anchor=near ticklabel, at={{(ticklabel cs:0.5)}}, anchor={near ticklabel}, font={{\fontsize{16 pt}{20.8 pt}\selectfont}}, color={rgb,1:red,0.0;green,0.0;blue,0.0}, draw opacity={1.0}, rotate={0.0}}, xmajorgrids={true}, xmin={-6.469999999999999}, xmax={257.47}, xticklabels={{$0$,$50$,$100$,$150$,$200$,$250$}}, xtick={{0.0,50.0,100.0,150.0,200.0,250.0}}, xtick align={inside}, xticklabel style={font={{\fontsize{14 pt}{18.2 pt}\selectfont}}, color={rgb,1:red,0.0;green,0.0;blue,0.0}, draw opacity={1.0}, rotate={0.0}}, x grid style={color={rgb,1:red,0.0;green,0.0;blue,0.0}, draw opacity={0.1}, line width={0.5}, solid}, axis x line*={left}, x axis line style={color={rgb,1:red,0.0;green,0.0;blue,0.0}, draw opacity={1.0}, line width={1}, solid}, scaled y ticks={false}, ylabel={Angular velocity (radians / s)}, y tick style={color={rgb,1:red,0.0;green,0.0;blue,0.0}, opacity={1.0}}, y tick label style={color={rgb,1:red,0.0;green,0.0;blue,0.0}, opacity={1.0}, rotate={0}}, ylabel style={at={(ticklabel cs:0.5)}, anchor=near ticklabel, at={{(ticklabel cs:0.5)}}, anchor={near ticklabel}, font={{\fontsize{16 pt}{20.8 pt}\selectfont}}, color={rgb,1:red,0.0;green,0.0;blue,0.0}, draw opacity={1.0}, rotate={0.0}}, ymajorgrids={true}, ymin={-9.091178336630337}, ymax={8.437980430204165}, yticklabels={{$-7.5$,$-5.0$,$-2.5$,$0.0$,$2.5$,$5.0$,$7.5$}}, ytick={{-7.5,-5.0,-2.5,0.0,2.5,5.0,7.5}}, ytick align={inside}, yticklabel style={font={{\fontsize{14 pt}{18.2 pt}\selectfont}}, color={rgb,1:red,0.0;green,0.0;blue,0.0}, draw opacity={1.0}, rotate={0.0}}, y grid style={color={rgb,1:red,0.0;green,0.0;blue,0.0}, draw opacity={0.1}, line width={0.5}, solid}, axis y line*={left}, y axis line style={color={rgb,1:red,0.0;green,0.0;blue,0.0}, draw opacity={1.0}, line width={1}, solid}, colorbar={false}]
    \addplot[color={rgb,1:red,0.8824;green,0.3412;blue,0.349}, name path={f0c7c1c3-5970-4636-bea9-2514b67adc33}, draw opacity={1.0}, line width={2}, solid]
        table[row sep={\\}]
        {
            \\
            1.0  0.0  \\
            2.0  -0.11469369582103466  \\
            3.0  -0.2303421162258017  \\
            4.0  -0.34622588754774697  \\
            5.0  -0.46344571084153396  \\
            6.0  -0.5794872789774292  \\
            7.0  -0.6963370542011961  \\
            8.0  -0.8156115313764584  \\
            9.0  -0.9352072738369449  \\
            10.0  -1.057681152924293  \\
            11.0  -1.1791548565830803  \\
            12.0  -1.304355450806594  \\
            13.0  -1.4317670124279585  \\
            14.0  -1.5625859493065333  \\
            15.0  -1.697217167187706  \\
            16.0  -1.8348106276619283  \\
            17.0  -1.976880750611236  \\
            18.0  -2.1234380296882645  \\
            19.0  -2.2744163454539152  \\
            20.0  -2.432478143572495  \\
            21.0  -2.596714785006003  \\
            22.0  -2.767370975788402  \\
            23.0  -2.942598921923271  \\
            24.0  -3.1221088959503267  \\
            25.0  -3.3012568957356594  \\
            26.0  -3.477988206795535  \\
            27.0  -3.646343615865239  \\
            28.0  -3.796235689886352  \\
            29.0  -3.9181798967490584  \\
            30.0  -4.0078104858256935  \\
            31.0  -4.059617677185862  \\
            32.0  -4.074935577674803  \\
            33.0  -4.0565067367906575  \\
            34.0  -4.0109749830529005  \\
            35.0  -3.9489784589505286  \\
            36.0  -3.8785966431243124  \\
            37.0  -3.801695342721828  \\
            38.0  -3.725838666521876  \\
            39.0  -3.6505218296612236  \\
            40.0  -3.5789743666829823  \\
            41.0  -3.509147549986082  \\
            42.0  -3.4410999093027876  \\
            43.0  -3.3775101696388115  \\
            44.0  -3.3161891965565156  \\
            45.0  -3.2552427222880946  \\
            46.0  -3.194593690822886  \\
            47.0  -3.1338896735472765  \\
            48.0  -3.0701416467745415  \\
            49.0  -3.005679726541149  \\
            50.0  -2.937125581503674  \\
            51.0  -2.8659363977496928  \\
            52.0  -2.7890442999895275  \\
            53.0  -2.707857389605486  \\
            54.0  -2.6197472389448526  \\
            55.0  -2.522754061337334  \\
            56.0  -2.417291883121855  \\
            57.0  -2.305007970287432  \\
            58.0  -2.185496186863027  \\
            59.0  -2.056516137755376  \\
            60.0  -1.9201726788897706  \\
            61.0  -1.772667571587761  \\
            62.0  -1.6148552000751337  \\
            63.0  -1.4458516486359108  \\
            64.0  -1.265799397707936  \\
            65.0  -1.0755471481924308  \\
            66.0  -0.8750764684641791  \\
            67.0  -0.6605971457996626  \\
            68.0  -0.4362305210966639  \\
            69.0  -0.20059806406267858  \\
            70.0  0.04282335289058727  \\
            71.0  0.289602988801137  \\
            72.0  0.5339377555163622  \\
            73.0  0.7636677465158925  \\
            74.0  0.9641722090768065  \\
            75.0  1.110758458465914  \\
            76.0  1.1926014019147961  \\
            77.0  1.20004893073642  \\
            78.0  1.138046868721077  \\
            79.0  1.0225922526525066  \\
            80.0  0.8733823366680578  \\
            81.0  0.7046542615625162  \\
            82.0  0.5326883991609962  \\
            83.0  0.3649348239612217  \\
            84.0  0.20473725437382617  \\
            85.0  0.054054865382906124  \\
            86.0  -0.08399648608444148  \\
            87.0  -0.2105883679640817  \\
            88.0  -0.32405488662615567  \\
            89.0  -0.42684551255963554  \\
            90.0  -0.5196736443082666  \\
            91.0  -0.6013995291165614  \\
            92.0  -0.6746055262597548  \\
            93.0  -0.7375306396883535  \\
            94.0  -0.7915977928354512  \\
            95.0  -0.8382027840490303  \\
            96.0  -0.8774157889153973  \\
            97.0  -0.9078879886275556  \\
            98.0  -0.931972950125969  \\
            99.0  -0.9493664532617724  \\
            100.0  -0.9585878607249019  \\
            101.0  -0.9636826303100203  \\
            102.0  -0.9623529918080015  \\
            103.0  -0.9561537369931289  \\
            104.0  -0.9443971125242926  \\
            105.0  -0.9297409461996058  \\
            106.0  -0.9092722175093794  \\
            107.0  -0.8844098441224203  \\
            108.0  -0.85657592599216  \\
            109.0  -0.8238885307175626  \\
            110.0  -0.7868764777161443  \\
            111.0  -0.7494948212163158  \\
            112.0  -0.7090883558164797  \\
            113.0  -0.6624252836993054  \\
            114.0  -0.6174498364874885  \\
            115.0  -0.5706932657435145  \\
            116.0  -0.5222594150732089  \\
            117.0  -0.4708386355714777  \\
            118.0  -0.4181794846142812  \\
            119.0  -0.36364162480352447  \\
            120.0  -0.31072177965173414  \\
            121.0  -0.2570897457758618  \\
            122.0  -0.20237576172661356  \\
            123.0  -0.14518045840579788  \\
            124.0  -0.09059790570643785  \\
            125.0  -0.034560065088884005  \\
            126.0  0.020466073935859492  \\
            127.0  0.07597135961211722  \\
            128.0  0.13027823684360296  \\
            129.0  0.18243955361562558  \\
            130.0  0.23199571322438856  \\
            131.0  0.28236958013568825  \\
            132.0  0.33087944872712377  \\
            133.0  0.37677278076347753  \\
            134.0  0.42123186089103554  \\
            135.0  0.4624486591900846  \\
            136.0  0.5022371009479519  \\
            137.0  0.5387689351037152  \\
            138.0  0.5713553821175097  \\
            139.0  0.6001543759530066  \\
            140.0  0.627386020503483  \\
            141.0  0.6508543782395393  \\
            142.0  0.6690645320034623  \\
            143.0  0.6835297792079897  \\
            144.0  0.6952302037502394  \\
            145.0  0.6996577906377953  \\
            146.0  0.7003211334145267  \\
            147.0  0.6943886146116728  \\
            148.0  0.6835353915541688  \\
            149.0  0.6678820408001007  \\
            150.0  0.6425491457371817  \\
            151.0  0.6119406246300322  \\
            152.0  0.5748006278753754  \\
            153.0  0.5278668240022439  \\
            154.0  0.472418773835481  \\
            155.0  0.4087142403161307  \\
            156.0  0.33719049482166324  \\
            157.0  0.25635911613467993  \\
            158.0  0.16691113993083065  \\
            159.0  0.06622415047083736  \\
            160.0  -0.04258514258797661  \\
            161.0  -0.16000964029462988  \\
            162.0  -0.2851556753034746  \\
            163.0  -0.4179850247300051  \\
            164.0  -0.5513021131417009  \\
            165.0  -0.6820349048935124  \\
            166.0  -0.7982902213260555  \\
            167.0  -0.885208165545819  \\
            168.0  -0.9293434112857475  \\
            169.0  -0.9211110364685334  \\
            170.0  -0.8551005437280914  \\
            171.0  -0.7340531856794583  \\
            172.0  -0.5667116258029488  \\
            173.0  -0.36692010398802377  \\
            174.0  -0.14997229523902222  \\
            175.0  0.07450165508272173  \\
            176.0  0.3000633804290153  \\
            177.0  0.5221459537932684  \\
            178.0  0.7365350235740147  \\
            179.0  0.9425224245605714  \\
            180.0  1.1399490663766303  \\
            181.0  1.326271648929363  \\
            182.0  1.5060743997961685  \\
            183.0  1.6753874679808116  \\
            184.0  1.8356245961742579  \\
            185.0  1.9868195372769104  \\
            186.0  2.129344414101183  \\
            187.0  2.265220685841997  \\
            188.0  2.3929813278935828  \\
            189.0  2.5131067477954017  \\
            190.0  2.625716922478407  \\
            191.0  2.7334493393731387  \\
            192.0  2.835161608088676  \\
            193.0  2.933109759850051  \\
            194.0  3.026368080641237  \\
            195.0  3.115209533744372  \\
            196.0  3.2019352416637243  \\
            197.0  3.2878986103557115  \\
            198.0  3.371599343998376  \\
            199.0  3.4549262427720464  \\
            200.0  3.539671998211015  \\
            201.0  3.625609024602902  \\
            202.0  3.7108626739725126  \\
            203.0  3.7996501920242434  \\
            204.0  3.8877343958252735  \\
            205.0  3.97860174634788  \\
            206.0  4.063263714712888  \\
            207.0  4.142663017131904  \\
            208.0  4.207695198407714  \\
            209.0  4.2548180778073705  \\
            210.0  4.270888302601859  \\
            211.0  4.254087710333332  \\
            212.0  4.198905084974159  \\
            213.0  4.108152801288326  \\
            214.0  3.9874763487595306  \\
            215.0  3.843847338778264  \\
            216.0  3.6836700422482616  \\
            217.0  3.516386397788761  \\
            218.0  3.3443798420415534  \\
            219.0  3.1747687835982736  \\
            220.0  3.006819409685668  \\
            221.0  2.8449443994113945  \\
            222.0  2.6869911222979925  \\
            223.0  2.5331933692768005  \\
            224.0  2.384730291186175  \\
            225.0  2.242528718754081  \\
            226.0  2.101713957509452  \\
            227.0  1.9647873311787711  \\
            228.0  1.8309750220067058  \\
            229.0  1.7001239173215992  \\
            230.0  1.57122036112736  \\
            231.0  1.4438909797422157  \\
            232.0  1.3182297047322182  \\
            233.0  1.1930137005877524  \\
            234.0  1.0712496435246563  \\
            235.0  0.9474503706248942  \\
            236.0  0.8259511836433248  \\
            237.0  0.7032309391732701  \\
            238.0  0.581471378400904  \\
            239.0  0.4598109906604906  \\
            240.0  0.33804443189028877  \\
            241.0  0.21602027295401013  \\
            242.0  0.08998181303614222  \\
            243.0  -0.03502951670441838  \\
            244.0  -0.1594907498015014  \\
            245.0  -0.28551984037612227  \\
            246.0  -0.414659482487976  \\
            247.0  -0.5439620929481536  \\
            248.0  -0.675378840079946  \\
            249.0  -0.8097408270664973  \\
            250.0  -0.9462927962058011  \\
        }
        ;
    \addlegendentry {$\dot{\theta}_1$}
    \addplot[color={rgb,1:red,0.4627;green,0.7176;blue,0.698}, name path={58f00d7e-a4d1-4c79-ad8d-27a1a726d3e1}, draw opacity={1.0}, line width={2}, dashed]
        table[row sep={\\}]
        {
            \\
            1.0  0.0  \\
            2.0  0.06561075103066388  \\
            3.0  0.1312796182550023  \\
            4.0  0.19627365691735038  \\
            5.0  0.26432711134394754  \\
            6.0  0.33527133844672374  \\
            7.0  0.4094543110334141  \\
            8.0  0.4867704634141591  \\
            9.0  0.5703937224310937  \\
            10.0  0.6584554722372764  \\
            11.0  0.7517708614092733  \\
            12.0  0.8531003032023541  \\
            13.0  0.9632271428249128  \\
            14.0  1.0798769294385722  \\
            15.0  1.2058919166190711  \\
            16.0  1.343199661686743  \\
            17.0  1.490797060051735  \\
            18.0  1.6535090757681468  \\
            19.0  1.8274483865624656  \\
            20.0  2.01559971411268  \\
            21.0  2.2176974958440776  \\
            22.0  2.4339240040639845  \\
            23.0  2.6640346647000595  \\
            24.0  2.9008089313079473  \\
            25.0  3.141139863150367  \\
            26.0  3.3776072588711576  \\
            27.0  3.5951815418243966  \\
            28.0  3.777319356221099  \\
            29.0  3.9023385896187923  \\
            30.0  3.9524731695102524  \\
            31.0  3.9256759969695363  \\
            32.0  3.816549689291597  \\
            33.0  3.6373680132019506  \\
            34.0  3.4007247890190553  \\
            35.0  3.12244208523993  \\
            36.0  2.814472240896399  \\
            37.0  2.4944263956161334  \\
            38.0  2.168047616206861  \\
            39.0  1.8405227736522722  \\
            40.0  1.516417727930466  \\
            41.0  1.1931080204160158  \\
            42.0  0.8735547990046896  \\
            43.0  0.5607182600666099  \\
            44.0  0.2509247816481534  \\
            45.0  -0.05398340773753719  \\
            46.0  -0.3569451255042136  \\
            47.0  -0.6547216296226264  \\
            48.0  -0.9507713927221794  \\
            49.0  -1.2428701495923284  \\
            50.0  -1.5311968770831084  \\
            51.0  -1.816411853667134  \\
            52.0  -2.0960558133786735  \\
            53.0  -2.3717274032691384  \\
            54.0  -2.644780734060081  \\
            55.0  -2.9151272987384482  \\
            56.0  -3.1817060081817594  \\
            57.0  -3.4449826552660596  \\
            58.0  -3.705412242101849  \\
            59.0  -3.962385224069968  \\
            60.0  -4.219314001250959  \\
            61.0  -4.47482920206865  \\
            62.0  -4.733531927353915  \\
            63.0  -4.9946694157441645  \\
            64.0  -5.260423722653354  \\
            65.0  -5.5343177027035155  \\
            66.0  -5.817506235378878  \\
            67.0  -6.109567485921137  \\
            68.0  -6.413906676608645  \\
            69.0  -6.733332064431985  \\
            70.0  -7.064897124714967  \\
            71.0  -7.405863919317457  \\
            72.0  -7.744340231975508  \\
            73.0  -8.062853179624197  \\
            74.0  -8.3341111997839  \\
            75.0  -8.521705901803795  \\
            76.0  -8.595070069644455  \\
            77.0  -8.546426495926113  \\
            78.0  -8.381546245175317  \\
            79.0  -8.129364732959543  \\
            80.0  -7.81936820660994  \\
            81.0  -7.487728846558257  \\
            82.0  -7.149565921528105  \\
            83.0  -6.8210924916960485  \\
            84.0  -6.5098692402140745  \\
            85.0  -6.215665453313591  \\
            86.0  -5.938729383729261  \\
            87.0  -5.678925805680125  \\
            88.0  -5.431336800941772  \\
            89.0  -5.199318938197323  \\
            90.0  -4.979391464605305  \\
            91.0  -4.768872743798065  \\
            92.0  -4.566782936568377  \\
            93.0  -4.371364055654769  \\
            94.0  -4.184131391007341  \\
            95.0  -4.00234057167186  \\
            96.0  -3.8240381254676974  \\
            97.0  -3.650947152903731  \\
            98.0  -3.482865011314678  \\
            99.0  -3.3176892932289577  \\
            100.0  -3.154200096979302  \\
            101.0  -2.9949660369282247  \\
            102.0  -2.8373416922099004  \\
            103.0  -2.6828829871194553  \\
            104.0  -2.529833191418322  \\
            105.0  -2.379500240021144  \\
            106.0  -2.2320238723964065  \\
            107.0  -2.086229962975256  \\
            108.0  -1.9437553815447037  \\
            109.0  -1.8012539474322364  \\
            110.0  -1.6623507297172555  \\
            111.0  -1.5245980788358677  \\
            112.0  -1.3859655835879996  \\
            113.0  -1.2506561931471534  \\
            114.0  -1.1153285839264377  \\
            115.0  -0.983698691921002  \\
            116.0  -0.8532189527583628  \\
            117.0  -0.722805080008211  \\
            118.0  -0.5949897926967866  \\
            119.0  -0.4671866831830483  \\
            120.0  -0.3391328172689367  \\
            121.0  -0.21308542095941388  \\
            122.0  -0.08851972662962355  \\
            123.0  0.0387124595524646  \\
            124.0  0.16523340483470755  \\
            125.0  0.2894474505138102  \\
            126.0  0.4168592959116265  \\
            127.0  0.5402388207830539  \\
            128.0  0.6651126774941646  \\
            129.0  0.788330010521965  \\
            130.0  0.9137983716423885  \\
            131.0  1.0410317047901698  \\
            132.0  1.1686650076757379  \\
            133.0  1.296451452556609  \\
            134.0  1.4262603365345088  \\
            135.0  1.557056392086338  \\
            136.0  1.6889028827888348  \\
            137.0  1.8218248991755708  \\
            138.0  1.9552259022524296  \\
            139.0  2.090463705768385  \\
            140.0  2.228061786905526  \\
            141.0  2.36703618189395  \\
            142.0  2.508776556874719  \\
            143.0  2.652551882410584  \\
            144.0  2.7995475296934553  \\
            145.0  2.9510960432372224  \\
            146.0  3.10351798726883  \\
            147.0  3.256743559742156  \\
            148.0  3.4159695583616343  \\
            149.0  3.5782040423159587  \\
            150.0  3.7460732596746493  \\
            151.0  3.918024451108469  \\
            152.0  4.096886587721988  \\
            153.0  4.280213609540859  \\
            154.0  4.471610992179753  \\
            155.0  4.670221709180633  \\
            156.0  4.8780711251915205  \\
            157.0  5.0948654704549226  \\
            158.0  5.324664761052817  \\
            159.0  5.565564001673021  \\
            160.0  5.819849007656344  \\
            161.0  6.088478376489354  \\
            162.0  6.368177750254779  \\
            163.0  6.659956208982515  \\
            164.0  6.9542225929592725  \\
            165.0  7.241767328263497  \\
            166.0  7.507701487529667  \\
            167.0  7.728875212046091  \\
            168.0  7.879303968843611  \\
            169.0  7.9418721632182825  \\
            170.0  7.901730305319663  \\
            171.0  7.772395907780991  \\
            172.0  7.561962498871173  \\
            173.0  7.302393582482706  \\
            174.0  7.014605687535104  \\
            175.0  6.712826236032062  \\
            176.0  6.4119168643842395  \\
            177.0  6.116155337059315  \\
            178.0  5.8271941921364085  \\
            179.0  5.54740771340651  \\
            180.0  5.2738037946478755  \\
            181.0  5.007390415398314  \\
            182.0  4.746291454663917  \\
            183.0  4.485941677233922  \\
            184.0  4.226225855609385  \\
            185.0  3.96901493560427  \\
            186.0  3.709173772185523  \\
            187.0  3.4465384813885787  \\
            188.0  3.184134048294976  \\
            189.0  2.9149375537784925  \\
            190.0  2.644596415004213  \\
            191.0  2.3693654792815  \\
            192.0  2.0891728006074874  \\
            193.0  1.802022655845248  \\
            194.0  1.5115798270222367  \\
            195.0  1.2151759294141014  \\
            196.0  0.9133632588036125  \\
            197.0  0.6075039193202733  \\
            198.0  0.29628781956877487  \\
            199.0  -0.021613015527838203  \\
            200.0  -0.3406149977135845  \\
            201.0  -0.6684943380283683  \\
            202.0  -0.998701022902418  \\
            203.0  -1.3318195420544228  \\
            204.0  -1.6643756710447528  \\
            205.0  -1.9919470396936434  \\
            206.0  -2.310657290246491  \\
            207.0  -2.6110418398497526  \\
            208.0  -2.8796284621845945  \\
            209.0  -3.1008507054182384  \\
            210.0  -3.2597184808118365  \\
            211.0  -3.3429667173313895  \\
            212.0  -3.3424134513488633  \\
            213.0  -3.264217928912508  \\
            214.0  -3.1199559592489505  \\
            215.0  -2.925655131831742  \\
            216.0  -2.6993169099554386  \\
            217.0  -2.4570650581012967  \\
            218.0  -2.207532193260275  \\
            219.0  -1.9576134007744703  \\
            220.0  -1.7190529140784152  \\
            221.0  -1.490382841561922  \\
            222.0  -1.2729806289529977  \\
            223.0  -1.0698212671380896  \\
            224.0  -0.8785263520222449  \\
            225.0  -0.7016153415803198  \\
            226.0  -0.535417863640908  \\
            227.0  -0.38064767159675206  \\
            228.0  -0.2364346627083335  \\
            229.0  -0.10146123052597221  \\
            230.0  0.026031165626789754  \\
            231.0  0.14551511333959521  \\
            232.0  0.2567069320136211  \\
            233.0  0.36028062440598724  \\
            234.0  0.4587131316164322  \\
            235.0  0.5537273181384772  \\
            236.0  0.6424235500199472  \\
            237.0  0.7290541862111373  \\
            238.0  0.8113479071990206  \\
            239.0  0.8918138261653396  \\
            240.0  0.9706160354608213  \\
            241.0  1.0477836054067478  \\
            242.0  1.1246937086594326  \\
            243.0  1.204685335173556  \\
            244.0  1.2850033259513665  \\
            245.0  1.3660964526189672  \\
            246.0  1.4502672074732348  \\
            247.0  1.538269701526228  \\
            248.0  1.6312182830673907  \\
            249.0  1.7299500664744036  \\
            250.0  1.8316593263389964  \\
        }
        ;
    \addlegendentry {$\dot{\theta}_2$}
\end{axis}
\end{tikzpicture}

        \includegraphics{contents/05-experiments/plots/nlds/03-pendulum_example_velocities.pdf}
    }
    \caption{Simulated evolution of the angular velocities $\dot{\theta}_1$ and $\dot{\theta}_2$ at each time step index $t$.}
    \label{fig:sim:pendulum_example_velocities}
  \end{subfigure}
  \caption{Simulated evolution of the double pendulum system state $s_t = (\theta_1, \theta_2, \dot{\theta}_1, \dot{\theta}_2)_t$ using the Runge-Kutta (RK4) method with a starting point of $s_1 = (1.2, 0.2, 0.0, 0.0)$ in discrete time steps.
    The time difference between measurements is set to be $0.01$ seconds.
    The masses $m_1$ and $m_2$ are set to be $14.715$ and $4.905$ respectively.
    The lengths $l_1$ and $l_2$ are equal and set to be $1.0$.
    The measurements $y$ only include the second component of the state vector.
    Other components of the state vector are not observed.
    The state transition noise $v$ is distributed according to the multivariate Normal
    distribution $\mathcal{N}(0, \Sigma)$, where the covariance matrix $\Sigma$ is a diagonal
    matrix with $10^{-6}$ values on the diagonal.
    The measurement noise signal $w$ is distributed according to the Normal distribution
    $\mathcal{N}(0, \Omega)$, where the variance $\Omega$ is set to be $0.3$.
    The figure shows the first $250$ time steps of the simulation.
  }
  \label{fig:sim:pendulum_example_states}
\end{figure}

%\begin{figure}
%\centering
%\resizebox{0.95\textwidth}{!}{% Recommended preamble:
% \usetikzlibrary{arrows.meta}
% \usetikzlibrary{backgrounds}
% \usepgfplotslibrary{patchplots}
% \usepgfplotslibrary{fillbetween}
% \pgfplotsset{%
%     layers/standard/.define layer set={%
%         background,axis background,axis grid,axis ticks,axis lines,axis tick labels,pre main,main,axis descriptions,axis foreground%
%     }{
%         grid style={/pgfplots/on layer=axis grid},%
%         tick style={/pgfplots/on layer=axis ticks},%
%         axis line style={/pgfplots/on layer=axis lines},%
%         label style={/pgfplots/on layer=axis descriptions},%
%         legend style={/pgfplots/on layer=axis descriptions},%
%         title style={/pgfplots/on layer=axis descriptions},%
%         colorbar style={/pgfplots/on layer=axis descriptions},%
%         ticklabel style={/pgfplots/on layer=axis tick labels},%
%         axis background@ style={/pgfplots/on layer=axis background},%
%         3d box foreground style={/pgfplots/on layer=axis foreground},%
%     },
% }

\begin{tikzpicture}[/tikz/background rectangle/.style={fill={rgb,1:red,1.0;green,1.0;blue,1.0}, fill opacity={1.0}, draw opacity={1.0}}, show background rectangle]
\begin{axis}[point meta max={nan}, point meta min={nan}, legend cell align={left}, legend columns={1}, title={}, title style={at={{(0.5,1)}}, anchor={south}, font={{\fontsize{18 pt}{23.400000000000002 pt}\selectfont}}, color={rgb,1:red,0.0;green,0.0;blue,0.0}, draw opacity={1.0}, rotate={0.0}, align={center}}, legend style={color={rgb,1:red,0.0;green,0.0;blue,0.0}, draw opacity={1.0}, line width={1}, solid, fill={rgb,1:red,1.0;green,1.0;blue,1.0}, fill opacity={1.0}, text opacity={1.0}, font={{\fontsize{16 pt}{20.8 pt}\selectfont}}, text={rgb,1:red,0.0;green,0.0;blue,0.0}, cells={anchor={center}}, at={(0.98, 0.02)}, anchor={south east}}, axis background/.style={fill={rgb,1:red,1.0;green,1.0;blue,1.0}, opacity={1.0}}, anchor={north west}, xshift={1.0mm}, yshift={-1.0mm}, width={201.2mm}, height={74.2mm}, scaled x ticks={false}, xlabel={Time step index}, x tick style={color={rgb,1:red,0.0;green,0.0;blue,0.0}, opacity={1.0}}, x tick label style={color={rgb,1:red,0.0;green,0.0;blue,0.0}, opacity={1.0}, rotate={0}}, xlabel style={at={(ticklabel cs:0.5)}, anchor=near ticklabel, at={{(ticklabel cs:0.5)}}, anchor={near ticklabel}, font={{\fontsize{18 pt}{23.400000000000002 pt}\selectfont}}, color={rgb,1:red,0.0;green,0.0;blue,0.0}, draw opacity={1.0}, rotate={0.0}}, xmajorgrids={true}, xmin={-13.970000000000027}, xmax={514.97}, xticklabels={{$0$,$100$,$200$,$300$,$400$,$500$}}, xtick={{0.0,100.0,200.0,300.0,400.0,500.0}}, xtick align={inside}, xticklabel style={font={{\fontsize{16 pt}{20.8 pt}\selectfont}}, color={rgb,1:red,0.0;green,0.0;blue,0.0}, draw opacity={1.0}, rotate={0.0}}, x grid style={color={rgb,1:red,0.0;green,0.0;blue,0.0}, draw opacity={0.1}, line width={0.5}, solid}, axis x line*={left}, x axis line style={color={rgb,1:red,0.0;green,0.0;blue,0.0}, draw opacity={1.0}, line width={1}, solid}, scaled y ticks={false}, ylabel={Angle (radians)}, y tick style={color={rgb,1:red,0.0;green,0.0;blue,0.0}, opacity={1.0}}, y tick label style={color={rgb,1:red,0.0;green,0.0;blue,0.0}, opacity={1.0}, rotate={0}}, ylabel style={at={(ticklabel cs:0.5)}, anchor=near ticklabel, at={{(ticklabel cs:0.5)}}, anchor={near ticklabel}, font={{\fontsize{18 pt}{23.400000000000002 pt}\selectfont}}, color={rgb,1:red,0.0;green,0.0;blue,0.0}, draw opacity={1.0}, rotate={0.0}}, ymajorgrids={true}, ymin={-3.1627985685809996}, ymax={4.001398297675946}, yticklabels={{$-2$,$0$,$2$,$4$}}, ytick={{-2.0,0.0,2.0,4.0}}, ytick align={inside}, yticklabel style={font={{\fontsize{16 pt}{20.8 pt}\selectfont}}, color={rgb,1:red,0.0;green,0.0;blue,0.0}, draw opacity={1.0}, rotate={0.0}}, y grid style={color={rgb,1:red,0.0;green,0.0;blue,0.0}, draw opacity={0.1}, line width={0.5}, solid}, axis y line*={left}, y axis line style={color={rgb,1:red,0.0;green,0.0;blue,0.0}, draw opacity={1.0}, line width={1}, solid}, colorbar={false}]
    \addplot[color={rgb,1:red,0.349;green,0.6314;blue,0.3098}, name path={20e679cb-e292-4d72-a30f-3c01ec8a30d4}, only marks, draw opacity={1.0}, line width={0}, solid, mark={*}, mark size={1.5 pt}, mark repeat={1}, mark options={color={rgb,1:red,0.0;green,0.0;blue,0.0}, draw opacity={1.0}, fill={rgb,1:red,0.349;green,0.6314;blue,0.3098}, fill opacity={1.0}, line width={0.0}, rotate={0}, solid}]
        table[row sep={\\}]
        {
            \\
            1.0  0.2547037461634025  \\
            2.0  0.13257046233559705  \\
            3.0  -0.03776758991926968  \\
            4.0  0.09161371400759022  \\
            5.0  0.03580303079397851  \\
            6.0  0.5243201219639236  \\
            7.0  -0.8317678104217855  \\
            8.0  0.7337639757938248  \\
            9.0  0.22525926075097388  \\
            10.0  0.060703235270812284  \\
            11.0  1.1737933912733125  \\
            12.0  0.7293091316300014  \\
            13.0  0.4526942096503235  \\
            14.0  0.23733115525947865  \\
            15.0  0.10045711610050653  \\
            16.0  0.010002203005235799  \\
            17.0  0.5849716363761392  \\
            18.0  1.0358206629387468  \\
            19.0  -0.06737686312785268  \\
            20.0  0.8199790894376653  \\
            21.0  0.11075360934320727  \\
            22.0  0.13090376179252466  \\
            23.0  0.6199697498576933  \\
            24.0  0.47159275449181753  \\
            25.0  0.5929405416150995  \\
            26.0  -1.2135665373223106  \\
            27.0  -0.44866295973229275  \\
            28.0  1.3265840688107517  \\
            29.0  -0.36583305536554445  \\
            30.0  0.949304748387459  \\
            31.0  1.1280175834461463  \\
            32.0  -0.18824106990492295  \\
            33.0  1.1197643355571476  \\
            34.0  0.5492803261703256  \\
            35.0  0.7686902631208499  \\
            36.0  0.3711370099283624  \\
            37.0  1.5122590414691222  \\
            38.0  0.6389043165711847  \\
            39.0  1.7274218873887848  \\
            40.0  0.7571412667515867  \\
            41.0  0.8204111857224877  \\
            42.0  0.7358486110893316  \\
            43.0  0.4968958503130323  \\
            44.0  2.5685992967612394  \\
            45.0  1.406056678762214  \\
            46.0  -0.7143217877382071  \\
            47.0  1.4991134982528218  \\
            48.0  0.46361233396621204  \\
            49.0  -0.10732966757856954  \\
            50.0  0.005352002113944487  \\
            51.0  0.4302888807554356  \\
            52.0  1.7421359411792186  \\
            53.0  1.8204722216757916  \\
            54.0  2.4952368321404834  \\
            55.0  1.3370275881377172  \\
            56.0  0.3529439649780819  \\
            57.0  0.33008697408953047  \\
            58.0  0.6228590344824246  \\
            59.0  0.6456893503624443  \\
            60.0  -0.03454432042297306  \\
            61.0  1.3902048320595508  \\
            62.0  0.3946371884144047  \\
            63.0  -0.5108757055489302  \\
            64.0  -0.3413719851201213  \\
            65.0  0.11345939121094634  \\
            66.0  -0.06852280893165702  \\
            67.0  1.042928613543596  \\
            68.0  0.016429464666352145  \\
            69.0  -0.6325896533488691  \\
            70.0  0.7132531520286924  \\
            71.0  -0.2589576993379058  \\
            72.0  -0.3270350833595384  \\
            73.0  -0.9987946037390639  \\
            74.0  0.08700004838123115  \\
            75.0  -0.8414776367231596  \\
            76.0  -0.9661575945971216  \\
            77.0  -0.3973354477736911  \\
            78.0  -0.9612560502306002  \\
            79.0  -0.10336496442805954  \\
            80.0  -0.09609815268853472  \\
            81.0  -1.1592087786938123  \\
            82.0  -1.752612003763668  \\
            83.0  -1.1469312919276016  \\
            84.0  -0.47904067512064585  \\
            85.0  0.10934428785343453  \\
            86.0  -1.1814654323630396  \\
            87.0  -1.7847524419801517  \\
            88.0  -1.0446277627615885  \\
            89.0  -1.8202833045766815  \\
            90.0  -1.3618410397311158  \\
            91.0  -1.6615797777525465  \\
            92.0  -0.20041513781277365  \\
            93.0  -1.8066344946051562  \\
            94.0  -1.9973256250305298  \\
            95.0  -1.5276086342422077  \\
            96.0  -1.9069607291602308  \\
            97.0  -2.283888732889661  \\
            98.0  -1.7495055467102847  \\
            99.0  -1.280413978460448  \\
            100.0  -1.794527938788751  \\
            101.0  -0.9350793519095448  \\
            102.0  -1.7801702059944355  \\
            103.0  -1.3307663563417451  \\
            104.0  -2.4753523322202415  \\
            105.0  -1.405132012217344  \\
            106.0  -1.2928011724102173  \\
            107.0  -1.6083470684035137  \\
            108.0  -1.6917811864631256  \\
            109.0  -1.120361859962426  \\
            110.0  -2.030965288679942  \\
            111.0  -2.242777682861987  \\
            112.0  -1.6518689048681314  \\
            113.0  -2.2793768057160637  \\
            114.0  -1.8821228665101208  \\
            115.0  -2.6957631794674164  \\
            116.0  -1.7461509806630824  \\
            117.0  -2.1419322564559584  \\
            118.0  -2.9600382799133507  \\
            119.0  -1.549348642096706  \\
            120.0  -1.7786924083727553  \\
            121.0  -2.279279300137501  \\
            122.0  -1.4197770384800101  \\
            123.0  -2.4870009676186786  \\
            124.0  -1.8939187940668558  \\
            125.0  -1.1452780024184255  \\
            126.0  -1.762804657436483  \\
            127.0  -2.958358586447785  \\
            128.0  -1.9678753036282253  \\
            129.0  -1.2866120734788664  \\
            130.0  -2.6288486206158783  \\
            131.0  -2.1318523277248858  \\
            132.0  -1.6020497590171445  \\
            133.0  -1.9787588277831722  \\
            134.0  -1.1479390192078547  \\
            135.0  -1.265486139945545  \\
            136.0  -2.1802697522359984  \\
            137.0  -1.9517398595235642  \\
            138.0  -1.7475681174680806  \\
            139.0  -1.9833710265259836  \\
            140.0  -2.1490718382039655  \\
            141.0  -1.3291352355050958  \\
            142.0  -1.743851573349185  \\
            143.0  -1.3600481155221495  \\
            144.0  -1.7714493216840912  \\
            145.0  -0.8962415563893462  \\
            146.0  -0.8234230389913816  \\
            147.0  -2.324907308996061  \\
            148.0  -1.254133825881493  \\
            149.0  -1.6097525414786458  \\
            150.0  -0.578114139934352  \\
            151.0  -1.3239620059485115  \\
            152.0  -0.7243928919061647  \\
            153.0  -1.283410064621941  \\
            154.0  -1.7473270629033948  \\
            155.0  -0.6986673644819154  \\
            156.0  -1.8626151500844275  \\
            157.0  -1.9347445137447257  \\
            158.0  -0.842063234393021  \\
            159.0  -2.05895557547552  \\
            160.0  -1.0435751654902805  \\
            161.0  -2.021011187277641  \\
            162.0  0.04549998660082011  \\
            163.0  -0.792534551761701  \\
            164.0  -0.9068480486991035  \\
            165.0  -0.2262425834735241  \\
            166.0  -0.9518746721201454  \\
            167.0  -0.22718942854916757  \\
            168.0  -0.08412107533518237  \\
            169.0  -1.3775567238188153  \\
            170.0  -0.5773270441455716  \\
            171.0  0.4548574286780537  \\
            172.0  -0.4913342999311329  \\
            173.0  -1.5196404104094683  \\
            174.0  0.04670881449122884  \\
            175.0  -0.03981830747880498  \\
            176.0  -0.03051172022122728  \\
            177.0  -0.7666882010123338  \\
            178.0  -0.8514670831496012  \\
            179.0  -0.31112845208766404  \\
            180.0  0.23461705449331224  \\
            181.0  -0.5667738317603577  \\
            182.0  0.3957519902448624  \\
            183.0  0.7909567080577389  \\
            184.0  0.35341347958062397  \\
            185.0  1.1364681196456983  \\
            186.0  -0.8866984983949993  \\
            187.0  0.8355489057574816  \\
            188.0  1.4286726142240131  \\
            189.0  0.6967053593922535  \\
            190.0  0.13057315453066154  \\
            191.0  0.003951425711167467  \\
            192.0  0.12893881621325853  \\
            193.0  0.3418514828675006  \\
            194.0  0.6623690670745918  \\
            195.0  0.971115257437804  \\
            196.0  0.8675616707231275  \\
            197.0  1.659762788216834  \\
            198.0  -0.22980701221832445  \\
            199.0  0.8158018319882907  \\
            200.0  1.1128360472584942  \\
            201.0  -0.29601005369727285  \\
            202.0  0.7023766852736714  \\
            203.0  0.9293406180086811  \\
            204.0  0.29015371371690346  \\
            205.0  0.7556310056529474  \\
            206.0  1.4344105171084804  \\
            207.0  1.4698620035432237  \\
            208.0  1.2941818852525873  \\
            209.0  0.3176034193205455  \\
            210.0  0.6785517639346975  \\
            211.0  0.792172500328094  \\
            212.0  1.2646419707916983  \\
            213.0  0.7960149273348547  \\
            214.0  0.3152646318989692  \\
            215.0  0.6548741529538871  \\
            216.0  -0.21144910077212714  \\
            217.0  -0.19414924581320875  \\
            218.0  -0.4740329810364304  \\
            219.0  0.9927697879645445  \\
            220.0  0.7241107825685815  \\
            221.0  1.6116659856676383  \\
            222.0  0.1350016249205366  \\
            223.0  0.6073804642947458  \\
            224.0  -0.6440673069791236  \\
            225.0  0.28335334619370584  \\
            226.0  0.35746396362649174  \\
            227.0  -0.4721405871607463  \\
            228.0  0.31718158750288894  \\
            229.0  0.9034357373574358  \\
            230.0  0.4162164899549826  \\
            231.0  0.6785569721554142  \\
            232.0  -0.7554189124192134  \\
            233.0  -0.07352504380112157  \\
            234.0  0.4799551408175747  \\
            235.0  0.6111606993526726  \\
            236.0  0.8035344971127392  \\
            237.0  0.4172396951092041  \\
            238.0  0.2264966528743171  \\
            239.0  -0.9837931596618397  \\
            240.0  0.49360922876195135  \\
            241.0  -0.20391346638707497  \\
            242.0  -0.04427114824239464  \\
            243.0  0.5320992830720863  \\
            244.0  0.7547897166086556  \\
            245.0  0.12664982304599093  \\
            246.0  -0.25291909862300027  \\
            247.0  0.05474471648881557  \\
            248.0  0.7012705881732797  \\
            249.0  1.1126540575872577  \\
            250.0  0.9564509782059143  \\
            251.0  0.6086049342251145  \\
            252.0  0.36497917307272715  \\
            253.0  0.8107120223693237  \\
            254.0  -0.14022250368494682  \\
            255.0  -0.5567382267815906  \\
            256.0  2.042790666922171  \\
            257.0  0.28553466984456144  \\
            258.0  1.0556365973219524  \\
            259.0  0.34777014128007483  \\
            260.0  -0.18067156125584405  \\
            261.0  1.997332567858869  \\
            262.0  1.684860640174111  \\
            263.0  0.12781814933239244  \\
            264.0  0.9343981681768515  \\
            265.0  0.8918032967660761  \\
            266.0  0.46063174159636844  \\
            267.0  0.7148957229897894  \\
            268.0  0.8443678383635875  \\
            269.0  0.8841012329654054  \\
            270.0  1.595037634437785  \\
            271.0  0.9208737925369979  \\
            272.0  1.706996691692529  \\
            273.0  2.2546304108994804  \\
            274.0  2.0122559426806443  \\
            275.0  1.3866010921131795  \\
            276.0  0.4660724748335717  \\
            277.0  1.6597815170707642  \\
            278.0  2.1037716408757894  \\
            279.0  1.0953964595497925  \\
            280.0  1.5607946722551016  \\
            281.0  0.9414944402504257  \\
            282.0  0.5607578326192387  \\
            283.0  1.4278986129947961  \\
            284.0  2.076966846601141  \\
            285.0  1.3999069772729673  \\
            286.0  0.8150626347309256  \\
            287.0  0.5485022388438334  \\
            288.0  1.5059141464901211  \\
            289.0  0.7290835898276607  \\
            290.0  1.6049076623581637  \\
            291.0  1.5697447834792064  \\
            292.0  0.45171024839277607  \\
            293.0  0.8293124007853085  \\
            294.0  0.846601310395398  \\
            295.0  0.8739146920926135  \\
            296.0  0.08488410227529886  \\
            297.0  0.002190553834938269  \\
            298.0  0.5598896749165143  \\
            299.0  0.6488627031233556  \\
            300.0  0.6626201431943712  \\
            301.0  1.550809263129851  \\
            302.0  1.2035366486807428  \\
            303.0  1.4391513598957628  \\
            304.0  0.9465325080806437  \\
            305.0  1.0363051989532819  \\
            306.0  0.23808535182725732  \\
            307.0  0.6627634639145569  \\
            308.0  0.07281148228098394  \\
            309.0  -0.5777034836980899  \\
            310.0  -0.12292021554616031  \\
            311.0  1.0502619194373501  \\
            312.0  -0.45659941413233857  \\
            313.0  0.15183854869504282  \\
            314.0  -0.6863180564401987  \\
            315.0  0.3230755997005984  \\
            316.0  -0.9453184589914141  \\
            317.0  -0.4585674228477514  \\
            318.0  -1.3071798370154104  \\
            319.0  -0.4978799443024999  \\
            320.0  -0.4088673706448102  \\
            321.0  -0.3128210562588147  \\
            322.0  -1.146473803498684  \\
            323.0  -0.7587892155610564  \\
            324.0  -1.4263939872180178  \\
            325.0  -0.7074078532701644  \\
            326.0  -1.022633900769508  \\
            327.0  -0.98774822932522  \\
            328.0  -0.6916193332964828  \\
            329.0  -2.0016733429279228  \\
            330.0  -1.4131814744671165  \\
            331.0  -1.7930584486992707  \\
            332.0  -1.0855561768133115  \\
            333.0  -1.7393989295487908  \\
            334.0  -1.3819726354505815  \\
            335.0  -1.1648785058254565  \\
            336.0  -1.1696073623634942  \\
            337.0  -2.22389591886838  \\
            338.0  -1.436309384075033  \\
            339.0  -1.4251929505875185  \\
            340.0  -2.776049799437655  \\
            341.0  -2.015551359451244  \\
            342.0  -1.4453997369175922  \\
            343.0  -1.6318896644401617  \\
            344.0  -1.3405557295883137  \\
            345.0  -1.938032837095029  \\
            346.0  -1.1602399262184786  \\
            347.0  -2.2341320955057733  \\
            348.0  -2.023659457762066  \\
            349.0  -1.3512464239084263  \\
            350.0  -1.2546849565308018  \\
            351.0  -2.789860757175241  \\
            352.0  -2.1450923087806646  \\
            353.0  -1.4082746958849548  \\
            354.0  -2.124991790703219  \\
            355.0  -2.524805156615292  \\
            356.0  -1.3570631992843396  \\
            357.0  -2.1583761447926655  \\
            358.0  -1.4119514670359934  \\
            359.0  -2.6025322083758513  \\
            360.0  -2.2737146590008903  \\
            361.0  -0.9001825844007502  \\
            362.0  -1.573369654329728  \\
            363.0  -0.599429902944882  \\
            364.0  -1.4189348800495303  \\
            365.0  -1.4594283988176981  \\
            366.0  -1.442131238522419  \\
            367.0  -1.9203528374277246  \\
            368.0  -2.163109350493304  \\
            369.0  -1.6720797183679454  \\
            370.0  -2.3683604047124254  \\
            371.0  -1.8192560001767353  \\
            372.0  -1.8596849452131368  \\
            373.0  -2.133720376870821  \\
            374.0  -2.1572688262127735  \\
            375.0  -2.0132201429760843  \\
            376.0  -2.251616277999592  \\
            377.0  -1.6512530881801881  \\
            378.0  -2.5752184245881256  \\
            379.0  -1.4988184096798005  \\
            380.0  -1.5770600580654228  \\
            381.0  -1.4176315937500161  \\
            382.0  -0.3842147119966244  \\
            383.0  -1.3145227316277073  \\
            384.0  -1.5181413561303954  \\
            385.0  -0.9843052329262914  \\
            386.0  -1.3493778679300799  \\
            387.0  -0.8600089106646416  \\
            388.0  -0.9397972935597814  \\
            389.0  -1.647682186971911  \\
            390.0  -0.24146190170522108  \\
            391.0  -1.149427215835684  \\
            392.0  -0.6135056898672351  \\
            393.0  -0.23704227857604454  \\
            394.0  -0.7266293222284332  \\
            395.0  -0.6469705578152578  \\
            396.0  -0.3302973842437824  \\
            397.0  0.026955363776290597  \\
            398.0  -0.6066017396824744  \\
            399.0  -0.28600718047321994  \\
            400.0  -0.4411287947197255  \\
            401.0  -0.8407578459391283  \\
            402.0  -0.8566991229381864  \\
            403.0  -0.11663955661405484  \\
            404.0  -0.2318327999287174  \\
            405.0  -1.3079709713328675  \\
            406.0  -0.373487161529459  \\
            407.0  -0.7188321977668082  \\
            408.0  -0.23123651433818074  \\
            409.0  -0.1856290459735493  \\
            410.0  -0.540336242094907  \\
            411.0  0.8337033209990051  \\
            412.0  -0.3396091540246816  \\
            413.0  -0.42910998394416744  \\
            414.0  -0.09462292850916931  \\
            415.0  -0.15017075721075504  \\
            416.0  0.41213988117037437  \\
            417.0  0.36491289243153135  \\
            418.0  0.337896049834181  \\
            419.0  -0.14159182180049262  \\
            420.0  0.3100023522725418  \\
            421.0  -0.15411792005010652  \\
            422.0  -0.04805806938270364  \\
            423.0  -0.1694380388120736  \\
            424.0  -0.20075377742702127  \\
            425.0  -0.008950829491479642  \\
            426.0  -0.4149538639880728  \\
            427.0  0.1315005462593218  \\
            428.0  -0.0978205171395651  \\
            429.0  -0.32781175892481035  \\
            430.0  -0.5416378726445532  \\
            431.0  -0.592027995888375  \\
            432.0  -0.0072412499821009335  \\
            433.0  1.0757461134062702  \\
            434.0  0.3187614023975534  \\
            435.0  -0.3462311843273594  \\
            436.0  0.7877353816789289  \\
            437.0  0.6063977877619431  \\
            438.0  -0.6253099371736408  \\
            439.0  -0.5251760267252457  \\
            440.0  -0.7042190087360131  \\
            441.0  0.501730289547447  \\
            442.0  -0.5432339546632522  \\
            443.0  0.46240715276019245  \\
            444.0  -1.3638831432538736  \\
            445.0  0.27692021444472936  \\
            446.0  -0.28368394816613585  \\
            447.0  0.03578460381578301  \\
            448.0  -0.22517247272992194  \\
            449.0  -0.5598223791153143  \\
            450.0  1.2340688123060086  \\
            451.0  0.7872944419691128  \\
            452.0  0.2722569244292567  \\
            453.0  0.4254568179708278  \\
            454.0  0.1969275217683756  \\
            455.0  1.133441032672545  \\
            456.0  0.6658599460644867  \\
            457.0  1.9985518620530995  \\
            458.0  0.016524650050032896  \\
            459.0  0.807295983156374  \\
            460.0  0.7250361007337011  \\
            461.0  0.5332407655761877  \\
            462.0  -0.7119212761095042  \\
            463.0  2.4026773260289414  \\
            464.0  0.029620102395093495  \\
            465.0  0.7926953815630983  \\
            466.0  2.0601968140143283  \\
            467.0  0.5597883347518665  \\
            468.0  0.2514432720609753  \\
            469.0  0.6302679706671976  \\
            470.0  1.108173209534797  \\
            471.0  0.8255730143149258  \\
            472.0  1.3672865928237867  \\
            473.0  2.127190053908513  \\
            474.0  0.7028778760573932  \\
            475.0  1.0740457652865356  \\
            476.0  1.3401585553280866  \\
            477.0  1.3221688074315678  \\
            478.0  2.0436794384366976  \\
            479.0  1.504102864120624  \\
            480.0  1.680958305960178  \\
            481.0  1.1510945551751592  \\
            482.0  1.0373728076856932  \\
            483.0  0.9736528078858427  \\
            484.0  2.323150574619187  \\
            485.0  1.6585870925096187  \\
            486.0  1.4809574354758377  \\
            487.0  0.4707693339277854  \\
            488.0  1.7715124020594613  \\
            489.0  2.2854953468975143  \\
            490.0  1.4861928161488789  \\
            491.0  1.5555742747797223  \\
            492.0  1.842983651332346  \\
            493.0  3.7986380090082967  \\
            494.0  2.6034814916128366  \\
            495.0  2.6757511607228572  \\
            496.0  1.6688285847992965  \\
            497.0  2.2405816958585785  \\
            498.0  1.9152722958693746  \\
            499.0  1.7944875204012671  \\
            500.0  1.960266375197512  \\
        }
        ;
    \addlegendentry {$y = \theta_2 + \omega$}
    \addplot[color={rgb,1:red,0.949;green,0.5569;blue,0.1686}, name path={d5feaefa-7394-43f6-b2be-5a8419165467}, draw opacity={1.0}, line width={2}, dashed]
        table[row sep={\\}]
        {
            \\
            1.0  0.2  \\
            2.0  0.20077121306979956  \\
            3.0  0.20244421310869218  \\
            4.0  0.20538450261381772  \\
            5.0  0.2076748169724466  \\
            6.0  0.21004613114216075  \\
            7.0  0.21198732109905916  \\
            8.0  0.21872872777404256  \\
            9.0  0.22323289221781725  \\
            10.0  0.22943985449754048  \\
            11.0  0.23581566502370901  \\
            12.0  0.24315735276246045  \\
            13.0  0.25284533768152134  \\
            14.0  0.26333740397862243  \\
            15.0  0.2767267762686036  \\
            16.0  0.29025284273577634  \\
            17.0  0.30537834643258177  \\
            18.0  0.3223457673611555  \\
            19.0  0.33790075820583876  \\
            20.0  0.35864484846938993  \\
            21.0  0.381008210128867  \\
            22.0  0.40503304101881643  \\
            23.0  0.4297012844324011  \\
            24.0  0.45813855895874606  \\
            25.0  0.48682693931483995  \\
            26.0  0.5206129038195051  \\
            27.0  0.5559471357422973  \\
            28.0  0.5922734827511258  \\
            29.0  0.6295904162401603  \\
            30.0  0.6681568582309619  \\
            31.0  0.7069617380962739  \\
            32.0  0.7458108999643405  \\
            33.0  0.7837109003387867  \\
            34.0  0.8199399855636275  \\
            35.0  0.8502841255822313  \\
            36.0  0.8783688485478693  \\
            37.0  0.9035997250689355  \\
            38.0  0.9251420935601686  \\
            39.0  0.9447342773893882  \\
            40.0  0.9621520056848557  \\
            41.0  0.9766089837799784  \\
            42.0  0.9890431780302547  \\
            43.0  0.9949241636544784  \\
            44.0  0.9969713618207969  \\
            45.0  0.9973295439251917  \\
            46.0  0.9932927022359279  \\
            47.0  0.9864562676330484  \\
            48.0  0.9780658607032465  \\
            49.0  0.9663855710137768  \\
            50.0  0.9536149015339549  \\
            51.0  0.9353859590904313  \\
            52.0  0.9147743221649417  \\
            53.0  0.8924033613981605  \\
            54.0  0.8670071208395579  \\
            55.0  0.8401128720204886  \\
            56.0  0.80859996527425  \\
            57.0  0.7750602396433758  \\
            58.0  0.7390795630859462  \\
            59.0  0.700158369569577  \\
            60.0  0.6588167666960625  \\
            61.0  0.6137401324392603  \\
            62.0  0.5654269061868354  \\
            63.0  0.5151341678076061  \\
            64.0  0.46252447496209453  \\
            65.0  0.40651420138940986  \\
            66.0  0.3505493080310132  \\
            67.0  0.2921725223792161  \\
            68.0  0.23005615438543553  \\
            69.0  0.16228513577315584  \\
            70.0  0.09250511552723756  \\
            71.0  0.019513296113997818  \\
            72.0  -0.057542073684514736  \\
            73.0  -0.137917451369763  \\
            74.0  -0.21952606427583504  \\
            75.0  -0.30457893498341554  \\
            76.0  -0.3915912846654537  \\
            77.0  -0.47728396192190414  \\
            78.0  -0.5636963639110109  \\
            79.0  -0.6444670567618447  \\
            80.0  -0.7239413402630823  \\
            81.0  -0.7989819317719744  \\
            82.0  -0.8715307328990217  \\
            83.0  -0.940233108432891  \\
            84.0  -1.005239787419807  \\
            85.0  -1.0685953502753918  \\
            86.0  -1.1283227352474465  \\
            87.0  -1.185359026976821  \\
            88.0  -1.2394906751023496  \\
            89.0  -1.2943304148202577  \\
            90.0  -1.3441062924771507  \\
            91.0  -1.3902859791631426  \\
            92.0  -1.4366803682827065  \\
            93.0  -1.4811529284215403  \\
            94.0  -1.5225448964617248  \\
            95.0  -1.563730261262378  \\
            96.0  -1.5997131249515735  \\
            97.0  -1.6366441866414128  \\
            98.0  -1.6733136776466513  \\
            99.0  -1.7056024983702138  \\
            100.0  -1.737847797377272  \\
            101.0  -1.768615713947156  \\
            102.0  -1.7973953650595775  \\
            103.0  -1.8266478016459915  \\
            104.0  -1.8525142454289871  \\
            105.0  -1.875355680191903  \\
            106.0  -1.8985285772418155  \\
            107.0  -1.9200193226034352  \\
            108.0  -1.9382098190965413  \\
            109.0  -1.955117629532639  \\
            110.0  -1.9696443312177885  \\
            111.0  -1.9839163127155266  \\
            112.0  -1.9971876830609512  \\
            113.0  -2.010110211562051  \\
            114.0  -2.0209182957170873  \\
            115.0  -2.0325631009319194  \\
            116.0  -2.0404197873851406  \\
            117.0  -2.047105285757217  \\
            118.0  -2.05406118715393  \\
            119.0  -2.0578465063172775  \\
            120.0  -2.061754863949012  \\
            121.0  -2.0623956324894124  \\
            122.0  -2.0635020145097647  \\
            123.0  -2.065016572927728  \\
            124.0  -2.0642986961816012  \\
            125.0  -2.062082836640928  \\
            126.0  -2.0572354604704226  \\
            127.0  -2.051457063599901  \\
            128.0  -2.0453135650132035  \\
            129.0  -2.0377082511655993  \\
            130.0  -2.0271684177566027  \\
            131.0  -2.0174163118592103  \\
            132.0  -2.006627255161083  \\
            133.0  -1.99458661776932  \\
            134.0  -1.981714592694915  \\
            135.0  -1.9657357644053175  \\
            136.0  -1.95008811649449  \\
            137.0  -1.9303573794981208  \\
            138.0  -1.9110237806162047  \\
            139.0  -1.891032236293788  \\
            140.0  -1.86892276396149  \\
            141.0  -1.8459024476627526  \\
            142.0  -1.8240516541245848  \\
            143.0  -1.7969981244184352  \\
            144.0  -1.7689196262694191  \\
            145.0  -1.7410262603456275  \\
            146.0  -1.7107785978513035  \\
            147.0  -1.6775932275228054  \\
            148.0  -1.6435335559227948  \\
            149.0  -1.6087777293433998  \\
            150.0  -1.5715761528918881  \\
            151.0  -1.5333355225829213  \\
            152.0  -1.4924470485598136  \\
            153.0  -1.4505696989582664  \\
            154.0  -1.4081074652001293  \\
            155.0  -1.3631752509782977  \\
            156.0  -1.315047608485016  \\
            157.0  -1.2671404222558629  \\
            158.0  -1.2145856333462213  \\
            159.0  -1.1590012892917456  \\
            160.0  -1.1007170055946283  \\
            161.0  -1.0412272270808436  \\
            162.0  -0.9788028842153407  \\
            163.0  -0.9154362282771672  \\
            164.0  -0.8467195634224802  \\
            165.0  -0.7740772228420397  \\
            166.0  -0.6991025525069955  \\
            167.0  -0.6219897733785096  \\
            168.0  -0.544075776265649  \\
            169.0  -0.4647533959092831  \\
            170.0  -0.3871719296897495  \\
            171.0  -0.30871861039755644  \\
            172.0  -0.23228660994473999  \\
            173.0  -0.16006993457316293  \\
            174.0  -0.09104079554576809  \\
            175.0  -0.023836038553904634  \\
            176.0  0.04065941764398388  \\
            177.0  0.10298433719679366  \\
            178.0  0.16191786460486846  \\
            179.0  0.21717664729431627  \\
            180.0  0.27196657770121696  \\
            181.0  0.32217097417301116  \\
            182.0  0.3704057719112007  \\
            183.0  0.41381936139150055  \\
            184.0  0.45562076321800266  \\
            185.0  0.4956701607299468  \\
            186.0  0.533427051320366  \\
            187.0  0.568639337023961  \\
            188.0  0.6009351087159419  \\
            189.0  0.6309386763830759  \\
            190.0  0.6559845555974834  \\
            191.0  0.6807106299250723  \\
            192.0  0.702832881211831  \\
            193.0  0.7215784638941529  \\
            194.0  0.7368913573391841  \\
            195.0  0.7489647617236266  \\
            196.0  0.7587190342979474  \\
            197.0  0.7647364823399241  \\
            198.0  0.7678485327456598  \\
            199.0  0.7664416622206628  \\
            200.0  0.7638649055908637  \\
            201.0  0.7571631057135545  \\
            202.0  0.7459260731431627  \\
            203.0  0.7330132005437374  \\
            204.0  0.7166998253727788  \\
            205.0  0.6970059767715231  \\
            206.0  0.6728385877341402  \\
            207.0  0.6478060532225469  \\
            208.0  0.6195863531767273  \\
            209.0  0.5868529561778992  \\
            210.0  0.5565162004508684  \\
            211.0  0.5235131226371783  \\
            212.0  0.49046125601779067  \\
            213.0  0.45908445524916563  \\
            214.0  0.4287847959908928  \\
            215.0  0.3998148291027864  \\
            216.0  0.37347240850929825  \\
            217.0  0.3480287471251013  \\
            218.0  0.3267248542670755  \\
            219.0  0.30876219079313677  \\
            220.0  0.29228402246562163  \\
            221.0  0.27752050815829726  \\
            222.0  0.2665193975217233  \\
            223.0  0.2587725529804194  \\
            224.0  0.2529307282775175  \\
            225.0  0.2483180748127668  \\
            226.0  0.2440885148853471  \\
            227.0  0.2427798059909984  \\
            228.0  0.2426666754505097  \\
            229.0  0.24357407307853082  \\
            230.0  0.24644402757467024  \\
            231.0  0.24939693899033108  \\
            232.0  0.2543840911518738  \\
            233.0  0.2601560785761774  \\
            234.0  0.2675938233072742  \\
            235.0  0.2761793284140923  \\
            236.0  0.2849108055903707  \\
            237.0  0.29213702863752405  \\
            238.0  0.30335718171378145  \\
            239.0  0.31517148765300046  \\
            240.0  0.3259089236880202  \\
            241.0  0.3399682838979915  \\
            242.0  0.352352226201167  \\
            243.0  0.36610976576424914  \\
            244.0  0.3811338156002569  \\
            245.0  0.39720331331081  \\
            246.0  0.41544809635028374  \\
            247.0  0.4342584566805917  \\
            248.0  0.45318580254310387  \\
            249.0  0.47051640079546203  \\
            250.0  0.4909670906812182  \\
            251.0  0.5123169411543128  \\
            252.0  0.5341537477526385  \\
            253.0  0.5573820680188972  \\
            254.0  0.5835830581812941  \\
            255.0  0.6107251247397203  \\
            256.0  0.6406161162667461  \\
            257.0  0.6702865430626567  \\
            258.0  0.7016032452233902  \\
            259.0  0.7347960859908554  \\
            260.0  0.7705628308338417  \\
            261.0  0.8059160272118696  \\
            262.0  0.84288803856595  \\
            263.0  0.8811221824524031  \\
            264.0  0.9198144901495555  \\
            265.0  0.95884434700868  \\
            266.0  0.9954583838804504  \\
            267.0  1.0298741377702962  \\
            268.0  1.0620963085025146  \\
            269.0  1.0939848148453273  \\
            270.0  1.1237094248280324  \\
            271.0  1.1499473336236519  \\
            272.0  1.1762164321933248  \\
            273.0  1.19810910993671  \\
            274.0  1.216592320085004  \\
            275.0  1.2340240970896694  \\
            276.0  1.2490729120116455  \\
            277.0  1.2616507087092939  \\
            278.0  1.2731459764365671  \\
            279.0  1.2790286057274227  \\
            280.0  1.2826761038801717  \\
            281.0  1.2818506515993286  \\
            282.0  1.2821127224324627  \\
            283.0  1.2780850357853335  \\
            284.0  1.271882537167133  \\
            285.0  1.2634332162823554  \\
            286.0  1.2525107018951074  \\
            287.0  1.2381118944893745  \\
            288.0  1.2226291302564298  \\
            289.0  1.2041263887955536  \\
            290.0  1.1820500585997478  \\
            291.0  1.1576983384623114  \\
            292.0  1.1307560022316532  \\
            293.0  1.1024438194823365  \\
            294.0  1.071669706706363  \\
            295.0  1.0368412239102822  \\
            296.0  0.9992564558933096  \\
            297.0  0.9611198859474601  \\
            298.0  0.9206250823234695  \\
            299.0  0.8754309651938195  \\
            300.0  0.8296117610952934  \\
            301.0  0.7816644017332739  \\
            302.0  0.7304266423142528  \\
            303.0  0.6756067306473944  \\
            304.0  0.6190265780459794  \\
            305.0  0.5584983854187708  \\
            306.0  0.4945959772946861  \\
            307.0  0.42809256646303506  \\
            308.0  0.35743390304190054  \\
            309.0  0.28237060763882155  \\
            310.0  0.20635408477251946  \\
            311.0  0.1250553670513947  \\
            312.0  0.04180154609516395  \\
            313.0  -0.04551155106840756  \\
            314.0  -0.1345143377334055  \\
            315.0  -0.22336764701950043  \\
            316.0  -0.3125542934537056  \\
            317.0  -0.4006237103481264  \\
            318.0  -0.48579671222006027  \\
            319.0  -0.5661325834512474  \\
            320.0  -0.6426821090518738  \\
            321.0  -0.7172673894809637  \\
            322.0  -0.7873213024989413  \\
            323.0  -0.8535899716351228  \\
            324.0  -0.9169131725885248  \\
            325.0  -0.9763768999174139  \\
            326.0  -1.0338852035448125  \\
            327.0  -1.0891228139966511  \\
            328.0  -1.1416932335586711  \\
            329.0  -1.193887252163816  \\
            330.0  -1.2430722378361234  \\
            331.0  -1.2883536780403297  \\
            332.0  -1.329396889152606  \\
            333.0  -1.3721015380533579  \\
            334.0  -1.410587058006288  \\
            335.0  -1.4492797618722968  \\
            336.0  -1.486674589986813  \\
            337.0  -1.5206389114801222  \\
            338.0  -1.5501233600278685  \\
            339.0  -1.5796125431703836  \\
            340.0  -1.605352478506891  \\
            341.0  -1.6305648993554678  \\
            342.0  -1.6556095801916717  \\
            343.0  -1.6776966573688799  \\
            344.0  -1.6997700496991055  \\
            345.0  -1.7191881263115145  \\
            346.0  -1.7369253843635197  \\
            347.0  -1.7515632743001506  \\
            348.0  -1.7643092546648584  \\
            349.0  -1.7755572892747071  \\
            350.0  -1.7858686698589785  \\
            351.0  -1.7934184436420966  \\
            352.0  -1.7990947955853844  \\
            353.0  -1.8028736089432493  \\
            354.0  -1.8053315454855539  \\
            355.0  -1.8079919924695234  \\
            356.0  -1.8093904029978138  \\
            357.0  -1.8078166392243034  \\
            358.0  -1.8055371340062794  \\
            359.0  -1.8019405037544185  \\
            360.0  -1.794975486009442  \\
            361.0  -1.7905792287527662  \\
            362.0  -1.7827217879847146  \\
            363.0  -1.772925501738634  \\
            364.0  -1.7632655353784426  \\
            365.0  -1.7520106091188579  \\
            366.0  -1.7383000493614214  \\
            367.0  -1.7234026804900597  \\
            368.0  -1.708974424622292  \\
            369.0  -1.69236497220162  \\
            370.0  -1.6755435472232991  \\
            371.0  -1.6571146092150943  \\
            372.0  -1.6374132718613716  \\
            373.0  -1.6163102211069915  \\
            374.0  -1.5929513801862383  \\
            375.0  -1.5687634294720114  \\
            376.0  -1.5428182225696192  \\
            377.0  -1.514806187896199  \\
            378.0  -1.4864729695322472  \\
            379.0  -1.4598740006553625  \\
            380.0  -1.428238813862558  \\
            381.0  -1.3966106475504818  \\
            382.0  -1.3633237176332025  \\
            383.0  -1.3297460029459516  \\
            384.0  -1.2927014569083286  \\
            385.0  -1.2559255313171707  \\
            386.0  -1.2176065190303096  \\
            387.0  -1.177734348758448  \\
            388.0  -1.1380828389474167  \\
            389.0  -1.0965241030332809  \\
            390.0  -1.0544048855806913  \\
            391.0  -1.0104318668487824  \\
            392.0  -0.9668637422171491  \\
            393.0  -0.9210986512713824  \\
            394.0  -0.8759473281918401  \\
            395.0  -0.8308703080506649  \\
            396.0  -0.7879880374857822  \\
            397.0  -0.7426043444842876  \\
            398.0  -0.6983254732221218  \\
            399.0  -0.6550811625572704  \\
            400.0  -0.6132201649598005  \\
            401.0  -0.5706226728514276  \\
            402.0  -0.5322993731682073  \\
            403.0  -0.49292381085620945  \\
            404.0  -0.4577807272342732  \\
            405.0  -0.4229313927409816  \\
            406.0  -0.38872015090073975  \\
            407.0  -0.35614586208550836  \\
            408.0  -0.32391746151118594  \\
            409.0  -0.2961991504149649  \\
            410.0  -0.2688703373225751  \\
            411.0  -0.24300339968906806  \\
            412.0  -0.21841312482110195  \\
            413.0  -0.195745558629374  \\
            414.0  -0.1758067598105461  \\
            415.0  -0.15669856498070192  \\
            416.0  -0.14225439628330064  \\
            417.0  -0.1296002340183511  \\
            418.0  -0.1183458774384559  \\
            419.0  -0.10776289230030738  \\
            420.0  -0.09928133750765428  \\
            421.0  -0.09505392176962664  \\
            422.0  -0.0907450575845551  \\
            423.0  -0.0875819912094041  \\
            424.0  -0.08625446621392481  \\
            425.0  -0.08602303669183034  \\
            426.0  -0.08713274729147846  \\
            427.0  -0.08967431522213234  \\
            428.0  -0.09140430098031307  \\
            429.0  -0.09296106125317692  \\
            430.0  -0.09622687805733805  \\
            431.0  -0.09684639699937946  \\
            432.0  -0.09642500882333452  \\
            433.0  -0.0967136950350474  \\
            434.0  -0.09573907530773336  \\
            435.0  -0.09341218857559279  \\
            436.0  -0.0904342350941615  \\
            437.0  -0.08456954983150432  \\
            438.0  -0.07583762207941781  \\
            439.0  -0.06742353933148973  \\
            440.0  -0.05631315723911538  \\
            441.0  -0.040785202322536226  \\
            442.0  -0.0235748776051294  \\
            443.0  -0.004782089065188591  \\
            444.0  0.01590566076613401  \\
            445.0  0.03950056776861398  \\
            446.0  0.06543046370744351  \\
            447.0  0.09492198194121704  \\
            448.0  0.12610558953312836  \\
            449.0  0.1586309851294126  \\
            450.0  0.19355706702781691  \\
            451.0  0.230025694199613  \\
            452.0  0.2679324797687284  \\
            453.0  0.3064775006398816  \\
            454.0  0.3482269644060056  \\
            455.0  0.39180595773331695  \\
            456.0  0.4350604666675338  \\
            457.0  0.4795282897482437  \\
            458.0  0.5257240292112495  \\
            459.0  0.5734869196666108  \\
            460.0  0.6223792324979419  \\
            461.0  0.6705189697226674  \\
            462.0  0.7192056655973864  \\
            463.0  0.7692543645849823  \\
            464.0  0.819610148030468  \\
            465.0  0.8695307437193631  \\
            466.0  0.9174735544541  \\
            467.0  0.9636516621615923  \\
            468.0  1.008535049737471  \\
            469.0  1.052868083983837  \\
            470.0  1.0960269081926686  \\
            471.0  1.1395459744608987  \\
            472.0  1.1810358753381913  \\
            473.0  1.222315868793669  \\
            474.0  1.2609254354362214  \\
            475.0  1.2982144544178071  \\
            476.0  1.3325148108960736  \\
            477.0  1.365792021915171  \\
            478.0  1.3988889620021583  \\
            479.0  1.429979927736979  \\
            480.0  1.459990136142756  \\
            481.0  1.4888925458881535  \\
            482.0  1.5153678924227227  \\
            483.0  1.5421780279299844  \\
            484.0  1.5663501709612253  \\
            485.0  1.5895042054268662  \\
            486.0  1.6126836645202955  \\
            487.0  1.633957416388249  \\
            488.0  1.6536515230672766  \\
            489.0  1.6717639042044081  \\
            490.0  1.6870349753785274  \\
            491.0  1.7031595765222853  \\
            492.0  1.7191392810467103  \\
            493.0  1.7320970606146595  \\
            494.0  1.746670888367036  \\
            495.0  1.7600274770352307  \\
            496.0  1.7697038593710774  \\
            497.0  1.7790393784436769  \\
            498.0  1.787770529010681  \\
            499.0  1.7958635671581844  \\
            500.0  1.8008777402744447  \\
        }
        ;
    \addlegendentry {$\theta_2$}
\end{axis}
\end{tikzpicture}
}
%\caption{Simulated measurements of the double pendulum system on Figure~\ref{fig:sim:pendulum_example_states}.
%We assume $g(s_t) = \theta_2$, thus measurements contain only a second component of the state
%vector.
%Other components of the state vector cannot be observed.
%The measurement noise $\omega$ is distributed according to the Normal distribution
%$\mathcal{N}(0, \Omega)$, where the variance $\Omega$ is set to be $0.3$.
%The figure shows only first $500$ time steps.
%}
%\label{fig:sim:pendulum_example_observations}
%\end{figure}

\subsection{The probabilistic model and the inference specification}

Listing~\ref{lst:sim:double_pendulum_model_specification} provides an example of the specification of the 
probabilistic model for the double pendulum system with nonlinear
dynamics~\eqref{eq:sim:nlds_model} using the RxInfer
framework.
\begin{figure*}
  \begin{adjustbox}{minipage=\textwidth,margin=0pt \smallskipamount,center}
    \jlinputlisting[label={lst:sim:double_pendulum_model_specification}, caption={
          An example of the specification of the probabilistic model for the double pendulum system with nonlinear dynamics~\eqref{eq:sim:nlds_model}.
        },captionpos=b]{contents/05-experiments/code/03-double-pendulum-model.jl}
  \end{adjustbox}
\end{figure*}
As part of the inference specification, we also introduce extra factorization constraints for
the variational family of distributions $Q_{B}$ using the \jlinl{@constraints} macro, as shown
in Listing~\eqref{lst:sim:double_pendulum_constraints}. 

These constraints assume that the states $\bm{s}$ and the precision of the measurement noise
are jointly independent.
\begin{figure*}
  \begin{adjustbox}{minipage=\textwidth,margin=0pt \smallskipamount,center}
    \jlinputlisting[label={lst:sim:double_pendulum_constraints}, caption={
          Extra factorization constraints for the variational family of distributions $Q_{B}$ in the probabilistic model of the double pendulum dynamics, which is defined in Listing~\eqref{lst:sim:double_pendulum_model_specification}.
        },captionpos=b]{contents/05-experiments/code/03-double-pendulum-constraints.jl}
  \end{adjustbox}
\end{figure*}
We utilize the \jlinl{@meta} macro to define an approximate inference strategy for the factor
$f$ in the model, as obtaining an exact solution is not feasible.
For our approximation method, we employ the \jlinl{Linearization()} method, which is a
first-order Taylor series expansion approximation.
The method locally approximates the nonlinearity with a linear function and performs exact
inference on the approximated factor \citep[Section~5.2]{sarkka_bayesian_2013}.
The \jlinl{Linearization()} method is provided by the RxInfer framework.
Other approximation strategies are possible, for example, Unscented transform
\citep[Section~5.5]{sarkka_bayesian_2013} or the Conjugate-Computation variational inference 
\citep{khan_conjugate-computation_2017,akbayrak_probabilistic_2022}.
\begin{figure*}
  \begin{adjustbox}{minipage=\textwidth,margin=0pt \smallskipamount,center}
    \jlinputlisting[label={lst:sim:double_pendulum_meta}, caption={
          Approximation method using \jlinl{Linearization()} for the nonlinear factor $f$ in the probabilistic model of the double pendulum dynamics defined in Listing~\eqref{lst:sim:double_pendulum_model_specification}.
          The \jlinl{Linearization()} method, which is a first-order Taylor series expansion approximation, is provided by the RxInfer framework.
        },captionpos=b]{contents/05-experiments/code/03-double-pendulum-meta.jl}
  \end{adjustbox}
\end{figure*}
In order to execute the inference procedure we simply call the \jlinl{inference()} function,
see Listing~\eqref{lst:sim:double_pendulum_inference}.
Figure~\ref{fig:sim:pendulum_example_inference_states} presents an example of the inference
task, showing the inferred posterior distributions over the states along with their
respective uncertainties.
Implementing the model and inference specifications for this type of model requires
approximately 30 lines of code (\hyperlink{experiments:userfriendliness}{\emph{User-friendliness}}).
\begin{figure*}
  \begin{adjustbox}{minipage=\textwidth,margin=0pt \smallskipamount,center}
    \jlinputlisting[label={lst:sim:double_pendulum_inference}, caption={
          An example of executing the inference procedure for the probabilistic model of the double pendulum dynamics defined in Listing~\eqref{lst:sim:double_pendulum_model_specification}, with constraints and approximation methods specified in Listing~\ref{lst:sim:double_pendulum_constraints} and Listing~\ref{lst:sim:double_pendulum_meta}, respectively.
        },captionpos=b]{contents/05-experiments/code/03-double-pendulum-inference.jl}
  \end{adjustbox}
\end{figure*}

%\begin{figure*}
%\begin{adjustbox}{minipage=\textwidth,margin=0pt \smallskipamount,center}
%\jlinputlisting[label={lst:sim:double_pendulum_full_example}, caption={An example of the probabilistic model and inference specification for the double pendulum non-linear dynamics~\eqref{eq:sim:nlds}-\eqref{eq:sim:pendulum_state_transition}.
%},captionpos=b]{contents/05-experiments/code/03-double-pendulum-full-example.jl}
%\end{adjustbox}
%\end{figure*}

\begin{figure}
  \centering
  \begin{subfigure}[t]{0.315\textwidth}
    \centering
    \resizebox{\textwidth}{!}{
        % % Recommended preamble:
% \usetikzlibrary{arrows.meta}
% \usetikzlibrary{backgrounds}
% \usepgfplotslibrary{patchplots}
% \usepgfplotslibrary{fillbetween}
% \pgfplotsset{%
%     layers/standard/.define layer set={%
%         background,axis background,axis grid,axis ticks,axis lines,axis tick labels,pre main,main,axis descriptions,axis foreground%
%     }{
%         grid style={/pgfplots/on layer=axis grid},%
%         tick style={/pgfplots/on layer=axis ticks},%
%         axis line style={/pgfplots/on layer=axis lines},%
%         label style={/pgfplots/on layer=axis descriptions},%
%         legend style={/pgfplots/on layer=axis descriptions},%
%         title style={/pgfplots/on layer=axis descriptions},%
%         colorbar style={/pgfplots/on layer=axis descriptions},%
%         ticklabel style={/pgfplots/on layer=axis tick labels},%
%         axis background@ style={/pgfplots/on layer=axis background},%
%         3d box foreground style={/pgfplots/on layer=axis foreground},%
%     },
% }

\begin{tikzpicture}[/tikz/background rectangle/.style={fill={rgb,1:red,1.0;green,1.0;blue,1.0}, fill opacity={1.0}, draw opacity={1.0}}, show background rectangle]
\begin{axis}[point meta max={nan}, point meta min={nan}, legend cell align={left}, legend columns={1}, title={}, title style={at={{(0.5,1)}}, anchor={south}, font={{\fontsize{18 pt}{23.400000000000002 pt}\selectfont}}, color={rgb,1:red,0.0;green,0.0;blue,0.0}, draw opacity={1.0}, rotate={0.0}, align={center}}, legend style={color={rgb,1:red,0.0;green,0.0;blue,0.0}, draw opacity={1.0}, line width={1}, solid, fill={rgb,1:red,1.0;green,1.0;blue,1.0}, fill opacity={1.0}, text opacity={1.0}, font={{\fontsize{14 pt}{18.2 pt}\selectfont}}, text={rgb,1:red,0.0;green,0.0;blue,0.0}, cells={anchor={center}}, at={(0.02, 0.02)}, anchor={south west}}, axis background/.style={fill={rgb,1:red,1.0;green,1.0;blue,1.0}, opacity={1.0}}, anchor={north west}, xshift={1.0mm}, yshift={-1.0mm}, width={99.6mm}, height={74.2mm}, scaled x ticks={false}, xlabel={Time step index}, x tick style={color={rgb,1:red,0.0;green,0.0;blue,0.0}, opacity={1.0}}, x tick label style={color={rgb,1:red,0.0;green,0.0;blue,0.0}, opacity={1.0}, rotate={0}}, xlabel style={at={(ticklabel cs:0.5)}, anchor=near ticklabel, at={{(ticklabel cs:0.5)}}, anchor={near ticklabel}, font={{\fontsize{16 pt}{20.8 pt}\selectfont}}, color={rgb,1:red,0.0;green,0.0;blue,0.0}, draw opacity={1.0}, rotate={0.0}}, xmajorgrids={true}, xmin={-6.469999999999999}, xmax={257.47}, xticklabels={{$0$,$50$,$100$,$150$,$200$,$250$}}, xtick={{0.0,50.0,100.0,150.0,200.0,250.0}}, xtick align={inside}, xticklabel style={font={{\fontsize{14 pt}{18.2 pt}\selectfont}}, color={rgb,1:red,0.0;green,0.0;blue,0.0}, draw opacity={1.0}, rotate={0.0}}, x grid style={color={rgb,1:red,0.0;green,0.0;blue,0.0}, draw opacity={0.1}, line width={0.5}, solid}, axis x line*={left}, x axis line style={color={rgb,1:red,0.0;green,0.0;blue,0.0}, draw opacity={1.0}, line width={1}, solid}, scaled y ticks={false}, ylabel={Angle (radians)}, y tick style={color={rgb,1:red,0.0;green,0.0;blue,0.0}, opacity={1.0}}, y tick label style={color={rgb,1:red,0.0;green,0.0;blue,0.0}, opacity={1.0}, rotate={0}}, ylabel style={at={(ticklabel cs:0.5)}, anchor=near ticklabel, at={{(ticklabel cs:0.5)}}, anchor={near ticklabel}, font={{\fontsize{16 pt}{20.8 pt}\selectfont}}, color={rgb,1:red,0.0;green,0.0;blue,0.0}, draw opacity={1.0}, rotate={0.0}}, ymajorgrids={true}, ymin={-4.064075544819777}, ymax={2.3669820967456285}, yticklabels={{$-4$,$-3$,$-2$,$-1$,$0$,$1$,$2$}}, ytick={{-4.0,-3.0,-2.0,-1.0,0.0,1.0,2.0}}, ytick align={inside}, yticklabel style={font={{\fontsize{14 pt}{18.2 pt}\selectfont}}, color={rgb,1:red,0.0;green,0.0;blue,0.0}, draw opacity={1.0}, rotate={0.0}}, y grid style={color={rgb,1:red,0.0;green,0.0;blue,0.0}, draw opacity={0.1}, line width={0.5}, solid}, axis y line*={left}, y axis line style={color={rgb,1:red,0.0;green,0.0;blue,0.0}, draw opacity={1.0}, line width={1}, solid}, colorbar={false}]
    \addplot+[line width={0}, draw opacity={0}, fill={rgb,1:red,0.8824;green,0.3412;blue,0.349}, fill opacity={0.5}, mark={none}, forget plot]
        coordinates {
            (1,1.1684620654295985)
            (2,1.1723934370800164)
            (3,1.174189896401998)
            (4,1.1742456488486275)
            (5,1.1725627849895122)
            (6,1.1691969857234075)
            (7,1.1641717510311234)
            (8,1.1574927864631517)
            (9,1.1491695495092458)
            (10,1.1391402878836963)
            (11,1.1273321405153243)
            (12,1.1136996165568733)
            (13,1.098348267838681)
            (14,1.0812207623594392)
            (15,1.0620165832945592)
            (16,1.0405921697456293)
            (17,1.0169219690399185)
            (18,0.9911319148636832)
            (19,0.9634854165117018)
            (20,0.9341082069087829)
            (21,0.9031059759877039)
            (22,0.8706007886354012)
            (23,0.8367268876695846)
            (24,0.8015972752852359)
            (25,0.7653753401433059)
            (26,0.7282696269701487)
            (27,0.6903483473595186)
            (28,0.6517633330445693)
            (29,0.6128816166922897)
            (30,0.5740128575430689)
            (31,0.5353689542893891)
            (32,0.49716100524022977)
            (33,0.45944415943711286)
            (34,0.42217991098603175)
            (35,0.3854015697437212)
            (36,0.349152577999913)
            (37,0.31344861985503586)
            (38,0.2782914106467327)
            (39,0.24367015593832586)
            (40,0.2095698380905346)
            (41,0.1759825270921614)
            (42,0.14289830719697763)
            (43,0.11031526303467981)
            (44,0.07823821157114501)
            (45,0.046668906347046235)
            (46,0.01562804472565035)
            (47,-0.014842266901310429)
            (48,-0.04469810306378706)
            (49,-0.07388871253824947)
            (50,-0.102355038834288)
            (51,-0.13003492774196324)
            (52,-0.1568597649340207)
            (53,-0.18275309419693803)
            (54,-0.20763327116265792)
            (55,-0.23141122597493688)
            (56,-0.2539936203149264)
            (57,-0.2752860399070448)
            (58,-0.29519578124546264)
            (59,-0.3136160252575994)
            (60,-0.330426247570315)
            (61,-0.34551392443985474)
            (62,-0.35877925445148035)
            (63,-0.3701353129036664)
            (64,-0.3794348375646544)
            (65,-0.38650560045414173)
            (66,-0.3912095788900051)
            (67,-0.3933703051109696)
            (68,-0.39284986650658543)
            (69,-0.3896246966970606)
            (70,-0.38368590070826425)
            (71,-0.37507420391896096)
            (72,-0.3641478663995805)
            (73,-0.351578300234441)
            (74,-0.3382054358314164)
            (75,-0.32498709451773705)
            (76,-0.3127383982822008)
            (77,-0.30203696300147453)
            (78,-0.29317908429058465)
            (79,-0.2861995810816395)
            (80,-0.2810805416183223)
            (81,-0.27777685741582586)
            (82,-0.27621551303521635)
            (83,-0.27630556132332673)
            (84,-0.27794294268188247)
            (85,-0.28099766259323417)
            (86,-0.2853316534570666)
            (87,-0.2908183201000722)
            (88,-0.29733482489042695)
            (89,-0.30475930960813274)
            (90,-0.3129744397086793)
            (91,-0.321882022495301)
            (92,-0.33139307179689903)
            (93,-0.34141783342333565)
            (94,-0.3518754453655576)
            (95,-0.36268604478433475)
            (96,-0.37376712402263357)
            (97,-0.38503942361659266)
            (98,-0.3964286994427101)
            (99,-0.40787077173874603)
            (100,-0.41931254624732667)
            (101,-0.4307024662433299)
            (102,-0.44198495427192486)
            (103,-0.45310707955558455)
            (104,-0.4640201829193735)
            (105,-0.4746785645225312)
            (106,-0.48504115686324906)
            (107,-0.4950676481182369)
            (108,-0.5047208333987061)
            (109,-0.5139679116554192)
            (110,-0.522778072622607)
            (111,-0.531124930658389)
            (112,-0.5389833020780798)
            (113,-0.5463282869424065)
            (114,-0.5531389242335379)
            (115,-0.5593989803832607)
            (116,-0.5650953572549223)
            (117,-0.5702170484901375)
            (118,-0.5747543121643969)
            (119,-0.578698796344267)
            (120,-0.5820442931128118)
            (121,-0.5847869310379269)
            (122,-0.5869254527034546)
            (123,-0.588459660837291)
            (124,-0.5893911687009826)
            (125,-0.5897239991744033)
            (126,-0.5894637728404195)
            (127,-0.5886174958627453)
            (128,-0.5871938258108506)
            (129,-0.5852053225787665)
            (130,-0.5826664594702683)
            (131,-0.5795913187951244)
            (132,-0.5759968600766823)
            (133,-0.5719024055110707)
            (134,-0.5673309144038609)
            (135,-0.5623073316174307)
            (136,-0.5568583711330003)
            (137,-0.5510141711625326)
            (138,-0.5448076620984703)
            (139,-0.5382733422918684)
            (140,-0.5314482034215263)
            (141,-0.5243712040313214)
            (142,-0.5170855360429609)
            (143,-0.509638459782994)
            (144,-0.5020815245658733)
            (145,-0.4944681784001883)
            (146,-0.4868530702276602)
            (147,-0.4792960860894861)
            (148,-0.47185801812751954)
            (149,-0.4646026153278246)
            (150,-0.45760094379784705)
            (151,-0.4509274275480017)
            (152,-0.44465944479219394)
            (153,-0.43887862996749905)
            (154,-0.43367229287421766)
            (155,-0.4291341659234812)
            (156,-0.42536202246626265)
            (157,-0.4224580567364321)
            (158,-0.42052756155762877)
            (159,-0.4196852411934716)
            (160,-0.4200511757642021)
            (161,-0.42173823289945966)
            (162,-0.4248220646686962)
            (163,-0.4293532073721908)
            (164,-0.4353540779023874)
            (165,-0.44273144451895413)
            (166,-0.45127767217092357)
            (167,-0.46065549510715725)
            (168,-0.47037434144631285)
            (169,-0.47981026472807875)
            (170,-0.4882990971011925)
            (171,-0.4952722273568802)
            (172,-0.5003109338296524)
            (173,-0.5031530179369117)
            (174,-0.5036710964735998)
            (175,-0.5018133949587681)
            (176,-0.49758795783854926)
            (177,-0.49104997948962065)
            (178,-0.4822805649024758)
            (179,-0.47136735620300835)
            (180,-0.45840303891462253)
            (181,-0.44348922862150125)
            (182,-0.42673318434826846)
            (183,-0.4082441709407492)
            (184,-0.388126354036324)
            (185,-0.3664600652045722)
            (186,-0.3433207254377717)
            (187,-0.318793354367488)
            (188,-0.2929597919529978)
            (189,-0.26589911383459697)
            (190,-0.2376871360785799)
            (191,-0.20839411057934856)
            (192,-0.17809357162293984)
            (193,-0.14684716104704448)
            (194,-0.11467884644416497)
            (195,-0.08160441260411126)
            (196,-0.04764066158510331)
            (197,-0.012799601374379101)
            (198,0.02291320853793495)
            (199,0.05949083934322415)
            (200,0.0969386738614862)
            (201,0.13528836439546876)
            (202,0.17457607728578872)
            (203,0.21483511612775055)
            (204,0.25606969231690624)
            (205,0.2982555150727469)
            (206,0.34134136264004816)
            (207,0.38525108346562675)
            (208,0.4298150753663038)
            (209,0.47474613787408004)
            (210,0.519662179295802)
            (211,0.5641349775847555)
            (212,0.6077480540018608)
            (213,0.6501353778470986)
            (214,0.6910160570979739)
            (215,0.7302194851356303)
            (216,0.7676610902649481)
            (217,0.803313715734488)
            (218,0.837192146825965)
            (219,0.8693366276072692)
            (220,0.8998017250650836)
            (221,0.9286460761418802)
            (222,0.9559257183272065)
            (223,0.9816923817060574)
            (224,1.00599301877092)
            (225,1.0288687000700112)
            (226,1.0503569874676457)
            (227,1.0704913356305026)
            (228,1.0892984942794903)
            (229,1.1068030351049116)
            (230,1.1230259311710469)
            (231,1.1379812828985796)
            (232,1.1516822862576344)
            (233,1.164137791158527)
            (234,1.1753526394455585)
            (235,1.1853346728307481)
            (236,1.1940911003633636)
            (237,1.2016274664821907)
            (238,1.2079441128466988)
            (239,1.2130372452422722)
            (240,1.216904397911528)
            (241,1.2195432793029775)
            (242,1.2209483179217793)
            (243,1.221111894754575)
            (244,1.2200243847274628)
            (245,1.2176722573206078)
            (246,1.2140423677578511)
            (247,1.2091189583574147)
            (248,1.2028842804702486)
            (249,1.1953181112993307)
            (250,1.1863949031211196)
            (250,0.8829227350024286)
            (249,0.8997327703409771)
            (248,0.9146817347985918)
            (247,0.9278247221116738)
            (246,0.939211290545604)
            (245,0.9488862231124804)
            (244,0.9568903825676558)
            (243,0.9632590518641351)
            (242,0.9680209199652252)
            (241,0.9712029014116159)
            (240,0.9728284723836037)
            (239,0.9729175041921824)
            (238,0.9714861584689571)
            (237,0.9685475580012655)
            (236,0.9641135846495825)
            (235,0.9581923302659874)
            (234,0.9507862199993218)
            (233,0.9418936305086925)
            (232,0.9315095462729228)
            (231,0.9196303683029575)
            (230,0.9062474619318501)
            (229,0.8913457524812081)
            (228,0.874907945474585)
            (227,0.8569053031912974)
            (226,0.837301407477547)
            (225,0.8160548860982224)
            (224,0.7931105531030156)
            (223,0.7684040388648377)
            (222,0.7418643110228459)
            (221,0.7134093254108391)
            (220,0.6829492313133928)
            (219,0.6503876273397429)
            (218,0.6156275905190167)
            (217,0.5785948543642859)
            (216,0.5392684736055074)
            (215,0.4977240317592756)
            (214,0.45414845048531993)
            (213,0.4088415009299732)
            (212,0.36232306603505715)
            (211,0.31540518128910106)
            (210,0.26888825988690773)
            (209,0.22345921648001915)
            (208,0.17962395282958904)
            (207,0.13766041874894083)
            (206,0.09756503478712503)
            (205,0.05894151741636888)
            (204,0.021428318904967714)
            (203,-0.01512712729133775)
            (202,-0.05091894604256869)
            (201,-0.08602762687094256)
            (200,-0.12044654612994908)
            (199,-0.15416702201582866)
            (198,-0.18717164472130682)
            (197,-0.2194354099537539)
            (196,-0.25092934056448407)
            (195,-0.28161547867660763)
            (194,-0.3114476263048334)
            (193,-0.34036922591785135)
            (192,-0.36832737779647917)
            (191,-0.39526567105084554)
            (190,-0.4211194913442867)
            (189,-0.445820113542975)
            (188,-0.4692984369612311)
            (187,-0.49148456612112423)
            (186,-0.5123098554150474)
            (185,-0.5317001511180599)
            (184,-0.5495789775827595)
            (183,-0.565880209224777)
            (182,-0.5805423434711947)
            (181,-0.5935135481994808)
            (180,-0.6047520393965465)
            (179,-0.6142308067089748)
            (178,-0.6219418968158841)
            (177,-0.6279076160236451)
            (176,-0.6321837030576181)
            (175,-0.63485404090369)
            (174,-0.6360260748586521)
            (173,-0.6358128961921561)
            (172,-0.6342697859223896)
            (171,-0.6313244294268272)
            (170,-0.626865918549339)
            (169,-0.6208665187440711)
            (168,-0.6136077211707504)
            (167,-0.6057346591246713)
            (166,-0.5980437476104716)
            (165,-0.5913241597275878)
            (164,-0.5861177415040597)
            (163,-0.5827703220156815)
            (162,-0.5813558392860136)
            (161,-0.5817796115775722)
            (160,-0.5839333863539903)
            (159,-0.5876623289082774)
            (158,-0.5927843526371944)
            (157,-0.5991171556264543)
            (156,-0.6064976799522791)
            (155,-0.6147820415399123)
            (154,-0.6238350976220294)
            (153,-0.6335305059879363)
            (152,-0.6437488571418424)
            (151,-0.6543819076937527)
            (150,-0.6653313224198067)
            (149,-0.6765051028388862)
            (148,-0.6878155018775505)
            (147,-0.6991790587553603)
            (146,-0.7105251168241334)
            (145,-0.7217878838830663)
            (144,-0.7329005031480358)
            (143,-0.7438045925626489)
            (142,-0.754442377106741)
            (141,-0.7647560592153777)
            (140,-0.7746946642519763)
            (139,-0.7842132504186441)
            (138,-0.793275600770722)
            (137,-0.8018462607966215)
            (136,-0.8098922075056172)
            (135,-0.8173801491989212)
            (134,-0.8242794346594957)
            (133,-0.8305665085619555)
            (132,-0.8362180748108765)
            (131,-0.8412115927870445)
            (130,-0.845531904587021)
            (129,-0.8491683408639351)
            (128,-0.852112977735054)
            (127,-0.8543533236890952)
            (126,-0.85587460175758)
            (125,-0.8566736162231864)
            (124,-0.8567464621284653)
            (123,-0.8560895082304667)
            (122,-0.854698291473617)
            (121,-0.8525721521759873)
            (120,-0.8497114671123474)
            (119,-0.8461150980095724)
            (118,-0.8417845143521236)
            (117,-0.8367254944386957)
            (116,-0.8309447271855952)
            (115,-0.8244500084168193)
            (114,-0.8172481494724664)
            (113,-0.809348902167669)
            (112,-0.8007655076175819)
            (111,-0.7915173971751343)
            (110,-0.7816282383836353)
            (109,-0.7711210815148274)
            (108,-0.7600185509475814)
            (107,-0.7483482904728445)
            (106,-0.7361399970604312)
            (105,-0.7234270039650452)
            (104,-0.710246591294601)
            (103,-0.6966369034372859)
            (102,-0.6826405496820378)
            (101,-0.6683035964030011)
            (100,-0.6536760627420193)
            (99,-0.6388117494478087)
            (98,-0.6237634639153944)
            (97,-0.6085859558651718)
            (96,-0.593343533511356)
            (95,-0.5781098556536404)
            (94,-0.5629639289374965)
            (93,-0.5479889654616501)
            (92,-0.5332686168017831)
            (91,-0.5188903747998066)
            (90,-0.5049504595803496)
            (89,-0.4915471298585508)
            (88,-0.478790353868647)
            (87,-0.4668073590385788)
            (86,-0.4557332492997831)
            (85,-0.4457103833895895)
            (84,-0.43689282240779503)
            (83,-0.4294541848223595)
            (82,-0.42356808417752134)
            (81,-0.4194124722421809)
            (80,-0.41716147627136874)
            (79,-0.41698909938537354)
            (78,-0.4190346213286292)
            (77,-0.42342268434304964)
            (76,-0.4301902130420908)
            (75,-0.43908952152998293)
            (74,-0.44958349959567945)
            (73,-0.4608505408048593)
            (72,-0.4720144567012352)
            (71,-0.48231749362020926)
            (70,-0.49120526434370915)
            (69,-0.4984721034781206)
            (68,-0.5040507688708085)
            (67,-0.5077579708851522)
            (66,-0.5094525400817447)
            (65,-0.5091036834384504)
            (64,-0.5067196292057784)
            (63,-0.5023091478229235)
            (62,-0.49595830442372724)
            (61,-0.4877548420162772)
            (60,-0.4777263072624712)
            (59,-0.46592592578459396)
            (58,-0.45243740756379336)
            (57,-0.43737100659021366)
            (56,-0.4208264806260655)
            (55,-0.4028849399824577)
            (54,-0.38363674693303607)
            (53,-0.3631790982758871)
            (52,-0.34160912188248543)
            (51,-0.3190156158946831)
            (50,-0.29548478925328014)
            (49,-0.2710935819438207)
            (48,-0.24591476212248348)
            (47,-0.2200263475962896)
            (46,-0.19349774748466578)
            (45,-0.16639397080495766)
            (44,-0.13878826727928123)
            (43,-0.11071499255029982)
            (42,-0.08216449536581669)
            (41,-0.053147992683330536)
            (40,-0.02367006204197758)
            (39,0.00632255371818774)
            (38,0.03686061508396102)
            (37,0.06794378583218375)
            (36,0.09962164678631619)
            (35,0.1319594539666979)
            (34,0.16501093740980471)
            (33,0.19881829044185)
            (32,0.233406050988411)
            (31,0.26880961197031056)
            (30,0.30505657834973793)
            (29,0.3420788991675071)
            (28,0.37974554446946934)
            (27,0.417754913288498)
            (26,0.45569980587387543)
            (25,0.49342920782802097)
            (24,0.5308964868985537)
            (23,0.5678381225782984)
            (22,0.6040278990369158)
            (21,0.639320405038813)
            (20,0.6735529645449354)
            (19,0.7065504090141721)
            (18,0.7381484117557555)
            (17,0.7681602858997012)
            (16,0.7962125618483642)
            (15,0.8220850575574473)
            (14,0.8456675012134012)
            (13,0.8669052702043317)
            (12,0.8859963367237453)
            (11,0.9030774236091625)
            (10,0.918130274957954)
            (9,0.9311912907443805)
            (8,0.942314499423422)
            (7,0.9515525880343685)
            (6,0.9588923281413013)
            (5,0.9643188112231794)
            (4,0.9678082350388932)
            (3,0.9693071287757883)
            (2,0.968813402378395)
            (1,0.9659791826877318)
            (1,1.1684620654295985)
        }
        ;
    \addplot+[line width={0}, draw opacity={0}, fill={rgb,1:red,0.8824;green,0.3412;blue,0.349}, fill opacity={0.5}, mark={none}, forget plot]
        coordinates {
            (1,1.1684620654295985)
            (2,1.1723934370800164)
            (3,1.174189896401998)
            (4,1.1742456488486275)
            (5,1.1725627849895122)
            (6,1.1691969857234075)
            (7,1.1641717510311234)
            (8,1.1574927864631517)
            (9,1.1491695495092458)
            (10,1.1391402878836963)
            (11,1.1273321405153243)
            (12,1.1136996165568733)
            (13,1.098348267838681)
            (14,1.0812207623594392)
            (15,1.0620165832945592)
            (16,1.0405921697456293)
            (17,1.0169219690399185)
            (18,0.9911319148636832)
            (19,0.9634854165117018)
            (20,0.9341082069087829)
            (21,0.9031059759877039)
            (22,0.8706007886354012)
            (23,0.8367268876695846)
            (24,0.8015972752852359)
            (25,0.7653753401433059)
            (26,0.7282696269701487)
            (27,0.6903483473595186)
            (28,0.6517633330445693)
            (29,0.6128816166922897)
            (30,0.5740128575430689)
            (31,0.5353689542893891)
            (32,0.49716100524022977)
            (33,0.45944415943711286)
            (34,0.42217991098603175)
            (35,0.3854015697437212)
            (36,0.349152577999913)
            (37,0.31344861985503586)
            (38,0.2782914106467327)
            (39,0.24367015593832586)
            (40,0.2095698380905346)
            (41,0.1759825270921614)
            (42,0.14289830719697763)
            (43,0.11031526303467981)
            (44,0.07823821157114501)
            (45,0.046668906347046235)
            (46,0.01562804472565035)
            (47,-0.014842266901310429)
            (48,-0.04469810306378706)
            (49,-0.07388871253824947)
            (50,-0.102355038834288)
            (51,-0.13003492774196324)
            (52,-0.1568597649340207)
            (53,-0.18275309419693803)
            (54,-0.20763327116265792)
            (55,-0.23141122597493688)
            (56,-0.2539936203149264)
            (57,-0.2752860399070448)
            (58,-0.29519578124546264)
            (59,-0.3136160252575994)
            (60,-0.330426247570315)
            (61,-0.34551392443985474)
            (62,-0.35877925445148035)
            (63,-0.3701353129036664)
            (64,-0.3794348375646544)
            (65,-0.38650560045414173)
            (66,-0.3912095788900051)
            (67,-0.3933703051109696)
            (68,-0.39284986650658543)
            (69,-0.3896246966970606)
            (70,-0.38368590070826425)
            (71,-0.37507420391896096)
            (72,-0.3641478663995805)
            (73,-0.351578300234441)
            (74,-0.3382054358314164)
            (75,-0.32498709451773705)
            (76,-0.3127383982822008)
            (77,-0.30203696300147453)
            (78,-0.29317908429058465)
            (79,-0.2861995810816395)
            (80,-0.2810805416183223)
            (81,-0.27777685741582586)
            (82,-0.27621551303521635)
            (83,-0.27630556132332673)
            (84,-0.27794294268188247)
            (85,-0.28099766259323417)
            (86,-0.2853316534570666)
            (87,-0.2908183201000722)
            (88,-0.29733482489042695)
            (89,-0.30475930960813274)
            (90,-0.3129744397086793)
            (91,-0.321882022495301)
            (92,-0.33139307179689903)
            (93,-0.34141783342333565)
            (94,-0.3518754453655576)
            (95,-0.36268604478433475)
            (96,-0.37376712402263357)
            (97,-0.38503942361659266)
            (98,-0.3964286994427101)
            (99,-0.40787077173874603)
            (100,-0.41931254624732667)
            (101,-0.4307024662433299)
            (102,-0.44198495427192486)
            (103,-0.45310707955558455)
            (104,-0.4640201829193735)
            (105,-0.4746785645225312)
            (106,-0.48504115686324906)
            (107,-0.4950676481182369)
            (108,-0.5047208333987061)
            (109,-0.5139679116554192)
            (110,-0.522778072622607)
            (111,-0.531124930658389)
            (112,-0.5389833020780798)
            (113,-0.5463282869424065)
            (114,-0.5531389242335379)
            (115,-0.5593989803832607)
            (116,-0.5650953572549223)
            (117,-0.5702170484901375)
            (118,-0.5747543121643969)
            (119,-0.578698796344267)
            (120,-0.5820442931128118)
            (121,-0.5847869310379269)
            (122,-0.5869254527034546)
            (123,-0.588459660837291)
            (124,-0.5893911687009826)
            (125,-0.5897239991744033)
            (126,-0.5894637728404195)
            (127,-0.5886174958627453)
            (128,-0.5871938258108506)
            (129,-0.5852053225787665)
            (130,-0.5826664594702683)
            (131,-0.5795913187951244)
            (132,-0.5759968600766823)
            (133,-0.5719024055110707)
            (134,-0.5673309144038609)
            (135,-0.5623073316174307)
            (136,-0.5568583711330003)
            (137,-0.5510141711625326)
            (138,-0.5448076620984703)
            (139,-0.5382733422918684)
            (140,-0.5314482034215263)
            (141,-0.5243712040313214)
            (142,-0.5170855360429609)
            (143,-0.509638459782994)
            (144,-0.5020815245658733)
            (145,-0.4944681784001883)
            (146,-0.4868530702276602)
            (147,-0.4792960860894861)
            (148,-0.47185801812751954)
            (149,-0.4646026153278246)
            (150,-0.45760094379784705)
            (151,-0.4509274275480017)
            (152,-0.44465944479219394)
            (153,-0.43887862996749905)
            (154,-0.43367229287421766)
            (155,-0.4291341659234812)
            (156,-0.42536202246626265)
            (157,-0.4224580567364321)
            (158,-0.42052756155762877)
            (159,-0.4196852411934716)
            (160,-0.4200511757642021)
            (161,-0.42173823289945966)
            (162,-0.4248220646686962)
            (163,-0.4293532073721908)
            (164,-0.4353540779023874)
            (165,-0.44273144451895413)
            (166,-0.45127767217092357)
            (167,-0.46065549510715725)
            (168,-0.47037434144631285)
            (169,-0.47981026472807875)
            (170,-0.4882990971011925)
            (171,-0.4952722273568802)
            (172,-0.5003109338296524)
            (173,-0.5031530179369117)
            (174,-0.5036710964735998)
            (175,-0.5018133949587681)
            (176,-0.49758795783854926)
            (177,-0.49104997948962065)
            (178,-0.4822805649024758)
            (179,-0.47136735620300835)
            (180,-0.45840303891462253)
            (181,-0.44348922862150125)
            (182,-0.42673318434826846)
            (183,-0.4082441709407492)
            (184,-0.388126354036324)
            (185,-0.3664600652045722)
            (186,-0.3433207254377717)
            (187,-0.318793354367488)
            (188,-0.2929597919529978)
            (189,-0.26589911383459697)
            (190,-0.2376871360785799)
            (191,-0.20839411057934856)
            (192,-0.17809357162293984)
            (193,-0.14684716104704448)
            (194,-0.11467884644416497)
            (195,-0.08160441260411126)
            (196,-0.04764066158510331)
            (197,-0.012799601374379101)
            (198,0.02291320853793495)
            (199,0.05949083934322415)
            (200,0.0969386738614862)
            (201,0.13528836439546876)
            (202,0.17457607728578872)
            (203,0.21483511612775055)
            (204,0.25606969231690624)
            (205,0.2982555150727469)
            (206,0.34134136264004816)
            (207,0.38525108346562675)
            (208,0.4298150753663038)
            (209,0.47474613787408004)
            (210,0.519662179295802)
            (211,0.5641349775847555)
            (212,0.6077480540018608)
            (213,0.6501353778470986)
            (214,0.6910160570979739)
            (215,0.7302194851356303)
            (216,0.7676610902649481)
            (217,0.803313715734488)
            (218,0.837192146825965)
            (219,0.8693366276072692)
            (220,0.8998017250650836)
            (221,0.9286460761418802)
            (222,0.9559257183272065)
            (223,0.9816923817060574)
            (224,1.00599301877092)
            (225,1.0288687000700112)
            (226,1.0503569874676457)
            (227,1.0704913356305026)
            (228,1.0892984942794903)
            (229,1.1068030351049116)
            (230,1.1230259311710469)
            (231,1.1379812828985796)
            (232,1.1516822862576344)
            (233,1.164137791158527)
            (234,1.1753526394455585)
            (235,1.1853346728307481)
            (236,1.1940911003633636)
            (237,1.2016274664821907)
            (238,1.2079441128466988)
            (239,1.2130372452422722)
            (240,1.216904397911528)
            (241,1.2195432793029775)
            (242,1.2209483179217793)
            (243,1.221111894754575)
            (244,1.2200243847274628)
            (245,1.2176722573206078)
            (246,1.2140423677578511)
            (247,1.2091189583574147)
            (248,1.2028842804702486)
            (249,1.1953181112993307)
            (250,1.1863949031211196)
            (250,1.4898670712398108)
            (249,1.4909034522576845)
            (248,1.4910868261419055)
            (247,1.4904131946031556)
            (246,1.4888734449700982)
            (245,1.4864582915287352)
            (244,1.4831583868872698)
            (243,1.4789647376450148)
            (242,1.4738757158783333)
            (241,1.467883657194339)
            (240,1.4609803234394523)
            (239,1.453156986292362)
            (238,1.4444020672244404)
            (237,1.4347073749631158)
            (236,1.4240686160771447)
            (235,1.4124770153955089)
            (234,1.3999190588917951)
            (233,1.3863819518083613)
            (232,1.371855026242346)
            (231,1.3563321974942015)
            (230,1.3398044004102436)
            (229,1.3222603177286152)
            (228,1.3036890430843957)
            (227,1.2840773680697077)
            (226,1.2634125674577446)
            (225,1.2416825140418)
            (224,1.2188754844388245)
            (223,1.1949807245472772)
            (222,1.169987125631567)
            (221,1.1438828268729213)
            (220,1.1166542188167745)
            (219,1.0882856278747954)
            (218,1.0587567031329133)
            (217,1.0280325771046899)
            (216,0.9960537069243888)
            (215,0.962714938511985)
            (214,0.9278836637106278)
            (213,0.8914292547642241)
            (212,0.8531730419686645)
            (211,0.81286477388041)
            (210,0.7704360987046963)
            (209,0.726033059268141)
            (208,0.6800061979030185)
            (207,0.6328417481823126)
            (206,0.5851176904929714)
            (205,0.5375695127291249)
            (204,0.49071106572884476)
            (203,0.44479735954683886)
            (202,0.40007110061414614)
            (201,0.3566043556618801)
            (200,0.3143238938529215)
            (199,0.2731487007022769)
            (198,0.2329980617971767)
            (197,0.1938362072049957)
            (196,0.15564801739427744)
            (195,0.11840665346838512)
            (194,0.08208993341650346)
            (193,0.04667490382376238)
            (192,0.012140234550599482)
            (191,-0.021522550107851618)
            (190,-0.054254780812873105)
            (189,-0.08597811412621889)
            (188,-0.11662114694476455)
            (187,-0.14610214261385177)
            (186,-0.17433159546049604)
            (185,-0.2012199792910845)
            (184,-0.22667373048988848)
            (183,-0.25060813265672144)
            (182,-0.2729240252253422)
            (181,-0.29346490904352174)
            (180,-0.3120540384326985)
            (179,-0.32850390569704185)
            (178,-0.34261923298906743)
            (177,-0.3541923429555962)
            (176,-0.36299221261948045)
            (175,-0.36877274901384616)
            (174,-0.37131611808854736)
            (173,-0.3704931396816673)
            (172,-0.3663520817369152)
            (171,-0.3592200252869332)
            (170,-0.34973227565304593)
            (169,-0.33875401071208644)
            (168,-0.32714096172187535)
            (167,-0.3155763310896433)
            (166,-0.3045115967313755)
            (165,-0.29413872931032053)
            (164,-0.2845904143007151)
            (163,-0.27593609272870007)
            (162,-0.2682882900513788)
            (161,-0.26169685422134714)
            (160,-0.2561689651744139)
            (159,-0.2517081534786658)
            (158,-0.2482707704780631)
            (157,-0.24579895784640995)
            (156,-0.24422636498024622)
            (155,-0.24348629030705007)
            (154,-0.24350948812640597)
            (153,-0.24422675394706184)
            (152,-0.2455700324425455)
            (151,-0.24747294740225073)
            (150,-0.24987056517588738)
            (149,-0.25270012781676293)
            (148,-0.25590053437748855)
            (147,-0.2594131134236119)
            (146,-0.263181023631187)
            (145,-0.26714847291731036)
            (144,-0.2712625459837108)
            (143,-0.2754723270033393)
            (142,-0.27972869497918085)
            (141,-0.2839863488472651)
            (140,-0.2882017425910764)
            (139,-0.29233343416509266)
            (138,-0.29633972342621867)
            (137,-0.3001820815284437)
            (136,-0.30382453476038335)
            (135,-0.3072345140359402)
            (134,-0.31038239414822605)
            (133,-0.313238302460186)
            (132,-0.31577564534248803)
            (131,-0.31797104480320426)
            (130,-0.3198010143535156)
            (129,-0.3212423042935978)
            (128,-0.32227467388664705)
            (127,-0.3228816680363953)
            (126,-0.323052943923259)
            (125,-0.32277438212562026)
            (124,-0.3220358752735)
            (123,-0.32082981344411515)
            (122,-0.31915261393329225)
            (121,-0.31700170989986654)
            (120,-0.3143771191132762)
            (119,-0.31128249467896163)
            (118,-0.3077241099766701)
            (117,-0.3037086025415794)
            (116,-0.2992459873242495)
            (115,-0.29434795234970224)
            (114,-0.2890296989946095)
            (113,-0.28330767171714416)
            (112,-0.2772010965385776)
            (111,-0.27073246414164365)
            (110,-0.26392790686157874)
            (109,-0.25681474179601105)
            (108,-0.2494231158498308)
            (107,-0.24178700576362927)
            (106,-0.23394231666606685)
            (105,-0.22593012508001725)
            (104,-0.21779377454414606)
            (103,-0.20957725567388322)
            (102,-0.20132935886181197)
            (101,-0.19310133608365881)
            (100,-0.1849490297526341)
            (99,-0.17692979402968328)
            (98,-0.1690939349700258)
            (97,-0.16149289136801354)
            (96,-0.1541907145339111)
            (95,-0.14726223391502907)
            (94,-0.14078696179361871)
            (93,-0.13484670138502114)
            (92,-0.12951752679201506)
            (91,-0.12487367019079537)
            (90,-0.12099841983700904)
            (89,-0.11797148935771468)
            (88,-0.11587929591220691)
            (87,-0.1148292811615656)
            (86,-0.11493005761435016)
            (85,-0.11628494179687882)
            (84,-0.11899306295596987)
            (83,-0.12315693782429399)
            (82,-0.12886294189291136)
            (81,-0.1361412425894708)
            (80,-0.14499960696527586)
            (79,-0.1554100627779055)
            (78,-0.16732354725254012)
            (77,-0.1806512416598994)
            (76,-0.19528658352231076)
            (75,-0.21088466750549117)
            (74,-0.22682737206715334)
            (73,-0.24230605966402263)
            (72,-0.25628127609792584)
            (71,-0.26783091421771266)
            (70,-0.27616653707281935)
            (69,-0.28077728991600065)
            (68,-0.28164896414236246)
            (67,-0.278982639336787)
            (66,-0.2729666176982655)
            (65,-0.26390751746983315)
            (64,-0.2521500459235304)
            (63,-0.2379614779844094)
            (62,-0.22160020447923343)
            (61,-0.20327300686343225)
            (60,-0.18312618787815887)
            (59,-0.16130612473060485)
            (58,-0.1379541549271319)
            (57,-0.11320107322387599)
            (56,-0.08716076000378731)
            (55,-0.059937511967416035)
            (54,-0.03162979539227978)
            (53,-0.0023270901179889003)
            (52,0.02788959201444402)
            (51,0.05894576041075661)
            (50,0.09077471158470414)
            (49,0.12331615686732181)
            (48,0.15651855599490938)
            (47,0.19034181379366874)
            (46,0.22475383693596648)
            (45,0.25973178349905013)
            (44,0.2952646904215713)
            (43,0.33134551861965944)
            (42,0.3679611097597719)
            (41,0.4051130468676533)
            (40,0.44280973822304676)
            (39,0.481017758158464)
            (38,0.5197222062095044)
            (37,0.558953453877888)
            (36,0.5986835092135099)
            (35,0.6388436855207444)
            (34,0.6793488845622588)
            (33,0.7200700284323758)
            (32,0.7609159594920485)
            (31,0.8019282966084678)
            (30,0.8429691367363998)
            (29,0.8836843342170723)
            (28,0.9237811216196694)
            (27,0.9629417814305391)
            (26,1.000839448066422)
            (25,1.0373214724585909)
            (24,1.0722980636719182)
            (23,1.105615652760871)
            (22,1.1371736782338866)
            (21,1.1668915469365948)
            (20,1.1946634492726305)
            (19,1.2204204240092316)
            (18,1.244115417971611)
            (17,1.2656836521801358)
            (16,1.2849717776428944)
            (15,1.301948109031671)
            (14,1.3167740235054772)
            (13,1.32979126547303)
            (12,1.3414028963900013)
            (11,1.351586857421486)
            (10,1.3601503008094384)
            (9,1.367147808274111)
            (8,1.3726710735028813)
            (7,1.3767909140278782)
            (6,1.3795016433055136)
            (5,1.380806758755845)
            (4,1.3806830626583617)
            (3,1.3790726640282074)
            (2,1.3759734717816379)
            (1,1.3709449481714653)
            (1,1.1684620654295985)
        }
        ;
    \addplot[color={rgb,1:red,0.8824;green,0.3412;blue,0.349}, name path={7ce786d3-bc8f-42f7-8246-770ba45fb8c2}, legend image code/.code={{
    \draw[fill={rgb,1:red,0.8824;green,0.3412;blue,0.349}, fill opacity={0.5}] (0cm,-0.1cm) rectangle (0.6cm,0.1cm);
    }}, draw opacity={1.0}, line width={1}, solid]
        table[row sep={\\}]
        {
            \\
            1.0  1.1684620654295985  \\
            2.0  1.1723934370800164  \\
            3.0  1.174189896401998  \\
            4.0  1.1742456488486275  \\
            5.0  1.1725627849895122  \\
            6.0  1.1691969857234075  \\
            7.0  1.1641717510311234  \\
            8.0  1.1574927864631517  \\
            9.0  1.1491695495092458  \\
            10.0  1.1391402878836963  \\
            11.0  1.1273321405153243  \\
            12.0  1.1136996165568733  \\
            13.0  1.098348267838681  \\
            14.0  1.0812207623594392  \\
            15.0  1.0620165832945592  \\
            16.0  1.0405921697456293  \\
            17.0  1.0169219690399185  \\
            18.0  0.9911319148636832  \\
            19.0  0.9634854165117018  \\
            20.0  0.9341082069087829  \\
            21.0  0.9031059759877039  \\
            22.0  0.8706007886354012  \\
            23.0  0.8367268876695846  \\
            24.0  0.8015972752852359  \\
            25.0  0.7653753401433059  \\
            26.0  0.7282696269701487  \\
            27.0  0.6903483473595186  \\
            28.0  0.6517633330445693  \\
            29.0  0.6128816166922897  \\
            30.0  0.5740128575430689  \\
            31.0  0.5353689542893891  \\
            32.0  0.49716100524022977  \\
            33.0  0.45944415943711286  \\
            34.0  0.42217991098603175  \\
            35.0  0.3854015697437212  \\
            36.0  0.349152577999913  \\
            37.0  0.31344861985503586  \\
            38.0  0.2782914106467327  \\
            39.0  0.24367015593832586  \\
            40.0  0.2095698380905346  \\
            41.0  0.1759825270921614  \\
            42.0  0.14289830719697763  \\
            43.0  0.11031526303467981  \\
            44.0  0.07823821157114501  \\
            45.0  0.046668906347046235  \\
            46.0  0.01562804472565035  \\
            47.0  -0.014842266901310429  \\
            48.0  -0.04469810306378706  \\
            49.0  -0.07388871253824947  \\
            50.0  -0.102355038834288  \\
            51.0  -0.13003492774196324  \\
            52.0  -0.1568597649340207  \\
            53.0  -0.18275309419693803  \\
            54.0  -0.20763327116265792  \\
            55.0  -0.23141122597493688  \\
            56.0  -0.2539936203149264  \\
            57.0  -0.2752860399070448  \\
            58.0  -0.29519578124546264  \\
            59.0  -0.3136160252575994  \\
            60.0  -0.330426247570315  \\
            61.0  -0.34551392443985474  \\
            62.0  -0.35877925445148035  \\
            63.0  -0.3701353129036664  \\
            64.0  -0.3794348375646544  \\
            65.0  -0.38650560045414173  \\
            66.0  -0.3912095788900051  \\
            67.0  -0.3933703051109696  \\
            68.0  -0.39284986650658543  \\
            69.0  -0.3896246966970606  \\
            70.0  -0.38368590070826425  \\
            71.0  -0.37507420391896096  \\
            72.0  -0.3641478663995805  \\
            73.0  -0.351578300234441  \\
            74.0  -0.3382054358314164  \\
            75.0  -0.32498709451773705  \\
            76.0  -0.3127383982822008  \\
            77.0  -0.30203696300147453  \\
            78.0  -0.29317908429058465  \\
            79.0  -0.2861995810816395  \\
            80.0  -0.2810805416183223  \\
            81.0  -0.27777685741582586  \\
            82.0  -0.27621551303521635  \\
            83.0  -0.27630556132332673  \\
            84.0  -0.27794294268188247  \\
            85.0  -0.28099766259323417  \\
            86.0  -0.2853316534570666  \\
            87.0  -0.2908183201000722  \\
            88.0  -0.29733482489042695  \\
            89.0  -0.30475930960813274  \\
            90.0  -0.3129744397086793  \\
            91.0  -0.321882022495301  \\
            92.0  -0.33139307179689903  \\
            93.0  -0.34141783342333565  \\
            94.0  -0.3518754453655576  \\
            95.0  -0.36268604478433475  \\
            96.0  -0.37376712402263357  \\
            97.0  -0.38503942361659266  \\
            98.0  -0.3964286994427101  \\
            99.0  -0.40787077173874603  \\
            100.0  -0.41931254624732667  \\
            101.0  -0.4307024662433299  \\
            102.0  -0.44198495427192486  \\
            103.0  -0.45310707955558455  \\
            104.0  -0.4640201829193735  \\
            105.0  -0.4746785645225312  \\
            106.0  -0.48504115686324906  \\
            107.0  -0.4950676481182369  \\
            108.0  -0.5047208333987061  \\
            109.0  -0.5139679116554192  \\
            110.0  -0.522778072622607  \\
            111.0  -0.531124930658389  \\
            112.0  -0.5389833020780798  \\
            113.0  -0.5463282869424065  \\
            114.0  -0.5531389242335379  \\
            115.0  -0.5593989803832607  \\
            116.0  -0.5650953572549223  \\
            117.0  -0.5702170484901375  \\
            118.0  -0.5747543121643969  \\
            119.0  -0.578698796344267  \\
            120.0  -0.5820442931128118  \\
            121.0  -0.5847869310379269  \\
            122.0  -0.5869254527034546  \\
            123.0  -0.588459660837291  \\
            124.0  -0.5893911687009826  \\
            125.0  -0.5897239991744033  \\
            126.0  -0.5894637728404195  \\
            127.0  -0.5886174958627453  \\
            128.0  -0.5871938258108506  \\
            129.0  -0.5852053225787665  \\
            130.0  -0.5826664594702683  \\
            131.0  -0.5795913187951244  \\
            132.0  -0.5759968600766823  \\
            133.0  -0.5719024055110707  \\
            134.0  -0.5673309144038609  \\
            135.0  -0.5623073316174307  \\
            136.0  -0.5568583711330003  \\
            137.0  -0.5510141711625326  \\
            138.0  -0.5448076620984703  \\
            139.0  -0.5382733422918684  \\
            140.0  -0.5314482034215263  \\
            141.0  -0.5243712040313214  \\
            142.0  -0.5170855360429609  \\
            143.0  -0.509638459782994  \\
            144.0  -0.5020815245658733  \\
            145.0  -0.4944681784001883  \\
            146.0  -0.4868530702276602  \\
            147.0  -0.4792960860894861  \\
            148.0  -0.47185801812751954  \\
            149.0  -0.4646026153278246  \\
            150.0  -0.45760094379784705  \\
            151.0  -0.4509274275480017  \\
            152.0  -0.44465944479219394  \\
            153.0  -0.43887862996749905  \\
            154.0  -0.43367229287421766  \\
            155.0  -0.4291341659234812  \\
            156.0  -0.42536202246626265  \\
            157.0  -0.4224580567364321  \\
            158.0  -0.42052756155762877  \\
            159.0  -0.4196852411934716  \\
            160.0  -0.4200511757642021  \\
            161.0  -0.42173823289945966  \\
            162.0  -0.4248220646686962  \\
            163.0  -0.4293532073721908  \\
            164.0  -0.4353540779023874  \\
            165.0  -0.44273144451895413  \\
            166.0  -0.45127767217092357  \\
            167.0  -0.46065549510715725  \\
            168.0  -0.47037434144631285  \\
            169.0  -0.47981026472807875  \\
            170.0  -0.4882990971011925  \\
            171.0  -0.4952722273568802  \\
            172.0  -0.5003109338296524  \\
            173.0  -0.5031530179369117  \\
            174.0  -0.5036710964735998  \\
            175.0  -0.5018133949587681  \\
            176.0  -0.49758795783854926  \\
            177.0  -0.49104997948962065  \\
            178.0  -0.4822805649024758  \\
            179.0  -0.47136735620300835  \\
            180.0  -0.45840303891462253  \\
            181.0  -0.44348922862150125  \\
            182.0  -0.42673318434826846  \\
            183.0  -0.4082441709407492  \\
            184.0  -0.388126354036324  \\
            185.0  -0.3664600652045722  \\
            186.0  -0.3433207254377717  \\
            187.0  -0.318793354367488  \\
            188.0  -0.2929597919529978  \\
            189.0  -0.26589911383459697  \\
            190.0  -0.2376871360785799  \\
            191.0  -0.20839411057934856  \\
            192.0  -0.17809357162293984  \\
            193.0  -0.14684716104704448  \\
            194.0  -0.11467884644416497  \\
            195.0  -0.08160441260411126  \\
            196.0  -0.04764066158510331  \\
            197.0  -0.012799601374379101  \\
            198.0  0.02291320853793495  \\
            199.0  0.05949083934322415  \\
            200.0  0.0969386738614862  \\
            201.0  0.13528836439546876  \\
            202.0  0.17457607728578872  \\
            203.0  0.21483511612775055  \\
            204.0  0.25606969231690624  \\
            205.0  0.2982555150727469  \\
            206.0  0.34134136264004816  \\
            207.0  0.38525108346562675  \\
            208.0  0.4298150753663038  \\
            209.0  0.47474613787408004  \\
            210.0  0.519662179295802  \\
            211.0  0.5641349775847555  \\
            212.0  0.6077480540018608  \\
            213.0  0.6501353778470986  \\
            214.0  0.6910160570979739  \\
            215.0  0.7302194851356303  \\
            216.0  0.7676610902649481  \\
            217.0  0.803313715734488  \\
            218.0  0.837192146825965  \\
            219.0  0.8693366276072692  \\
            220.0  0.8998017250650836  \\
            221.0  0.9286460761418802  \\
            222.0  0.9559257183272065  \\
            223.0  0.9816923817060574  \\
            224.0  1.00599301877092  \\
            225.0  1.0288687000700112  \\
            226.0  1.0503569874676457  \\
            227.0  1.0704913356305026  \\
            228.0  1.0892984942794903  \\
            229.0  1.1068030351049116  \\
            230.0  1.1230259311710469  \\
            231.0  1.1379812828985796  \\
            232.0  1.1516822862576344  \\
            233.0  1.164137791158527  \\
            234.0  1.1753526394455585  \\
            235.0  1.1853346728307481  \\
            236.0  1.1940911003633636  \\
            237.0  1.2016274664821907  \\
            238.0  1.2079441128466988  \\
            239.0  1.2130372452422722  \\
            240.0  1.216904397911528  \\
            241.0  1.2195432793029775  \\
            242.0  1.2209483179217793  \\
            243.0  1.221111894754575  \\
            244.0  1.2200243847274628  \\
            245.0  1.2176722573206078  \\
            246.0  1.2140423677578511  \\
            247.0  1.2091189583574147  \\
            248.0  1.2028842804702486  \\
            249.0  1.1953181112993307  \\
            250.0  1.1863949031211196  \\
        }
        ;
    \addlegendentry {$q(\theta_1)$}
    \addplot+[line width={0}, draw opacity={0}, fill={rgb,1:red,0.4627;green,0.7176;blue,0.698}, fill opacity={0.5}, mark={none}, forget plot]
        coordinates {
            (1,0.288801629884363)
            (2,0.2851868540470555)
            (3,0.28396589520588095)
            (4,0.28451980187161185)
            (5,0.2868306890198159)
            (6,0.29071582680848923)
            (7,0.2962061339743432)
            (8,0.30304612531791086)
            (9,0.3116215380783149)
            (10,0.3218661372601797)
            (11,0.3338979185912772)
            (12,0.34787422953730607)
            (13,0.3637317673398286)
            (14,0.3811827065077648)
            (15,0.4008143063328079)
            (16,0.4229327308874569)
            (17,0.4478865991997015)
            (18,0.4752091858710409)
            (19,0.5045176087819724)
            (20,0.5358703805747638)
            (21,0.5688504542399024)
            (22,0.6034384891911477)
            (23,0.639242392630285)
            (24,0.6761402913626464)
            (25,0.7135228567223307)
            (26,0.7512495444472292)
            (27,0.7893709443287692)
            (28,0.8271288469947391)
            (29,0.8637299270694238)
            (30,0.8986949568763597)
            (31,0.9314859467438529)
            (32,0.9614966540729184)
            (33,0.9890558107806864)
            (34,1.0140727252449036)
            (35,1.0362745746119513)
            (36,1.055517633530571)
            (37,1.0717364610694684)
            (38,1.0847806203285804)
            (39,1.0948007962153627)
            (40,1.1019834750308155)
            (41,1.1058272752257743)
            (42,1.1065161407704893)
            (43,1.1042879320437995)
            (44,1.0988926127313625)
            (45,1.0902522562327632)
            (46,1.0788638176788026)
            (47,1.0646787520607977)
            (48,1.04764196760894)
            (49,1.0278574760878472)
            (50,1.0053496413965886)
            (51,0.9799584764432836)
            (52,0.9518680818390618)
            (53,0.9210541840563797)
            (54,0.8875767692318762)
            (55,0.8515393049323308)
            (56,0.8128637281896582)
            (57,0.7715103451910373)
            (58,0.7274060621577746)
            (59,0.6808496976695958)
            (60,0.6317081944740004)
            (61,0.5797838005673409)
            (62,0.5249234711388415)
            (63,0.4670730927185575)
            (64,0.4067129535032382)
            (65,0.34317959395944897)
            (66,0.2764893341150099)
            (67,0.20646486921090157)
            (68,0.13261607540445772)
            (69,0.05480570235989872)
            (70,-0.026649176341472857)
            (71,-0.11178866344959444)
            (72,-0.20030895322104372)
            (73,-0.29112798438464627)
            (74,-0.38253476734271136)
            (75,-0.4729387606212669)
            (76,-0.5612168289778652)
            (77,-0.6462236367308318)
            (78,-0.7279203886480669)
            (79,-0.8064953367283141)
            (80,-0.8816048758308739)
            (81,-0.9534121657223039)
            (82,-1.0217260452871246)
            (83,-1.0867612333836743)
            (84,-1.1484821436865127)
            (85,-1.207736439978349)
            (86,-1.2645220692083448)
            (87,-1.3189567408507936)
            (88,-1.3712541571753778)
            (89,-1.4215918009094324)
            (90,-1.4700154947595718)
            (91,-1.5162214045783753)
            (92,-1.5604799694733638)
            (93,-1.6028476387026955)
            (94,-1.6431979653752047)
            (95,-1.6818690129457101)
            (96,-1.7189278429410741)
            (97,-1.754399516218416)
            (98,-1.7882700625682972)
            (99,-1.8203599609845962)
            (100,-1.8505077918439667)
            (101,-1.878916545194505)
            (102,-1.905922113740772)
            (103,-1.931343966214438)
            (104,-1.9552980111818505)
            (105,-1.9777483092445272)
            (106,-1.9985988339005156)
            (107,-2.018241825986563)
            (108,-2.0363780558985436)
            (109,-2.053160714182053)
            (110,-2.0685746778047953)
            (111,-2.0824421527766774)
            (112,-2.0951882342656662)
            (113,-2.10674424677744)
            (114,-2.117020313799527)
            (115,-2.1258803041489673)
            (116,-2.1333688222030083)
            (117,-2.1393670799987774)
            (118,-2.1441913119578353)
            (119,-2.147663015462556)
            (120,-2.149995032018119)
            (121,-2.150851889424392)
            (122,-2.150452122752704)
            (123,-2.1488514818886975)
            (124,-2.1460511618073426)
            (125,-2.1419073520424634)
            (126,-2.136694044423)
            (127,-2.130271861146461)
            (128,-2.122867071839735)
            (129,-2.1137327722654518)
            (130,-2.103716536254889)
            (131,-2.092462206687125)
            (132,-2.0801908965150875)
            (133,-2.06658073818662)
            (134,-2.0515978199783684)
            (135,-2.0353213047166654)
            (136,-2.0177738031147836)
            (137,-1.9986522842193286)
            (138,-1.9781752689984382)
            (139,-1.9563000284033274)
            (140,-1.933060993392011)
            (141,-1.9085342831964136)
            (142,-1.8823044155467477)
            (143,-1.8545324830533618)
            (144,-1.8248670975508021)
            (145,-1.793802827683336)
            (146,-1.7611004342672376)
            (147,-1.726680348028616)
            (148,-1.6909095897163997)
            (149,-1.6534290992372211)
            (150,-1.6140398259902506)
            (151,-1.5729509776120287)
            (152,-1.5300385865899113)
            (153,-1.4853533939202275)
            (154,-1.438611233850019)
            (155,-1.389804814036423)
            (156,-1.3389360318287387)
            (157,-1.2858813767528254)
            (158,-1.230589874304256)
            (159,-1.1725575360256923)
            (160,-1.1118583157288937)
            (161,-1.0483210424109042)
            (162,-0.9821921424063321)
            (163,-0.9128753163637935)
            (164,-0.8404067620319411)
            (165,-0.7651520870308822)
            (166,-0.6871341373066174)
            (167,-0.6070333633019992)
            (168,-0.5254127597654266)
            (169,-0.4433480654470204)
            (170,-0.36198378792589553)
            (171,-0.2824043020387443)
            (172,-0.20510021561724887)
            (173,-0.13070078245386682)
            (174,-0.059296908780217726)
            (175,0.009194413688979554)
            (176,0.07468767501231674)
            (177,0.13714080983693355)
            (178,0.1966463867348033)
            (179,0.25350282900503524)
            (180,0.3077025294479853)
            (181,0.3590586526968253)
            (182,0.40750414524681144)
            (183,0.453109235212781)
            (184,0.4959743641260574)
            (185,0.5366390435837181)
            (186,0.5745555841430044)
            (187,0.6098366902942566)
            (188,0.642201428074195)
            (189,0.6716383066276159)
            (190,0.6981127616110719)
            (191,0.7215925524592061)
            (192,0.7418991804956352)
            (193,0.759156674195354)
            (194,0.7735380034607499)
            (195,0.7847834772383896)
            (196,0.7930533320633262)
            (197,0.7979258241729669)
            (198,0.799493281901929)
            (199,0.7973510527192232)
            (200,0.7917005803919287)
            (201,0.7826007548864886)
            (202,0.7697552790149347)
            (203,0.7535247006367878)
            (204,0.7339945888985691)
            (205,0.7110580117035398)
            (206,0.6847817081234928)
            (207,0.6556637356791192)
            (208,0.6238657616268)
            (209,0.5901981651906343)
            (210,0.5550813873800912)
            (211,0.5199129735491435)
            (212,0.48534913454369155)
            (213,0.4523764679562938)
            (214,0.4215272870291472)
            (215,0.3930397431358262)
            (216,0.36713886705335236)
            (217,0.3440593921953835)
            (218,0.3238071519625035)
            (219,0.3064862825374745)
            (220,0.2915811941248822)
            (221,0.27883687204571334)
            (222,0.2685099770593875)
            (223,0.260303474587578)
            (224,0.2538471224059268)
            (225,0.24922880434856445)
            (226,0.24569885906039715)
            (227,0.24354485267252474)
            (228,0.2427856404927769)
            (229,0.2429676068196037)
            (230,0.24434762870960727)
            (231,0.24698684415121916)
            (232,0.25046577774706963)
            (233,0.255262020763648)
            (234,0.2609599671754724)
            (235,0.26731930690030253)
            (236,0.2743865973451987)
            (237,0.2820851327290235)
            (238,0.2907064887412431)
            (239,0.300108590801535)
            (240,0.31017314315918076)
            (241,0.3208457738691458)
            (242,0.3322968364421827)
            (243,0.344377999047322)
            (244,0.3572732382983439)
            (245,0.37103533828774593)
            (246,0.38538973478216637)
            (247,0.40066757109155904)
            (248,0.41650760464972764)
            (249,0.4331700944638147)
            (250,0.4509564286863325)
            (250,0.013907966025970497)
            (249,0.010641350789823867)
            (248,0.008443204129649995)
            (247,0.006973223670602102)
            (246,0.005955663166347824)
            (245,0.005730878388000238)
            (244,0.005943727001748833)
            (243,0.006842642056104908)
            (242,0.008322372049461069)
            (241,0.010159283328160906)
            (240,0.01245925230242062)
            (239,0.015005204650694659)
            (238,0.017807957307326483)
            (237,0.020922996264876537)
            (236,0.02440549382486787)
            (235,0.027880035682385396)
            (234,0.03134431216341266)
            (233,0.03465995262655547)
            (232,0.0379555096738686)
            (231,0.04150531289655976)
            (230,0.04469462318662784)
            (229,0.047815203340443785)
            (228,0.05068724405551378)
            (227,0.052961136813009624)
            (226,0.05502370515575192)
            (225,0.05682488945005623)
            (224,0.05808279589846527)
            (223,0.05959237025771344)
            (222,0.06135537858406037)
            (221,0.06386959864725195)
            (220,0.06759919396068284)
            (219,0.07248395384251605)
            (218,0.0790013205240715)
            (217,0.08795039446800651)
            (216,0.09958785092449396)
            (215,0.11433998129094403)
            (214,0.13242215600701135)
            (213,0.15400620450281094)
            (212,0.1792936287233245)
            (211,0.2082778017374114)
            (210,0.2402796870749906)
            (209,0.27471127233258325)
            (208,0.3101653250947679)
            (207,0.3461360375702443)
            (206,0.3815339976605926)
            (205,0.415593686106697)
            (204,0.4472168937791662)
            (203,0.47605643231719885)
            (202,0.5017879184165521)
            (201,0.5239951300778904)
            (200,0.5422167657102571)
            (199,0.5567080871837446)
            (198,0.5673642830868145)
            (197,0.5739691428852696)
            (196,0.5769134211514202)
            (195,0.5760803549363995)
            (194,0.5718575809800949)
            (193,0.5640279317200472)
            (192,0.5528002765137565)
            (191,0.5379389648204653)
            (190,0.5192643688242494)
            (189,0.4968975934424944)
            (188,0.4708020767874787)
            (187,0.4409397280165154)
            (186,0.407250351368528)
            (185,0.36992432096010847)
            (184,0.32873201411887093)
            (183,0.2840523985133252)
            (182,0.23513059197078828)
            (181,0.18172502014129444)
            (180,0.12367791165129813)
            (179,0.06101727103057364)
            (178,-0.006115213785566398)
            (177,-0.07775683592301924)
            (176,-0.15408954672249275)
            (175,-0.23481265747292615)
            (174,-0.3192306462414752)
            (173,-0.4063558816484212)
            (172,-0.4951254753889818)
            (171,-0.5842239922374568)
            (170,-0.6721560126014604)
            (169,-0.758044603402926)
            (168,-0.8408679713392521)
            (167,-0.919837428302147)
            (166,-0.9942869581416824)
            (165,-1.064338959369256)
            (164,-1.1301696178593303)
            (163,-1.1926111274126427)
            (162,-1.251995961255905)
            (161,-1.3086016424731914)
            (160,-1.3632298652861263)
            (159,-1.4157402777902606)
            (158,-1.46630295100786)
            (157,-1.5148202941313413)
            (156,-1.5617511883304458)
            (155,-1.6070857185641065)
            (154,-1.6508879860600683)
            (153,-1.6931035208110843)
            (152,-1.7336982439016602)
            (151,-1.7729131063544346)
            (150,-1.8106543849439962)
            (149,-1.8470094432652215)
            (148,-1.881741971788306)
            (147,-1.9150303700249696)
            (146,-1.947201219798198)
            (145,-1.9778594055730547)
            (144,-2.0070696537124344)
            (143,-2.035051612231222)
            (142,-2.06129874220963)
            (141,-2.0861578954425486)
            (140,-2.1094594908273567)
            (139,-2.1316085521391734)
            (138,-2.1525130577969995)
            (137,-2.1721270158834622)
            (136,-2.190483489087748)
            (135,-2.20735814069922)
            (134,-2.223049291765497)
            (133,-2.2375259361086632)
            (132,-2.2507030674601523)
            (131,-2.262610603910458)
            (130,-2.273563487951008)
            (129,-2.2833323677375166)
            (128,-2.2922659916255466)
            (127,-2.2995128722511726)
            (126,-2.3058208826828626)
            (125,-2.3109582016137282)
            (124,-2.3150609399227866)
            (123,-2.3178526503808974)
            (122,-2.319476649427301)
            (121,-2.3199304388031567)
            (120,-2.3191575349592273)
            (119,-2.316939735637153)
            (118,-2.3136129468909976)
            (117,-2.3089641887439165)
            (116,-2.3031724272377083)
            (115,-2.295922015652484)
            (114,-2.2873320744998966)
            (113,-2.2773583656889804)
            (112,-2.2661377019480917)
            (111,-2.2537625018998946)
            (110,-2.2403065638005026)
            (109,-2.2253496643157447)
            (108,-2.209071673540053)
            (107,-2.191493376581623)
            (106,-2.1724674448967796)
            (105,-2.1523012418355116)
            (104,-2.1306142277766718)
            (103,-2.1075083080868566)
            (102,-2.0830301694671522)
            (101,-2.0570760907457055)
            (100,-2.02984154101451)
            (99,-2.0010078360420462)
            (98,-1.9703737215268802)
            (97,-1.9380985470810217)
            (96,-1.9043745689504243)
            (95,-1.869243364913901)
            (94,-1.8327144931655424)
            (93,-1.7947597925037804)
            (92,-1.7550690931299124)
            (91,-1.713787323310961)
            (90,-1.6708922744632546)
            (89,-1.6261361953477507)
            (88,-1.57985697623396)
            (87,-1.532095144254038)
            (86,-1.482770820854131)
            (85,-1.4317612276540812)
            (84,-1.3790383946267557)
            (83,-1.3247333717686323)
            (82,-1.2680307030361337)
            (81,-1.2088191257752434)
            (80,-1.146639240302377)
            (79,-1.0814009157111402)
            (78,-1.0126312082889357)
            (77,-0.9404323292566081)
            (76,-0.8643945118656315)
            (75,-0.7837689987217187)
            (74,-0.6986784649120859)
            (73,-0.6093598668650799)
            (72,-0.5169737402543347)
            (71,-0.4232596135696165)
            (70,-0.32935520394310713)
            (69,-0.23641305001469254)
            (68,-0.14599679589429745)
            (67,-0.059479013442013784)
            (66,0.022762458999721402)
            (65,0.10070504054784488)
            (64,0.17421281736891392)
            (63,0.24321150823425214)
            (62,0.3084587322781979)
            (61,0.36957910903078345)
            (60,0.4266856621606711)
            (59,0.4800053197760201)
            (58,0.5298342891052352)
            (57,0.5764235596758984)
            (56,0.6195599017113675)
            (55,0.6593656934340785)
            (54,0.6959347288023499)
            (53,0.729403437700433)
            (52,0.7597224106202669)
            (51,0.7868718624518185)
            (50,0.8109113570362582)
            (49,0.8316866993326799)
            (48,0.8493832637665819)
            (47,0.8640020078605645)
            (46,0.8754589576547243)
            (45,0.8838235578799484)
            (44,0.8891484140730888)
            (43,0.890950649511496)
            (42,0.8893465373003415)
            (41,0.8846088589578006)
            (40,0.8765204170174365)
            (39,0.8649730031025638)
            (38,0.85052035827252)
            (37,0.8329765187827547)
            (36,0.8122508415254603)
            (35,0.7885808395567064)
            (34,0.762112060685576)
            (33,0.7330778987527837)
            (32,0.7018052327125585)
            (31,0.6683739068415901)
            (30,0.6325310191430344)
            (29,0.5950207168857871)
            (28,0.5564547157149893)
            (27,0.5173102196511917)
            (26,0.4783173684868078)
            (25,0.4402432387169475)
            (24,0.40310164115042857)
            (23,0.3669915504880317)
            (22,0.33245844744529374)
            (21,0.2996226838589353)
            (20,0.2688862236005705)
            (19,0.24023534970201182)
            (18,0.21402840131078182)
            (17,0.19013470777460256)
            (16,0.1687847762474206)
            (15,0.1502917743315476)
            (14,0.13407230050664848)
            (13,0.11947604422144736)
            (12,0.1056419633237572)
            (11,0.09279498072414571)
            (10,0.08098170658356652)
            (9,0.06991789209201549)
            (8,0.05936149174873542)
            (7,0.04928324880118934)
            (6,0.03927107285537296)
            (5,0.029597642707792304)
            (4,0.02029175317607934)
            (3,0.011646706172449728)
            (2,0.0037934788771329386)
            (1,-0.0022921073506344536)
            (1,0.288801629884363)
        }
        ;
    \addplot+[line width={0}, draw opacity={0}, fill={rgb,1:red,0.4627;green,0.7176;blue,0.698}, fill opacity={0.5}, mark={none}, forget plot]
        coordinates {
            (1,0.288801629884363)
            (2,0.2851868540470555)
            (3,0.28396589520588095)
            (4,0.28451980187161185)
            (5,0.2868306890198159)
            (6,0.29071582680848923)
            (7,0.2962061339743432)
            (8,0.30304612531791086)
            (9,0.3116215380783149)
            (10,0.3218661372601797)
            (11,0.3338979185912772)
            (12,0.34787422953730607)
            (13,0.3637317673398286)
            (14,0.3811827065077648)
            (15,0.4008143063328079)
            (16,0.4229327308874569)
            (17,0.4478865991997015)
            (18,0.4752091858710409)
            (19,0.5045176087819724)
            (20,0.5358703805747638)
            (21,0.5688504542399024)
            (22,0.6034384891911477)
            (23,0.639242392630285)
            (24,0.6761402913626464)
            (25,0.7135228567223307)
            (26,0.7512495444472292)
            (27,0.7893709443287692)
            (28,0.8271288469947391)
            (29,0.8637299270694238)
            (30,0.8986949568763597)
            (31,0.9314859467438529)
            (32,0.9614966540729184)
            (33,0.9890558107806864)
            (34,1.0140727252449036)
            (35,1.0362745746119513)
            (36,1.055517633530571)
            (37,1.0717364610694684)
            (38,1.0847806203285804)
            (39,1.0948007962153627)
            (40,1.1019834750308155)
            (41,1.1058272752257743)
            (42,1.1065161407704893)
            (43,1.1042879320437995)
            (44,1.0988926127313625)
            (45,1.0902522562327632)
            (46,1.0788638176788026)
            (47,1.0646787520607977)
            (48,1.04764196760894)
            (49,1.0278574760878472)
            (50,1.0053496413965886)
            (51,0.9799584764432836)
            (52,0.9518680818390618)
            (53,0.9210541840563797)
            (54,0.8875767692318762)
            (55,0.8515393049323308)
            (56,0.8128637281896582)
            (57,0.7715103451910373)
            (58,0.7274060621577746)
            (59,0.6808496976695958)
            (60,0.6317081944740004)
            (61,0.5797838005673409)
            (62,0.5249234711388415)
            (63,0.4670730927185575)
            (64,0.4067129535032382)
            (65,0.34317959395944897)
            (66,0.2764893341150099)
            (67,0.20646486921090157)
            (68,0.13261607540445772)
            (69,0.05480570235989872)
            (70,-0.026649176341472857)
            (71,-0.11178866344959444)
            (72,-0.20030895322104372)
            (73,-0.29112798438464627)
            (74,-0.38253476734271136)
            (75,-0.4729387606212669)
            (76,-0.5612168289778652)
            (77,-0.6462236367308318)
            (78,-0.7279203886480669)
            (79,-0.8064953367283141)
            (80,-0.8816048758308739)
            (81,-0.9534121657223039)
            (82,-1.0217260452871246)
            (83,-1.0867612333836743)
            (84,-1.1484821436865127)
            (85,-1.207736439978349)
            (86,-1.2645220692083448)
            (87,-1.3189567408507936)
            (88,-1.3712541571753778)
            (89,-1.4215918009094324)
            (90,-1.4700154947595718)
            (91,-1.5162214045783753)
            (92,-1.5604799694733638)
            (93,-1.6028476387026955)
            (94,-1.6431979653752047)
            (95,-1.6818690129457101)
            (96,-1.7189278429410741)
            (97,-1.754399516218416)
            (98,-1.7882700625682972)
            (99,-1.8203599609845962)
            (100,-1.8505077918439667)
            (101,-1.878916545194505)
            (102,-1.905922113740772)
            (103,-1.931343966214438)
            (104,-1.9552980111818505)
            (105,-1.9777483092445272)
            (106,-1.9985988339005156)
            (107,-2.018241825986563)
            (108,-2.0363780558985436)
            (109,-2.053160714182053)
            (110,-2.0685746778047953)
            (111,-2.0824421527766774)
            (112,-2.0951882342656662)
            (113,-2.10674424677744)
            (114,-2.117020313799527)
            (115,-2.1258803041489673)
            (116,-2.1333688222030083)
            (117,-2.1393670799987774)
            (118,-2.1441913119578353)
            (119,-2.147663015462556)
            (120,-2.149995032018119)
            (121,-2.150851889424392)
            (122,-2.150452122752704)
            (123,-2.1488514818886975)
            (124,-2.1460511618073426)
            (125,-2.1419073520424634)
            (126,-2.136694044423)
            (127,-2.130271861146461)
            (128,-2.122867071839735)
            (129,-2.1137327722654518)
            (130,-2.103716536254889)
            (131,-2.092462206687125)
            (132,-2.0801908965150875)
            (133,-2.06658073818662)
            (134,-2.0515978199783684)
            (135,-2.0353213047166654)
            (136,-2.0177738031147836)
            (137,-1.9986522842193286)
            (138,-1.9781752689984382)
            (139,-1.9563000284033274)
            (140,-1.933060993392011)
            (141,-1.9085342831964136)
            (142,-1.8823044155467477)
            (143,-1.8545324830533618)
            (144,-1.8248670975508021)
            (145,-1.793802827683336)
            (146,-1.7611004342672376)
            (147,-1.726680348028616)
            (148,-1.6909095897163997)
            (149,-1.6534290992372211)
            (150,-1.6140398259902506)
            (151,-1.5729509776120287)
            (152,-1.5300385865899113)
            (153,-1.4853533939202275)
            (154,-1.438611233850019)
            (155,-1.389804814036423)
            (156,-1.3389360318287387)
            (157,-1.2858813767528254)
            (158,-1.230589874304256)
            (159,-1.1725575360256923)
            (160,-1.1118583157288937)
            (161,-1.0483210424109042)
            (162,-0.9821921424063321)
            (163,-0.9128753163637935)
            (164,-0.8404067620319411)
            (165,-0.7651520870308822)
            (166,-0.6871341373066174)
            (167,-0.6070333633019992)
            (168,-0.5254127597654266)
            (169,-0.4433480654470204)
            (170,-0.36198378792589553)
            (171,-0.2824043020387443)
            (172,-0.20510021561724887)
            (173,-0.13070078245386682)
            (174,-0.059296908780217726)
            (175,0.009194413688979554)
            (176,0.07468767501231674)
            (177,0.13714080983693355)
            (178,0.1966463867348033)
            (179,0.25350282900503524)
            (180,0.3077025294479853)
            (181,0.3590586526968253)
            (182,0.40750414524681144)
            (183,0.453109235212781)
            (184,0.4959743641260574)
            (185,0.5366390435837181)
            (186,0.5745555841430044)
            (187,0.6098366902942566)
            (188,0.642201428074195)
            (189,0.6716383066276159)
            (190,0.6981127616110719)
            (191,0.7215925524592061)
            (192,0.7418991804956352)
            (193,0.759156674195354)
            (194,0.7735380034607499)
            (195,0.7847834772383896)
            (196,0.7930533320633262)
            (197,0.7979258241729669)
            (198,0.799493281901929)
            (199,0.7973510527192232)
            (200,0.7917005803919287)
            (201,0.7826007548864886)
            (202,0.7697552790149347)
            (203,0.7535247006367878)
            (204,0.7339945888985691)
            (205,0.7110580117035398)
            (206,0.6847817081234928)
            (207,0.6556637356791192)
            (208,0.6238657616268)
            (209,0.5901981651906343)
            (210,0.5550813873800912)
            (211,0.5199129735491435)
            (212,0.48534913454369155)
            (213,0.4523764679562938)
            (214,0.4215272870291472)
            (215,0.3930397431358262)
            (216,0.36713886705335236)
            (217,0.3440593921953835)
            (218,0.3238071519625035)
            (219,0.3064862825374745)
            (220,0.2915811941248822)
            (221,0.27883687204571334)
            (222,0.2685099770593875)
            (223,0.260303474587578)
            (224,0.2538471224059268)
            (225,0.24922880434856445)
            (226,0.24569885906039715)
            (227,0.24354485267252474)
            (228,0.2427856404927769)
            (229,0.2429676068196037)
            (230,0.24434762870960727)
            (231,0.24698684415121916)
            (232,0.25046577774706963)
            (233,0.255262020763648)
            (234,0.2609599671754724)
            (235,0.26731930690030253)
            (236,0.2743865973451987)
            (237,0.2820851327290235)
            (238,0.2907064887412431)
            (239,0.300108590801535)
            (240,0.31017314315918076)
            (241,0.3208457738691458)
            (242,0.3322968364421827)
            (243,0.344377999047322)
            (244,0.3572732382983439)
            (245,0.37103533828774593)
            (246,0.38538973478216637)
            (247,0.40066757109155904)
            (248,0.41650760464972764)
            (249,0.4331700944638147)
            (250,0.4509564286863325)
            (250,0.8880048913466945)
            (249,0.8556988381378056)
            (248,0.8245720051698053)
            (247,0.7943619185125159)
            (246,0.7648238063979849)
            (245,0.7363397981874916)
            (244,0.708602749594939)
            (243,0.6819133560385391)
            (242,0.6562713008349044)
            (241,0.6315322644101307)
            (240,0.6078870340159409)
            (239,0.5852119769523754)
            (238,0.5636050201751597)
            (237,0.5432472691931705)
            (236,0.5243677008655295)
            (235,0.5067585781182197)
            (234,0.4905756221875321)
            (233,0.47586408890074056)
            (232,0.46297604582027063)
            (231,0.4524683754058786)
            (230,0.44400063423258673)
            (229,0.4381200102987636)
            (228,0.43488403693004)
            (227,0.43412856853203985)
            (226,0.4363740129650424)
            (225,0.4416327192470727)
            (224,0.4496114489133883)
            (223,0.4610145789174426)
            (222,0.47566457553471464)
            (221,0.4938041454441747)
            (220,0.5155631942890816)
            (219,0.540488611232433)
            (218,0.5686129834009355)
            (217,0.6001683899227606)
            (216,0.6346898831822108)
            (215,0.6717395049807083)
            (214,0.7106324180512831)
            (213,0.7507467314097767)
            (212,0.7914046403640587)
            (211,0.8315481453608755)
            (210,0.8698830876851917)
            (209,0.9056850580486853)
            (208,0.937566198158832)
            (207,0.9651914337879941)
            (206,0.988029418586393)
            (205,1.0065223373003827)
            (204,1.0207722840179718)
            (203,1.0309929689563768)
            (202,1.0377226396133172)
            (201,1.0412063796950868)
            (200,1.0411843950736004)
            (199,1.0379940182547018)
            (198,1.0316222807170436)
            (197,1.0218825054606642)
            (196,1.0091932429752322)
            (195,0.9934865995403797)
            (194,0.9752184259414048)
            (193,0.9542854166706609)
            (192,0.9309980844775139)
            (191,0.9052461400979468)
            (190,0.8769611543978945)
            (189,0.8463790198127374)
            (188,0.8136007793609112)
            (187,0.7787336525719978)
            (186,0.741860816917481)
            (185,0.7033537662073277)
            (184,0.6632167141332439)
            (183,0.6221660719122368)
            (182,0.5798776985228347)
            (181,0.5363922852523562)
            (180,0.4917271472446725)
            (179,0.44598838697949683)
            (178,0.399407987255173)
            (177,0.35203845559688635)
            (176,0.30346489674712623)
            (175,0.2532014848508853)
            (174,0.20063682868103974)
            (173,0.1449543167406875)
            (172,0.08492504415448401)
            (171,0.019415388159968183)
            (170,-0.05181156325033065)
            (169,-0.1286515274911147)
            (168,-0.20995754819160106)
            (167,-0.2942292983018514)
            (166,-0.3799813164715525)
            (165,-0.46596521469250846)
            (164,-0.550643906204552)
            (163,-0.6331395053149442)
            (162,-0.7123883235567592)
            (161,-0.7880404423486169)
            (160,-0.8604867661716612)
            (159,-0.9293747942611241)
            (158,-0.9948767976006522)
            (157,-1.0569424593743095)
            (156,-1.1161208753270315)
            (155,-1.1725239095087394)
            (154,-1.2263344816399697)
            (153,-1.2776032670293707)
            (152,-1.3263789292781625)
            (151,-1.372988848869623)
            (150,-1.417425267036505)
            (149,-1.4598487552092207)
            (148,-1.5000772076444935)
            (147,-1.5383303260322623)
            (146,-1.5749996487362772)
            (145,-1.6097462497936172)
            (144,-1.6426645413891696)
            (143,-1.674013353875502)
            (142,-1.7033100888838655)
            (141,-1.7309106709502784)
            (140,-1.7566624959566655)
            (139,-1.7809915046674814)
            (138,-1.8038374801998769)
            (137,-1.8251775525551952)
            (136,-1.8450641171418192)
            (135,-1.8632844687341106)
            (134,-1.8801463481912395)
            (133,-1.8956355402645766)
            (132,-1.9096787255700227)
            (131,-1.9223138094637922)
            (130,-1.9338695845587701)
            (129,-1.944133176793387)
            (128,-1.9534681520539232)
            (127,-1.9610308500417493)
            (126,-1.967567206163137)
            (125,-1.9728565024711984)
            (124,-1.9770413836918985)
            (123,-1.9798503133964978)
            (122,-1.9814275960781071)
            (121,-1.9817733400456279)
            (120,-1.9808325290770108)
            (119,-1.9783862952879585)
            (118,-1.974769677024673)
            (117,-1.9697699712536383)
            (116,-1.9635652171683082)
            (115,-1.9558385926454502)
            (114,-1.946708553099157)
            (113,-1.9361301278659002)
            (112,-1.924238766583241)
            (111,-1.9111218036534605)
            (110,-1.8968427918090878)
            (109,-1.8809717640483616)
            (108,-1.8636844382570346)
            (107,-1.8449902753915026)
            (106,-1.8247302229042517)
            (105,-1.803195376653543)
            (104,-1.7799817945870293)
            (103,-1.7551796243420197)
            (102,-1.7288140580143918)
            (101,-1.7007569996433045)
            (100,-1.6711740426734236)
            (99,-1.6397120859271461)
            (98,-1.6061664036097143)
            (97,-1.5707004853558104)
            (96,-1.533481116931724)
            (95,-1.4944946609775192)
            (94,-1.453681437584867)
            (93,-1.4109354849016107)
            (92,-1.365890845816815)
            (91,-1.3186554858457895)
            (90,-1.269138715055889)
            (89,-1.217047406471114)
            (88,-1.1626513381167956)
            (87,-1.1058183374475492)
            (86,-1.0462733175625585)
            (85,-0.9837116523026168)
            (84,-0.9179258927462696)
            (83,-0.8487890949987164)
            (82,-0.7754213875381155)
            (81,-0.6980052056693644)
            (80,-0.6165705113593706)
            (79,-0.5315897577454879)
            (78,-0.44320956900719805)
            (77,-0.35201494420505547)
            (76,-0.2580391460900988)
            (75,-0.1621085225208151)
            (74,-0.06639106977333681)
            (73,0.02710389809578745)
            (72,0.11635583381224732)
            (71,0.19968228667042764)
            (70,0.27605685126016144)
            (69,0.34602445473449)
            (68,0.41122894670321286)
            (67,0.4724087518638169)
            (66,0.5302162092302984)
            (65,0.585654147371053)
            (64,0.6392130896375625)
            (63,0.6909346772028628)
            (62,0.7413882099994851)
            (61,0.7899884921038984)
            (60,0.8367307267873296)
            (59,0.8816940755631715)
            (58,0.924977835210314)
            (57,0.9665971307061763)
            (56,1.006167554667949)
            (55,1.0437129164305832)
            (54,1.0792188096614026)
            (53,1.1127049304123264)
            (52,1.1440137530578567)
            (51,1.1730450904347487)
            (50,1.199787925756919)
            (49,1.2240282528430144)
            (48,1.245900671451298)
            (47,1.265355496261031)
            (46,1.2822686777028809)
            (45,1.296680954585578)
            (44,1.3086368113896363)
            (43,1.317625214576103)
            (42,1.3236857442406371)
            (41,1.327045691493748)
            (40,1.3274465330441945)
            (39,1.3246285893281617)
            (38,1.3190408823846407)
            (37,1.3104964033561821)
            (36,1.2987844255356815)
            (35,1.283968309667196)
            (34,1.266033389804231)
            (33,1.245033722808589)
            (32,1.2211880754332785)
            (31,1.1945979866461156)
            (30,1.164858894609685)
            (29,1.1324391372530604)
            (28,1.097802978274489)
            (27,1.0614316690063466)
            (26,1.0241817204076507)
            (25,0.986802474727714)
            (24,0.9491789415748643)
            (23,0.9114932347725384)
            (22,0.8744185309370017)
            (21,0.8380782246208696)
            (20,0.8028545375489571)
            (19,0.768799867861933)
            (18,0.7363899704313)
            (17,0.7056384906248003)
            (16,0.6770806855274932)
            (15,0.6513368383340682)
            (14,0.6282931125088811)
            (13,0.6079874904582099)
            (12,0.5901064957508549)
            (11,0.5750008564584087)
            (10,0.5627505679367929)
            (9,0.5533251840646143)
            (8,0.5467307588870863)
            (7,0.5431290191474971)
            (6,0.5421605807616054)
            (5,0.5440637353318395)
            (4,0.5487478505671444)
            (3,0.5562850842393121)
            (2,0.5665802292169781)
            (1,0.5798953671193604)
            (1,0.288801629884363)
        }
        ;
    \addplot[color={rgb,1:red,0.4627;green,0.7176;blue,0.698}, name path={cf0ab353-1b04-4fae-a7a9-6df884ec8dc3}, legend image code/.code={{
    \draw[fill={rgb,1:red,0.4627;green,0.7176;blue,0.698}, fill opacity={0.5}] (0cm,-0.1cm) rectangle (0.6cm,0.1cm);
    }}, draw opacity={1.0}, line width={1}, solid]
        table[row sep={\\}]
        {
            \\
            1.0  0.288801629884363  \\
            2.0  0.2851868540470555  \\
            3.0  0.28396589520588095  \\
            4.0  0.28451980187161185  \\
            5.0  0.2868306890198159  \\
            6.0  0.29071582680848923  \\
            7.0  0.2962061339743432  \\
            8.0  0.30304612531791086  \\
            9.0  0.3116215380783149  \\
            10.0  0.3218661372601797  \\
            11.0  0.3338979185912772  \\
            12.0  0.34787422953730607  \\
            13.0  0.3637317673398286  \\
            14.0  0.3811827065077648  \\
            15.0  0.4008143063328079  \\
            16.0  0.4229327308874569  \\
            17.0  0.4478865991997015  \\
            18.0  0.4752091858710409  \\
            19.0  0.5045176087819724  \\
            20.0  0.5358703805747638  \\
            21.0  0.5688504542399024  \\
            22.0  0.6034384891911477  \\
            23.0  0.639242392630285  \\
            24.0  0.6761402913626464  \\
            25.0  0.7135228567223307  \\
            26.0  0.7512495444472292  \\
            27.0  0.7893709443287692  \\
            28.0  0.8271288469947391  \\
            29.0  0.8637299270694238  \\
            30.0  0.8986949568763597  \\
            31.0  0.9314859467438529  \\
            32.0  0.9614966540729184  \\
            33.0  0.9890558107806864  \\
            34.0  1.0140727252449036  \\
            35.0  1.0362745746119513  \\
            36.0  1.055517633530571  \\
            37.0  1.0717364610694684  \\
            38.0  1.0847806203285804  \\
            39.0  1.0948007962153627  \\
            40.0  1.1019834750308155  \\
            41.0  1.1058272752257743  \\
            42.0  1.1065161407704893  \\
            43.0  1.1042879320437995  \\
            44.0  1.0988926127313625  \\
            45.0  1.0902522562327632  \\
            46.0  1.0788638176788026  \\
            47.0  1.0646787520607977  \\
            48.0  1.04764196760894  \\
            49.0  1.0278574760878472  \\
            50.0  1.0053496413965886  \\
            51.0  0.9799584764432836  \\
            52.0  0.9518680818390618  \\
            53.0  0.9210541840563797  \\
            54.0  0.8875767692318762  \\
            55.0  0.8515393049323308  \\
            56.0  0.8128637281896582  \\
            57.0  0.7715103451910373  \\
            58.0  0.7274060621577746  \\
            59.0  0.6808496976695958  \\
            60.0  0.6317081944740004  \\
            61.0  0.5797838005673409  \\
            62.0  0.5249234711388415  \\
            63.0  0.4670730927185575  \\
            64.0  0.4067129535032382  \\
            65.0  0.34317959395944897  \\
            66.0  0.2764893341150099  \\
            67.0  0.20646486921090157  \\
            68.0  0.13261607540445772  \\
            69.0  0.05480570235989872  \\
            70.0  -0.026649176341472857  \\
            71.0  -0.11178866344959444  \\
            72.0  -0.20030895322104372  \\
            73.0  -0.29112798438464627  \\
            74.0  -0.38253476734271136  \\
            75.0  -0.4729387606212669  \\
            76.0  -0.5612168289778652  \\
            77.0  -0.6462236367308318  \\
            78.0  -0.7279203886480669  \\
            79.0  -0.8064953367283141  \\
            80.0  -0.8816048758308739  \\
            81.0  -0.9534121657223039  \\
            82.0  -1.0217260452871246  \\
            83.0  -1.0867612333836743  \\
            84.0  -1.1484821436865127  \\
            85.0  -1.207736439978349  \\
            86.0  -1.2645220692083448  \\
            87.0  -1.3189567408507936  \\
            88.0  -1.3712541571753778  \\
            89.0  -1.4215918009094324  \\
            90.0  -1.4700154947595718  \\
            91.0  -1.5162214045783753  \\
            92.0  -1.5604799694733638  \\
            93.0  -1.6028476387026955  \\
            94.0  -1.6431979653752047  \\
            95.0  -1.6818690129457101  \\
            96.0  -1.7189278429410741  \\
            97.0  -1.754399516218416  \\
            98.0  -1.7882700625682972  \\
            99.0  -1.8203599609845962  \\
            100.0  -1.8505077918439667  \\
            101.0  -1.878916545194505  \\
            102.0  -1.905922113740772  \\
            103.0  -1.931343966214438  \\
            104.0  -1.9552980111818505  \\
            105.0  -1.9777483092445272  \\
            106.0  -1.9985988339005156  \\
            107.0  -2.018241825986563  \\
            108.0  -2.0363780558985436  \\
            109.0  -2.053160714182053  \\
            110.0  -2.0685746778047953  \\
            111.0  -2.0824421527766774  \\
            112.0  -2.0951882342656662  \\
            113.0  -2.10674424677744  \\
            114.0  -2.117020313799527  \\
            115.0  -2.1258803041489673  \\
            116.0  -2.1333688222030083  \\
            117.0  -2.1393670799987774  \\
            118.0  -2.1441913119578353  \\
            119.0  -2.147663015462556  \\
            120.0  -2.149995032018119  \\
            121.0  -2.150851889424392  \\
            122.0  -2.150452122752704  \\
            123.0  -2.1488514818886975  \\
            124.0  -2.1460511618073426  \\
            125.0  -2.1419073520424634  \\
            126.0  -2.136694044423  \\
            127.0  -2.130271861146461  \\
            128.0  -2.122867071839735  \\
            129.0  -2.1137327722654518  \\
            130.0  -2.103716536254889  \\
            131.0  -2.092462206687125  \\
            132.0  -2.0801908965150875  \\
            133.0  -2.06658073818662  \\
            134.0  -2.0515978199783684  \\
            135.0  -2.0353213047166654  \\
            136.0  -2.0177738031147836  \\
            137.0  -1.9986522842193286  \\
            138.0  -1.9781752689984382  \\
            139.0  -1.9563000284033274  \\
            140.0  -1.933060993392011  \\
            141.0  -1.9085342831964136  \\
            142.0  -1.8823044155467477  \\
            143.0  -1.8545324830533618  \\
            144.0  -1.8248670975508021  \\
            145.0  -1.793802827683336  \\
            146.0  -1.7611004342672376  \\
            147.0  -1.726680348028616  \\
            148.0  -1.6909095897163997  \\
            149.0  -1.6534290992372211  \\
            150.0  -1.6140398259902506  \\
            151.0  -1.5729509776120287  \\
            152.0  -1.5300385865899113  \\
            153.0  -1.4853533939202275  \\
            154.0  -1.438611233850019  \\
            155.0  -1.389804814036423  \\
            156.0  -1.3389360318287387  \\
            157.0  -1.2858813767528254  \\
            158.0  -1.230589874304256  \\
            159.0  -1.1725575360256923  \\
            160.0  -1.1118583157288937  \\
            161.0  -1.0483210424109042  \\
            162.0  -0.9821921424063321  \\
            163.0  -0.9128753163637935  \\
            164.0  -0.8404067620319411  \\
            165.0  -0.7651520870308822  \\
            166.0  -0.6871341373066174  \\
            167.0  -0.6070333633019992  \\
            168.0  -0.5254127597654266  \\
            169.0  -0.4433480654470204  \\
            170.0  -0.36198378792589553  \\
            171.0  -0.2824043020387443  \\
            172.0  -0.20510021561724887  \\
            173.0  -0.13070078245386682  \\
            174.0  -0.059296908780217726  \\
            175.0  0.009194413688979554  \\
            176.0  0.07468767501231674  \\
            177.0  0.13714080983693355  \\
            178.0  0.1966463867348033  \\
            179.0  0.25350282900503524  \\
            180.0  0.3077025294479853  \\
            181.0  0.3590586526968253  \\
            182.0  0.40750414524681144  \\
            183.0  0.453109235212781  \\
            184.0  0.4959743641260574  \\
            185.0  0.5366390435837181  \\
            186.0  0.5745555841430044  \\
            187.0  0.6098366902942566  \\
            188.0  0.642201428074195  \\
            189.0  0.6716383066276159  \\
            190.0  0.6981127616110719  \\
            191.0  0.7215925524592061  \\
            192.0  0.7418991804956352  \\
            193.0  0.759156674195354  \\
            194.0  0.7735380034607499  \\
            195.0  0.7847834772383896  \\
            196.0  0.7930533320633262  \\
            197.0  0.7979258241729669  \\
            198.0  0.799493281901929  \\
            199.0  0.7973510527192232  \\
            200.0  0.7917005803919287  \\
            201.0  0.7826007548864886  \\
            202.0  0.7697552790149347  \\
            203.0  0.7535247006367878  \\
            204.0  0.7339945888985691  \\
            205.0  0.7110580117035398  \\
            206.0  0.6847817081234928  \\
            207.0  0.6556637356791192  \\
            208.0  0.6238657616268  \\
            209.0  0.5901981651906343  \\
            210.0  0.5550813873800912  \\
            211.0  0.5199129735491435  \\
            212.0  0.48534913454369155  \\
            213.0  0.4523764679562938  \\
            214.0  0.4215272870291472  \\
            215.0  0.3930397431358262  \\
            216.0  0.36713886705335236  \\
            217.0  0.3440593921953835  \\
            218.0  0.3238071519625035  \\
            219.0  0.3064862825374745  \\
            220.0  0.2915811941248822  \\
            221.0  0.27883687204571334  \\
            222.0  0.2685099770593875  \\
            223.0  0.260303474587578  \\
            224.0  0.2538471224059268  \\
            225.0  0.24922880434856445  \\
            226.0  0.24569885906039715  \\
            227.0  0.24354485267252474  \\
            228.0  0.2427856404927769  \\
            229.0  0.2429676068196037  \\
            230.0  0.24434762870960727  \\
            231.0  0.24698684415121916  \\
            232.0  0.25046577774706963  \\
            233.0  0.255262020763648  \\
            234.0  0.2609599671754724  \\
            235.0  0.26731930690030253  \\
            236.0  0.2743865973451987  \\
            237.0  0.2820851327290235  \\
            238.0  0.2907064887412431  \\
            239.0  0.300108590801535  \\
            240.0  0.31017314315918076  \\
            241.0  0.3208457738691458  \\
            242.0  0.3322968364421827  \\
            243.0  0.344377999047322  \\
            244.0  0.3572732382983439  \\
            245.0  0.37103533828774593  \\
            246.0  0.38538973478216637  \\
            247.0  0.40066757109155904  \\
            248.0  0.41650760464972764  \\
            249.0  0.4331700944638147  \\
            250.0  0.4509564286863325  \\
        }
        ;
    \addlegendentry {$q(\theta_2)$}
    \addplot[color={rgb,1:red,0.3059;green,0.4745;blue,0.6549}, name path={539d0543-a02c-4a71-9a7f-b20fd3dfc8ee}, draw opacity={1.0}, line width={2}, solid]
        table[row sep={\\}]
        {
            \\
            1.0  1.2  \\
            2.0  1.1975151335062217  \\
            3.0  1.1970915600789533  \\
            4.0  1.19534634473405  \\
            5.0  1.1915766509805357  \\
            6.0  1.1848493575796306  \\
            7.0  1.1770800678424318  \\
            8.0  1.171864165361489  \\
            9.0  1.1621801659387943  \\
            10.0  1.1535629165712973  \\
            11.0  1.141411216245086  \\
            12.0  1.1287961854949984  \\
            13.0  1.115313529253476  \\
            14.0  1.1005697733298465  \\
            15.0  1.0847299364321075  \\
            16.0  1.0682234391516674  \\
            17.0  1.0499067813055099  \\
            18.0  1.0303343823088553  \\
            19.0  1.0073565051419662  \\
            20.0  0.9825366453297035  \\
            21.0  0.9576164478756791  \\
            22.0  0.9294183232794121  \\
            23.0  0.9010540594657842  \\
            24.0  0.871177584792846  \\
            25.0  0.8387534517016493  \\
            26.0  0.8057038102328733  \\
            27.0  0.7685330020538078  \\
            28.0  0.7292676576998472  \\
            29.0  0.6897696092836253  \\
            30.0  0.6510980021951602  \\
            31.0  0.612423474596998  \\
            32.0  0.5716349931446794  \\
            33.0  0.5312005456391712  \\
            34.0  0.49082456757771825  \\
            35.0  0.4510744962823734  \\
            36.0  0.41196698302997964  \\
            37.0  0.3741257502378474  \\
            38.0  0.33682834653492205  \\
            39.0  0.29906848098745864  \\
            40.0  0.2632068497472198  \\
            41.0  0.228512722781819  \\
            42.0  0.19392671449094387  \\
            43.0  0.1597211144056951  \\
            44.0  0.1262304434907114  \\
            45.0  0.0924920952566912  \\
            46.0  0.06290372087302695  \\
            47.0  0.030759781377157607  \\
            48.0  -0.00019681684207249295  \\
            49.0  -0.03048593511435672  \\
            50.0  -0.059864055765992644  \\
            51.0  -0.08846075775352978  \\
            52.0  -0.11801535576010934  \\
            53.0  -0.14606175876149094  \\
            54.0  -0.1728002036841634  \\
            55.0  -0.19811453282468772  \\
            56.0  -0.22321336513983644  \\
            57.0  -0.24676826356595563  \\
            58.0  -0.2690219043706597  \\
            59.0  -0.28879821866571687  \\
            60.0  -0.30827741896017696  \\
            61.0  -0.3258030996845972  \\
            62.0  -0.34399531620454155  \\
            63.0  -0.35874384274678023  \\
            64.0  -0.37165970519829494  \\
            65.0  -0.38383748327804124  \\
            66.0  -0.39509490023155414  \\
            67.0  -0.40329931377336636  \\
            68.0  -0.40857216473046554  \\
            69.0  -0.4113500548127559  \\
            70.0  -0.41034053709838464  \\
            71.0  -0.4068516442375843  \\
            72.0  -0.40361054944639163  \\
            73.0  -0.39695295224771066  \\
            74.0  -0.38882945373319316  \\
            75.0  -0.3782519152019781  \\
            76.0  -0.3679323730802571  \\
            77.0  -0.35794380182176166  \\
            78.0  -0.34539970103964324  \\
            79.0  -0.3355173266761075  \\
            80.0  -0.3280702268312168  \\
            81.0  -0.32082298318352864  \\
            82.0  -0.3145960820564577  \\
            83.0  -0.3095664231771796  \\
            84.0  -0.30610422223427197  \\
            85.0  -0.30360272750690004  \\
            86.0  -0.3036187943229742  \\
            87.0  -0.30626703747376666  \\
            88.0  -0.3087172643195013  \\
            89.0  -0.31180449343857075  \\
            90.0  -0.3150305159677781  \\
            91.0  -0.3207579598205615  \\
            92.0  -0.32807003198091367  \\
            93.0  -0.33593202804874084  \\
            94.0  -0.3429605640923118  \\
            95.0  -0.35197137548101587  \\
            96.0  -0.3608155036416177  \\
            97.0  -0.3711962350839144  \\
            98.0  -0.38022804039756897  \\
            99.0  -0.38996544739969935  \\
            100.0  -0.399223926489659  \\
            101.0  -0.410014479203897  \\
            102.0  -0.4207589595400376  \\
            103.0  -0.4298825773447677  \\
            104.0  -0.43863619739852805  \\
            105.0  -0.44892265326297776  \\
            106.0  -0.4570540785038736  \\
            107.0  -0.4658356094534848  \\
            108.0  -0.47472761287526904  \\
            109.0  -0.48320172886313545  \\
            110.0  -0.4908051366828858  \\
            111.0  -0.49667123555427767  \\
            112.0  -0.5033095749392856  \\
            113.0  -0.5121771516371678  \\
            114.0  -0.5181976331508031  \\
            115.0  -0.523015155874087  \\
            116.0  -0.528859929132881  \\
            117.0  -0.5340502551787951  \\
            118.0  -0.53954838141458  \\
            119.0  -0.5423968975831579  \\
            120.0  -0.5436024976987137  \\
            121.0  -0.5468403142975105  \\
            122.0  -0.5482744013999638  \\
            123.0  -0.55013538025475  \\
            124.0  -0.5510837075435782  \\
            125.0  -0.553417724267389  \\
            126.0  -0.5533321642799661  \\
            127.0  -0.5528945058438742  \\
            128.0  -0.5514719283944148  \\
            129.0  -0.5488966930627701  \\
            130.0  -0.5461860421463555  \\
            131.0  -0.5440932976565519  \\
            132.0  -0.5392222044124598  \\
            133.0  -0.5343142081793605  \\
            134.0  -0.5306700169185303  \\
            135.0  -0.525843515465054  \\
            136.0  -0.5219267630618546  \\
            137.0  -0.5155884501298923  \\
            138.0  -0.5100456345018718  \\
            139.0  -0.5056072946383906  \\
            140.0  -0.4990695777308651  \\
            141.0  -0.49360968509416325  \\
            142.0  -0.48615619142021466  \\
            143.0  -0.47990071453724625  \\
            144.0  -0.4744550326720059  \\
            145.0  -0.46673524036556174  \\
            146.0  -0.4602419992410883  \\
            147.0  -0.4516059844059045  \\
            148.0  -0.4444233154634801  \\
            149.0  -0.4388371444137217  \\
            150.0  -0.4338432428599476  \\
            151.0  -0.4268186710022263  \\
            152.0  -0.42014954014142314  \\
            153.0  -0.41526093790691315  \\
            154.0  -0.4116755896080328  \\
            155.0  -0.40784835083540943  \\
            156.0  -0.40398383906468993  \\
            157.0  -0.4030891539728474  \\
            158.0  -0.40058540807119236  \\
            159.0  -0.40018734408082496  \\
            160.0  -0.4002452702260852  \\
            161.0  -0.4010309351600541  \\
            162.0  -0.40396888226884936  \\
            163.0  -0.40545692218719254  \\
            164.0  -0.40948236016498035  \\
            165.0  -0.41563059451908363  \\
            166.0  -0.42379810682412267  \\
            167.0  -0.4319152742285834  \\
            168.0  -0.44038192248955943  \\
            169.0  -0.45013963064179674  \\
            170.0  -0.4579857936211864  \\
            171.0  -0.4664674266139653  \\
            172.0  -0.47164864520534816  \\
            173.0  -0.4758253168111328  \\
            174.0  -0.4770860226799992  \\
            175.0  -0.47686592673089057  \\
            176.0  -0.4747980752011135  \\
            177.0  -0.4705216624384824  \\
            178.0  -0.4642965721751808  \\
            179.0  -0.45455622902970894  \\
            180.0  -0.4449589425799307  \\
            181.0  -0.4327109363087674  \\
            182.0  -0.4176722130844967  \\
            183.0  -0.4022921555397081  \\
            184.0  -0.3845046845167431  \\
            185.0  -0.3650606021349645  \\
            186.0  -0.3453750561269453  \\
            187.0  -0.32212769817558845  \\
            188.0  -0.2988035149071209  \\
            189.0  -0.27436868212581755  \\
            190.0  -0.25017233635191016  \\
            191.0  -0.22200383919331307  \\
            192.0  -0.19374017883719744  \\
            193.0  -0.16571344954797784  \\
            194.0  -0.13661098102940178  \\
            195.0  -0.10529876448674946  \\
            196.0  -0.0734953542951545  \\
            197.0  -0.0410537980389718  \\
            198.0  -0.008238733484454912  \\
            199.0  0.024791282082862708  \\
            200.0  0.06017033974980184  \\
            201.0  0.09793684459262833  \\
            202.0  0.13394133858774057  \\
            203.0  0.17176996466755437  \\
            204.0  0.21039892071287483  \\
            205.0  0.2510154143778975  \\
            206.0  0.2898169784372966  \\
            207.0  0.3319145967300665  \\
            208.0  0.37414428096452823  \\
            209.0  0.4183206931283684  \\
            210.0  0.462891318183566  \\
            211.0  0.5071840476177338  \\
            212.0  0.5487511253636029  \\
            213.0  0.5899835137553773  \\
            214.0  0.6299724567803545  \\
            215.0  0.6690026388252036  \\
            216.0  0.705351438212168  \\
            217.0  0.7405442912925966  \\
            218.0  0.7758875285510587  \\
            219.0  0.8076618348447357  \\
            220.0  0.8384725883894171  \\
            221.0  0.8691619888023103  \\
            222.0  0.8974837790254814  \\
            223.0  0.9225428268700723  \\
            224.0  0.9450031709106218  \\
            225.0  0.9685303427123059  \\
            226.0  0.9905253372222244  \\
            227.0  1.0114990165398179  \\
            228.0  1.028479007046821  \\
            229.0  1.0458429273393943  \\
            230.0  1.061442756914741  \\
            231.0  1.0777222154155037  \\
            232.0  1.0914119009102001  \\
            233.0  1.1052414921812326  \\
            234.0  1.116962920234714  \\
            235.0  1.1265067072366524  \\
            236.0  1.135099849358075  \\
            237.0  1.1431523497406946  \\
            238.0  1.1497604532512318  \\
            239.0  1.1555509991269148  \\
            240.0  1.1602030937033547  \\
            241.0  1.1621115986257473  \\
            242.0  1.162251490184189  \\
            243.0  1.1618570748091008  \\
            244.0  1.161508175347564  \\
            245.0  1.1596898332611592  \\
            246.0  1.1546940205367722  \\
            247.0  1.150852301705998  \\
            248.0  1.1457446674304295  \\
            249.0  1.1384683215921845  \\
            250.0  1.129549151674926  \\
        }
        ;
    \addlegendentry {$\theta_1$}
    \addplot[color={rgb,1:red,0.949;green,0.5569;blue,0.1686}, name path={9fb2c365-cdd6-4ddf-86a6-41d3fea0cff5}, draw opacity={1.0}, line width={2}, dashed]
        table[row sep={\\}]
        {
            \\
            1.0  0.2  \\
            2.0  0.20014795282996192  \\
            3.0  0.2015892702791613  \\
            4.0  0.2045304988955912  \\
            5.0  0.20629802635064515  \\
            6.0  0.2099446717416698  \\
            7.0  0.2138006248954691  \\
            8.0  0.2186281467489561  \\
            9.0  0.22459386508769225  \\
            10.0  0.22939330864455088  \\
            11.0  0.23713790355167816  \\
            12.0  0.2439054505923744  \\
            13.0  0.2514137618449681  \\
            14.0  0.2629415136011364  \\
            15.0  0.27354101306426004  \\
            16.0  0.28543310267026956  \\
            17.0  0.298792504365207  \\
            18.0  0.3155287916474878  \\
            19.0  0.33356701919655113  \\
            20.0  0.3527934275601199  \\
            21.0  0.37578007440250727  \\
            22.0  0.4002368756250385  \\
            23.0  0.4262635892360739  \\
            24.0  0.4540431688387428  \\
            25.0  0.4838727309174653  \\
            26.0  0.5167826492437508  \\
            27.0  0.5519668617808965  \\
            28.0  0.588053994698271  \\
            29.0  0.6249524241731168  \\
            30.0  0.663541982949194  \\
            31.0  0.7018194162753675  \\
            32.0  0.7404503599503693  \\
            33.0  0.7778837908371161  \\
            34.0  0.8127863212507864  \\
            35.0  0.8466768314361579  \\
            36.0  0.8748698650163873  \\
            37.0  0.9027697552624105  \\
            38.0  0.9264315000129855  \\
            39.0  0.945924374025345  \\
            40.0  0.9629028681602438  \\
            41.0  0.9739748814949962  \\
            42.0  0.9848964706159017  \\
            43.0  0.9937184392096085  \\
            44.0  0.9980606527356798  \\
            45.0  0.9990878224065965  \\
            46.0  0.9953825026774266  \\
            47.0  0.9915038132737892  \\
            48.0  0.9842015097910931  \\
            49.0  0.9735136458423738  \\
            50.0  0.9586968409858955  \\
            51.0  0.9432410948716828  \\
            52.0  0.922808901312495  \\
            53.0  0.9017747709206567  \\
            54.0  0.8760247114692843  \\
            55.0  0.8478089251027414  \\
            56.0  0.818004858422634  \\
            57.0  0.7851674696206382  \\
            58.0  0.7487242197785741  \\
            59.0  0.7113490307997925  \\
            60.0  0.6697121759588444  \\
            61.0  0.626877981756683  \\
            62.0  0.5801265660403458  \\
            63.0  0.5319217376821287  \\
            64.0  0.4805633684660777  \\
            65.0  0.4263328306632688  \\
            66.0  0.3698727974055543  \\
            67.0  0.3095117840687658  \\
            68.0  0.24628455720822065  \\
            69.0  0.18136033185672965  \\
            70.0  0.11207893216634322  \\
            71.0  0.03881554879661123  \\
            72.0  -0.037402651242302375  \\
            73.0  -0.11586515882907368  \\
            74.0  -0.1995395433915428  \\
            75.0  -0.2839508407160594  \\
            76.0  -0.3682575035224065  \\
            77.0  -0.45420936321701577  \\
            78.0  -0.5382768135862344  \\
            79.0  -0.6210585715568504  \\
            80.0  -0.7004763025692795  \\
            81.0  -0.776712001227956  \\
            82.0  -0.8515343081096088  \\
            83.0  -0.9214892520443937  \\
            84.0  -0.9900464643373615  \\
            85.0  -1.0534256056697218  \\
            86.0  -1.1154071289844663  \\
            87.0  -1.171992416944711  \\
            88.0  -1.2273660586138282  \\
            89.0  -1.2807507727602123  \\
            90.0  -1.3328100171253912  \\
            91.0  -1.3814856543603447  \\
            92.0  -1.4287493892500025  \\
            93.0  -1.4727743384898777  \\
            94.0  -1.5148899902493198  \\
            95.0  -1.5559741920222443  \\
            96.0  -1.5940677229470062  \\
            97.0  -1.6312634329015123  \\
            98.0  -1.666807152491991  \\
            99.0  -1.701145488786691  \\
            100.0  -1.733080684550524  \\
            101.0  -1.7645363950731643  \\
            102.0  -1.7956372071487925  \\
            103.0  -1.8236141039029647  \\
            104.0  -1.851552191312872  \\
            105.0  -1.8751132928999443  \\
            106.0  -1.8965620921958326  \\
            107.0  -1.9193907800329624  \\
            108.0  -1.9393352645690394  \\
            109.0  -1.9590921544678312  \\
            110.0  -1.9761907268477568  \\
            111.0  -1.9915158082313058  \\
            112.0  -2.0044096664982916  \\
            113.0  -2.0147177742712237  \\
            114.0  -2.026204760941373  \\
            115.0  -2.0373534075240882  \\
            116.0  -2.044262048517907  \\
            117.0  -2.0537575315997136  \\
            118.0  -2.05965125045974  \\
            119.0  -2.065353670736444  \\
            120.0  -2.0687601566861575  \\
            121.0  -2.0693349636101055  \\
            122.0  -2.072066163568076  \\
            123.0  -2.070398015185167  \\
            124.0  -2.071838015087258  \\
            125.0  -2.069400313356204  \\
            126.0  -2.0652774520802  \\
            127.0  -2.0589310748191303  \\
            128.0  -2.0520338243308127  \\
            129.0  -2.043922617311372  \\
            130.0  -2.0359038379552086  \\
            131.0  -2.0263998778555314  \\
            132.0  -2.016561004709457  \\
            133.0  -2.0048760850481853  \\
            134.0  -1.992756234532112  \\
            135.0  -1.9794420052131425  \\
            136.0  -1.9633727499578038  \\
            137.0  -1.945651290554721  \\
            138.0  -1.926665741531999  \\
            139.0  -1.9055981126466544  \\
            140.0  -1.8819257878939737  \\
            141.0  -1.8591495850116222  \\
            142.0  -1.8345104783747905  \\
            143.0  -1.8092600504878071  \\
            144.0  -1.7816587778763533  \\
            145.0  -1.7524444303691145  \\
            146.0  -1.7228664385200716  \\
            147.0  -1.688684929778569  \\
            148.0  -1.6562895526996102  \\
            149.0  -1.6217506812387135  \\
            150.0  -1.5848973436753144  \\
            151.0  -1.5459660004189906  \\
            152.0  -1.5039164953461193  \\
            153.0  -1.4614902409015877  \\
            154.0  -1.4171333971580777  \\
            155.0  -1.3710010490386064  \\
            156.0  -1.3233369237635857  \\
            157.0  -1.2736828141239902  \\
            158.0  -1.2214386760793725  \\
            159.0  -1.167260818532911  \\
            160.0  -1.1111516944497613  \\
            161.0  -1.0506373508988947  \\
            162.0  -0.988160795962799  \\
            163.0  -0.9225268317059587  \\
            164.0  -0.8546415861941944  \\
            165.0  -0.7843151467382317  \\
            166.0  -0.7104904355204124  \\
            167.0  -0.6339901661737442  \\
            168.0  -0.5565930772259681  \\
            169.0  -0.4773143742598667  \\
            170.0  -0.3980910653622422  \\
            171.0  -0.3178105308863873  \\
            172.0  -0.24060114448414854  \\
            173.0  -0.1654682557226259  \\
            174.0  -0.09414335347680591  \\
            175.0  -0.023848473812658373  \\
            176.0  0.04067352660202277  \\
            177.0  0.10114936009143856  \\
            178.0  0.1600438260256045  \\
            179.0  0.21590888885467965  \\
            180.0  0.27051381001312214  \\
            181.0  0.32024601454184626  \\
            182.0  0.3688655171413613  \\
            183.0  0.4166246633613116  \\
            184.0  0.4594926774239902  \\
            185.0  0.5028304074982022  \\
            186.0  0.540569940491248  \\
            187.0  0.577560331603903  \\
            188.0  0.6105478870991597  \\
            189.0  0.6421585888022179  \\
            190.0  0.6706159041419734  \\
            191.0  0.6970500799111491  \\
            192.0  0.7197435094036865  \\
            193.0  0.7378551980813102  \\
            194.0  0.7542124843678307  \\
            195.0  0.7683047546682596  \\
            196.0  0.7789386775231583  \\
            197.0  0.7868957504121493  \\
            198.0  0.7931073460366627  \\
            199.0  0.7949246770230469  \\
            200.0  0.7936005050218228  \\
            201.0  0.7876767104576587  \\
            202.0  0.779403956411658  \\
            203.0  0.7677783320264244  \\
            204.0  0.7512274884741433  \\
            205.0  0.7330989643798353  \\
            206.0  0.7113177420516267  \\
            207.0  0.6869604421044824  \\
            208.0  0.6584226637486279  \\
            209.0  0.6282798814320983  \\
            210.0  0.597040090763739  \\
            211.0  0.563408595273468  \\
            212.0  0.5308882639127307  \\
            213.0  0.4979971109363854  \\
            214.0  0.46745131373259935  \\
            215.0  0.4366339640803057  \\
            216.0  0.4090669652652999  \\
            217.0  0.3819664646070613  \\
            218.0  0.3561757980325981  \\
            219.0  0.33644602672112117  \\
            220.0  0.3190420517392564  \\
            221.0  0.3023093379197373  \\
            222.0  0.2901469208248329  \\
            223.0  0.27831479583644775  \\
            224.0  0.2695358598517463  \\
            225.0  0.2625652018648371  \\
            226.0  0.2558374592161741  \\
            227.0  0.24875346147758706  \\
            228.0  0.24456479842304105  \\
            229.0  0.2424772766150704  \\
            230.0  0.24126154788867626  \\
            231.0  0.24009929568431823  \\
            232.0  0.24093024019387352  \\
            233.0  0.24460884483487844  \\
            234.0  0.2479604518592741  \\
            235.0  0.2529483639549797  \\
            236.0  0.2592313065893545  \\
            237.0  0.2660505277266568  \\
            238.0  0.2753968809006906  \\
            239.0  0.2826702483163156  \\
            240.0  0.29083210440505375  \\
            241.0  0.30121957178765885  \\
            242.0  0.31062819768866534  \\
            243.0  0.322404289926968  \\
            244.0  0.3351003414524155  \\
            245.0  0.34836689517394465  \\
            246.0  0.3628691781222756  \\
            247.0  0.37882112822957353  \\
            248.0  0.39517863432842987  \\
            249.0  0.41133787688550866  \\
            250.0  0.4269921433178552  \\
        }
        ;
    \addlegendentry {$\theta_1$}
    \addplot[color={rgb,1:red,0.349;green,0.6314;blue,0.3098}, name path={63234387-f0e0-4047-8a19-1d0f94dacfcb}, only marks, draw opacity={0.5}, line width={0}, solid, mark={*}, mark size={1.5 pt}, mark repeat={1}, mark options={color={rgb,1:red,0.0;green,0.0;blue,0.0}, draw opacity={0.5}, fill={rgb,1:red,0.349;green,0.6314;blue,0.3098}, fill opacity={0.5}, line width={0.0}, rotate={0}, solid}]
        table[row sep={\\}]
        {
            \\
            1.0  -0.10463079207969089  \\
            2.0  0.9709541278698491  \\
            3.0  0.3826891498147659  \\
            4.0  0.41641585082867305  \\
            5.0  0.659856718876079  \\
            6.0  0.37309108060081797  \\
            7.0  1.277659227214526  \\
            8.0  -0.19303706078989558  \\
            9.0  0.35580802260263017  \\
            10.0  0.4154877776578587  \\
            11.0  0.08475922338824993  \\
            12.0  -0.8452966573500975  \\
            13.0  0.3417902753703519  \\
            14.0  0.6069223705843  \\
            15.0  0.9297344880935607  \\
            16.0  0.05025408188589675  \\
            17.0  1.0287920904818406  \\
            18.0  1.0302846133497814  \\
            19.0  0.06568379158028509  \\
            20.0  0.8363837594455024  \\
            21.0  0.10036847257368126  \\
            22.0  0.4902930378946067  \\
            23.0  0.11581875588086227  \\
            24.0  1.2054264474448024  \\
            25.0  0.3269833139162768  \\
            26.0  -0.3934508922246426  \\
            27.0  1.2118855803455573  \\
            28.0  1.3977119591295202  \\
            29.0  0.805660614027555  \\
            30.0  1.0722720007441147  \\
            31.0  1.5340629219074977  \\
            32.0  -0.1529852672033296  \\
            33.0  0.4166049780884919  \\
            34.0  1.0565797344620997  \\
            35.0  1.0988549858273748  \\
            36.0  1.077992652514692  \\
            37.0  1.447692159945075  \\
            38.0  0.8868147842381108  \\
            39.0  0.38251603360326836  \\
            40.0  2.1849710314183057  \\
            41.0  1.4986435686209565  \\
            42.0  0.7461763966081623  \\
            43.0  1.7120526583371927  \\
            44.0  2.0358056971170706  \\
            45.0  0.3546316352120128  \\
            46.0  0.6201515860348237  \\
            47.0  0.9060009540870649  \\
            48.0  0.6330473980926208  \\
            49.0  0.6509003809063695  \\
            50.0  1.3056413524415742  \\
            51.0  0.7380761998265967  \\
            52.0  0.9115803251671858  \\
            53.0  0.7776977937893889  \\
            54.0  0.49400826563649414  \\
            55.0  0.8280253506114502  \\
            56.0  0.9714094042840613  \\
            57.0  1.183765627779868  \\
            58.0  0.07592202000607662  \\
            59.0  0.45914001558124645  \\
            60.0  0.9607804316607527  \\
            61.0  1.2336600316546493  \\
            62.0  1.1673807166198131  \\
            63.0  -0.9745249497644445  \\
            64.0  0.7214391034396188  \\
            65.0  -0.001432125820121699  \\
            66.0  -0.3036598110254172  \\
            67.0  0.5642033723340689  \\
            68.0  0.727415173156427  \\
            69.0  -0.6721386463273321  \\
            70.0  -0.5751161680881756  \\
            71.0  0.21352473008639106  \\
            72.0  0.20441010136123186  \\
            73.0  -0.9613505515301143  \\
            74.0  -0.9739260380969228  \\
            75.0  -0.2572256202840269  \\
            76.0  -1.1397447369682545  \\
            77.0  0.18301729638287145  \\
            78.0  0.8650257617696033  \\
            79.0  -0.5799973020685869  \\
            80.0  -0.3971568241477475  \\
            81.0  -1.6814343213445573  \\
            82.0  -1.6732638995861895  \\
            83.0  -2.5418885217799705  \\
            84.0  -0.27636187996809414  \\
            85.0  -0.9231920405064601  \\
            86.0  -1.0591269235261427  \\
            87.0  -0.7771359229045115  \\
            88.0  -0.5346802424307127  \\
            89.0  -0.6866123267619486  \\
            90.0  -2.0885354989893004  \\
            91.0  -1.461152671011556  \\
            92.0  -1.520009175098537  \\
            93.0  -2.215458310401248  \\
            94.0  -1.2366537650821887  \\
            95.0  -1.1562225264394663  \\
            96.0  -1.2065511655000467  \\
            97.0  -1.389958566399675  \\
            98.0  -2.177687870929832  \\
            99.0  -2.926971894410806  \\
            100.0  -2.371137091309751  \\
            101.0  -1.3013116850635476  \\
            102.0  -2.0431616700834043  \\
            103.0  -1.740967559636707  \\
            104.0  -1.969208196003724  \\
            105.0  -2.41270639445722  \\
            106.0  -1.119984053064603  \\
            107.0  -2.264662487848284  \\
            108.0  -1.8155388740812481  \\
            109.0  -1.9542334298197457  \\
            110.0  -2.668756471899233  \\
            111.0  -1.2461301485985095  \\
            112.0  -1.5590886485275022  \\
            113.0  -1.9473757719467382  \\
            114.0  -2.496220551341682  \\
            115.0  -2.400648370731073  \\
            116.0  -2.876140856309562  \\
            117.0  -1.8080737089351717  \\
            118.0  -2.4813346519059043  \\
            119.0  -1.76610190688205  \\
            120.0  -2.9816347420156153  \\
            121.0  -2.233223410114454  \\
            122.0  -2.0530288103186023  \\
            123.0  -2.058620473390166  \\
            124.0  -2.5703252305408393  \\
            125.0  -1.5986356245803166  \\
            126.0  -2.0799102465591672  \\
            127.0  -1.2661824824090342  \\
            128.0  -3.8820644794924544  \\
            129.0  -0.8546056103667701  \\
            130.0  -2.0753790412416167  \\
            131.0  -1.2568377704510545  \\
            132.0  -2.348198772850747  \\
            133.0  -2.407369200176149  \\
            134.0  -2.0633093642665985  \\
            135.0  -1.9166002561120732  \\
            136.0  -2.9100093244455425  \\
            137.0  -2.053455833714235  \\
            138.0  -2.117548361727147  \\
            139.0  -1.9029797235214092  \\
            140.0  -1.536492888762661  \\
            141.0  -2.898355421550715  \\
            142.0  -2.214935322062658  \\
            143.0  -3.349044932408097  \\
            144.0  -1.4775331187598062  \\
            145.0  -2.1755224474791066  \\
            146.0  -2.3125359128953202  \\
            147.0  -0.8362768272825517  \\
            148.0  -1.9011186253621393  \\
            149.0  -2.4106296759824364  \\
            150.0  -1.4477499239119813  \\
            151.0  -1.6219099226531968  \\
            152.0  -1.1457375473677036  \\
            153.0  -1.8212811814703276  \\
            154.0  -1.4832783831944556  \\
            155.0  -1.055653924478813  \\
            156.0  -1.017627623061574  \\
            157.0  -0.680321860896667  \\
            158.0  -1.8741281396154394  \\
            159.0  -0.959974588389475  \\
            160.0  -0.8943162137694862  \\
            161.0  0.4744610650331067  \\
            162.0  -1.2372107490081337  \\
            163.0  -0.822216892536155  \\
            164.0  0.08394014227663427  \\
            165.0  -0.8997289730400873  \\
            166.0  -0.323576080571593  \\
            167.0  -1.1195279781672451  \\
            168.0  -0.9291308289479432  \\
            169.0  -0.7534991634579427  \\
            170.0  -0.1915876143961938  \\
            171.0  -0.7889922531747685  \\
            172.0  0.05624533461853887  \\
            173.0  -0.21366226029138208  \\
            174.0  -0.709217945675594  \\
            175.0  -0.32474821943429893  \\
            176.0  0.11549861280817  \\
            177.0  0.15324606369064422  \\
            178.0  -0.5838885445064906  \\
            179.0  -0.28702774061416314  \\
            180.0  0.6306369298733958  \\
            181.0  1.0911798032965283  \\
            182.0  1.0038739911907122  \\
            183.0  0.7249651502556405  \\
            184.0  -1.2056104668213805  \\
            185.0  0.6713789385525382  \\
            186.0  0.2038464317135404  \\
            187.0  1.1009422161169056  \\
            188.0  1.0244883117754235  \\
            189.0  1.0050564588618194  \\
            190.0  0.9600955551163666  \\
            191.0  1.4151152616000602  \\
            192.0  0.798267677108591  \\
            193.0  0.023530418089968586  \\
            194.0  0.768019165947492  \\
            195.0  0.012259833547316301  \\
            196.0  1.307046099642534  \\
            197.0  0.769818918405917  \\
            198.0  1.9583321872685426  \\
            199.0  0.9799215451099285  \\
            200.0  0.5454496843225098  \\
            201.0  1.3376672541521293  \\
            202.0  -0.06211304160335973  \\
            203.0  -0.15118658328258594  \\
            204.0  0.581362995283576  \\
            205.0  0.8988238078265551  \\
            206.0  -0.09180328054864095  \\
            207.0  0.7800519163536174  \\
            208.0  0.206726114429426  \\
            209.0  1.672690567205215  \\
            210.0  -0.005868794724597559  \\
            211.0  0.8211341545860765  \\
            212.0  0.08252195686467967  \\
            213.0  0.3123176837795772  \\
            214.0  0.7645364139490456  \\
            215.0  0.6110715264079557  \\
            216.0  -0.022368096496587386  \\
            217.0  -0.17690173949807964  \\
            218.0  -0.8896881097911429  \\
            219.0  0.45571695558439634  \\
            220.0  0.8220430928282134  \\
            221.0  -0.6222498497117714  \\
            222.0  -0.0977104706261217  \\
            223.0  0.7057951763840716  \\
            224.0  -0.0779363561885717  \\
            225.0  2.120245049593542  \\
            226.0  0.683973437771664  \\
            227.0  0.23929765220699353  \\
            228.0  1.4820770392265583  \\
            229.0  0.2451633723494544  \\
            230.0  -0.24854178917955855  \\
            231.0  0.9691593285067869  \\
            232.0  -0.9425277370733105  \\
            233.0  0.3384207654328142  \\
            234.0  0.9901574941768617  \\
            235.0  0.6455508571123019  \\
            236.0  0.76087010639677  \\
            237.0  -0.3708762578591095  \\
            238.0  0.07408114749492126  \\
            239.0  0.4362943209412486  \\
            240.0  0.5846691681922602  \\
            241.0  -0.02201894994977066  \\
            242.0  0.5258171736388496  \\
            243.0  -0.07316167779978161  \\
            244.0  -0.16833891147331403  \\
            245.0  0.9017184566057852  \\
            246.0  -0.1381641450105242  \\
            247.0  1.2946107805691316  \\
            248.0  0.5608361674494352  \\
            249.0  -0.27132892612125253  \\
            250.0  -0.3063152021371821  \\
        }
        ;
    \addlegendentry {$y$}
\end{axis}
\end{tikzpicture}

        \includegraphics{contents/05-experiments/plots/nlds/03-pendulum_example_inference_angles.pdf}
    }
    \caption{Simulated evolution of the angles $\theta_1$ and $\theta_2$ and their corresponding inferred posterior distributions.}
    \label{fig:sim:pendulum_example_inference_angles}
  \end{subfigure}
  \hfill
  \begin{subfigure}[t]{0.315\textwidth}
    \centering
    \resizebox{\textwidth}{!}{
        % % Recommended preamble:
% \usetikzlibrary{arrows.meta}
% \usetikzlibrary{backgrounds}
% \usepgfplotslibrary{patchplots}
% \usepgfplotslibrary{fillbetween}
% \pgfplotsset{%
%     layers/standard/.define layer set={%
%         background,axis background,axis grid,axis ticks,axis lines,axis tick labels,pre main,main,axis descriptions,axis foreground%
%     }{
%         grid style={/pgfplots/on layer=axis grid},%
%         tick style={/pgfplots/on layer=axis ticks},%
%         axis line style={/pgfplots/on layer=axis lines},%
%         label style={/pgfplots/on layer=axis descriptions},%
%         legend style={/pgfplots/on layer=axis descriptions},%
%         title style={/pgfplots/on layer=axis descriptions},%
%         colorbar style={/pgfplots/on layer=axis descriptions},%
%         ticklabel style={/pgfplots/on layer=axis tick labels},%
%         axis background@ style={/pgfplots/on layer=axis background},%
%         3d box foreground style={/pgfplots/on layer=axis foreground},%
%     },
% }

\begin{tikzpicture}[/tikz/background rectangle/.style={fill={rgb,1:red,1.0;green,1.0;blue,1.0}, fill opacity={1.0}, draw opacity={1.0}}, show background rectangle]
\begin{axis}[point meta max={nan}, point meta min={nan}, legend cell align={left}, legend columns={1}, title={}, title style={at={{(0.5,1)}}, anchor={south}, font={{\fontsize{18 pt}{23.400000000000002 pt}\selectfont}}, color={rgb,1:red,0.0;green,0.0;blue,0.0}, draw opacity={1.0}, rotate={0.0}, align={center}}, legend style={color={rgb,1:red,0.0;green,0.0;blue,0.0}, draw opacity={1.0}, line width={1}, solid, fill={rgb,1:red,1.0;green,1.0;blue,1.0}, fill opacity={1.0}, text opacity={1.0}, font={{\fontsize{14 pt}{18.2 pt}\selectfont}}, text={rgb,1:red,0.0;green,0.0;blue,0.0}, cells={anchor={center}}, at={(0.02, 0.02)}, anchor={south west}}, axis background/.style={fill={rgb,1:red,1.0;green,1.0;blue,1.0}, opacity={1.0}}, anchor={north west}, xshift={1.0mm}, yshift={-1.0mm}, width={99.6mm}, height={74.2mm}, scaled x ticks={false}, xlabel={Time step index}, x tick style={color={rgb,1:red,0.0;green,0.0;blue,0.0}, opacity={1.0}}, x tick label style={color={rgb,1:red,0.0;green,0.0;blue,0.0}, opacity={1.0}, rotate={0}}, xlabel style={at={(ticklabel cs:0.5)}, anchor=near ticklabel, at={{(ticklabel cs:0.5)}}, anchor={near ticklabel}, font={{\fontsize{16 pt}{20.8 pt}\selectfont}}, color={rgb,1:red,0.0;green,0.0;blue,0.0}, draw opacity={1.0}, rotate={0.0}}, xmajorgrids={true}, xmin={-6.469999999999999}, xmax={257.47}, xticklabels={{$0$,$50$,$100$,$150$,$200$,$250$}}, xtick={{0.0,50.0,100.0,150.0,200.0,250.0}}, xtick align={inside}, xticklabel style={font={{\fontsize{14 pt}{18.2 pt}\selectfont}}, color={rgb,1:red,0.0;green,0.0;blue,0.0}, draw opacity={1.0}, rotate={0.0}}, x grid style={color={rgb,1:red,0.0;green,0.0;blue,0.0}, draw opacity={0.1}, line width={0.5}, solid}, axis x line*={left}, x axis line style={color={rgb,1:red,0.0;green,0.0;blue,0.0}, draw opacity={1.0}, line width={1}, solid}, scaled y ticks={false}, ylabel={Angular velocity (radians / s)}, y tick style={color={rgb,1:red,0.0;green,0.0;blue,0.0}, opacity={1.0}}, y tick label style={color={rgb,1:red,0.0;green,0.0;blue,0.0}, opacity={1.0}, rotate={0}}, ylabel style={at={(ticklabel cs:0.5)}, anchor=near ticklabel, at={{(ticklabel cs:0.5)}}, anchor={near ticklabel}, font={{\fontsize{16 pt}{20.8 pt}\selectfont}}, color={rgb,1:red,0.0;green,0.0;blue,0.0}, draw opacity={1.0}, rotate={0.0}}, ymajorgrids={true}, ymin={-10.60944972164631}, ymax={9.49356914336381}, yticklabels={{$-10$,$-5$,$0$,$5$}}, ytick={{-10.0,-5.0,0.0,5.0}}, ytick align={inside}, yticklabel style={font={{\fontsize{14 pt}{18.2 pt}\selectfont}}, color={rgb,1:red,0.0;green,0.0;blue,0.0}, draw opacity={1.0}, rotate={0.0}}, y grid style={color={rgb,1:red,0.0;green,0.0;blue,0.0}, draw opacity={0.1}, line width={0.5}, solid}, axis y line*={left}, y axis line style={color={rgb,1:red,0.0;green,0.0;blue,0.0}, draw opacity={1.0}, line width={1}, solid}, colorbar={false}]
    \addplot+[line width={0}, draw opacity={0}, fill={rgb,1:red,0.9294;green,0.7882;blue,0.2824}, fill opacity={0.5}, mark={none}, forget plot]
        coordinates {
            (1,0.5057003674436129)
            (2,0.2504921463300286)
            (3,0.07443339898316603)
            (4,-0.10187494611822312)
            (5,-0.2773359931401739)
            (6,-0.4420911961040989)
            (7,-0.6126920421319734)
            (8,-0.7757723707851779)
            (9,-0.9444882326801859)
            (10,-1.1194741626544111)
            (11,-1.3024309822585596)
            (12,-1.4865241015405122)
            (13,-1.6494793637283058)
            (14,-1.8428277233516137)
            (15,-2.064463747651597)
            (16,-2.2849611399050573)
            (17,-2.5147052809921884)
            (18,-2.7074312422081106)
            (19,-2.885639175452421)
            (20,-3.0526467160756288)
            (21,-3.208962634970845)
            (22,-3.3514897899903597)
            (23,-3.4802697534583795)
            (24,-3.6001358139644624)
            (25,-3.6945209582782326)
            (26,-3.77355276554482)
            (27,-3.854641851185924)
            (28,-3.9022318781762055)
            (29,-3.909292416265611)
            (30,-3.895679353841624)
            (31,-3.8605337621593123)
            (32,-3.805226803445588)
            (33,-3.7580806184714315)
            (34,-3.7107050532829176)
            (35,-3.657581059220253)
            (36,-3.6018706312043074)
            (37,-3.5458867267426166)
            (38,-3.490316430643327)
            (39,-3.436484115309542)
            (40,-3.3838320811340528)
            (41,-3.3323412180127754)
            (42,-3.2817614910922743)
            (43,-3.23063094037422)
            (44,-3.179306314016141)
            (45,-3.127964993781282)
            (46,-3.072451634563064)
            (47,-3.0127348516664805)
            (48,-2.9484986854337167)
            (49,-2.87868184887121)
            (50,-2.8026788051233327)
            (51,-2.7204252968314706)
            (52,-2.6306852769824114)
            (53,-2.5331070018851434)
            (54,-2.4269765652323585)
            (55,-2.3114872320549202)
            (56,-2.1865440631877022)
            (57,-2.052013420019967)
            (58,-1.9084151760541628)
            (59,-1.752199277473769)
            (60,-1.5842123590810038)
            (61,-1.4053715576387718)
            (62,-1.2174711660498618)
            (63,-1.0212976691686393)
            (64,-0.8028600502215164)
            (65,-0.5725224040825968)
            (66,-0.32607337405447445)
            (67,-0.0601504465771435)
            (68,0.2121799533441337)
            (69,0.4815818161721393)
            (70,0.7538788742554509)
            (71,1.0109645363906574)
            (72,1.2094035190128767)
            (73,1.3323226339470489)
            (74,1.3623724132357025)
            (75,1.297859176275387)
            (76,1.167935260747374)
            (77,0.9898426068652764)
            (78,0.7994219062936576)
            (79,0.6125446173918478)
            (80,0.42680895633472465)
            (81,0.24917835366483163)
            (82,0.07946723827142213)
            (83,-0.07924818894316961)
            (84,-0.22706521818233152)
            (85,-0.360340030239815)
            (86,-0.48061896970655976)
            (87,-0.5886667781031858)
            (88,-0.6846488566797584)
            (89,-0.7687014352980184)
            (90,-0.8415176106105599)
            (91,-0.9057946922897121)
            (92,-0.9609207997984707)
            (93,-1.0073690896541536)
            (94,-1.0463172242071799)
            (95,-1.0770060667952088)
            (96,-1.0996713506900764)
            (97,-1.1146894868200188)
            (98,-1.1226306333241733)
            (99,-1.1248346614644913)
            (100,-1.122121978726929)
            (101,-1.1140555385852229)
            (102,-1.100366327530073)
            (103,-1.0817796064924838)
            (104,-1.0584329654176121)
            (105,-1.0307554311745055)
            (106,-0.9992207983387998)
            (107,-0.9635478104224401)
            (108,-0.9245763615364597)
            (109,-0.8823494086193665)
            (110,-0.8372053219140403)
            (111,-0.7896848492339498)
            (112,-0.7394881722264257)
            (113,-0.686931232572252)
            (114,-0.63246376733532)
            (115,-0.5765900760573962)
            (116,-0.5194436195255927)
            (117,-0.46133947184267515)
            (118,-0.4022001990658437)
            (119,-0.34239777506306623)
            (120,-0.2819678503434523)
            (121,-0.22138555557291084)
            (122,-0.16069526214740654)
            (123,-0.10006212654895993)
            (124,-0.03968034508982251)
            (125,0.02014068888245825)
            (126,0.07942096955319085)
            (127,0.13785791007789402)
            (128,0.19546666428465725)
            (129,0.25135420349921905)
            (130,0.3061103742367206)
            (131,0.3592032655216958)
            (132,0.41062756228824693)
            (133,0.4598877114847005)
            (134,0.506735098660088)
            (135,0.5510186926864539)
            (136,0.5925335015417634)
            (137,0.6307597971530663)
            (138,0.6656621437051816)
            (139,0.696953002694503)
            (140,0.7244192941925056)
            (141,0.7478840685270151)
            (142,0.7666493609093108)
            (143,0.7806000624801746)
            (144,0.7889806434808994)
            (145,0.7921781441913645)
            (146,0.7895710635810992)
            (147,0.780744547663588)
            (148,0.7659472563815838)
            (149,0.7443465334590392)
            (150,0.7153260870284119)
            (151,0.6788524997113154)
            (152,0.6343838908226693)
            (153,0.5816231308477435)
            (154,0.5197354558791557)
            (155,0.4482904786531244)
            (156,0.3669139027163726)
            (157,0.27507329217407134)
            (158,0.1726406319586643)
            (159,0.05780411071041772)
            (160,-0.06864651424430386)
            (161,-0.20618239227389693)
            (162,-0.34820781525192906)
            (163,-0.4964613891716636)
            (164,-0.6433956428583079)
            (165,-0.774404733024641)
            (166,-0.8804432205870428)
            (167,-0.9452902966310877)
            (168,-0.9533113306836494)
            (169,-0.894815415383851)
            (170,-0.7716419345847425)
            (171,-0.5997335014852023)
            (172,-0.3918000608960472)
            (173,-0.1657836776528368)
            (174,0.06903760974965716)
            (175,0.3063613768724077)
            (176,0.5403319015422959)
            (177,0.767208667794218)
            (178,0.9853868498067119)
            (179,1.1950292833692842)
            (180,1.394974032635724)
            (181,1.5845957178552816)
            (182,1.7633661498798463)
            (183,1.9313031563293044)
            (184,2.0893082956517595)
            (185,2.241096686559258)
            (186,2.384132527150984)
            (187,2.518924881868163)
            (188,2.6457535851493925)
            (189,2.764824525474562)
            (190,2.876540214617972)
            (191,2.9815685913315955)
            (192,3.078612076409345)
            (193,3.1712593336214248)
            (194,3.263299829089983)
            (195,3.352667372767862)
            (196,3.441002335340975)
            (197,3.528044515313178)
            (198,3.6151206869550303)
            (199,3.701081295431029)
            (200,3.789155447211714)
            (201,3.8812300085117584)
            (202,3.976697241713574)
            (203,4.074642114493901)
            (204,4.17011563896359)
            (205,4.262822106891359)
            (206,4.347522402742728)
            (207,4.423614023135122)
            (208,4.473529252874323)
            (209,4.491095334989266)
            (210,4.466426663949905)
            (211,4.398624616346552)
            (212,4.291979961864283)
            (213,4.152243044118836)
            (214,3.990150690802343)
            (215,3.816365217726847)
            (216,3.6375256191926857)
            (217,3.4585378962730013)
            (218,3.2827593793613032)
            (219,3.1119801883008336)
            (220,2.94715733641778)
            (221,2.788252502079023)
            (222,2.6348884806313126)
            (223,2.4865882389981224)
            (224,2.342877174269553)
            (225,2.2030713594608)
            (226,2.0670814852709687)
            (227,1.9341016483337203)
            (228,1.8035581074529865)
            (229,1.6755203934927416)
            (230,1.5491375576220512)
            (231,1.4238820923344653)
            (232,1.3000781143055233)
            (233,1.1765405524643504)
            (234,1.0536924612986551)
            (235,0.9316723342130492)
            (236,0.8101551768002568)
            (237,0.6890843842887018)
            (238,0.5674802315360022)
            (239,0.44553484601282095)
            (240,0.32332035603096004)
            (241,0.2007744963080772)
            (242,0.07731202575742778)
            (243,-0.046889959376795125)
            (244,-0.1723903467715202)
            (245,-0.29936632765460336)
            (246,-0.42760074756266186)
            (247,-0.5577584491288576)
            (248,-0.6896871667851121)
            (249,-0.824040855204218)
            (250,-0.9610709884369425)
            (250,-1.5397853661750882)
            (249,-1.4019100523368784)
            (248,-1.2665623366547807)
            (247,-1.134448908813562)
            (246,-1.005811357878275)
            (245,-0.8792123152148572)
            (244,-0.7551715313245758)
            (243,-0.6328070014041747)
            (242,-0.512053907248398)
            (241,-0.39301119619602964)
            (240,-0.27536123281593455)
            (239,-0.1591857984707719)
            (238,-0.044384956217875526)
            (237,0.06924989235868506)
            (236,0.1819864989260782)
            (235,0.29364587836196143)
            (234,0.40431403154431667)
            (233,0.5138922631837376)
            (232,0.6229479743204452)
            (231,0.7327856588148498)
            (230,0.8430112961189031)
            (229,0.9550830391598768)
            (228,1.069666386568795)
            (227,1.1870233210192174)
            (226,1.3087755695518903)
            (225,1.4356064738825216)
            (224,1.5677352310985617)
            (223,1.7065055603735835)
            (222,1.8521545826485228)
            (221,2.0049211039485466)
            (220,2.1646777801530814)
            (219,2.331677475120875)
            (218,2.50540915607427)
            (217,2.6837617368344926)
            (216,2.8653442711421326)
            (215,3.0470725787037543)
            (214,3.223867542572685)
            (213,3.3885920460108614)
            (212,3.5322840561648667)
            (211,3.647359120252647)
            (210,3.7219875403152063)
            (209,3.7629316633527763)
            (208,3.7481762280582465)
            (207,3.7088596353335275)
            (206,3.64330592304728)
            (205,3.5758442990450243)
            (204,3.499428333999637)
            (203,3.4195260004218335)
            (202,3.3330944929331716)
            (201,3.2428700680848515)
            (200,3.156613501789089)
            (199,3.0709385280836416)
            (198,2.9861065094211403)
            (197,2.903889819155334)
            (196,2.819926639887049)
            (195,2.735433664996828)
            (194,2.644721516681426)
            (193,2.5492350721443073)
            (192,2.4474182296751024)
            (191,2.3383455406679343)
            (190,2.22116748531484)
            (189,2.095537253726634)
            (188,1.9612887068259952)
            (187,1.8186250435219438)
            (186,1.6684362003798179)
            (185,1.5069583186829987)
            (184,1.336480360881923)
            (183,1.1519006729510597)
            (182,0.9584782524726796)
            (181,0.7581417654773988)
            (180,0.5529067118306359)
            (179,0.34259293251264633)
            (178,0.12491847303817172)
            (177,-0.09938018292178574)
            (176,-0.3271979428138918)
            (175,-0.5566321393964715)
            (174,-0.7849468726155605)
            (173,-1.006937498072917)
            (172,-1.2150020185257637)
            (171,-1.399758156119911)
            (170,-1.5466936170120469)
            (169,-1.6429701974676676)
            (168,-1.6765451396142512)
            (167,-1.6470220156097075)
            (166,-1.5644330565071978)
            (165,-1.4419656430789494)
            (164,-1.2965572174033966)
            (163,-1.1369225177264959)
            (162,-0.9774198388861204)
            (161,-0.8251759850322898)
            (160,-0.679334619162819)
            (159,-0.5452716496259495)
            (158,-0.4233499685542944)
            (157,-0.31398519809151043)
            (156,-0.2157188950300648)
            (155,-0.12819858144854318)
            (154,-0.050795911777307245)
            (153,0.016924800605953028)
            (152,0.07537724315782413)
            (151,0.12544892114575912)
            (150,0.16741833664159245)
            (149,0.20186819741334228)
            (148,0.22864438431961587)
            (147,0.24843981010662042)
            (146,0.26235427958176494)
            (145,0.2697925641001183)
            (144,0.27119458122200635)
            (143,0.2673736721494908)
            (142,0.2574384627156566)
            (141,0.24242920469897966)
            (140,0.22231580227749959)
            (139,0.19807821437387318)
            (138,0.16983623531409742)
            (137,0.13776418584888495)
            (136,0.10216929938848973)
            (135,0.06292740226068666)
            (134,0.02074182235961719)
            (133,-0.024198348133474046)
            (132,-0.07178667538154954)
            (131,-0.12175563228028796)
            (130,-0.17347072789046036)
            (129,-0.2270178391852073)
            (128,-0.2816818913934579)
            (127,-0.33844148766820337)
            (126,-0.3961283885026604)
            (125,-0.45481878062977993)
            (124,-0.5141734413043116)
            (123,-0.5742797942972588)
            (122,-0.6347972303599023)
            (121,-0.6955328050939926)
            (120,-0.7563205128999304)
            (119,-0.8170593825375994)
            (118,-0.8773473787891766)
            (117,-0.9370719216284993)
            (116,-0.9959385718802141)
            (115,-1.0538451099585986)
            (114,-1.1105213290098348)
            (113,-1.165780343443276)
            (112,-1.2192443032475788)
            (111,-1.2705671513203767)
            (110,-1.3195134232851913)
            (109,-1.366008970843783)
            (108,-1.4096851895993545)
            (107,-1.4502393230285606)
            (106,-1.4874839032566576)
            (105,-1.520921876965444)
            (104,-1.5504524688296444)
            (103,-1.5757160613933339)
            (102,-1.5963508279491925)
            (101,-1.6121216649866816)
            (100,-1.6225352060617544)
            (99,-1.6275379808553212)
            (98,-1.6273410399605948)
            (97,-1.621336259623342)
            (96,-1.6084495834093517)
            (95,-1.588182860052053)
            (94,-1.5601973329176395)
            (93,-1.5242680028228066)
            (92,-1.4809256175214247)
            (91,-1.4291926394648722)
            (90,-1.3686532741607074)
            (89,-1.2998146451883335)
            (88,-1.2203398472411424)
            (87,-1.1296609362305947)
            (86,-1.0277132744977828)
            (85,-0.9143989293006406)
            (84,-0.7891031580985992)
            (83,-0.6505935600431506)
            (82,-0.5020322364334016)
            (81,-0.3433984819962852)
            (80,-0.17754051312860314)
            (79,-0.004750399830273033)
            (78,0.16816957765953156)
            (77,0.34179514301435643)
            (76,0.5001327918108052)
            (75,0.6103021711007329)
            (74,0.6533968058525101)
            (73,0.6050670196510644)
            (72,0.46961908605155867)
            (71,0.2599207100921579)
            (70,-0.005890095011200969)
            (69,-0.28074593742466675)
            (68,-0.54791228316681)
            (67,-0.8122462791376641)
            (66,-1.065744760866191)
            (65,-1.299029841225857)
            (64,-1.5158038506244509)
            (63,-1.7189621818298089)
            (62,-1.904093849069427)
            (61,-2.0791451258262548)
            (60,-2.244000247462658)
            (59,-2.3983362163494792)
            (58,-2.542723569552511)
            (57,-2.6769330926262223)
            (56,-2.8026606599099844)
            (55,-2.919929696053692)
            (54,-3.0291024889503584)
            (53,-3.1304743071119714)
            (52,-3.224440923541608)
            (51,-3.311532855424418)
            (50,-3.3921013361806764)
            (49,-3.4670637910147106)
            (48,-3.5367008693976034)
            (47,-3.6015283183317557)
            (46,-3.662483639810281)
            (45,-3.7199706217604476)
            (44,-3.774146243900467)
            (43,-3.8286386919472095)
            (42,-3.8833112505730973)
            (41,-3.938430184269181)
            (40,-3.9948988128395024)
            (39,-4.051821599497787)
            (38,-4.111698077126283)
            (37,-4.174359347872156)
            (36,-4.237015833084917)
            (35,-4.299584336284462)
            (34,-4.359295260533507)
            (33,-4.412683981045094)
            (32,-4.4685640675640315)
            (31,-4.5401767252675755)
            (30,-4.5868052761586116)
            (29,-4.609510682660918)
            (28,-4.611002548016779)
            (27,-4.56656661249572)
            (26,-4.486485881052949)
            (25,-4.41397077866757)
            (24,-4.328524493412005)
            (23,-4.216292899649282)
            (22,-4.098227060231374)
            (21,-3.9683511693906395)
            (20,-3.82724649754693)
            (19,-3.6776246435714164)
            (18,-3.5191141945569715)
            (17,-3.3479335261517287)
            (16,-3.1328246574480643)
            (15,-2.931503493890081)
            (14,-2.7181716211225364)
            (13,-2.5252626848639204)
            (12,-2.3588517651484207)
            (11,-2.18107191043837)
            (10,-2.0044612951998886)
            (9,-1.8332036880368747)
            (8,-1.6654944235751596)
            (7,-1.5014174151866997)
            (6,-1.3265131486509554)
            (5,-1.1546850410920741)
            (4,-0.9702740049420194)
            (3,-0.7833989333977597)
            (2,-0.5956652520575864)
            (1,-0.32926413425403145)
            (1,0.5057003674436129)
        }
        ;
    \addplot+[line width={0}, draw opacity={0}, fill={rgb,1:red,0.9294;green,0.7882;blue,0.2824}, fill opacity={0.5}, mark={none}, forget plot]
        coordinates {
            (1,0.5057003674436129)
            (2,0.2504921463300286)
            (3,0.07443339898316603)
            (4,-0.10187494611822312)
            (5,-0.2773359931401739)
            (6,-0.4420911961040989)
            (7,-0.6126920421319734)
            (8,-0.7757723707851779)
            (9,-0.9444882326801859)
            (10,-1.1194741626544111)
            (11,-1.3024309822585596)
            (12,-1.4865241015405122)
            (13,-1.6494793637283058)
            (14,-1.8428277233516137)
            (15,-2.064463747651597)
            (16,-2.2849611399050573)
            (17,-2.5147052809921884)
            (18,-2.7074312422081106)
            (19,-2.885639175452421)
            (20,-3.0526467160756288)
            (21,-3.208962634970845)
            (22,-3.3514897899903597)
            (23,-3.4802697534583795)
            (24,-3.6001358139644624)
            (25,-3.6945209582782326)
            (26,-3.77355276554482)
            (27,-3.854641851185924)
            (28,-3.9022318781762055)
            (29,-3.909292416265611)
            (30,-3.895679353841624)
            (31,-3.8605337621593123)
            (32,-3.805226803445588)
            (33,-3.7580806184714315)
            (34,-3.7107050532829176)
            (35,-3.657581059220253)
            (36,-3.6018706312043074)
            (37,-3.5458867267426166)
            (38,-3.490316430643327)
            (39,-3.436484115309542)
            (40,-3.3838320811340528)
            (41,-3.3323412180127754)
            (42,-3.2817614910922743)
            (43,-3.23063094037422)
            (44,-3.179306314016141)
            (45,-3.127964993781282)
            (46,-3.072451634563064)
            (47,-3.0127348516664805)
            (48,-2.9484986854337167)
            (49,-2.87868184887121)
            (50,-2.8026788051233327)
            (51,-2.7204252968314706)
            (52,-2.6306852769824114)
            (53,-2.5331070018851434)
            (54,-2.4269765652323585)
            (55,-2.3114872320549202)
            (56,-2.1865440631877022)
            (57,-2.052013420019967)
            (58,-1.9084151760541628)
            (59,-1.752199277473769)
            (60,-1.5842123590810038)
            (61,-1.4053715576387718)
            (62,-1.2174711660498618)
            (63,-1.0212976691686393)
            (64,-0.8028600502215164)
            (65,-0.5725224040825968)
            (66,-0.32607337405447445)
            (67,-0.0601504465771435)
            (68,0.2121799533441337)
            (69,0.4815818161721393)
            (70,0.7538788742554509)
            (71,1.0109645363906574)
            (72,1.2094035190128767)
            (73,1.3323226339470489)
            (74,1.3623724132357025)
            (75,1.297859176275387)
            (76,1.167935260747374)
            (77,0.9898426068652764)
            (78,0.7994219062936576)
            (79,0.6125446173918478)
            (80,0.42680895633472465)
            (81,0.24917835366483163)
            (82,0.07946723827142213)
            (83,-0.07924818894316961)
            (84,-0.22706521818233152)
            (85,-0.360340030239815)
            (86,-0.48061896970655976)
            (87,-0.5886667781031858)
            (88,-0.6846488566797584)
            (89,-0.7687014352980184)
            (90,-0.8415176106105599)
            (91,-0.9057946922897121)
            (92,-0.9609207997984707)
            (93,-1.0073690896541536)
            (94,-1.0463172242071799)
            (95,-1.0770060667952088)
            (96,-1.0996713506900764)
            (97,-1.1146894868200188)
            (98,-1.1226306333241733)
            (99,-1.1248346614644913)
            (100,-1.122121978726929)
            (101,-1.1140555385852229)
            (102,-1.100366327530073)
            (103,-1.0817796064924838)
            (104,-1.0584329654176121)
            (105,-1.0307554311745055)
            (106,-0.9992207983387998)
            (107,-0.9635478104224401)
            (108,-0.9245763615364597)
            (109,-0.8823494086193665)
            (110,-0.8372053219140403)
            (111,-0.7896848492339498)
            (112,-0.7394881722264257)
            (113,-0.686931232572252)
            (114,-0.63246376733532)
            (115,-0.5765900760573962)
            (116,-0.5194436195255927)
            (117,-0.46133947184267515)
            (118,-0.4022001990658437)
            (119,-0.34239777506306623)
            (120,-0.2819678503434523)
            (121,-0.22138555557291084)
            (122,-0.16069526214740654)
            (123,-0.10006212654895993)
            (124,-0.03968034508982251)
            (125,0.02014068888245825)
            (126,0.07942096955319085)
            (127,0.13785791007789402)
            (128,0.19546666428465725)
            (129,0.25135420349921905)
            (130,0.3061103742367206)
            (131,0.3592032655216958)
            (132,0.41062756228824693)
            (133,0.4598877114847005)
            (134,0.506735098660088)
            (135,0.5510186926864539)
            (136,0.5925335015417634)
            (137,0.6307597971530663)
            (138,0.6656621437051816)
            (139,0.696953002694503)
            (140,0.7244192941925056)
            (141,0.7478840685270151)
            (142,0.7666493609093108)
            (143,0.7806000624801746)
            (144,0.7889806434808994)
            (145,0.7921781441913645)
            (146,0.7895710635810992)
            (147,0.780744547663588)
            (148,0.7659472563815838)
            (149,0.7443465334590392)
            (150,0.7153260870284119)
            (151,0.6788524997113154)
            (152,0.6343838908226693)
            (153,0.5816231308477435)
            (154,0.5197354558791557)
            (155,0.4482904786531244)
            (156,0.3669139027163726)
            (157,0.27507329217407134)
            (158,0.1726406319586643)
            (159,0.05780411071041772)
            (160,-0.06864651424430386)
            (161,-0.20618239227389693)
            (162,-0.34820781525192906)
            (163,-0.4964613891716636)
            (164,-0.6433956428583079)
            (165,-0.774404733024641)
            (166,-0.8804432205870428)
            (167,-0.9452902966310877)
            (168,-0.9533113306836494)
            (169,-0.894815415383851)
            (170,-0.7716419345847425)
            (171,-0.5997335014852023)
            (172,-0.3918000608960472)
            (173,-0.1657836776528368)
            (174,0.06903760974965716)
            (175,0.3063613768724077)
            (176,0.5403319015422959)
            (177,0.767208667794218)
            (178,0.9853868498067119)
            (179,1.1950292833692842)
            (180,1.394974032635724)
            (181,1.5845957178552816)
            (182,1.7633661498798463)
            (183,1.9313031563293044)
            (184,2.0893082956517595)
            (185,2.241096686559258)
            (186,2.384132527150984)
            (187,2.518924881868163)
            (188,2.6457535851493925)
            (189,2.764824525474562)
            (190,2.876540214617972)
            (191,2.9815685913315955)
            (192,3.078612076409345)
            (193,3.1712593336214248)
            (194,3.263299829089983)
            (195,3.352667372767862)
            (196,3.441002335340975)
            (197,3.528044515313178)
            (198,3.6151206869550303)
            (199,3.701081295431029)
            (200,3.789155447211714)
            (201,3.8812300085117584)
            (202,3.976697241713574)
            (203,4.074642114493901)
            (204,4.17011563896359)
            (205,4.262822106891359)
            (206,4.347522402742728)
            (207,4.423614023135122)
            (208,4.473529252874323)
            (209,4.491095334989266)
            (210,4.466426663949905)
            (211,4.398624616346552)
            (212,4.291979961864283)
            (213,4.152243044118836)
            (214,3.990150690802343)
            (215,3.816365217726847)
            (216,3.6375256191926857)
            (217,3.4585378962730013)
            (218,3.2827593793613032)
            (219,3.1119801883008336)
            (220,2.94715733641778)
            (221,2.788252502079023)
            (222,2.6348884806313126)
            (223,2.4865882389981224)
            (224,2.342877174269553)
            (225,2.2030713594608)
            (226,2.0670814852709687)
            (227,1.9341016483337203)
            (228,1.8035581074529865)
            (229,1.6755203934927416)
            (230,1.5491375576220512)
            (231,1.4238820923344653)
            (232,1.3000781143055233)
            (233,1.1765405524643504)
            (234,1.0536924612986551)
            (235,0.9316723342130492)
            (236,0.8101551768002568)
            (237,0.6890843842887018)
            (238,0.5674802315360022)
            (239,0.44553484601282095)
            (240,0.32332035603096004)
            (241,0.2007744963080772)
            (242,0.07731202575742778)
            (243,-0.046889959376795125)
            (244,-0.1723903467715202)
            (245,-0.29936632765460336)
            (246,-0.42760074756266186)
            (247,-0.5577584491288576)
            (248,-0.6896871667851121)
            (249,-0.824040855204218)
            (250,-0.9610709884369425)
            (250,-0.3823566106987968)
            (249,-0.24617165807155772)
            (248,-0.11281199691544352)
            (247,0.018932010555846923)
            (246,0.1506098627529514)
            (245,0.2804796599056505)
            (244,0.4103908377815354)
            (243,0.5390270826505845)
            (242,0.6666779587632536)
            (241,0.794560188812184)
            (240,0.9220019448778547)
            (239,1.0502554904964139)
            (238,1.17934541928988)
            (237,1.3089188762187185)
            (236,1.4383238546744355)
            (235,1.569698790064137)
            (234,1.7030708910529935)
            (233,1.839188841744963)
            (232,1.9772082542906015)
            (231,2.114978525854081)
            (230,2.255263819125199)
            (229,2.3959577478256064)
            (228,2.537449828337178)
            (227,2.6811799756482233)
            (226,2.825387400990047)
            (225,2.970536245039078)
            (224,3.1180191174405447)
            (223,3.2666709176226614)
            (222,3.4176223786141025)
            (221,3.5715839002095)
            (220,3.7296368926824788)
            (219,3.892282901480792)
            (218,4.060109602648336)
            (217,4.23331405571151)
            (216,4.409706967243238)
            (215,4.58565785674994)
            (214,4.756433839032002)
            (213,4.91589404222681)
            (212,5.051675867563699)
            (211,5.149890112440458)
            (210,5.210865787584604)
            (209,5.219259006625756)
            (208,5.1988822776904)
            (207,5.1383684109367165)
            (206,5.051738882438175)
            (205,4.949799914737693)
            (204,4.840802943927542)
            (203,4.729758228565969)
            (202,4.620299990493977)
            (201,4.519589948938665)
            (200,4.421697392634338)
            (199,4.331224062778416)
            (198,4.24413486448892)
            (197,4.152199211471022)
            (196,4.062078030794901)
            (195,3.9699010805388966)
            (194,3.8818781414985395)
            (193,3.793283595098542)
            (192,3.709805923143588)
            (191,3.6247916419952566)
            (190,3.531912943921104)
            (189,3.43411179722249)
            (188,3.33021846347279)
            (187,3.2192247202143824)
            (186,3.09982885392215)
            (185,2.9752350544355175)
            (184,2.842136230421596)
            (183,2.710705639707549)
            (182,2.568254047287013)
            (181,2.4110496702331643)
            (180,2.2370413534408122)
            (179,2.047465634225922)
            (178,1.8458552265752521)
            (177,1.6337975185102218)
            (176,1.4078617458984837)
            (175,1.169354893141287)
            (174,0.9230220921148747)
            (173,0.6753701427672436)
            (172,0.4314018967336693)
            (171,0.2002911531495063)
            (170,0.003409747842561872)
            (169,-0.14666063330003432)
            (168,-0.23007752175304752)
            (167,-0.2435585776524678)
            (166,-0.19645338466688778)
            (165,-0.10684382297033268)
            (164,0.009765931686780971)
            (163,0.14399973938316868)
            (162,0.2810042083822622)
            (161,0.41281120048449604)
            (160,0.5420415906742112)
            (159,0.6608798710467849)
            (158,0.768631232471623)
            (157,0.864131782439653)
            (156,0.94954670046281)
            (155,1.024779538754792)
            (154,1.0902668235356185)
            (153,1.146321461089534)
            (152,1.1933905384875145)
            (151,1.2322560782768717)
            (150,1.2632338374152314)
            (149,1.2868248695047362)
            (148,1.3032501284435516)
            (147,1.3130492852205555)
            (146,1.3167878475804335)
            (145,1.3145637242826105)
            (144,1.3067667057397925)
            (143,1.2938264528108583)
            (142,1.275860259102965)
            (141,1.2533389323550503)
            (140,1.2265227861075116)
            (139,1.195827791015133)
            (138,1.1614880520962658)
            (137,1.1237554084572476)
            (136,1.0828977036950371)
            (135,1.039109983112221)
            (134,0.9927283749605589)
            (133,0.9439737711028751)
            (132,0.8930417999580433)
            (131,0.8401621633236795)
            (130,0.7856914763639016)
            (129,0.7297262461836453)
            (128,0.6726152199627724)
            (127,0.6141573078239915)
            (126,0.5549703276090421)
            (125,0.4951001583946964)
            (124,0.4348127511246666)
            (123,0.374155541199339)
            (122,0.3134067060650893)
            (121,0.25276169394817094)
            (120,0.19238481221302584)
            (119,0.13226383241146683)
            (118,0.07294698065748917)
            (117,0.014392977943148977)
            (116,-0.0429486671709714)
            (115,-0.09933504215619393)
            (114,-0.1544062056608051)
            (113,-0.20808212170122786)
            (112,-0.25973204120527266)
            (111,-0.30880254714752287)
            (110,-0.35489722054288936)
            (109,-0.39868984639495)
            (108,-0.439467533473565)
            (107,-0.4768562978163196)
            (106,-0.5109576934209422)
            (105,-0.540588985383567)
            (104,-0.5664134620055798)
            (103,-0.5878431515916338)
            (102,-0.6043818271109536)
            (101,-0.6159894121837641)
            (100,-0.6217087513921036)
            (99,-0.6221313420736615)
            (98,-0.6179202266877519)
            (97,-0.6080427140166955)
            (96,-0.590893117970801)
            (95,-0.5658292735383648)
            (94,-0.5324371154967202)
            (93,-0.4904701764855006)
            (92,-0.44091598207551674)
            (91,-0.382396745114552)
            (90,-0.31438194706041234)
            (89,-0.2375882254077033)
            (88,-0.14895786611837436)
            (87,-0.047672619975776875)
            (86,0.06647533508466325)
            (85,0.19371886882101064)
            (84,0.3349727217339362)
            (83,0.49209718215681136)
            (82,0.6609667129762458)
            (81,0.8417551893259485)
            (80,1.0311584257980524)
            (79,1.2298396346139686)
            (78,1.4306742349277837)
            (77,1.6378900707161963)
            (76,1.835737729683943)
            (75,1.985416181450041)
            (74,2.0713480206188946)
            (73,2.0595782482430334)
            (72,1.9491879519741948)
            (71,1.762008362689157)
            (70,1.5136478435221026)
            (69,1.2439095697689453)
            (68,0.9722721898550775)
            (67,0.6919453859833771)
            (66,0.41359801275724206)
            (65,0.15398503306066336)
            (64,-0.08991624981858182)
            (63,-0.3236331565074697)
            (62,-0.5308484830302966)
            (61,-0.731597989451289)
            (60,-0.9244244706993492)
            (59,-1.1060623385980584)
            (58,-1.2741067825558146)
            (57,-1.4270937474137115)
            (56,-1.57042746646542)
            (55,-1.7030447680561487)
            (54,-1.8248506415143586)
            (53,-1.9357396966583154)
            (52,-2.0369296304232147)
            (51,-2.129317738238523)
            (50,-2.213256274065989)
            (49,-2.2902999067277094)
            (48,-2.36029650146983)
            (47,-2.4239413850012053)
            (46,-2.4824196293158467)
            (45,-2.5359593658021167)
            (44,-2.584466384131815)
            (43,-2.6326231888012304)
            (42,-2.6802117316114513)
            (41,-2.7262522517563696)
            (40,-2.772765349428603)
            (39,-2.8211466311212976)
            (38,-2.8689347841603716)
            (37,-2.9174141056130773)
            (36,-2.9667254293236978)
            (35,-3.0155777821560443)
            (34,-3.0621148460323275)
            (33,-3.103477255897769)
            (32,-3.1418895393271447)
            (31,-3.180890799051049)
            (30,-3.204553431524636)
            (29,-3.2090741498703035)
            (28,-3.1934612083356324)
            (27,-3.142717089876128)
            (26,-3.060619650036691)
            (25,-2.9750711378888943)
            (24,-2.8717471345169194)
            (23,-2.7442466072674767)
            (22,-2.6047525197493457)
            (21,-2.4495741005510503)
            (20,-2.2780469346043275)
            (19,-2.0936537073334254)
            (18,-1.8957482898592497)
            (17,-1.681477035832648)
            (16,-1.4370976223620504)
            (15,-1.1974240014131128)
            (14,-0.9674838255806908)
            (13,-0.7736960425926913)
            (12,-0.6141964379326037)
            (11,-0.42379005407874937)
            (10,-0.23448703010893357)
            (9,-0.05577277732349717)
            (8,0.11394968200480371)
            (7,0.27603333092275295)
            (6,0.4423307564427575)
            (5,0.6000130548117264)
            (4,0.7665241127055732)
            (3,0.9322657313640917)
            (2,1.0966495447176436)
            (1,1.3406648691412573)
            (1,0.5057003674436129)
        }
        ;
    \addplot[color={rgb,1:red,0.9294;green,0.7882;blue,0.2824}, name path={f71f3b0b-eef4-4bc8-b5d6-2f01482f2df1}, legend image code/.code={{
    \draw[fill={rgb,1:red,0.9294;green,0.7882;blue,0.2824}, fill opacity={0.5}] (0cm,-0.1cm) rectangle (0.6cm,0.1cm);
    }}, draw opacity={1.0}, line width={1}, solid]
        table[row sep={\\}]
        {
            \\
            1.0  0.5057003674436129  \\
            2.0  0.2504921463300286  \\
            3.0  0.07443339898316603  \\
            4.0  -0.10187494611822312  \\
            5.0  -0.2773359931401739  \\
            6.0  -0.4420911961040989  \\
            7.0  -0.6126920421319734  \\
            8.0  -0.7757723707851779  \\
            9.0  -0.9444882326801859  \\
            10.0  -1.1194741626544111  \\
            11.0  -1.3024309822585596  \\
            12.0  -1.4865241015405122  \\
            13.0  -1.6494793637283058  \\
            14.0  -1.8428277233516137  \\
            15.0  -2.064463747651597  \\
            16.0  -2.2849611399050573  \\
            17.0  -2.5147052809921884  \\
            18.0  -2.7074312422081106  \\
            19.0  -2.885639175452421  \\
            20.0  -3.0526467160756288  \\
            21.0  -3.208962634970845  \\
            22.0  -3.3514897899903597  \\
            23.0  -3.4802697534583795  \\
            24.0  -3.6001358139644624  \\
            25.0  -3.6945209582782326  \\
            26.0  -3.77355276554482  \\
            27.0  -3.854641851185924  \\
            28.0  -3.9022318781762055  \\
            29.0  -3.909292416265611  \\
            30.0  -3.895679353841624  \\
            31.0  -3.8605337621593123  \\
            32.0  -3.805226803445588  \\
            33.0  -3.7580806184714315  \\
            34.0  -3.7107050532829176  \\
            35.0  -3.657581059220253  \\
            36.0  -3.6018706312043074  \\
            37.0  -3.5458867267426166  \\
            38.0  -3.490316430643327  \\
            39.0  -3.436484115309542  \\
            40.0  -3.3838320811340528  \\
            41.0  -3.3323412180127754  \\
            42.0  -3.2817614910922743  \\
            43.0  -3.23063094037422  \\
            44.0  -3.179306314016141  \\
            45.0  -3.127964993781282  \\
            46.0  -3.072451634563064  \\
            47.0  -3.0127348516664805  \\
            48.0  -2.9484986854337167  \\
            49.0  -2.87868184887121  \\
            50.0  -2.8026788051233327  \\
            51.0  -2.7204252968314706  \\
            52.0  -2.6306852769824114  \\
            53.0  -2.5331070018851434  \\
            54.0  -2.4269765652323585  \\
            55.0  -2.3114872320549202  \\
            56.0  -2.1865440631877022  \\
            57.0  -2.052013420019967  \\
            58.0  -1.9084151760541628  \\
            59.0  -1.752199277473769  \\
            60.0  -1.5842123590810038  \\
            61.0  -1.4053715576387718  \\
            62.0  -1.2174711660498618  \\
            63.0  -1.0212976691686393  \\
            64.0  -0.8028600502215164  \\
            65.0  -0.5725224040825968  \\
            66.0  -0.32607337405447445  \\
            67.0  -0.0601504465771435  \\
            68.0  0.2121799533441337  \\
            69.0  0.4815818161721393  \\
            70.0  0.7538788742554509  \\
            71.0  1.0109645363906574  \\
            72.0  1.2094035190128767  \\
            73.0  1.3323226339470489  \\
            74.0  1.3623724132357025  \\
            75.0  1.297859176275387  \\
            76.0  1.167935260747374  \\
            77.0  0.9898426068652764  \\
            78.0  0.7994219062936576  \\
            79.0  0.6125446173918478  \\
            80.0  0.42680895633472465  \\
            81.0  0.24917835366483163  \\
            82.0  0.07946723827142213  \\
            83.0  -0.07924818894316961  \\
            84.0  -0.22706521818233152  \\
            85.0  -0.360340030239815  \\
            86.0  -0.48061896970655976  \\
            87.0  -0.5886667781031858  \\
            88.0  -0.6846488566797584  \\
            89.0  -0.7687014352980184  \\
            90.0  -0.8415176106105599  \\
            91.0  -0.9057946922897121  \\
            92.0  -0.9609207997984707  \\
            93.0  -1.0073690896541536  \\
            94.0  -1.0463172242071799  \\
            95.0  -1.0770060667952088  \\
            96.0  -1.0996713506900764  \\
            97.0  -1.1146894868200188  \\
            98.0  -1.1226306333241733  \\
            99.0  -1.1248346614644913  \\
            100.0  -1.122121978726929  \\
            101.0  -1.1140555385852229  \\
            102.0  -1.100366327530073  \\
            103.0  -1.0817796064924838  \\
            104.0  -1.0584329654176121  \\
            105.0  -1.0307554311745055  \\
            106.0  -0.9992207983387998  \\
            107.0  -0.9635478104224401  \\
            108.0  -0.9245763615364597  \\
            109.0  -0.8823494086193665  \\
            110.0  -0.8372053219140403  \\
            111.0  -0.7896848492339498  \\
            112.0  -0.7394881722264257  \\
            113.0  -0.686931232572252  \\
            114.0  -0.63246376733532  \\
            115.0  -0.5765900760573962  \\
            116.0  -0.5194436195255927  \\
            117.0  -0.46133947184267515  \\
            118.0  -0.4022001990658437  \\
            119.0  -0.34239777506306623  \\
            120.0  -0.2819678503434523  \\
            121.0  -0.22138555557291084  \\
            122.0  -0.16069526214740654  \\
            123.0  -0.10006212654895993  \\
            124.0  -0.03968034508982251  \\
            125.0  0.02014068888245825  \\
            126.0  0.07942096955319085  \\
            127.0  0.13785791007789402  \\
            128.0  0.19546666428465725  \\
            129.0  0.25135420349921905  \\
            130.0  0.3061103742367206  \\
            131.0  0.3592032655216958  \\
            132.0  0.41062756228824693  \\
            133.0  0.4598877114847005  \\
            134.0  0.506735098660088  \\
            135.0  0.5510186926864539  \\
            136.0  0.5925335015417634  \\
            137.0  0.6307597971530663  \\
            138.0  0.6656621437051816  \\
            139.0  0.696953002694503  \\
            140.0  0.7244192941925056  \\
            141.0  0.7478840685270151  \\
            142.0  0.7666493609093108  \\
            143.0  0.7806000624801746  \\
            144.0  0.7889806434808994  \\
            145.0  0.7921781441913645  \\
            146.0  0.7895710635810992  \\
            147.0  0.780744547663588  \\
            148.0  0.7659472563815838  \\
            149.0  0.7443465334590392  \\
            150.0  0.7153260870284119  \\
            151.0  0.6788524997113154  \\
            152.0  0.6343838908226693  \\
            153.0  0.5816231308477435  \\
            154.0  0.5197354558791557  \\
            155.0  0.4482904786531244  \\
            156.0  0.3669139027163726  \\
            157.0  0.27507329217407134  \\
            158.0  0.1726406319586643  \\
            159.0  0.05780411071041772  \\
            160.0  -0.06864651424430386  \\
            161.0  -0.20618239227389693  \\
            162.0  -0.34820781525192906  \\
            163.0  -0.4964613891716636  \\
            164.0  -0.6433956428583079  \\
            165.0  -0.774404733024641  \\
            166.0  -0.8804432205870428  \\
            167.0  -0.9452902966310877  \\
            168.0  -0.9533113306836494  \\
            169.0  -0.894815415383851  \\
            170.0  -0.7716419345847425  \\
            171.0  -0.5997335014852023  \\
            172.0  -0.3918000608960472  \\
            173.0  -0.1657836776528368  \\
            174.0  0.06903760974965716  \\
            175.0  0.3063613768724077  \\
            176.0  0.5403319015422959  \\
            177.0  0.767208667794218  \\
            178.0  0.9853868498067119  \\
            179.0  1.1950292833692842  \\
            180.0  1.394974032635724  \\
            181.0  1.5845957178552816  \\
            182.0  1.7633661498798463  \\
            183.0  1.9313031563293044  \\
            184.0  2.0893082956517595  \\
            185.0  2.241096686559258  \\
            186.0  2.384132527150984  \\
            187.0  2.518924881868163  \\
            188.0  2.6457535851493925  \\
            189.0  2.764824525474562  \\
            190.0  2.876540214617972  \\
            191.0  2.9815685913315955  \\
            192.0  3.078612076409345  \\
            193.0  3.1712593336214248  \\
            194.0  3.263299829089983  \\
            195.0  3.352667372767862  \\
            196.0  3.441002335340975  \\
            197.0  3.528044515313178  \\
            198.0  3.6151206869550303  \\
            199.0  3.701081295431029  \\
            200.0  3.789155447211714  \\
            201.0  3.8812300085117584  \\
            202.0  3.976697241713574  \\
            203.0  4.074642114493901  \\
            204.0  4.17011563896359  \\
            205.0  4.262822106891359  \\
            206.0  4.347522402742728  \\
            207.0  4.423614023135122  \\
            208.0  4.473529252874323  \\
            209.0  4.491095334989266  \\
            210.0  4.466426663949905  \\
            211.0  4.398624616346552  \\
            212.0  4.291979961864283  \\
            213.0  4.152243044118836  \\
            214.0  3.990150690802343  \\
            215.0  3.816365217726847  \\
            216.0  3.6375256191926857  \\
            217.0  3.4585378962730013  \\
            218.0  3.2827593793613032  \\
            219.0  3.1119801883008336  \\
            220.0  2.94715733641778  \\
            221.0  2.788252502079023  \\
            222.0  2.6348884806313126  \\
            223.0  2.4865882389981224  \\
            224.0  2.342877174269553  \\
            225.0  2.2030713594608  \\
            226.0  2.0670814852709687  \\
            227.0  1.9341016483337203  \\
            228.0  1.8035581074529865  \\
            229.0  1.6755203934927416  \\
            230.0  1.5491375576220512  \\
            231.0  1.4238820923344653  \\
            232.0  1.3000781143055233  \\
            233.0  1.1765405524643504  \\
            234.0  1.0536924612986551  \\
            235.0  0.9316723342130492  \\
            236.0  0.8101551768002568  \\
            237.0  0.6890843842887018  \\
            238.0  0.5674802315360022  \\
            239.0  0.44553484601282095  \\
            240.0  0.32332035603096004  \\
            241.0  0.2007744963080772  \\
            242.0  0.07731202575742778  \\
            243.0  -0.046889959376795125  \\
            244.0  -0.1723903467715202  \\
            245.0  -0.29936632765460336  \\
            246.0  -0.42760074756266186  \\
            247.0  -0.5577584491288576  \\
            248.0  -0.6896871667851121  \\
            249.0  -0.824040855204218  \\
            250.0  -0.9610709884369425  \\
        }
        ;
    \addlegendentry {$q(\dot{\theta}_1)$}
    \addplot+[line width={0}, draw opacity={0}, fill={rgb,1:red,0.6902;green,0.4784;blue,0.6314}, fill opacity={0.5}, mark={none}, forget plot]
        coordinates {
            (1,-0.5422820752264088)
            (2,-0.20352905814886255)
            (3,-0.02194319626261613)
            (4,0.15984604853383844)
            (5,0.33962141996377343)
            (6,0.4981970237044287)
            (7,0.6676032097263133)
            (8,0.8253300560574858)
            (9,0.988818863370236)
            (10,1.1633541045013345)
            (11,1.3529455331531846)
            (12,1.5401661823612445)
            (13,1.667259312494245)
            (14,1.8583865310593202)
            (15,2.1139158671951774)
            (16,2.3804517156636056)
            (17,2.661223095593254)
            (18,2.8836333281696853)
            (19,3.089332836950342)
            (20,3.266681203230791)
            (21,3.4299945605202944)
            (22,3.5604724163254278)
            (23,3.664634484041398)
            (24,3.7479274005601004)
            (25,3.7868609087986567)
            (26,3.7926293805293194)
            (27,3.799743860932126)
            (28,3.7400526226668678)
            (29,3.5963990803819503)
            (30,3.4072745842930843)
            (31,3.1695888872193945)
            (32,2.8844131248637637)
            (33,2.6140349190000007)
            (34,2.341644380342883)
            (35,2.052320546215192)
            (36,1.7526556237051045)
            (37,1.4482153434706861)
            (38,1.1389661442438874)
            (39,0.8321465203123617)
            (40,0.5310081496231359)
            (41,0.22631131023944548)
            (42,-0.07696261309570798)
            (43,-0.37667685802029693)
            (44,-0.6749009552153893)
            (45,-0.9713329094202279)
            (46,-1.2648020460369436)
            (47,-1.555209333064123)
            (48,-1.842636732027791)
            (49,-2.1266698145504237)
            (50,-2.407064813818604)
            (51,-2.6847659291291284)
            (52,-2.9590247941061585)
            (53,-3.2302241457801935)
            (54,-3.498399112466531)
            (55,-3.7632063892763847)
            (56,-4.026244583269717)
            (57,-4.288862768843701)
            (58,-4.552266466089247)
            (59,-4.815752533849394)
            (60,-5.081806241240661)
            (61,-5.353803421283845)
            (62,-5.63197541835675)
            (63,-5.915367889600238)
            (64,-6.217188503712363)
            (65,-6.533186588086235)
            (66,-6.86972682184527)
            (67,-7.235052115317705)
            (68,-7.6138652526659945)
            (69,-7.9914618249701235)
            (70,-8.379749761970979)
            (71,-8.749521581714948)
            (72,-9.024930912118917)
            (73,-9.175129844708515)
            (74,-9.16979489558563)
            (75,-9.004921681031199)
            (76,-8.734301824523254)
            (77,-8.38736827584921)
            (78,-8.026611859007255)
            (79,-7.671036273701679)
            (80,-7.320073436423652)
            (81,-6.982251000772408)
            (82,-6.663162008259533)
            (83,-6.364524598464302)
            (84,-6.085055705430396)
            (85,-5.823423241779386)
            (86,-5.576811535044428)
            (87,-5.34315519917995)
            (88,-5.119954116425445)
            (89,-4.904684281097795)
            (90,-4.696060507043128)
            (91,-4.496868289019906)
            (92,-4.304070685959521)
            (93,-4.116866429359571)
            (94,-3.935803322633494)
            (95,-3.758258773579386)
            (96,-3.583605008399653)
            (97,-3.4115910504617437)
            (98,-3.2423360273805364)
            (99,-3.0770409935182914)
            (100,-2.916151024754196)
            (101,-2.758244576367656)
            (102,-2.6021872505160832)
            (103,-2.448572609065687)
            (104,-2.297053442467834)
            (105,-2.147749931690745)
            (106,-2.0008343533123)
            (107,-1.8555105129516034)
            (108,-1.7124157193159897)
            (109,-1.5712396653656997)
            (110,-1.4319997018001338)
            (111,-1.2948437707136478)
            (112,-1.1592779151200696)
            (113,-1.0253886207390808)
            (114,-0.8931876865010528)
            (115,-0.7625752600988353)
            (116,-0.6334031211407234)
            (117,-0.505546794218402)
            (118,-0.3788370076500876)
            (119,-0.25314777221888357)
            (120,-0.1283460040954351)
            (121,-0.004270391974425494)
            (122,0.11922372808408717)
            (123,0.24230237852829173)
            (124,0.36512152829134775)
            (125,0.4878082232180966)
            (126,0.6105797446682275)
            (127,0.733566543459102)
            (128,0.8567510235985323)
            (129,0.9805874799464283)
            (130,1.105117047470001)
            (131,1.2304736023278566)
            (132,1.3564590805254026)
            (133,1.4837004367579385)
            (134,1.6123423641303525)
            (135,1.742379595002893)
            (136,1.8738890470607965)
            (137,2.007300959587729)
            (138,2.1425258091017105)
            (139,2.2797063841745735)
            (140,2.418971768021841)
            (141,2.560391413950443)
            (142,2.704273822103549)
            (143,2.8506936585094578)
            (144,2.9990338092044757)
            (145,3.150822070546683)
            (146,3.305689682909945)
            (147,3.4637203006838213)
            (148,3.6267778976415763)
            (149,3.794230744708276)
            (150,3.9660286506349767)
            (151,4.143816086545286)
            (152,4.3280851245042085)
            (153,4.520117161029373)
            (154,4.72029566539819)
            (155,4.930113627147287)
            (156,5.151195104684626)
            (157,5.384849188234543)
            (158,5.6321164775243115)
            (159,5.895863494341302)
            (160,6.176669372976446)
            (161,6.474898199823136)
            (162,6.780266180180415)
            (163,7.097682981365564)
            (164,7.415267984569382)
            (165,7.707496157151219)
            (166,7.960812540469158)
            (167,8.14651740775401)
            (168,8.237798862618176)
            (169,8.217273673613061)
            (170,8.088273975835996)
            (171,7.879373032538527)
            (172,7.612757062738506)
            (173,7.31917111509473)
            (174,7.0129381757623115)
            (175,6.702548030720478)
            (176,6.396034057193899)
            (177,6.097647633658021)
            (178,5.807725909180408)
            (179,5.523647435390516)
            (180,5.245492255794991)
            (181,4.972679124553904)
            (182,4.7040432783461315)
            (183,4.437471873886245)
            (184,4.170561256907859)
            (185,3.9004051720049167)
            (186,3.6278142015584143)
            (187,3.351659278952029)
            (188,3.0716192930070063)
            (189,2.7869981998359736)
            (190,2.497311324860183)
            (191,2.2022653750511325)
            (192,1.901349306482041)
            (193,1.5951451554662028)
            (194,1.284181142094997)
            (195,0.9675332399302787)
            (196,0.6455324252129031)
            (197,0.3173661421724175)
            (198,-0.016672094478872888)
            (199,-0.35880533680005)
            (200,-0.7076873000729562)
            (201,-1.062029637320592)
            (202,-1.4249786706831746)
            (203,-1.7844534616598087)
            (204,-2.133300634284433)
            (205,-2.4712079511694753)
            (206,-2.7865584197072497)
            (207,-3.077764084866208)
            (208,-3.3086069881875337)
            (209,-3.466946554704305)
            (210,-3.5326076796583927)
            (211,-3.5047814467016174)
            (212,-3.3927959730213617)
            (213,-3.2092468693694154)
            (214,-2.9767638962460112)
            (215,-2.717958970658403)
            (216,-2.447326526892953)
            (217,-2.1758053246738682)
            (218,-1.911177531066256)
            (219,-1.657880760977959)
            (220,-1.4189346873964999)
            (221,-1.195008475611377)
            (222,-0.9858739594283809)
            (223,-0.7911306254301584)
            (224,-0.6101637782505153)
            (225,-0.4419724017256664)
            (226,-0.2862888495068494)
            (227,-0.14181823759589438)
            (228,-0.0075075450702524315)
            (229,0.11659385056392806)
            (230,0.2321776501883873)
            (231,0.34043763630163504)
            (232,0.44090228554555116)
            (233,0.5362235551574948)
            (234,0.6258067158191759)
            (235,0.7096323637181846)
            (236,0.7886584629287603)
            (237,0.8632889437341577)
            (238,0.9361429733848136)
            (239,1.0070783072568943)
            (240,1.076267418305157)
            (241,1.1441870075931722)
            (242,1.2125182367058935)
            (243,1.281170759942497)
            (244,1.3517188643624538)
            (245,1.4248216328272305)
            (246,1.5003260046132179)
            (247,1.5799224773779696)
            (248,1.66353682261924)
            (249,1.7527905552034146)
            (250,1.8484871798340152)
            (250,0.8168954577408227)
            (249,0.706113011994496)
            (248,0.6022136670174987)
            (247,0.502484163212993)
            (246,0.4034573123067482)
            (245,0.3084042345932916)
            (244,0.2133985722555123)
            (243,0.12085891441143093)
            (242,0.03006842841349666)
            (241,-0.06152978230699624)
            (240,-0.1526788987660792)
            (239,-0.24575887865055734)
            (238,-0.3406122043721521)
            (237,-0.43647254662576807)
            (236,-0.5321993099786011)
            (235,-0.630811775046834)
            (234,-0.7316770856002738)
            (233,-0.8348063827970231)
            (232,-0.9389143718034298)
            (231,-1.042574235309726)
            (230,-1.1477800070985116)
            (229,-1.2536921996932615)
            (228,-1.3612878081592494)
            (227,-1.4721177432366717)
            (226,-1.5869615546612874)
            (225,-1.7076395781006404)
            (224,-1.836299060983154)
            (223,-1.9747985480980186)
            (222,-2.1252477956608145)
            (221,-2.289436505368971)
            (220,-2.468788421561565)
            (219,-2.6642287925410715)
            (218,-2.8757551842934737)
            (217,-3.1012710203006866)
            (216,-3.3367394943617246)
            (215,-3.5747729721907073)
            (214,-3.8045923637894132)
            (213,-4.011993609528329)
            (212,-4.174589175512747)
            (211,-4.269620727763664)
            (210,-4.285715440943464)
            (209,-4.21123091081423)
            (208,-4.05139663684856)
            (207,-3.8217167894073794)
            (206,-3.5353662422544594)
            (205,-3.2270566917461663)
            (204,-2.898559939429763)
            (203,-2.561606261406972)
            (202,-2.217746995357328)
            (201,-1.8733581323194874)
            (200,-1.5383560178883982)
            (199,-1.2101539118556843)
            (198,-0.8894333653818276)
            (197,-0.5772105174021804)
            (196,-0.2717550413420422)
            (195,0.027070037443393424)
            (194,0.320558467149636)
            (193,0.6092551653068663)
            (192,0.8968871586225986)
            (191,1.1826240130777683)
            (190,1.464356971745782)
            (189,1.7442135024891443)
            (188,2.0225765620449763)
            (187,2.2998942203826256)
            (186,2.576491517189419)
            (185,2.8547925126144724)
            (184,3.1344611855927194)
            (183,3.419645789787715)
            (182,3.7071898139179043)
            (181,3.9978230334984217)
            (180,4.292845689746386)
            (179,4.593550171562604)
            (178,4.900801190573362)
            (177,5.214227197141554)
            (176,5.5359247175386646)
            (175,5.865511278885179)
            (174,6.198697251015995)
            (173,6.527368354867513)
            (172,6.843383005802555)
            (171,7.132667600436366)
            (170,7.362442352539692)
            (169,7.512010652268522)
            (168,7.550981945976599)
            (167,7.476323250708383)
            (166,7.30604397608629)
            (165,7.071165001658044)
            (164,6.798488691853035)
            (163,6.501149665457071)
            (162,6.203242250400363)
            (161,5.916600353289037)
            (160,5.633099100598614)
            (159,5.3649591294237835)
            (158,5.1119056416329895)
            (157,4.87493032345936)
            (156,4.6492792543148935)
            (155,4.4345065728626025)
            (154,4.229433521812747)
            (153,4.033068431398893)
            (152,3.8443817599474075)
            (151,3.662213317668232)
            (150,3.4858636226303776)
            (149,3.314679479489375)
            (148,3.148035346654866)
            (147,2.9854817426051175)
            (146,2.8266451907658157)
            (145,2.6710410412983108)
            (144,2.518475545223029)
            (143,2.368728378847786)
            (142,2.2218522667531166)
            (141,2.0775422130154895)
            (140,1.936302731955876)
            (139,1.7967144051436978)
            (138,1.659012071811265)
            (137,1.5232356454405793)
            (136,1.3892035925869224)
            (135,1.257608494057042)
            (134,1.127341846613503)
            (133,0.9984341532179343)
            (132,0.8711098910213433)
            (131,0.7451894892138411)
            (130,0.6194510500652204)
            (129,0.494550948229028)
            (128,0.36992472920985797)
            (127,0.2466260143608644)
            (126,0.12329294355135556)
            (125,0.000243601643576008)
            (124,-0.12271103097436425)
            (123,-0.24564017986583575)
            (122,-0.36872748308534664)
            (121,-0.4921174472967451)
            (120,-0.6159800948613801)
            (119,-0.7405204819650739)
            (118,-0.8658605060263582)
            (117,-0.9922060225984844)
            (116,-1.1196104916749414)
            (115,-1.2484447163770782)
            (114,-1.3787894614241973)
            (113,-1.510834702207569)
            (112,-1.6446559401025853)
            (111,-1.7801789012441933)
            (110,-1.9171510110789654)
            (109,-2.0565666796472377)
            (108,-2.198075576661944)
            (107,-2.3416345245310684)
            (106,-2.4877695964110864)
            (105,-2.635358249002501)
            (104,-2.7858439745680603)
            (103,-2.9388887496104923)
            (102,-3.0942999022794764)
            (101,-3.2526276223800275)
            (100,-3.412906442556668)
            (99,-3.5769247500372536)
            (98,-3.746370146933376)
            (97,-3.9205179932310035)
            (96,-4.09802320422396)
            (95,-4.278737854185932)
            (94,-4.462897253108798)
            (93,-4.6511663663192815)
            (92,-4.846611467404189)
            (91,-5.048519846130856)
            (90,-5.257739497275763)
            (89,-5.4774355555509295)
            (88,-5.7047882255887865)
            (87,-5.941030668441122)
            (86,-6.188689992434032)
            (85,-6.450346888967081)
            (84,-6.728078356583663)
            (83,-7.0245431046554305)
            (82,-7.34124038677814)
            (81,-7.678920994002067)
            (80,-8.034921224580158)
            (79,-8.405188148664264)
            (78,-8.780439980982159)
            (77,-9.165349405556743)
            (76,-9.53878457844746)
            (75,-9.832264180605634)
            (74,-10.019551916872853)
            (73,-10.04049635754225)
            (72,-9.898393853685798)
            (71,-9.632568665463763)
            (70,-9.273676933006218)
            (69,-8.894569386998409)
            (68,-8.52665275393937)
            (67,-8.15622406011952)
            (66,-7.797645786042585)
            (65,-7.46895235609891)
            (64,-7.161368567499993)
            (63,-6.865703261100323)
            (62,-6.593157082361495)
            (61,-6.322993639365325)
            (60,-6.055363704161019)
            (59,-5.790506593994087)
            (58,-5.5267318124891975)
            (57,-5.2631665393282265)
            (56,-4.99726269907757)
            (55,-4.7284175047477)
            (54,-4.455898413873849)
            (53,-4.179471734844784)
            (52,-3.8990805650759612)
            (51,-3.614648356937132)
            (50,-3.326561622375589)
            (49,-3.0344176295836034)
            (48,-2.7390910254235776)
            (47,-2.4410837981716966)
            (46,-2.1404225351932595)
            (45,-1.837644620483588)
            (44,-1.5332024323513522)
            (43,-1.2261497587307533)
            (42,-0.9172424744464142)
            (41,-0.6057009502490296)
            (40,-0.2928793246148552)
            (39,0.017056033977721707)
            (38,0.3311444015454392)
            (37,0.6467061050924672)
            (36,0.9571445718086271)
            (35,1.2620944122908129)
            (34,1.556084068749834)
            (33,1.832557386989764)
            (32,2.1052294398734808)
            (31,2.3894870498490817)
            (30,2.6267577137744222)
            (29,2.815129000936142)
            (28,2.9570707333874693)
            (27,3.015527336805415)
            (26,3.0068667489851837)
            (25,2.997052946024199)
            (24,2.952209306172165)
            (23,2.862350623651107)
            (22,2.749726709486684)
            (21,2.609247843834928)
            (20,2.434097481159008)
            (19,2.243457099262207)
            (18,2.0228227462413866)
            (17,1.7843917195631647)
            (16,1.4879023951125352)
            (15,1.2040700820561583)
            (14,0.9312766354731389)
            (13,0.7223435506753708)
            (12,0.5771972606727098)
            (11,0.3726242472385728)
            (10,0.1658891598695642)
            (9,-0.025330103043297303)
            (8,-0.20477491626720712)
            (7,-0.37751516624846004)
            (6,-0.5606957064990483)
            (5,-0.731781973620188)
            (4,-0.922664844573249)
            (3,-1.1142064345928744)
            (2,-1.3042062522745586)
            (1,-1.6503464474005671)
            (1,-0.5422820752264088)
        }
        ;
    \addplot+[line width={0}, draw opacity={0}, fill={rgb,1:red,0.6902;green,0.4784;blue,0.6314}, fill opacity={0.5}, mark={none}, forget plot]
        coordinates {
            (1,-0.5422820752264088)
            (2,-0.20352905814886255)
            (3,-0.02194319626261613)
            (4,0.15984604853383844)
            (5,0.33962141996377343)
            (6,0.4981970237044287)
            (7,0.6676032097263133)
            (8,0.8253300560574858)
            (9,0.988818863370236)
            (10,1.1633541045013345)
            (11,1.3529455331531846)
            (12,1.5401661823612445)
            (13,1.667259312494245)
            (14,1.8583865310593202)
            (15,2.1139158671951774)
            (16,2.3804517156636056)
            (17,2.661223095593254)
            (18,2.8836333281696853)
            (19,3.089332836950342)
            (20,3.266681203230791)
            (21,3.4299945605202944)
            (22,3.5604724163254278)
            (23,3.664634484041398)
            (24,3.7479274005601004)
            (25,3.7868609087986567)
            (26,3.7926293805293194)
            (27,3.799743860932126)
            (28,3.7400526226668678)
            (29,3.5963990803819503)
            (30,3.4072745842930843)
            (31,3.1695888872193945)
            (32,2.8844131248637637)
            (33,2.6140349190000007)
            (34,2.341644380342883)
            (35,2.052320546215192)
            (36,1.7526556237051045)
            (37,1.4482153434706861)
            (38,1.1389661442438874)
            (39,0.8321465203123617)
            (40,0.5310081496231359)
            (41,0.22631131023944548)
            (42,-0.07696261309570798)
            (43,-0.37667685802029693)
            (44,-0.6749009552153893)
            (45,-0.9713329094202279)
            (46,-1.2648020460369436)
            (47,-1.555209333064123)
            (48,-1.842636732027791)
            (49,-2.1266698145504237)
            (50,-2.407064813818604)
            (51,-2.6847659291291284)
            (52,-2.9590247941061585)
            (53,-3.2302241457801935)
            (54,-3.498399112466531)
            (55,-3.7632063892763847)
            (56,-4.026244583269717)
            (57,-4.288862768843701)
            (58,-4.552266466089247)
            (59,-4.815752533849394)
            (60,-5.081806241240661)
            (61,-5.353803421283845)
            (62,-5.63197541835675)
            (63,-5.915367889600238)
            (64,-6.217188503712363)
            (65,-6.533186588086235)
            (66,-6.86972682184527)
            (67,-7.235052115317705)
            (68,-7.6138652526659945)
            (69,-7.9914618249701235)
            (70,-8.379749761970979)
            (71,-8.749521581714948)
            (72,-9.024930912118917)
            (73,-9.175129844708515)
            (74,-9.16979489558563)
            (75,-9.004921681031199)
            (76,-8.734301824523254)
            (77,-8.38736827584921)
            (78,-8.026611859007255)
            (79,-7.671036273701679)
            (80,-7.320073436423652)
            (81,-6.982251000772408)
            (82,-6.663162008259533)
            (83,-6.364524598464302)
            (84,-6.085055705430396)
            (85,-5.823423241779386)
            (86,-5.576811535044428)
            (87,-5.34315519917995)
            (88,-5.119954116425445)
            (89,-4.904684281097795)
            (90,-4.696060507043128)
            (91,-4.496868289019906)
            (92,-4.304070685959521)
            (93,-4.116866429359571)
            (94,-3.935803322633494)
            (95,-3.758258773579386)
            (96,-3.583605008399653)
            (97,-3.4115910504617437)
            (98,-3.2423360273805364)
            (99,-3.0770409935182914)
            (100,-2.916151024754196)
            (101,-2.758244576367656)
            (102,-2.6021872505160832)
            (103,-2.448572609065687)
            (104,-2.297053442467834)
            (105,-2.147749931690745)
            (106,-2.0008343533123)
            (107,-1.8555105129516034)
            (108,-1.7124157193159897)
            (109,-1.5712396653656997)
            (110,-1.4319997018001338)
            (111,-1.2948437707136478)
            (112,-1.1592779151200696)
            (113,-1.0253886207390808)
            (114,-0.8931876865010528)
            (115,-0.7625752600988353)
            (116,-0.6334031211407234)
            (117,-0.505546794218402)
            (118,-0.3788370076500876)
            (119,-0.25314777221888357)
            (120,-0.1283460040954351)
            (121,-0.004270391974425494)
            (122,0.11922372808408717)
            (123,0.24230237852829173)
            (124,0.36512152829134775)
            (125,0.4878082232180966)
            (126,0.6105797446682275)
            (127,0.733566543459102)
            (128,0.8567510235985323)
            (129,0.9805874799464283)
            (130,1.105117047470001)
            (131,1.2304736023278566)
            (132,1.3564590805254026)
            (133,1.4837004367579385)
            (134,1.6123423641303525)
            (135,1.742379595002893)
            (136,1.8738890470607965)
            (137,2.007300959587729)
            (138,2.1425258091017105)
            (139,2.2797063841745735)
            (140,2.418971768021841)
            (141,2.560391413950443)
            (142,2.704273822103549)
            (143,2.8506936585094578)
            (144,2.9990338092044757)
            (145,3.150822070546683)
            (146,3.305689682909945)
            (147,3.4637203006838213)
            (148,3.6267778976415763)
            (149,3.794230744708276)
            (150,3.9660286506349767)
            (151,4.143816086545286)
            (152,4.3280851245042085)
            (153,4.520117161029373)
            (154,4.72029566539819)
            (155,4.930113627147287)
            (156,5.151195104684626)
            (157,5.384849188234543)
            (158,5.6321164775243115)
            (159,5.895863494341302)
            (160,6.176669372976446)
            (161,6.474898199823136)
            (162,6.780266180180415)
            (163,7.097682981365564)
            (164,7.415267984569382)
            (165,7.707496157151219)
            (166,7.960812540469158)
            (167,8.14651740775401)
            (168,8.237798862618176)
            (169,8.217273673613061)
            (170,8.088273975835996)
            (171,7.879373032538527)
            (172,7.612757062738506)
            (173,7.31917111509473)
            (174,7.0129381757623115)
            (175,6.702548030720478)
            (176,6.396034057193899)
            (177,6.097647633658021)
            (178,5.807725909180408)
            (179,5.523647435390516)
            (180,5.245492255794991)
            (181,4.972679124553904)
            (182,4.7040432783461315)
            (183,4.437471873886245)
            (184,4.170561256907859)
            (185,3.9004051720049167)
            (186,3.6278142015584143)
            (187,3.351659278952029)
            (188,3.0716192930070063)
            (189,2.7869981998359736)
            (190,2.497311324860183)
            (191,2.2022653750511325)
            (192,1.901349306482041)
            (193,1.5951451554662028)
            (194,1.284181142094997)
            (195,0.9675332399302787)
            (196,0.6455324252129031)
            (197,0.3173661421724175)
            (198,-0.016672094478872888)
            (199,-0.35880533680005)
            (200,-0.7076873000729562)
            (201,-1.062029637320592)
            (202,-1.4249786706831746)
            (203,-1.7844534616598087)
            (204,-2.133300634284433)
            (205,-2.4712079511694753)
            (206,-2.7865584197072497)
            (207,-3.077764084866208)
            (208,-3.3086069881875337)
            (209,-3.466946554704305)
            (210,-3.5326076796583927)
            (211,-3.5047814467016174)
            (212,-3.3927959730213617)
            (213,-3.2092468693694154)
            (214,-2.9767638962460112)
            (215,-2.717958970658403)
            (216,-2.447326526892953)
            (217,-2.1758053246738682)
            (218,-1.911177531066256)
            (219,-1.657880760977959)
            (220,-1.4189346873964999)
            (221,-1.195008475611377)
            (222,-0.9858739594283809)
            (223,-0.7911306254301584)
            (224,-0.6101637782505153)
            (225,-0.4419724017256664)
            (226,-0.2862888495068494)
            (227,-0.14181823759589438)
            (228,-0.0075075450702524315)
            (229,0.11659385056392806)
            (230,0.2321776501883873)
            (231,0.34043763630163504)
            (232,0.44090228554555116)
            (233,0.5362235551574948)
            (234,0.6258067158191759)
            (235,0.7096323637181846)
            (236,0.7886584629287603)
            (237,0.8632889437341577)
            (238,0.9361429733848136)
            (239,1.0070783072568943)
            (240,1.076267418305157)
            (241,1.1441870075931722)
            (242,1.2125182367058935)
            (243,1.281170759942497)
            (244,1.3517188643624538)
            (245,1.4248216328272305)
            (246,1.5003260046132179)
            (247,1.5799224773779696)
            (248,1.66353682261924)
            (249,1.7527905552034146)
            (250,1.8484871798340152)
            (250,2.8800789019272077)
            (249,2.7994680984123335)
            (248,2.724859978220981)
            (247,2.6573607915429465)
            (246,2.5971946969196873)
            (245,2.5412390310611697)
            (244,2.490039156469395)
            (243,2.4414826054735634)
            (242,2.3949680449982904)
            (241,2.3499037974933406)
            (240,2.3052137353763933)
            (239,2.259915493164346)
            (238,2.212898151141779)
            (237,2.1630504340940835)
            (236,2.1095162358361215)
            (235,2.0500765024832033)
            (234,1.9832905172386255)
            (233,1.9072534931120126)
            (232,1.820718942894532)
            (231,1.7234495079129961)
            (230,1.612135307475286)
            (229,1.4868799008211178)
            (228,1.3462727180187446)
            (227,1.188481268044883)
            (226,1.0143838556475886)
            (225,0.8236947746493077)
            (224,0.6159715044821236)
            (223,0.3925372972377017)
            (222,0.15349987680405264)
            (221,-0.10058044585378334)
            (220,-0.3690809532314345)
            (219,-0.6515327294148461)
            (218,-0.9465998778390383)
            (217,-1.2503396290470499)
            (216,-1.5579135594241815)
            (215,-1.8611449691260984)
            (214,-2.1489354287026092)
            (213,-2.406500129210502)
            (212,-2.611002770529977)
            (211,-2.7399421656395706)
            (210,-2.7794999183733218)
            (209,-2.7226621985943793)
            (208,-2.5658173395265074)
            (207,-2.3338113803250367)
            (206,-2.03775059716004)
            (205,-1.7153592105927844)
            (204,-1.3680413291391034)
            (203,-1.0073006619126452)
            (202,-0.6322103460090209)
            (201,-0.2507011423216966)
            (200,0.12298141774248572)
            (199,0.4925432382555842)
            (198,0.8560891764240819)
            (197,1.2119428017470153)
            (196,1.5628198917678484)
            (195,1.907996442417164)
            (194,2.247803817040358)
            (193,2.581035145625539)
            (192,2.9058114543414835)
            (191,3.2219067370244967)
            (190,3.530265677974584)
            (189,3.829782897182803)
            (188,4.120662023969036)
            (187,4.403424337521432)
            (186,4.679136885927409)
            (185,4.946017831395361)
            (184,5.206661328222998)
            (183,5.455297957984774)
            (182,5.700896742774359)
            (181,5.947535215609387)
            (180,6.198138821843596)
            (179,6.453744699218428)
            (178,6.714650627787454)
            (177,6.981068070174489)
            (176,7.256143396849133)
            (175,7.539584782555777)
            (174,7.827179100508628)
            (173,8.110973875321946)
            (172,8.382131119674458)
            (171,8.626078464640688)
            (170,8.8141055991323)
            (169,8.9225366949576)
            (168,8.924615779259751)
            (167,8.816711564799636)
            (166,8.615581104852026)
            (165,8.343827312644393)
            (164,8.032047277285729)
            (163,7.694216297274057)
            (162,7.357290109960466)
            (161,7.033196046357236)
            (160,6.720239645354277)
            (159,6.426767859258821)
            (158,6.1523273134156335)
            (157,5.8947680530097255)
            (156,5.653110955054358)
            (155,5.425720681431972)
            (154,5.211157808983633)
            (153,5.007165890659853)
            (152,4.811788489061009)
            (151,4.625418855422339)
            (150,4.446193678639576)
            (149,4.273782009927177)
            (148,4.105520448628287)
            (147,3.941958858762525)
            (146,3.7847341750540746)
            (145,3.630603099795055)
            (144,3.4795920731859225)
            (143,3.3326589381711296)
            (142,3.1866953774539812)
            (141,3.0432406148853963)
            (140,2.9016408040878066)
            (139,2.762698363205449)
            (138,2.626039546392156)
            (137,2.491366273734879)
            (136,2.3585745015346706)
            (135,2.227150695948744)
            (134,2.0973428816472017)
            (133,1.9689667202979426)
            (132,1.841808270029462)
            (131,1.7157577154418722)
            (130,1.5907830448747815)
            (129,1.4666240116638285)
            (128,1.3435773179872066)
            (127,1.2205070725573397)
            (126,1.0978665457850996)
            (125,0.9753728447926171)
            (124,0.8529540875570598)
            (123,0.7302449369224192)
            (122,0.607174939253521)
            (121,0.4835766633478942)
            (120,0.3592880866705098)
            (119,0.23422493752730672)
            (118,0.10818649072618308)
            (117,-0.01888756583831952)
            (116,-0.14719575060650536)
            (115,-0.2767058038205922)
            (114,-0.40758591157790824)
            (113,-0.5399425392705927)
            (112,-0.673899890137554)
            (111,-0.8095086401831021)
            (110,-0.9468483925213024)
            (109,-1.0859126510841617)
            (108,-1.2267558619700352)
            (107,-1.3693865013721385)
            (106,-1.513899110213514)
            (105,-1.6601416143789893)
            (104,-1.8082629103676076)
            (103,-1.958256468520882)
            (102,-2.11007459875269)
            (101,-2.263861530355285)
            (100,-2.419395606951724)
            (99,-2.577157236999329)
            (98,-2.738301907827697)
            (97,-2.902664107692484)
            (96,-3.069186812575346)
            (95,-3.2377796929728393)
            (94,-3.408709392158191)
            (93,-3.582566492399861)
            (92,-3.761529904514852)
            (91,-3.945216731908956)
            (90,-4.1343815168104925)
            (89,-4.331933006644661)
            (88,-4.535120007262104)
            (87,-4.745279729918779)
            (86,-4.964933077654823)
            (85,-5.196499594591691)
            (84,-5.44203305427713)
            (83,-5.704506092273174)
            (82,-5.9850836297409264)
            (81,-6.285581007542748)
            (80,-6.605225648267146)
            (79,-6.936884398739095)
            (78,-7.272783737032352)
            (77,-7.609387146141679)
            (76,-7.929819070599047)
            (75,-8.177579181456764)
            (74,-8.320037874298407)
            (73,-8.30976333187478)
            (72,-8.151467970552035)
            (71,-7.8664744979661325)
            (70,-7.485822590935739)
            (69,-7.088354262941838)
            (68,-6.701077751392618)
            (67,-6.3138801705158905)
            (66,-5.941807857647955)
            (65,-5.597420820073561)
            (64,-5.273008439924732)
            (63,-4.965032518100153)
            (62,-4.670793754352005)
            (61,-4.384613203202365)
            (60,-4.108248778320302)
            (59,-3.8409984737047003)
            (58,-3.577801119689296)
            (57,-3.3145589983591752)
            (56,-3.0552264674618645)
            (55,-2.7979952738050695)
            (54,-2.5408998110592123)
            (53,-2.2809765567156037)
            (52,-2.0189690231363557)
            (51,-1.7548835013211246)
            (50,-1.487568005261619)
            (49,-1.218921999517244)
            (48,-0.9461824386320042)
            (47,-0.6693348679565496)
            (46,-0.38918155688062783)
            (45,-0.10502119835686785)
            (44,0.18340052192057366)
            (43,0.4727960426901593)
            (42,0.7633172482549982)
            (41,1.0583235707279206)
            (40,1.3548956238611272)
            (39,1.6472370066470017)
            (38,1.9467878869423356)
            (37,2.249724581848905)
            (36,2.548166675601582)
            (35,2.8425466801395713)
            (34,3.127204691935932)
            (33,3.3955124510102372)
            (32,3.6635968098540466)
            (31,3.9496907245897073)
            (30,4.187791454811746)
            (29,4.377669159827759)
            (28,4.523034511946266)
            (27,4.583960385058837)
            (26,4.578392012073455)
            (25,4.576668871573114)
            (24,4.543645494948036)
            (23,4.466918344431689)
            (22,4.371218123164171)
            (21,4.250741277205661)
            (20,4.099264925302574)
            (19,3.935208574638477)
            (18,3.744443910097984)
            (17,3.538054471623343)
            (16,3.273001036214676)
            (15,3.023761652334197)
            (14,2.7854964266455013)
            (13,2.612175074313119)
            (12,2.503135104049779)
            (11,2.3332668190677963)
            (10,2.1608190491331047)
            (9,2.002967829783769)
            (8,1.8554350283821788)
            (7,1.7127215857010865)
            (6,1.5570897539079056)
            (5,1.411024813547735)
            (4,1.242356941640926)
            (3,1.070320042067642)
            (2,0.8971481359768335)
            (1,0.5657822969477495)
            (1,-0.5422820752264088)
        }
        ;
    \addplot[color={rgb,1:red,0.6902;green,0.4784;blue,0.6314}, name path={319dfa49-01ec-47d2-bf26-b9d5a54d9654}, legend image code/.code={{
    \draw[fill={rgb,1:red,0.6902;green,0.4784;blue,0.6314}, fill opacity={0.5}] (0cm,-0.1cm) rectangle (0.6cm,0.1cm);
    }}, draw opacity={1.0}, line width={1}, solid]
        table[row sep={\\}]
        {
            \\
            1.0  -0.5422820752264088  \\
            2.0  -0.20352905814886255  \\
            3.0  -0.02194319626261613  \\
            4.0  0.15984604853383844  \\
            5.0  0.33962141996377343  \\
            6.0  0.4981970237044287  \\
            7.0  0.6676032097263133  \\
            8.0  0.8253300560574858  \\
            9.0  0.988818863370236  \\
            10.0  1.1633541045013345  \\
            11.0  1.3529455331531846  \\
            12.0  1.5401661823612445  \\
            13.0  1.667259312494245  \\
            14.0  1.8583865310593202  \\
            15.0  2.1139158671951774  \\
            16.0  2.3804517156636056  \\
            17.0  2.661223095593254  \\
            18.0  2.8836333281696853  \\
            19.0  3.089332836950342  \\
            20.0  3.266681203230791  \\
            21.0  3.4299945605202944  \\
            22.0  3.5604724163254278  \\
            23.0  3.664634484041398  \\
            24.0  3.7479274005601004  \\
            25.0  3.7868609087986567  \\
            26.0  3.7926293805293194  \\
            27.0  3.799743860932126  \\
            28.0  3.7400526226668678  \\
            29.0  3.5963990803819503  \\
            30.0  3.4072745842930843  \\
            31.0  3.1695888872193945  \\
            32.0  2.8844131248637637  \\
            33.0  2.6140349190000007  \\
            34.0  2.341644380342883  \\
            35.0  2.052320546215192  \\
            36.0  1.7526556237051045  \\
            37.0  1.4482153434706861  \\
            38.0  1.1389661442438874  \\
            39.0  0.8321465203123617  \\
            40.0  0.5310081496231359  \\
            41.0  0.22631131023944548  \\
            42.0  -0.07696261309570798  \\
            43.0  -0.37667685802029693  \\
            44.0  -0.6749009552153893  \\
            45.0  -0.9713329094202279  \\
            46.0  -1.2648020460369436  \\
            47.0  -1.555209333064123  \\
            48.0  -1.842636732027791  \\
            49.0  -2.1266698145504237  \\
            50.0  -2.407064813818604  \\
            51.0  -2.6847659291291284  \\
            52.0  -2.9590247941061585  \\
            53.0  -3.2302241457801935  \\
            54.0  -3.498399112466531  \\
            55.0  -3.7632063892763847  \\
            56.0  -4.026244583269717  \\
            57.0  -4.288862768843701  \\
            58.0  -4.552266466089247  \\
            59.0  -4.815752533849394  \\
            60.0  -5.081806241240661  \\
            61.0  -5.353803421283845  \\
            62.0  -5.63197541835675  \\
            63.0  -5.915367889600238  \\
            64.0  -6.217188503712363  \\
            65.0  -6.533186588086235  \\
            66.0  -6.86972682184527  \\
            67.0  -7.235052115317705  \\
            68.0  -7.6138652526659945  \\
            69.0  -7.9914618249701235  \\
            70.0  -8.379749761970979  \\
            71.0  -8.749521581714948  \\
            72.0  -9.024930912118917  \\
            73.0  -9.175129844708515  \\
            74.0  -9.16979489558563  \\
            75.0  -9.004921681031199  \\
            76.0  -8.734301824523254  \\
            77.0  -8.38736827584921  \\
            78.0  -8.026611859007255  \\
            79.0  -7.671036273701679  \\
            80.0  -7.320073436423652  \\
            81.0  -6.982251000772408  \\
            82.0  -6.663162008259533  \\
            83.0  -6.364524598464302  \\
            84.0  -6.085055705430396  \\
            85.0  -5.823423241779386  \\
            86.0  -5.576811535044428  \\
            87.0  -5.34315519917995  \\
            88.0  -5.119954116425445  \\
            89.0  -4.904684281097795  \\
            90.0  -4.696060507043128  \\
            91.0  -4.496868289019906  \\
            92.0  -4.304070685959521  \\
            93.0  -4.116866429359571  \\
            94.0  -3.935803322633494  \\
            95.0  -3.758258773579386  \\
            96.0  -3.583605008399653  \\
            97.0  -3.4115910504617437  \\
            98.0  -3.2423360273805364  \\
            99.0  -3.0770409935182914  \\
            100.0  -2.916151024754196  \\
            101.0  -2.758244576367656  \\
            102.0  -2.6021872505160832  \\
            103.0  -2.448572609065687  \\
            104.0  -2.297053442467834  \\
            105.0  -2.147749931690745  \\
            106.0  -2.0008343533123  \\
            107.0  -1.8555105129516034  \\
            108.0  -1.7124157193159897  \\
            109.0  -1.5712396653656997  \\
            110.0  -1.4319997018001338  \\
            111.0  -1.2948437707136478  \\
            112.0  -1.1592779151200696  \\
            113.0  -1.0253886207390808  \\
            114.0  -0.8931876865010528  \\
            115.0  -0.7625752600988353  \\
            116.0  -0.6334031211407234  \\
            117.0  -0.505546794218402  \\
            118.0  -0.3788370076500876  \\
            119.0  -0.25314777221888357  \\
            120.0  -0.1283460040954351  \\
            121.0  -0.004270391974425494  \\
            122.0  0.11922372808408717  \\
            123.0  0.24230237852829173  \\
            124.0  0.36512152829134775  \\
            125.0  0.4878082232180966  \\
            126.0  0.6105797446682275  \\
            127.0  0.733566543459102  \\
            128.0  0.8567510235985323  \\
            129.0  0.9805874799464283  \\
            130.0  1.105117047470001  \\
            131.0  1.2304736023278566  \\
            132.0  1.3564590805254026  \\
            133.0  1.4837004367579385  \\
            134.0  1.6123423641303525  \\
            135.0  1.742379595002893  \\
            136.0  1.8738890470607965  \\
            137.0  2.007300959587729  \\
            138.0  2.1425258091017105  \\
            139.0  2.2797063841745735  \\
            140.0  2.418971768021841  \\
            141.0  2.560391413950443  \\
            142.0  2.704273822103549  \\
            143.0  2.8506936585094578  \\
            144.0  2.9990338092044757  \\
            145.0  3.150822070546683  \\
            146.0  3.305689682909945  \\
            147.0  3.4637203006838213  \\
            148.0  3.6267778976415763  \\
            149.0  3.794230744708276  \\
            150.0  3.9660286506349767  \\
            151.0  4.143816086545286  \\
            152.0  4.3280851245042085  \\
            153.0  4.520117161029373  \\
            154.0  4.72029566539819  \\
            155.0  4.930113627147287  \\
            156.0  5.151195104684626  \\
            157.0  5.384849188234543  \\
            158.0  5.6321164775243115  \\
            159.0  5.895863494341302  \\
            160.0  6.176669372976446  \\
            161.0  6.474898199823136  \\
            162.0  6.780266180180415  \\
            163.0  7.097682981365564  \\
            164.0  7.415267984569382  \\
            165.0  7.707496157151219  \\
            166.0  7.960812540469158  \\
            167.0  8.14651740775401  \\
            168.0  8.237798862618176  \\
            169.0  8.217273673613061  \\
            170.0  8.088273975835996  \\
            171.0  7.879373032538527  \\
            172.0  7.612757062738506  \\
            173.0  7.31917111509473  \\
            174.0  7.0129381757623115  \\
            175.0  6.702548030720478  \\
            176.0  6.396034057193899  \\
            177.0  6.097647633658021  \\
            178.0  5.807725909180408  \\
            179.0  5.523647435390516  \\
            180.0  5.245492255794991  \\
            181.0  4.972679124553904  \\
            182.0  4.7040432783461315  \\
            183.0  4.437471873886245  \\
            184.0  4.170561256907859  \\
            185.0  3.9004051720049167  \\
            186.0  3.6278142015584143  \\
            187.0  3.351659278952029  \\
            188.0  3.0716192930070063  \\
            189.0  2.7869981998359736  \\
            190.0  2.497311324860183  \\
            191.0  2.2022653750511325  \\
            192.0  1.901349306482041  \\
            193.0  1.5951451554662028  \\
            194.0  1.284181142094997  \\
            195.0  0.9675332399302787  \\
            196.0  0.6455324252129031  \\
            197.0  0.3173661421724175  \\
            198.0  -0.016672094478872888  \\
            199.0  -0.35880533680005  \\
            200.0  -0.7076873000729562  \\
            201.0  -1.062029637320592  \\
            202.0  -1.4249786706831746  \\
            203.0  -1.7844534616598087  \\
            204.0  -2.133300634284433  \\
            205.0  -2.4712079511694753  \\
            206.0  -2.7865584197072497  \\
            207.0  -3.077764084866208  \\
            208.0  -3.3086069881875337  \\
            209.0  -3.466946554704305  \\
            210.0  -3.5326076796583927  \\
            211.0  -3.5047814467016174  \\
            212.0  -3.3927959730213617  \\
            213.0  -3.2092468693694154  \\
            214.0  -2.9767638962460112  \\
            215.0  -2.717958970658403  \\
            216.0  -2.447326526892953  \\
            217.0  -2.1758053246738682  \\
            218.0  -1.911177531066256  \\
            219.0  -1.657880760977959  \\
            220.0  -1.4189346873964999  \\
            221.0  -1.195008475611377  \\
            222.0  -0.9858739594283809  \\
            223.0  -0.7911306254301584  \\
            224.0  -0.6101637782505153  \\
            225.0  -0.4419724017256664  \\
            226.0  -0.2862888495068494  \\
            227.0  -0.14181823759589438  \\
            228.0  -0.0075075450702524315  \\
            229.0  0.11659385056392806  \\
            230.0  0.2321776501883873  \\
            231.0  0.34043763630163504  \\
            232.0  0.44090228554555116  \\
            233.0  0.5362235551574948  \\
            234.0  0.6258067158191759  \\
            235.0  0.7096323637181846  \\
            236.0  0.7886584629287603  \\
            237.0  0.8632889437341577  \\
            238.0  0.9361429733848136  \\
            239.0  1.0070783072568943  \\
            240.0  1.076267418305157  \\
            241.0  1.1441870075931722  \\
            242.0  1.2125182367058935  \\
            243.0  1.281170759942497  \\
            244.0  1.3517188643624538  \\
            245.0  1.4248216328272305  \\
            246.0  1.5003260046132179  \\
            247.0  1.5799224773779696  \\
            248.0  1.66353682261924  \\
            249.0  1.7527905552034146  \\
            250.0  1.8484871798340152  \\
        }
        ;
    \addlegendentry {$q(\dot{\theta}_2)$}
    \addplot[color={rgb,1:red,0.8824;green,0.3412;blue,0.349}, name path={0e1712d3-65c8-4199-bca3-a3d617716696}, draw opacity={1.0}, line width={1}, solid]
        table[row sep={\\}]
        {
            \\
            1.0  0.0  \\
            2.0  -0.11469369582103466  \\
            3.0  -0.2303421162258017  \\
            4.0  -0.34622588754774697  \\
            5.0  -0.46344571084153396  \\
            6.0  -0.5794872789774292  \\
            7.0  -0.6963370542011961  \\
            8.0  -0.8156115313764584  \\
            9.0  -0.9352072738369449  \\
            10.0  -1.057681152924293  \\
            11.0  -1.1791548565830803  \\
            12.0  -1.304355450806594  \\
            13.0  -1.4317670124279585  \\
            14.0  -1.5625859493065333  \\
            15.0  -1.697217167187706  \\
            16.0  -1.8348106276619283  \\
            17.0  -1.976880750611236  \\
            18.0  -2.1234380296882645  \\
            19.0  -2.2744163454539152  \\
            20.0  -2.432478143572495  \\
            21.0  -2.596714785006003  \\
            22.0  -2.767370975788402  \\
            23.0  -2.942598921923271  \\
            24.0  -3.1221088959503267  \\
            25.0  -3.3012568957356594  \\
            26.0  -3.477988206795535  \\
            27.0  -3.646343615865239  \\
            28.0  -3.796235689886352  \\
            29.0  -3.9181798967490584  \\
            30.0  -4.0078104858256935  \\
            31.0  -4.059617677185862  \\
            32.0  -4.074935577674803  \\
            33.0  -4.0565067367906575  \\
            34.0  -4.0109749830529005  \\
            35.0  -3.9489784589505286  \\
            36.0  -3.8785966431243124  \\
            37.0  -3.801695342721828  \\
            38.0  -3.725838666521876  \\
            39.0  -3.6505218296612236  \\
            40.0  -3.5789743666829823  \\
            41.0  -3.509147549986082  \\
            42.0  -3.4410999093027876  \\
            43.0  -3.3775101696388115  \\
            44.0  -3.3161891965565156  \\
            45.0  -3.2552427222880946  \\
            46.0  -3.194593690822886  \\
            47.0  -3.1338896735472765  \\
            48.0  -3.0701416467745415  \\
            49.0  -3.005679726541149  \\
            50.0  -2.937125581503674  \\
            51.0  -2.8659363977496928  \\
            52.0  -2.7890442999895275  \\
            53.0  -2.707857389605486  \\
            54.0  -2.6197472389448526  \\
            55.0  -2.522754061337334  \\
            56.0  -2.417291883121855  \\
            57.0  -2.305007970287432  \\
            58.0  -2.185496186863027  \\
            59.0  -2.056516137755376  \\
            60.0  -1.9201726788897706  \\
            61.0  -1.772667571587761  \\
            62.0  -1.6148552000751337  \\
            63.0  -1.4458516486359108  \\
            64.0  -1.265799397707936  \\
            65.0  -1.0755471481924308  \\
            66.0  -0.8750764684641791  \\
            67.0  -0.6605971457996626  \\
            68.0  -0.4362305210966639  \\
            69.0  -0.20059806406267858  \\
            70.0  0.04282335289058727  \\
            71.0  0.289602988801137  \\
            72.0  0.5339377555163622  \\
            73.0  0.7636677465158925  \\
            74.0  0.9641722090768065  \\
            75.0  1.110758458465914  \\
            76.0  1.1926014019147961  \\
            77.0  1.20004893073642  \\
            78.0  1.138046868721077  \\
            79.0  1.0225922526525066  \\
            80.0  0.8733823366680578  \\
            81.0  0.7046542615625162  \\
            82.0  0.5326883991609962  \\
            83.0  0.3649348239612217  \\
            84.0  0.20473725437382617  \\
            85.0  0.054054865382906124  \\
            86.0  -0.08399648608444148  \\
            87.0  -0.2105883679640817  \\
            88.0  -0.32405488662615567  \\
            89.0  -0.42684551255963554  \\
            90.0  -0.5196736443082666  \\
            91.0  -0.6013995291165614  \\
            92.0  -0.6746055262597548  \\
            93.0  -0.7375306396883535  \\
            94.0  -0.7915977928354512  \\
            95.0  -0.8382027840490303  \\
            96.0  -0.8774157889153973  \\
            97.0  -0.9078879886275556  \\
            98.0  -0.931972950125969  \\
            99.0  -0.9493664532617724  \\
            100.0  -0.9585878607249019  \\
            101.0  -0.9636826303100203  \\
            102.0  -0.9623529918080015  \\
            103.0  -0.9561537369931289  \\
            104.0  -0.9443971125242926  \\
            105.0  -0.9297409461996058  \\
            106.0  -0.9092722175093794  \\
            107.0  -0.8844098441224203  \\
            108.0  -0.85657592599216  \\
            109.0  -0.8238885307175626  \\
            110.0  -0.7868764777161443  \\
            111.0  -0.7494948212163158  \\
            112.0  -0.7090883558164797  \\
            113.0  -0.6624252836993054  \\
            114.0  -0.6174498364874885  \\
            115.0  -0.5706932657435145  \\
            116.0  -0.5222594150732089  \\
            117.0  -0.4708386355714777  \\
            118.0  -0.4181794846142812  \\
            119.0  -0.36364162480352447  \\
            120.0  -0.31072177965173414  \\
            121.0  -0.2570897457758618  \\
            122.0  -0.20237576172661356  \\
            123.0  -0.14518045840579788  \\
            124.0  -0.09059790570643785  \\
            125.0  -0.034560065088884005  \\
            126.0  0.020466073935859492  \\
            127.0  0.07597135961211722  \\
            128.0  0.13027823684360296  \\
            129.0  0.18243955361562558  \\
            130.0  0.23199571322438856  \\
            131.0  0.28236958013568825  \\
            132.0  0.33087944872712377  \\
            133.0  0.37677278076347753  \\
            134.0  0.42123186089103554  \\
            135.0  0.4624486591900846  \\
            136.0  0.5022371009479519  \\
            137.0  0.5387689351037152  \\
            138.0  0.5713553821175097  \\
            139.0  0.6001543759530066  \\
            140.0  0.627386020503483  \\
            141.0  0.6508543782395393  \\
            142.0  0.6690645320034623  \\
            143.0  0.6835297792079897  \\
            144.0  0.6952302037502394  \\
            145.0  0.6996577906377953  \\
            146.0  0.7003211334145267  \\
            147.0  0.6943886146116728  \\
            148.0  0.6835353915541688  \\
            149.0  0.6678820408001007  \\
            150.0  0.6425491457371817  \\
            151.0  0.6119406246300322  \\
            152.0  0.5748006278753754  \\
            153.0  0.5278668240022439  \\
            154.0  0.472418773835481  \\
            155.0  0.4087142403161307  \\
            156.0  0.33719049482166324  \\
            157.0  0.25635911613467993  \\
            158.0  0.16691113993083065  \\
            159.0  0.06622415047083736  \\
            160.0  -0.04258514258797661  \\
            161.0  -0.16000964029462988  \\
            162.0  -0.2851556753034746  \\
            163.0  -0.4179850247300051  \\
            164.0  -0.5513021131417009  \\
            165.0  -0.6820349048935124  \\
            166.0  -0.7982902213260555  \\
            167.0  -0.885208165545819  \\
            168.0  -0.9293434112857475  \\
            169.0  -0.9211110364685334  \\
            170.0  -0.8551005437280914  \\
            171.0  -0.7340531856794583  \\
            172.0  -0.5667116258029488  \\
            173.0  -0.36692010398802377  \\
            174.0  -0.14997229523902222  \\
            175.0  0.07450165508272173  \\
            176.0  0.3000633804290153  \\
            177.0  0.5221459537932684  \\
            178.0  0.7365350235740147  \\
            179.0  0.9425224245605714  \\
            180.0  1.1399490663766303  \\
            181.0  1.326271648929363  \\
            182.0  1.5060743997961685  \\
            183.0  1.6753874679808116  \\
            184.0  1.8356245961742579  \\
            185.0  1.9868195372769104  \\
            186.0  2.129344414101183  \\
            187.0  2.265220685841997  \\
            188.0  2.3929813278935828  \\
            189.0  2.5131067477954017  \\
            190.0  2.625716922478407  \\
            191.0  2.7334493393731387  \\
            192.0  2.835161608088676  \\
            193.0  2.933109759850051  \\
            194.0  3.026368080641237  \\
            195.0  3.115209533744372  \\
            196.0  3.2019352416637243  \\
            197.0  3.2878986103557115  \\
            198.0  3.371599343998376  \\
            199.0  3.4549262427720464  \\
            200.0  3.539671998211015  \\
            201.0  3.625609024602902  \\
            202.0  3.7108626739725126  \\
            203.0  3.7996501920242434  \\
            204.0  3.8877343958252735  \\
            205.0  3.97860174634788  \\
            206.0  4.063263714712888  \\
            207.0  4.142663017131904  \\
            208.0  4.207695198407714  \\
            209.0  4.2548180778073705  \\
            210.0  4.270888302601859  \\
            211.0  4.254087710333332  \\
            212.0  4.198905084974159  \\
            213.0  4.108152801288326  \\
            214.0  3.9874763487595306  \\
            215.0  3.843847338778264  \\
            216.0  3.6836700422482616  \\
            217.0  3.516386397788761  \\
            218.0  3.3443798420415534  \\
            219.0  3.1747687835982736  \\
            220.0  3.006819409685668  \\
            221.0  2.8449443994113945  \\
            222.0  2.6869911222979925  \\
            223.0  2.5331933692768005  \\
            224.0  2.384730291186175  \\
            225.0  2.242528718754081  \\
            226.0  2.101713957509452  \\
            227.0  1.9647873311787711  \\
            228.0  1.8309750220067058  \\
            229.0  1.7001239173215992  \\
            230.0  1.57122036112736  \\
            231.0  1.4438909797422157  \\
            232.0  1.3182297047322182  \\
            233.0  1.1930137005877524  \\
            234.0  1.0712496435246563  \\
            235.0  0.9474503706248942  \\
            236.0  0.8259511836433248  \\
            237.0  0.7032309391732701  \\
            238.0  0.581471378400904  \\
            239.0  0.4598109906604906  \\
            240.0  0.33804443189028877  \\
            241.0  0.21602027295401013  \\
            242.0  0.08998181303614222  \\
            243.0  -0.03502951670441838  \\
            244.0  -0.1594907498015014  \\
            245.0  -0.28551984037612227  \\
            246.0  -0.414659482487976  \\
            247.0  -0.5439620929481536  \\
            248.0  -0.675378840079946  \\
            249.0  -0.8097408270664973  \\
            250.0  -0.9462927962058011  \\
        }
        ;
    \addlegendentry {$\dot{\theta}_1$}
    \addplot[color={rgb,1:red,0.4627;green,0.7176;blue,0.698}, name path={bb5e79a9-65df-454f-bdfa-0a451723b7b5}, draw opacity={1.0}, line width={1}, solid]
        table[row sep={\\}]
        {
            \\
            1.0  0.0  \\
            2.0  0.06561075103066388  \\
            3.0  0.1312796182550023  \\
            4.0  0.19627365691735038  \\
            5.0  0.26432711134394754  \\
            6.0  0.33527133844672374  \\
            7.0  0.4094543110334141  \\
            8.0  0.4867704634141591  \\
            9.0  0.5703937224310937  \\
            10.0  0.6584554722372764  \\
            11.0  0.7517708614092733  \\
            12.0  0.8531003032023541  \\
            13.0  0.9632271428249128  \\
            14.0  1.0798769294385722  \\
            15.0  1.2058919166190711  \\
            16.0  1.343199661686743  \\
            17.0  1.490797060051735  \\
            18.0  1.6535090757681468  \\
            19.0  1.8274483865624656  \\
            20.0  2.01559971411268  \\
            21.0  2.2176974958440776  \\
            22.0  2.4339240040639845  \\
            23.0  2.6640346647000595  \\
            24.0  2.9008089313079473  \\
            25.0  3.141139863150367  \\
            26.0  3.3776072588711576  \\
            27.0  3.5951815418243966  \\
            28.0  3.777319356221099  \\
            29.0  3.9023385896187923  \\
            30.0  3.9524731695102524  \\
            31.0  3.9256759969695363  \\
            32.0  3.816549689291597  \\
            33.0  3.6373680132019506  \\
            34.0  3.4007247890190553  \\
            35.0  3.12244208523993  \\
            36.0  2.814472240896399  \\
            37.0  2.4944263956161334  \\
            38.0  2.168047616206861  \\
            39.0  1.8405227736522722  \\
            40.0  1.516417727930466  \\
            41.0  1.1931080204160158  \\
            42.0  0.8735547990046896  \\
            43.0  0.5607182600666099  \\
            44.0  0.2509247816481534  \\
            45.0  -0.05398340773753719  \\
            46.0  -0.3569451255042136  \\
            47.0  -0.6547216296226264  \\
            48.0  -0.9507713927221794  \\
            49.0  -1.2428701495923284  \\
            50.0  -1.5311968770831084  \\
            51.0  -1.816411853667134  \\
            52.0  -2.0960558133786735  \\
            53.0  -2.3717274032691384  \\
            54.0  -2.644780734060081  \\
            55.0  -2.9151272987384482  \\
            56.0  -3.1817060081817594  \\
            57.0  -3.4449826552660596  \\
            58.0  -3.705412242101849  \\
            59.0  -3.962385224069968  \\
            60.0  -4.219314001250959  \\
            61.0  -4.47482920206865  \\
            62.0  -4.733531927353915  \\
            63.0  -4.9946694157441645  \\
            64.0  -5.260423722653354  \\
            65.0  -5.5343177027035155  \\
            66.0  -5.817506235378878  \\
            67.0  -6.109567485921137  \\
            68.0  -6.413906676608645  \\
            69.0  -6.733332064431985  \\
            70.0  -7.064897124714967  \\
            71.0  -7.405863919317457  \\
            72.0  -7.744340231975508  \\
            73.0  -8.062853179624197  \\
            74.0  -8.3341111997839  \\
            75.0  -8.521705901803795  \\
            76.0  -8.595070069644455  \\
            77.0  -8.546426495926113  \\
            78.0  -8.381546245175317  \\
            79.0  -8.129364732959543  \\
            80.0  -7.81936820660994  \\
            81.0  -7.487728846558257  \\
            82.0  -7.149565921528105  \\
            83.0  -6.8210924916960485  \\
            84.0  -6.5098692402140745  \\
            85.0  -6.215665453313591  \\
            86.0  -5.938729383729261  \\
            87.0  -5.678925805680125  \\
            88.0  -5.431336800941772  \\
            89.0  -5.199318938197323  \\
            90.0  -4.979391464605305  \\
            91.0  -4.768872743798065  \\
            92.0  -4.566782936568377  \\
            93.0  -4.371364055654769  \\
            94.0  -4.184131391007341  \\
            95.0  -4.00234057167186  \\
            96.0  -3.8240381254676974  \\
            97.0  -3.650947152903731  \\
            98.0  -3.482865011314678  \\
            99.0  -3.3176892932289577  \\
            100.0  -3.154200096979302  \\
            101.0  -2.9949660369282247  \\
            102.0  -2.8373416922099004  \\
            103.0  -2.6828829871194553  \\
            104.0  -2.529833191418322  \\
            105.0  -2.379500240021144  \\
            106.0  -2.2320238723964065  \\
            107.0  -2.086229962975256  \\
            108.0  -1.9437553815447037  \\
            109.0  -1.8012539474322364  \\
            110.0  -1.6623507297172555  \\
            111.0  -1.5245980788358677  \\
            112.0  -1.3859655835879996  \\
            113.0  -1.2506561931471534  \\
            114.0  -1.1153285839264377  \\
            115.0  -0.983698691921002  \\
            116.0  -0.8532189527583628  \\
            117.0  -0.722805080008211  \\
            118.0  -0.5949897926967866  \\
            119.0  -0.4671866831830483  \\
            120.0  -0.3391328172689367  \\
            121.0  -0.21308542095941388  \\
            122.0  -0.08851972662962355  \\
            123.0  0.0387124595524646  \\
            124.0  0.16523340483470755  \\
            125.0  0.2894474505138102  \\
            126.0  0.4168592959116265  \\
            127.0  0.5402388207830539  \\
            128.0  0.6651126774941646  \\
            129.0  0.788330010521965  \\
            130.0  0.9137983716423885  \\
            131.0  1.0410317047901698  \\
            132.0  1.1686650076757379  \\
            133.0  1.296451452556609  \\
            134.0  1.4262603365345088  \\
            135.0  1.557056392086338  \\
            136.0  1.6889028827888348  \\
            137.0  1.8218248991755708  \\
            138.0  1.9552259022524296  \\
            139.0  2.090463705768385  \\
            140.0  2.228061786905526  \\
            141.0  2.36703618189395  \\
            142.0  2.508776556874719  \\
            143.0  2.652551882410584  \\
            144.0  2.7995475296934553  \\
            145.0  2.9510960432372224  \\
            146.0  3.10351798726883  \\
            147.0  3.256743559742156  \\
            148.0  3.4159695583616343  \\
            149.0  3.5782040423159587  \\
            150.0  3.7460732596746493  \\
            151.0  3.918024451108469  \\
            152.0  4.096886587721988  \\
            153.0  4.280213609540859  \\
            154.0  4.471610992179753  \\
            155.0  4.670221709180633  \\
            156.0  4.8780711251915205  \\
            157.0  5.0948654704549226  \\
            158.0  5.324664761052817  \\
            159.0  5.565564001673021  \\
            160.0  5.819849007656344  \\
            161.0  6.088478376489354  \\
            162.0  6.368177750254779  \\
            163.0  6.659956208982515  \\
            164.0  6.9542225929592725  \\
            165.0  7.241767328263497  \\
            166.0  7.507701487529667  \\
            167.0  7.728875212046091  \\
            168.0  7.879303968843611  \\
            169.0  7.9418721632182825  \\
            170.0  7.901730305319663  \\
            171.0  7.772395907780991  \\
            172.0  7.561962498871173  \\
            173.0  7.302393582482706  \\
            174.0  7.014605687535104  \\
            175.0  6.712826236032062  \\
            176.0  6.4119168643842395  \\
            177.0  6.116155337059315  \\
            178.0  5.8271941921364085  \\
            179.0  5.54740771340651  \\
            180.0  5.2738037946478755  \\
            181.0  5.007390415398314  \\
            182.0  4.746291454663917  \\
            183.0  4.485941677233922  \\
            184.0  4.226225855609385  \\
            185.0  3.96901493560427  \\
            186.0  3.709173772185523  \\
            187.0  3.4465384813885787  \\
            188.0  3.184134048294976  \\
            189.0  2.9149375537784925  \\
            190.0  2.644596415004213  \\
            191.0  2.3693654792815  \\
            192.0  2.0891728006074874  \\
            193.0  1.802022655845248  \\
            194.0  1.5115798270222367  \\
            195.0  1.2151759294141014  \\
            196.0  0.9133632588036125  \\
            197.0  0.6075039193202733  \\
            198.0  0.29628781956877487  \\
            199.0  -0.021613015527838203  \\
            200.0  -0.3406149977135845  \\
            201.0  -0.6684943380283683  \\
            202.0  -0.998701022902418  \\
            203.0  -1.3318195420544228  \\
            204.0  -1.6643756710447528  \\
            205.0  -1.9919470396936434  \\
            206.0  -2.310657290246491  \\
            207.0  -2.6110418398497526  \\
            208.0  -2.8796284621845945  \\
            209.0  -3.1008507054182384  \\
            210.0  -3.2597184808118365  \\
            211.0  -3.3429667173313895  \\
            212.0  -3.3424134513488633  \\
            213.0  -3.264217928912508  \\
            214.0  -3.1199559592489505  \\
            215.0  -2.925655131831742  \\
            216.0  -2.6993169099554386  \\
            217.0  -2.4570650581012967  \\
            218.0  -2.207532193260275  \\
            219.0  -1.9576134007744703  \\
            220.0  -1.7190529140784152  \\
            221.0  -1.490382841561922  \\
            222.0  -1.2729806289529977  \\
            223.0  -1.0698212671380896  \\
            224.0  -0.8785263520222449  \\
            225.0  -0.7016153415803198  \\
            226.0  -0.535417863640908  \\
            227.0  -0.38064767159675206  \\
            228.0  -0.2364346627083335  \\
            229.0  -0.10146123052597221  \\
            230.0  0.026031165626789754  \\
            231.0  0.14551511333959521  \\
            232.0  0.2567069320136211  \\
            233.0  0.36028062440598724  \\
            234.0  0.4587131316164322  \\
            235.0  0.5537273181384772  \\
            236.0  0.6424235500199472  \\
            237.0  0.7290541862111373  \\
            238.0  0.8113479071990206  \\
            239.0  0.8918138261653396  \\
            240.0  0.9706160354608213  \\
            241.0  1.0477836054067478  \\
            242.0  1.1246937086594326  \\
            243.0  1.204685335173556  \\
            244.0  1.2850033259513665  \\
            245.0  1.3660964526189672  \\
            246.0  1.4502672074732348  \\
            247.0  1.538269701526228  \\
            248.0  1.6312182830673907  \\
            249.0  1.7299500664744036  \\
            250.0  1.8316593263389964  \\
        }
        ;
    \addlegendentry {$\dot{\theta}_2$}
\end{axis}
\end{tikzpicture}

        \includegraphics{contents/05-experiments/plots/nlds/03-pendulum_example_inference_velocities.pdf}
    }
    \caption{Simulated evolution of the angular velocities $\dot{\theta}_1$ and $\dot{\theta}_2$ and their corresponding inferred posterior distributions.}
    \label{fig:sim:pendulum_example_inference_velocities}
  \end{subfigure}
  \hfill
  \begin{subfigure}[t]{0.315\textwidth}
    \centering
    \resizebox{\textwidth}{!}{
        % % Recommended preamble:
% \usetikzlibrary{arrows.meta}
% \usetikzlibrary{backgrounds}
% \usepgfplotslibrary{patchplots}
% \usepgfplotslibrary{fillbetween}
% \pgfplotsset{%
%     layers/standard/.define layer set={%
%         background,axis background,axis grid,axis ticks,axis lines,axis tick labels,pre main,main,axis descriptions,axis foreground%
%     }{
%         grid style={/pgfplots/on layer=axis grid},%
%         tick style={/pgfplots/on layer=axis ticks},%
%         axis line style={/pgfplots/on layer=axis lines},%
%         label style={/pgfplots/on layer=axis descriptions},%
%         legend style={/pgfplots/on layer=axis descriptions},%
%         title style={/pgfplots/on layer=axis descriptions},%
%         colorbar style={/pgfplots/on layer=axis descriptions},%
%         ticklabel style={/pgfplots/on layer=axis tick labels},%
%         axis background@ style={/pgfplots/on layer=axis background},%
%         3d box foreground style={/pgfplots/on layer=axis foreground},%
%     },
% }

\begin{tikzpicture}[/tikz/background rectangle/.style={fill={rgb,1:red,1.0;green,1.0;blue,1.0}, fill opacity={1.0}, draw opacity={1.0}}, show background rectangle]
\begin{axis}[point meta max={nan}, point meta min={nan}, legend cell align={left}, legend columns={1}, title={}, title style={at={{(0.5,1)}}, anchor={south}, font={{\fontsize{18 pt}{23.400000000000002 pt}\selectfont}}, color={rgb,1:red,0.0;green,0.0;blue,0.0}, draw opacity={1.0}, rotate={0.0}, align={center}}, legend style={color={rgb,1:red,0.0;green,0.0;blue,0.0}, draw opacity={1.0}, line width={1}, solid, fill={rgb,1:red,1.0;green,1.0;blue,1.0}, fill opacity={1.0}, text opacity={1.0}, font={{\fontsize{14 pt}{18.2 pt}\selectfont}}, text={rgb,1:red,0.0;green,0.0;blue,0.0}, cells={anchor={center}}, at={(0.98, 0.98)}, anchor={north east}}, axis background/.style={fill={rgb,1:red,1.0;green,1.0;blue,1.0}, opacity={1.0}}, anchor={north west}, xshift={1.0mm}, yshift={-1.0mm}, width={99.6mm}, height={74.2mm}, scaled x ticks={false}, xlabel={Variational iteration index}, x tick style={color={rgb,1:red,0.0;green,0.0;blue,0.0}, opacity={1.0}}, x tick label style={color={rgb,1:red,0.0;green,0.0;blue,0.0}, opacity={1.0}, rotate={0}}, xlabel style={at={(ticklabel cs:0.5)}, anchor=near ticklabel, at={{(ticklabel cs:0.5)}}, anchor={near ticklabel}, font={{\fontsize{16 pt}{20.8 pt}\selectfont}}, color={rgb,1:red,0.0;green,0.0;blue,0.0}, draw opacity={1.0}, rotate={0.0}}, xmajorgrids={true}, xmin={0.8799999999999999}, xmax={5.12}, xticklabels={{$1$,$2$,$3$,$4$,$5$}}, xtick={{1.0,2.0,3.0,4.0,5.0}}, xtick align={inside}, xticklabel style={font={{\fontsize{14 pt}{18.2 pt}\selectfont}}, color={rgb,1:red,0.0;green,0.0;blue,0.0}, draw opacity={1.0}, rotate={0.0}}, x grid style={color={rgb,1:red,0.0;green,0.0;blue,0.0}, draw opacity={0.1}, line width={0.5}, solid}, axis x line*={left}, x axis line style={color={rgb,1:red,0.0;green,0.0;blue,0.0}, draw opacity={1.0}, line width={1}, solid}, scaled y ticks={false}, ylabel={Bethe Free Energy}, y tick style={color={rgb,1:red,0.0;green,0.0;blue,0.0}, opacity={1.0}}, y tick label style={color={rgb,1:red,0.0;green,0.0;blue,0.0}, opacity={1.0}, rotate={0}}, ylabel style={at={(ticklabel cs:0.5)}, anchor=near ticklabel, at={{(ticklabel cs:0.5)}}, anchor={near ticklabel}, font={{\fontsize{16 pt}{20.8 pt}\selectfont}}, color={rgb,1:red,0.0;green,0.0;blue,0.0}, draw opacity={1.0}, rotate={0.0}}, ymajorgrids={true}, ymin={241.8453564721857}, ymax={281.36375799586244}, yticklabels={{$250$,$260$,$270$,$280$}}, ytick={{250.0,260.0,270.0,280.0}}, ytick align={inside}, yticklabel style={font={{\fontsize{14 pt}{18.2 pt}\selectfont}}, color={rgb,1:red,0.0;green,0.0;blue,0.0}, draw opacity={1.0}, rotate={0.0}}, y grid style={color={rgb,1:red,0.0;green,0.0;blue,0.0}, draw opacity={0.1}, line width={0.5}, solid}, axis y line*={left}, y axis line style={color={rgb,1:red,0.0;green,0.0;blue,0.0}, draw opacity={1.0}, line width={1}, solid}, colorbar={false}]
    \addplot[color={rgb,1:red,0.0;green,0.6056;blue,0.9787}, name path={bb194839-9c0c-48b7-8c57-c747b5d197db}, draw opacity={1.0}, line width={1}, solid]
        table[row sep={\\}]
        {
            \\
            1.0  280.24531266972065  \\
            2.0  243.0990439174052  \\
            3.0  242.9638017983275  \\
            4.0  242.96402495101802  \\
            5.0  242.96402505824835  \\
        }
        ;
    \addlegendentry {Bethe Free Energy}
\end{axis}
\end{tikzpicture}

        \includegraphics{contents/05-experiments/plots/nlds/03-pendulum_example_inference_free_energy.pdf}
    }
    \caption{Bethe Free Energy convergence results.
      The x-axis represents the index of VMP iteration.
      The y-axis represents the Bethe Free Energy value at a specific VMP iteration.
    }
    \label{fig:sim:pendulum_example_inference_free_energy}
  \end{subfigure}
  \caption{
    Simulated evolution of the double pendulum system state $s_t = (\theta_1, \theta_2, \dot{\theta}_1, \dot{\theta}_2)_t$ using the Runge-Kutta (RK4) method with a starting point of $s_1 = (1.2, 0.2, 0.0, 0.0)$ in discrete time steps.
    The time difference between measurements is set to be $0.01$ seconds.
    The masses $m_1$ and $m_2$ are set to be $14.715$ and $4.905$ respectively.
    The lengths $l_1$ and $l_2$ are equal and set to be $1.0$.
    The measurements $y$ only include the second component of the state vector.
    Other components of the state vector are not observed.
    The state transition noise signal $v$ is distributed according to the multivariate Normal
    distribution $\mathcal{N}(0, \Sigma)$, where the covariance matrix $\Sigma$ is a diagonal
    matrix with $10^{-6}$ values on the diagonal.
    The measurement noise $w$ is distributed according to the Normal distribution
    $\mathcal{N}(0, \Omega)$, where the variance $w$ is set to be $0.3$.
    The figure shows the first $250$ time steps of the simulation.
    The shaded area shows three standard deviations of the inferred posteriors from
    Listing~\ref{lst:sim:double_pendulum_inference}.
  }
  \label{fig:sim:pendulum_example_inference_states}
\end{figure}

\subsection{Scalability and performance characteristics}

\begin{figure}
  \centering
  \resizebox{\textwidth}{!}{
    % % Recommended preamble:
% \usetikzlibrary{arrows.meta}
% \usetikzlibrary{backgrounds}
% \usepgfplotslibrary{patchplots}
% \usepgfplotslibrary{fillbetween}
% \pgfplotsset{%
%     layers/standard/.define layer set={%
%         background,axis background,axis grid,axis ticks,axis lines,axis tick labels,pre main,main,axis descriptions,axis foreground%
%     }{
%         grid style={/pgfplots/on layer=axis grid},%
%         tick style={/pgfplots/on layer=axis ticks},%
%         axis line style={/pgfplots/on layer=axis lines},%
%         label style={/pgfplots/on layer=axis descriptions},%
%         legend style={/pgfplots/on layer=axis descriptions},%
%         title style={/pgfplots/on layer=axis descriptions},%
%         colorbar style={/pgfplots/on layer=axis descriptions},%
%         ticklabel style={/pgfplots/on layer=axis tick labels},%
%         axis background@ style={/pgfplots/on layer=axis background},%
%         3d box foreground style={/pgfplots/on layer=axis foreground},%
%     },
% }

\begin{tikzpicture}[/tikz/background rectangle/.style={fill={rgb,1:red,1.0;green,1.0;blue,1.0}, fill opacity={1.0}, draw opacity={1.0}}, show background rectangle]
\begin{axis}[point meta max={nan}, point meta min={nan}, legend cell align={left}, legend columns={1}, title={}, title style={at={{(0.5,1)}}, anchor={south}, font={{\fontsize{18 pt}{23.400000000000002 pt}\selectfont}}, color={rgb,1:red,0.0;green,0.0;blue,0.0}, draw opacity={1.0}, rotate={0.0}, align={center}}, legend style={color={rgb,1:red,0.0;green,0.0;blue,0.0}, draw opacity={1.0}, line width={1}, solid, fill={rgb,1:red,1.0;green,1.0;blue,1.0}, fill opacity={1.0}, text opacity={1.0}, font={{\fontsize{14 pt}{18.2 pt}\selectfont}}, text={rgb,1:red,0.0;green,0.0;blue,0.0}, cells={anchor={west}}, at={(1.02, 0.5)}, anchor={west}}, axis background/.style={fill={rgb,1:red,1.0;green,1.0;blue,1.0}, opacity={1.0}}, anchor={north west}, xshift={1.0mm}, yshift={-1.0mm}, width={196.2mm}, height={99.6mm}, scaled x ticks={false}, xlabel={Number of observation (log-scale)}, x tick style={color={rgb,1:red,0.0;green,0.0;blue,0.0}, opacity={1.0}}, x tick label style={color={rgb,1:red,0.0;green,0.0;blue,0.0}, opacity={1.0}, rotate={0}}, xlabel style={at={(ticklabel cs:0.5)}, anchor=near ticklabel, at={{(ticklabel cs:0.5)}}, anchor={near ticklabel}, font={{\fontsize{16 pt}{20.8 pt}\selectfont}}, color={rgb,1:red,0.0;green,0.0;blue,0.0}, draw opacity={1.0}, rotate={0.0}}, xmode={log}, log basis x={10}, xmajorgrids={true}, xmin={7.961027060291775}, xmax={25122.386657566538}, xticklabels={{10,20,30,50,100,200,300,500,1000,2000,5000,10000,20000}}, xtick={{10,20,30,50,100,200,300,500,1000,2000,5000,10000,20000}}, xtick align={inside}, xticklabel style={font={{\fontsize{14 pt}{18.2 pt}\selectfont}}, color={rgb,1:red,0.0;green,0.0;blue,0.0}, draw opacity={1.0}, rotate={0.0}}, x grid style={color={rgb,1:red,0.0;green,0.0;blue,0.0}, draw opacity={0.1}, line width={0.5}, solid}, axis x line*={left}, x axis line style={color={rgb,1:red,0.0;green,0.0;blue,0.0}, draw opacity={1.0}, line width={1}, solid}, scaled y ticks={false}, ylabel={Time (log-scale)}, y tick style={color={rgb,1:red,0.0;green,0.0;blue,0.0}, opacity={1.0}}, y tick label style={color={rgb,1:red,0.0;green,0.0;blue,0.0}, opacity={1.0}, rotate={0}}, ylabel style={at={(ticklabel cs:0.5)}, anchor=near ticklabel, at={{(ticklabel cs:0.5)}}, anchor={near ticklabel}, font={{\fontsize{16 pt}{20.8 pt}\selectfont}}, color={rgb,1:red,0.0;green,0.0;blue,0.0}, draw opacity={1.0}, rotate={0.0}}, ymode={log}, log basis y={10}, ymajorgrids={true}, ymin={716377.854156554}, ymax={3.4982602494616534e11}, yticklabels={{0ms,0ms,0ms,0ms,0ms,4ms,58ms,930ms,14982ms,241463ms}}, ytick={{3.29,53.02624780435095,854.6452754432935,13774.660231147058,222011.71636399475,3.5782517590840347e6,5.767211686429936e7,9.2952460797806e8,1.498152042640937e10,2.4146316553699988e11}}, ytick align={inside}, yticklabel style={font={{\fontsize{14 pt}{18.2 pt}\selectfont}}, color={rgb,1:red,0.0;green,0.0;blue,0.0}, draw opacity={1.0}, rotate={0.0}}, y grid style={color={rgb,1:red,0.0;green,0.0;blue,0.0}, draw opacity={0.1}, line width={0.5}, solid}, axis y line*={left}, y axis line style={color={rgb,1:red,0.0;green,0.0;blue,0.0}, draw opacity={1.0}, line width={1}, solid}, colorbar={false}]
    \addplot[color={rgb,1:red,0.0;green,0.6056;blue,0.9787}, name path={828f29ab-e0f0-4ace-aa96-6ee9758a56c4}, draw opacity={1.0}, line width={1}, solid, mark={*}, mark size={3.0 pt}, mark repeat={1}, mark options={color={rgb,1:red,0.0;green,0.0;blue,0.0}, draw opacity={1.0}, fill={rgb,1:red,0.0;green,0.6056;blue,0.9787}, fill opacity={1.0}, line width={0.75}, rotate={0}, solid}]
        table[row sep={\\}]
        {
            \\
            10.0  1.037871e6  \\
            20.0  1.904484e6  \\
            30.0  2.994267e6  \\
            50.0  5.79057e6  \\
            100.0  1.278087e7  \\
            200.0  2.9771094e7  \\
            300.0  4.6191392e7  \\
            500.0  7.8187381e7  \\
            1000.0  1.4206314e8  \\
            2000.0  3.15449035e8  \\
            5000.0  7.91687319e8  \\
            10000.0  1.687034053e9  \\
            20000.0  3.367013979e9  \\
        }
        ;
    \addlegendentry {Reactive MP (3 iterations)}
    \addplot[color={rgb,1:red,0.8889;green,0.4356;blue,0.2781}, name path={5e8657fc-8fcc-4186-931f-b95d08304e5f}, draw opacity={1.0}, line width={1}, solid, mark={triangle*}, mark size={3.0 pt}, mark repeat={1}, mark options={color={rgb,1:red,0.0;green,0.0;blue,0.0}, draw opacity={1.0}, fill={rgb,1:red,0.8889;green,0.4356;blue,0.2781}, fill opacity={1.0}, line width={0.75}, rotate={0}, solid}]
        table[row sep={\\}]
        {
            \\
            10.0  1.148334e6  \\
            20.0  2.707808e6  \\
            30.0  3.625415e6  \\
            50.0  5.976353e6  \\
            100.0  1.3604546e7  \\
        }
        ;
    \addlegendentry {Scheduled MP (inference, 3 iterations)}
    \addplot[color={rgb,1:red,0.2422;green,0.6433;blue,0.3044}, name path={c38ac8a0-d68d-491c-9ec8-9856b31b7b19}, draw opacity={1.0}, line width={1}, solid, mark={square*}, mark size={3.0 pt}, mark repeat={1}, mark options={color={rgb,1:red,0.0;green,0.0;blue,0.0}, draw opacity={1.0}, fill={rgb,1:red,0.2422;green,0.6433;blue,0.3044}, fill opacity={1.0}, line width={0.75}, rotate={0}, solid}]
        table[row sep={\\}]
        {
            \\
            10.0  5.100003094e9  \\
            20.0  1.0929191537e10  \\
            30.0  1.5898139967e10  \\
            50.0  2.7344669115e10  \\
            100.0  5.1074718032e10  \\
        }
        ;
    \addlegendentry {Scheduled MP (compilation)}
    \addplot[color={rgb,1:red,0.7644;green,0.4441;blue,0.8243}, name path={56409a79-a18f-4d79-a680-ce67ec0e3e03}, draw opacity={1.0}, line width={1}, solid, mark={triangle*}, mark size={3.0 pt}, mark repeat={1}, mark options={color={rgb,1:red,0.0;green,0.0;blue,0.0}, draw opacity={1.0}, fill={rgb,1:red,0.7644;green,0.4441;blue,0.8243}, fill opacity={1.0}, line width={0.75}, rotate={180}, solid}]
        table[row sep={\\}]
        {
            \\
            10.0  1.1237186986e10  \\
            20.0  2.1346839126e10  \\
            30.0  3.9902297427e10  \\
            50.0  7.4127069701e10  \\
            100.0  2.41463165537e11  \\
        }
        ;
    \addlegendentry {NUTS (100)}
\end{axis}
\end{tikzpicture}

    \includegraphics{contents/05-experiments/plots/nlds/03-benchmark_comparison.pdf}
  }
  \caption{A comparison of run-time duration in milliseconds for automated Bayesian inference in the \ac{nlds} across different methods: reactive message passing (RxInfer), scheduled message passing (ForneyLab) and \ac{nuts} (Turing).
    The values in the figure show the minimum possible duration across multiple runs.
    The RxInfer timings include graph creation time.
    The ForneyLab pipeline consists of model compilation, followed by actual inference execution.
    Turing uses \ac{nuts} sampling with $100$ and $200$ samples respectively.
    We provide benchmark results for over $300$ observations exclusively for the RxInfer
    framework.
  }
  \label{fig:sim:nlds_performance_comparison}
\end{figure}

\begin{table}
  \centering
  \begin{tabular}{ |l||r|r|r| }
    \hline
                  & \multicolumn{3}{|c|}{Number of observations}                \\
    \hline
                  & 50                                           & 100  & 200   \\
    \hline
    VMP (5 iters) & 3.41                                         & 3.22 & 3.35  \\
    \hline
    NUTS (50)     & 3.68                                         & 3.11 & 2.16  \\
    NUTS (100)    & 3.83                                         & 3.58 & 2.135 \\
    \hline
  \end{tabular}
  \caption{
    Comparison of posterior result accuracy in terms of the metric~\eqref{eq:sim:average_mse} in the \ac{nlds} among different methods: message passing (RxInfer and ForneyLab) and \ac{nuts} (Turing).
    Lower values indicate better performance.
    Both RxInfer and ForneyLab employ \ac{cbfe} minimization through \ac{vmp} on the full graph.
    The number of \ac{vmp} iterations is set to 5.
    Turing utilizes \ac{nuts} sampling with $50$ and $100$ samples, respectively.
  }
  \label{table:sim:nlds_accuracy_comparison}
\end{table}

\begin{figure}
  \centering
  \begin{subfigure}[t]{\textwidth}
    \centering
    \resizebox{\textwidth}{!}{
        % % Recommended preamble:
% \usetikzlibrary{arrows.meta}
% \usetikzlibrary{backgrounds}
% \usepgfplotslibrary{patchplots}
% \usepgfplotslibrary{fillbetween}
% \pgfplotsset{%
%     layers/standard/.define layer set={%
%         background,axis background,axis grid,axis ticks,axis lines,axis tick labels,pre main,main,axis descriptions,axis foreground%
%     }{
%         grid style={/pgfplots/on layer=axis grid},%
%         tick style={/pgfplots/on layer=axis ticks},%
%         axis line style={/pgfplots/on layer=axis lines},%
%         label style={/pgfplots/on layer=axis descriptions},%
%         legend style={/pgfplots/on layer=axis descriptions},%
%         title style={/pgfplots/on layer=axis descriptions},%
%         colorbar style={/pgfplots/on layer=axis descriptions},%
%         ticklabel style={/pgfplots/on layer=axis tick labels},%
%         axis background@ style={/pgfplots/on layer=axis background},%
%         3d box foreground style={/pgfplots/on layer=axis foreground},%
%     },
% }

\begin{tikzpicture}[/tikz/background rectangle/.style={fill={rgb,1:red,1.0;green,1.0;blue,1.0}, fill opacity={1.0}, draw opacity={1.0}}, show background rectangle]
\begin{axis}[point meta max={nan}, point meta min={nan}, legend cell align={left}, legend columns={2}, title={}, title style={at={{(0.5,1)}}, anchor={south}, font={{\fontsize{18 pt}{23.400000000000002 pt}\selectfont}}, color={rgb,1:red,0.0;green,0.0;blue,0.0}, draw opacity={1.0}, rotate={0.0}, align={center}}, legend style={color={rgb,1:red,0.0;green,0.0;blue,0.0}, draw opacity={1.0}, line width={1}, solid, fill={rgb,1:red,1.0;green,1.0;blue,1.0}, fill opacity={1.0}, text opacity={1.0}, font={{\fontsize{14 pt}{18.2 pt}\selectfont}}, text={rgb,1:red,0.0;green,0.0;blue,0.0}, cells={anchor={west}}, at={(0.5, 1.02)}, anchor={south}}, axis background/.style={fill={rgb,1:red,1.0;green,1.0;blue,1.0}, opacity={1.0}}, anchor={north west}, xshift={1.0mm}, yshift={-1.0mm}, width={201.2mm}, height={74.2mm}, scaled x ticks={false}, xlabel={Number of observations in dataset (log10-scale)}, x tick style={color={rgb,1:red,0.0;green,0.0;blue,0.0}, opacity={1.0}}, x tick label style={color={rgb,1:red,0.0;green,0.0;blue,0.0}, opacity={1.0}, rotate={0}}, xlabel style={at={(ticklabel cs:0.5)}, anchor=near ticklabel, at={{(ticklabel cs:0.5)}}, anchor={near ticklabel}, font={{\fontsize{16 pt}{20.8 pt}\selectfont}}, color={rgb,1:red,0.0;green,0.0;blue,0.0}, draw opacity={1.0}, rotate={0.0}}, xmode={log}, log basis x={10}, xmajorgrids={true}, xmin={7.585775750291836}, xmax={131825.67385564075}, xticklabels={{$10^1$,$10^2$,$10^3$,$10^4$,$10^5$}}, xtick={{10,100,1000,10000,100000}}, xtick align={inside}, xticklabel style={font={{\fontsize{14 pt}{18.2 pt}\selectfont}}, color={rgb,1:red,0.0;green,0.0;blue,0.0}, draw opacity={1.0}, rotate={0.0}}, x grid style={color={rgb,1:red,0.0;green,0.0;blue,0.0}, draw opacity={0.1}, line width={0.5}, solid}, axis x line*={left}, x axis line style={color={rgb,1:red,0.0;green,0.0;blue,0.0}, draw opacity={1.0}, line width={1}, solid}, scaled y ticks={false}, ylabel={Time (in ms, log10-scale)}, y tick style={color={rgb,1:red,0.0;green,0.0;blue,0.0}, opacity={1.0}}, y tick label style={color={rgb,1:red,0.0;green,0.0;blue,0.0}, opacity={1.0}, rotate={0}}, ylabel style={at={(ticklabel cs:0.5)}, anchor=near ticklabel, at={{(ticklabel cs:0.5)}}, anchor={near ticklabel}, font={{\fontsize{16 pt}{20.8 pt}\selectfont}}, color={rgb,1:red,0.0;green,0.0;blue,0.0}, draw opacity={1.0}, rotate={0.0}}, ymode={log}, log basis y={10}, ymajorgrids={true}, ymin={0.1}, ymax={100000.0}, yticklabels={{$10^{-1}$,$10^{0}$,$10^{1}$,$10^{2}$,$10^{3}$,$10^{4}$,$10^{5}$}}, ytick={{0.1,1.0,10.0,100.0,1000.0,10000.0,100000.0}}, ytick align={inside}, yticklabel style={font={{\fontsize{14 pt}{18.2 pt}\selectfont}}, color={rgb,1:red,0.0;green,0.0;blue,0.0}, draw opacity={1.0}, rotate={0.0}}, y grid style={color={rgb,1:red,0.0;green,0.0;blue,0.0}, draw opacity={0.1}, line width={0.5}, solid}, axis y line*={left}, y axis line style={color={rgb,1:red,0.0;green,0.0;blue,0.0}, draw opacity={1.0}, line width={1}, solid}, colorbar={false}]
    \addplot[color={rgb,1:red,0.3059;green,0.4745;blue,0.6549}, name path={fb9c8855-f575-4c89-a1c0-792d4f74d4a6}, draw opacity={1.0}, line width={1}, solid, mark={diamond*}, mark size={3.0 pt}, mark repeat={1}, mark options={color={rgb,1:red,0.0;green,0.0;blue,0.0}, draw opacity={1.0}, fill={rgb,1:red,0.3059;green,0.4745;blue,0.6549}, fill opacity={1.0}, line width={0.75}, rotate={0}, solid}]
        table[row sep={\\}]
        {
            \\
            10.0  0.5771  \\
            20.0  1.1087  \\
            30.0  1.6199  \\
            50.0  2.6872  \\
            100.0  5.5775  \\
            200.0  11.0266  \\
            500.0  30.8133  \\
            1000.0  67.9527  \\
            2000.0  150.319  \\
            5000.0  416.4492  \\
            10000.0  810.4124  \\
            30000.0  2621.8948  \\
            50000.0  4437.6781  \\
            100000.0  9348.438  \\
        }
        ;
    \addlegendentry {3 iterations}
    \addplot[color={rgb,1:red,0.949;green,0.5569;blue,0.1686}, name path={95def110-e976-41f5-8db0-834df0953a88}, draw opacity={1.0}, line width={1}, dashed, mark={*}, mark size={3.0 pt}, mark repeat={1}, mark options={color={rgb,1:red,0.0;green,0.0;blue,0.0}, draw opacity={1.0}, fill={rgb,1:red,0.949;green,0.5569;blue,0.1686}, fill opacity={1.0}, line width={0.75}, rotate={0}, solid}]
        table[row sep={\\}]
        {
            \\
            10.0  0.7282  \\
            20.0  1.403  \\
            30.0  2.1627  \\
            50.0  3.4508  \\
            100.0  6.9593  \\
            200.0  14.0073  \\
            500.0  39.8393  \\
            1000.0  83.5032  \\
            2000.0  198.2632  \\
            5000.0  542.9426  \\
            10000.0  1087.8617  \\
            30000.0  3430.2486  \\
            50000.0  5888.246  \\
            100000.0  11942.3057  \\
        }
        ;
    \addlegendentry {5 iterations}
    \addplot[color={rgb,1:red,0.8824;green,0.3412;blue,0.349}, name path={d26ce8f8-8241-45e8-861e-efdaa51b5819}, draw opacity={1.0}, line width={1}, dotted, mark={square*}, mark size={3.0 pt}, mark repeat={1}, mark options={color={rgb,1:red,0.0;green,0.0;blue,0.0}, draw opacity={1.0}, fill={rgb,1:red,0.8824;green,0.3412;blue,0.349}, fill opacity={1.0}, line width={0.75}, rotate={0}, solid}]
        table[row sep={\\}]
        {
            \\
            10.0  1.0527  \\
            20.0  2.1517  \\
            30.0  3.2629  \\
            50.0  5.4348  \\
            100.0  10.6038  \\
            200.0  21.7778  \\
            500.0  59.2229  \\
            1000.0  132.6098  \\
            2000.0  312.2849  \\
            5000.0  840.3824  \\
            10000.0  1712.7523  \\
            30000.0  5596.2811  \\
            50000.0  9757.1388  \\
            100000.0  21210.9018  \\
        }
        ;
    \addlegendentry {10 iterations}
    \addplot[color={rgb,1:red,0.4627;green,0.7176;blue,0.698}, name path={0d487c65-743d-4076-965d-211c857347cf}, draw opacity={1.0}, line width={1}, dashdotted, mark={triangle*}, mark size={3.0 pt}, mark repeat={1}, mark options={color={rgb,1:red,0.0;green,0.0;blue,0.0}, draw opacity={1.0}, fill={rgb,1:red,0.4627;green,0.7176;blue,0.698}, fill opacity={1.0}, line width={0.75}, rotate={0}, solid}]
        table[row sep={\\}]
        {
            \\
            10.0  1.7039  \\
            20.0  3.5387  \\
            30.0  5.2579  \\
            50.0  8.7437  \\
            100.0  17.4783  \\
            200.0  35.363  \\
            500.0  96.7902  \\
            1000.0  219.589  \\
            2000.0  510.2803  \\
            5000.0  1466.1236  \\
            10000.0  3110.749  \\
            30000.0  9680.737  \\
            50000.0  16897.1484  \\
            100000.0  34550.4566  \\
        }
        ;
    \addlegendentry {20 iterations}
\end{axis}
\end{tikzpicture}

        \includegraphics{contents/05-experiments/plots/nlds/03-rxinfer_double_pendulum_scalability_size.pdf}
    }
    \caption{
      Scalability benchmark for different number of performed \ac{vmp} iterations with respect to number of observations in dataset.
    }
    \label{fig:sim:pendulum_example_scalability_size}
  \end{subfigure}
  \hfill
  \begin{subfigure}[t]{\textwidth}
    \centering
    \resizebox{\textwidth}{!}{
        % % Recommended preamble:
% \usetikzlibrary{arrows.meta}
% \usetikzlibrary{backgrounds}
% \usepgfplotslibrary{patchplots}
% \usepgfplotslibrary{fillbetween}
% \pgfplotsset{%
%     layers/standard/.define layer set={%
%         background,axis background,axis grid,axis ticks,axis lines,axis tick labels,pre main,main,axis descriptions,axis foreground%
%     }{
%         grid style={/pgfplots/on layer=axis grid},%
%         tick style={/pgfplots/on layer=axis ticks},%
%         axis line style={/pgfplots/on layer=axis lines},%
%         label style={/pgfplots/on layer=axis descriptions},%
%         legend style={/pgfplots/on layer=axis descriptions},%
%         title style={/pgfplots/on layer=axis descriptions},%
%         colorbar style={/pgfplots/on layer=axis descriptions},%
%         ticklabel style={/pgfplots/on layer=axis tick labels},%
%         axis background@ style={/pgfplots/on layer=axis background},%
%         3d box foreground style={/pgfplots/on layer=axis foreground},%
%     },
% }

\begin{tikzpicture}[/tikz/background rectangle/.style={fill={rgb,1:red,1.0;green,1.0;blue,1.0}, fill opacity={1.0}, draw opacity={1.0}}, show background rectangle]
\begin{axis}[point meta max={nan}, point meta min={nan}, legend cell align={left}, legend columns={2}, title={}, title style={at={{(0.5,1)}}, anchor={south}, font={{\fontsize{18 pt}{23.400000000000002 pt}\selectfont}}, color={rgb,1:red,0.0;green,0.0;blue,0.0}, draw opacity={1.0}, rotate={0.0}, align={center}}, legend style={color={rgb,1:red,0.0;green,0.0;blue,0.0}, draw opacity={1.0}, line width={1}, solid, fill={rgb,1:red,1.0;green,1.0;blue,1.0}, fill opacity={1.0}, text opacity={1.0}, font={{\fontsize{14 pt}{18.2 pt}\selectfont}}, text={rgb,1:red,0.0;green,0.0;blue,0.0}, cells={anchor={west}}, at={(0.5, 1.02)}, anchor={south}}, axis background/.style={fill={rgb,1:red,1.0;green,1.0;blue,1.0}, opacity={1.0}}, anchor={north west}, xshift={1.0mm}, yshift={-1.0mm}, width={201.2mm}, height={74.2mm}, scaled x ticks={false}, xlabel={Number of iterations}, x tick style={color={rgb,1:red,0.0;green,0.0;blue,0.0}, opacity={1.0}}, x tick label style={color={rgb,1:red,0.0;green,0.0;blue,0.0}, opacity={1.0}, rotate={0}}, xlabel style={at={(ticklabel cs:0.5)}, anchor=near ticklabel, at={{(ticklabel cs:0.5)}}, anchor={near ticklabel}, font={{\fontsize{16 pt}{20.8 pt}\selectfont}}, color={rgb,1:red,0.0;green,0.0;blue,0.0}, draw opacity={1.0}, rotate={0.0}}, xmajorgrids={true}, xmin={2.49}, xmax={20.509999999999998}, xticklabels={{3,5,10,20}}, xtick={{3,5,10,20}}, xtick align={inside}, xticklabel style={font={{\fontsize{14 pt}{18.2 pt}\selectfont}}, color={rgb,1:red,0.0;green,0.0;blue,0.0}, draw opacity={1.0}, rotate={0.0}}, x grid style={color={rgb,1:red,0.0;green,0.0;blue,0.0}, draw opacity={0.1}, line width={0.5}, solid}, axis x line*={left}, x axis line style={color={rgb,1:red,0.0;green,0.0;blue,0.0}, draw opacity={1.0}, line width={1}, solid}, scaled y ticks={false}, ylabel={Time (in ms, log10-scale)}, y tick style={color={rgb,1:red,0.0;green,0.0;blue,0.0}, opacity={1.0}}, y tick label style={color={rgb,1:red,0.0;green,0.0;blue,0.0}, opacity={1.0}, rotate={0}}, ylabel style={at={(ticklabel cs:0.5)}, anchor=near ticklabel, at={{(ticklabel cs:0.5)}}, anchor={near ticklabel}, font={{\fontsize{16 pt}{20.8 pt}\selectfont}}, color={rgb,1:red,0.0;green,0.0;blue,0.0}, draw opacity={1.0}, rotate={0.0}}, ymode={log}, log basis y={10}, ymajorgrids={true}, ymin={0.1}, ymax={100000.0}, yticklabels={{$10^{-1}$,$10^{0}$,$10^{1}$,$10^{2}$,$10^{3}$,$10^{4}$,$10^{5}$}}, ytick={{0.1,1.0,10.0,100.0,1000.0,10000.0,100000.0}}, ytick align={inside}, yticklabel style={font={{\fontsize{14 pt}{18.2 pt}\selectfont}}, color={rgb,1:red,0.0;green,0.0;blue,0.0}, draw opacity={1.0}, rotate={0.0}}, y grid style={color={rgb,1:red,0.0;green,0.0;blue,0.0}, draw opacity={0.1}, line width={0.5}, solid}, axis y line*={left}, y axis line style={color={rgb,1:red,0.0;green,0.0;blue,0.0}, draw opacity={1.0}, line width={1}, solid}, colorbar={false}]
    \addplot[color={rgb,1:red,0.3059;green,0.4745;blue,0.6549}, name path={b3e556c8-a279-41de-b094-3889ebbeb166}, draw opacity={1.0}, line width={1}, solid, mark={triangle*}, mark size={3.0 pt}, mark repeat={1}, mark options={color={rgb,1:red,0.0;green,0.0;blue,0.0}, draw opacity={1.0}, fill={rgb,1:red,0.3059;green,0.4745;blue,0.6549}, fill opacity={1.0}, line width={0.75}, rotate={0}, solid}]
        table[row sep={\\}]
        {
            \\
            3.0  0.5771  \\
            5.0  0.7282  \\
            10.0  1.0527  \\
            20.0  1.7039  \\
        }
        ;
    \addlegendentry {10 observations}
    \addplot[color={rgb,1:red,0.949;green,0.5569;blue,0.1686}, name path={b0da77fa-2d92-4900-8b63-df62fd8cee3f}, draw opacity={1.0}, line width={1}, dashed, mark={triangle*}, mark size={3.0 pt}, mark repeat={1}, mark options={color={rgb,1:red,0.0;green,0.0;blue,0.0}, draw opacity={1.0}, fill={rgb,1:red,0.949;green,0.5569;blue,0.1686}, fill opacity={1.0}, line width={0.75}, rotate={180}, solid}]
        table[row sep={\\}]
        {
            \\
            3.0  67.9527  \\
            5.0  83.5032  \\
            10.0  132.6098  \\
            20.0  219.589  \\
        }
        ;
    \addlegendentry {1000 observations}
    \addplot[color={rgb,1:red,0.8824;green,0.3412;blue,0.349}, name path={4662b2e2-a907-48b8-b991-aedcfcb62dd7}, draw opacity={1.0}, line width={1}, dotted, mark={triangle*}, mark size={3.0 pt}, mark repeat={1}, mark options={color={rgb,1:red,0.0;green,0.0;blue,0.0}, draw opacity={1.0}, fill={rgb,1:red,0.8824;green,0.3412;blue,0.349}, fill opacity={1.0}, line width={0.75}, rotate={270}, solid}]
        table[row sep={\\}]
        {
            \\
            3.0  810.4124  \\
            5.0  1087.8617  \\
            10.0  1712.7523  \\
            20.0  3110.749  \\
        }
        ;
    \addlegendentry {10000 observations}
    \addplot[color={rgb,1:red,0.4627;green,0.7176;blue,0.698}, name path={e1f1391d-129e-427b-bc3b-ecce6ba66685}, draw opacity={1.0}, line width={1}, dashdotted, mark={triangle*}, mark size={3.0 pt}, mark repeat={1}, mark options={color={rgb,1:red,0.0;green,0.0;blue,0.0}, draw opacity={1.0}, fill={rgb,1:red,0.4627;green,0.7176;blue,0.698}, fill opacity={1.0}, line width={0.75}, rotate={90}, solid}]
        table[row sep={\\}]
        {
            \\
            3.0  9348.438  \\
            5.0  11942.3057  \\
            10.0  21210.9018  \\
            20.0  34550.4566  \\
        }
        ;
    \addlegendentry {100000 observations}
\end{axis}
\end{tikzpicture}

        \includegraphics{contents/05-experiments/plots/nlds/03-rxinfer_double_pendulum_scalability_nits.pdf}
    }
    \caption{Scalability benchmark for different number of observations with respect to number of performed \ac{vmp} iterations.}
    \label{fig:sim:pendulum_example_scalability_nits}
  \end{subfigure}
  \caption{
    Benchmark results for the inference in the \ac{nlds} system using the RxInfer framework.
    The results highlight the excellent scalability of the RxInfer framework for varying numbers
    of observations in the dataset and different numbers of performed \ac{vmp} iterations.
    The values in the table represent the minimum durations obtained across multiple runs,
    including graph creation time.
  }
  \label{fig:sim:pendulum_example_scalability}
\end{figure}

This section presents benchmark results for the inference task in the \ac{nlds} model for the double
pendulum system.
The main results are shown in Figure~\ref{fig:sim:nlds_performance_comparison}, which compares
the performance of different inference methods.
The accuracy of the methods is evaluated using the metric~\eqref{eq:sim:average_mse}, and the
results are summarized in Table~\ref{table:sim:nlds_accuracy_comparison}.

Similarly to the previous example, the RxInfer framework demonstrates excellent performance and
scalability compared to alternative packages, both in model creation and inference execution.
However, in contrast to the previous example, we observe that the absolute inference execution
timings between scheduled message passing implemented in ForneyLab and reactive message
passing implemented in RxInfer are similar.
This is because the inference task in this example is more complex, and the additional
run-time costs for managing reactive streams are less noticeable.
However, it is worth noting that ForneyLab's model compilation times are high, and any change
in the model specification requires a full model recompilation.
However, while the reactive message passing architecture in RxInfer does not provide
significant performance advantages, it also does not incur considerable overhead in generating
the fixed global message passing schedule, which is required in ForneyLab.

We specifically focus on presenting benchmark results for a large number of observations using
the RxInfer framework, as executing the inference with the other compared methods would
require significantly longer computation times, which are not suitable for real-time
applications.
Despite the inherently challenging nature of inference tasks in \ac{nlds} models, the RxInfer
framework demonstrates impressive scalability and performance (\hyperlink{experiments:scalability}{\emph{Scalability}}).
In fact, RxInfer is capable of executing the inference task on a full model graph for
$100\,000$ observations in less than $25$ seconds, resulting in an average of less than $0.25$
milliseconds per observation, which is well suited for real-time applications in many
fields (\hyperlink{experiments:efficiency}{\emph{Run-time efficiency and speed}}).

Interestingly, the accuracy results, which are given in Table~\ref{table:sim:nlds_accuracy_comparison}, indicate
that \ac{nuts} performs better in estimating the posterior distributions compared to message
passing-based inference in terms of~\eqref{eq:sim:average_mse} (\hyperlink{experiments:accuracy}{\emph{Posterior accuracy}}).
This could be because both RxInfer and ForneyLab rely on the first-order Taylor approximation
for the state transition nonlinearity, which is a fast but not very accurate method of approximation.
On the other hand, the RxInfer framework provides a user-friendly approach to easily change
and modify approximation methods for different nodes, also allowing for flexibility in
selecting specific approximation methods for different parts of the model.
However, it is worth noting that the accuracy of \ac{nuts} does not improve with an increased
number of samples, suggesting challenges in performing inference for nonlinear models.
Additionally, \ac{nuts} exhibits poor scalability and, similar to the previous example, is not
suitable for real-time applications.

In the upcoming section, we will delve into an example of continuous inference tasks in a nonlinear hierarchical dynamical system.
The example involves a potentially infinite dynamic stream of data, further highlighting the
capabilities of RxInfer to handle continuous and evolving scenarios.
