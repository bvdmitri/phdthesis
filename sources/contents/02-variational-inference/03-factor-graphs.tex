\section{Factor graphs}\label{chapter-02:section:factor-graphs}

A \acf{fg} is an undirected bipartite graph with factor nodes $a \in \mathcal{V}$,
variable nodes $i \in \mathcal{E}$, and edges $e \in \mathcal{K}$.
It serves as a useful representation for a positive factorized function
\begin{equation}
\label{eq:ffg:factorization_assumption}
    f(\bm{s}) = \prod_{a \in \mathcal{V}}
    f_a(\bm{s}_a),
  \end{equation} 
where $\bm{s}_a$ collects the argument variables of factor $f_a$.
This form is equivalent to the factorized model $p(\bm{y}, \bm{s})$ defined in~\eqref{eq:bfe:factorized_p}.
However, in this section, to avoid clutter, we do not distinguish between observed states
$\bm{y}$ and hidden states $\bm{s}$ and simply use $\bm{s}$ to denote all variables.
In the factor graphs, each factor node $a \in \mathcal{V}$ represents the corresponding factor
$f_a$, and each variable node $i \in \mathcal{E}$ represents a single variable $s_i$.
A factor node $a$ is connected to a variable node $i$ if and only if $s_i \in \bm{s}_a$.
As an example, consider the factorized function \begin{equation}
  \label{eq:ffg:example1}
  f(s_1, s_2, s_3) = f_1(s_1, s_2, s_3)f_2(s_2, s_3)f_3(s_3)
\end{equation} for which the
corresponding \ac{fg} is shown on Figure~\ref{fig:ffg:example1}.
The \ac{fg} representation comprises three factor nodes and three variable nodes,
connected with edges according to the argument lists.

\begin{figure}
  \centering
  \resizebox{.75\textwidth}{!}{
\begin{tikzpicture}[]
  %Nodes
  \node[roundbox](x){$s_1$};
  \node[box](p1)[right=of x]{$f_1$};
  \node[roundbox](y)[below=of p1, yshift=14pt]{$s_2$};
  \node[box](p2)[right=of y]{$f_2$};
  \node[roundbox](z)[right=of p1]{$s_3$};
  \node[box](p3)[right=of z]{$f_3$};

  %Lines
  \path[line] (x) edge[-] (p1);
  \path[line] (p1) edge[-] (z);
  \path[line] (p1) edge[-] (y);
  \path[line] (z) edge[-] (p3);
  \path[line] (z) edge[-] (p2);
  \path[line] (y) edge[-] (p2);
\end{tikzpicture}
}
  \caption{An example of a factor graph for the factorized function $f(s_1, s_2, s_3) = f_1(s_1, s_2, s_3)f_2(s_2, s_3)f_3(s_3)$.
    The square blocks represent factor nodes, and the circled blocks represent variable nodes.
    A factor node is connected to a variable node if and only if the variable is present in the
    argument list of the corresponding factor.
  }
  \label{fig:ffg:example1}
\end{figure}

The original bi-partite formulation of \acp{fg} involves two distinct types of nodes:
factor nodes and variable nodes.
However, for the sake of clarity and simplicity in subsequent discussions, we will adopt a
specific instance of factor graphs known as \acp{tffg}.

\subsection{Terminated Forney-style factor graphs}

\Acp{tffg} simplify the original factor graph
representation by using only one type of node, namely factor nodes.
The variable nodes are replaced with edges in \acp{tffg}.
As a notational convention, factor nodes are indexed by $a, b, c$, and edges are indexed by
$i, j, k$, unless specified otherwise, see Figure~\ref{fig:ffg:notation}.
We use the notation $\mathcal{E}(a)$ to refer to the neighboring edges of a node $a$, and for
an edge $i \in \mathcal{E}$, $\mathcal{V}(i)$ collects all neighboring nodes.
An edge-induced subgraph is denoted by $\mathcal{G}(i) = (\mathcal{V}(i), i)$, while a
node-induced subgraph is denoted by $\mathcal{G}(a) = (a, \mathcal{E}(a))$.
Furthermore, a local subgraph is defined by $\mathcal{G}(a, i) = (\mathcal{V}(i),
  \mathcal{E}(a))$, which encompasses all local nodes and edges around both $i$ and $a$.
The \ac{tffg} representation of the previous example function~\eqref{eq:ffg:example1} is presented
in Figure~\ref{fig:ffg:example1_tffg}.

Replacing variable nodes with edges implies an implicit assumption that all variables $s_i$
are connected to exactly two factor nodes.
While this assumption may seem restrictive, it does not limit the representable class of
functions $p$.
This situation is analogous to our previous discussion in
Section~\ref{chapter-02:section:bethe-free-energy}, where we assumed $d_i = 2$.
The proof is similar to~\ref{appendix:proofs:cardinality_a_i_2} and introduces
\textit{equality factors} that restrict connected edges to carry identical beliefs, with the
implication that these beliefs can be made available to more than two factors.
An equality factor is defined as \begin{equation}
    \label{eq:ffg:equality_node}
    f_{=}(s, s', s'') = \delta(s - s')\delta(s - s''),
    \end{equation} and Figure~\ref{fig:ffg:equality_node}
illustrates the node-induced subgraph $\mathcal{G}(a)$ for this case.
In contrast, if a variable $s_i$ is connected to only one node, we can handle this scenario by
"terminating" the graph.
This is achieved by introducing an auxiliary factor $f_t(s_i) = 1$, which does not affect the
function $f$ but allows us to represent $s_i$ as an edge in the corresponding TFFG.

\begin{figure}
  \centering
  \begin{subfigure}[t]{0.265\textwidth}
    \centering
    \resizebox{\textwidth}{!}{\begin{tikzpicture}[]
  %Node
  \node[box] (pa) {$f_a$};
  \node[box] (pb) [right=of pa] {$f_b$};

  %Lines
  \path[line] (pa) edge[-] node[pos=0.5, anchor=south]{$s_i$} (pb);
\end{tikzpicture}
}
    \caption{An example of an edge-induced subgraph $\mathcal{G}(i) = (\mathcal{V}(i), i)$}
    \label{fig:ffg:g_i}
  \end{subfigure}
  \hfill
  \begin{subfigure}[t]{0.285\textwidth}
    \centering
    \resizebox{\textwidth}{!}{
\begin{tikzpicture}[]
  %Node
  \node[box] (pa) {$f_a$};
  \node[istyle] (i) [right=of pa] {};
  \node[istyle] (d) [left=of pa] {$\dots$};
  \node[istyle] (j) [left=of pa, yshift=10pt] {};
  \node[istyle] (k) [left=of pa, yshift=-10pt] {};

  %Lines
  \path[line] (pa) edge[-] node[pos=0.5, anchor=south]{$s_i$} (i);
  \path[line] (pa) edge[-] node[pos=0.5, anchor=south]{$s_j$} (j);
  \path[line] (pa) edge[-] node[pos=0.5, anchor=north]{$s_k$} (k);
\end{tikzpicture}
}
    \caption{An example of a node-induced subgraph $\mathcal{G}(a) = (a, \mathcal{E}(a))$}
    \label{fig:ffg:g_a}
  \end{subfigure}
  \hfill
  \begin{subfigure}[t]{0.355\textwidth}
    \centering
    \resizebox{\textwidth}{!}{
\begin{tikzpicture}[]
  %Node
  \node[box] (pa) {$f_a$};
  \node[box] (pb) [right=of pa] {$f_b$};
  \node[istyle] (d) [left=of pa] {$\dots$};
  \node[istyle] (j) [left=of pa, yshift=10pt] {};
  \node[istyle] (k) [left=of pa, yshift=-10pt] {};

  %Lines
  \path[line] (pa) edge[-] node[pos=0.5, anchor=south]{$s_i$} (pb);
  \path[line] (pa) edge[-] node[pos=0.5, anchor=south]{$s_j$} (j);
  \path[line] (pa) edge[-] node[pos=0.5, anchor=north]{$s_k$} (k);
\end{tikzpicture}
}
    \caption{An example of a local subgraph $\mathcal{G}(a, i) = (\mathcal{V}(i), \mathcal{E}(a))$}
    \label{fig:ffg:g_a_i}
  \end{subfigure}
  \caption{The notational convention for TFFGs.
    Factor nodes are indexed by $a, b, c$ and edges are indexed by $i, j, k$.
    Neighboring edges of a node $a$ are defined as $\mathcal{E}(a)$.
    Neighboring nodes of an edge $i$ are defined as $\mathcal{V}(i)$.
    Note that nodes can have an arbitrary number of connected edges, which is indicated by the
    $\cdots$ symbol.
  }
  \label{fig:ffg:notation}
\end{figure}

\begin{figure}
  \centering
  \begin{subfigure}[t]{0.575\textwidth}
    \centering
    \resizebox{1.0\textwidth}{!}{
\begin{tikzpicture}[]
  %Nodes
  \node[box] (pt) {$f_t$};
  \node[box] (p1) [right=of pt] {$f_1$};
  \node[smallbox] (peq) [right=of p1] {\small{=}};
  \node[box] (p2) [below=of peq, yshift=3mm] {$f_2$};
  \node[box] (p3) [right=of peq] {$f_3$};

  %Lines
  \path[line] (pt) edge[-] node[pos=0.5, anchor=south]{$s_1$}(p1);
  \path[line] (p1) edge[-] node[pos=0.5, anchor=south]{$s_3$}(peq);
  \path[line] (peq) edge[-] node[pos=0.5, anchor=south]{$s_3'$}(p3);
  \path[line] (peq) edge[-] node[pos=0.5, anchor=west]{$s_3''$}(p2);
  \draw (p1) |- node[pos=0.5, anchor=east]{$s_2$}(p2);
\end{tikzpicture}
}
    \caption{An example of a \ac{tffg} for the factorized function $f(s_1, s_2, s_3) = f_1(s_1, s_2, s_3)f_2(s_2, s_3)f_3(s_3)$, which is equivalent to $f_t(s_1)f_{=}(s_3, s_3', s_3'')f_1(s_1, s_2, s_3)f_2(s_2, s_3'')f_3(s_3')$.
      The square blocks represent factor nodes and the edges represent individual variables.
      A factor node is connected to an edge if and only if the associated variable is present in the
      argument list of the factor node.
    }
    \label{fig:ffg:example1_tffg}
  \end{subfigure}
  \hfill
  \begin{subfigure}[t]{0.375\textwidth}
    \centering
    \resizebox{1.0\textwidth}{!}{
\begin{tikzpicture}[]
  %Nodes
  \node[smallbox] (equality) {\small{=}};
  \node[](z)[left=of equality]{};
  \node[](zz)[right=of equality]{};
  \node[](zzz)[below=of equality, yshift=6pt]{};

  %Lines
  \path[line] (equality) edge[-] node[pos=0.5, anchor=south]{$s$} (z);
  \path[line] (equality) edge[-] node[pos=0.5, anchor=south]{$s'$}(zz);
  \path[line] (equality) edge[-] node[pos=0.5, anchor=west]{$s''$}(zzz);
\end{tikzpicture}
}
    \caption{A vizualization of the node induced subgraph for an equality node (\ref{eq:ffg:equality_node}).
      The symbol $=$ indicates that the functional form of the node is known to refer to the
      equality node $f_{=}(s, s', s'') = \delta(s - s')\delta(s - s'')$.
    }
    \label{fig:ffg:equality_node}
  \end{subfigure}
  % \caption{Terminated Forney-style Factor Graphs notation includes "equality" nodes and represents all variables as edges.}
  \label{fig:ffg:equality_node_and_tffg_example}
\end{figure}

Since \acp{tffg} are equivalent to regular \acp{fg} and do not limit the class of
representable functions, we assume that all further discussions are applicable to regular
\acp{fg} as well.
Therefore, we use both terms interchangeably, unless stated otherwise.
