\section{Conclusions}\label{chapter-02:section:conclusion}

This chapter aims to establish a strong and practical foundation for performing efficient
Bayesian inference in large models.
The chosen approach, \ac{cbfe}, provides a solid and mathematically sound framework that allows for
a flexible balance between accuracy and computational complexity.
In factorized models, the ability to use different sets of constraints in various parts of the
factor graph allows for precise control over the computational complexity of the message
passing scheme for specific applications.
This adaptability also means that constraints can be adjusted based on changes in the
environment or focus on relevant parts of the model at different times.
Additionally, the locality of the approach allows the option of performing fewer computations
in less important parts of the model, leading to efficiency gains compared to global
optimization procedures.

Factor graphs were chosen not only as a convenient graphical representation of large sparse
probabilistic models, but also as a useful visual framework for deriving efficient inference
procedures through message passing.
The message passing interpretation leverages the local properties of the \ac{cbfe} minimization,
which will be particularly useful for further discussion about reactive and continual
inference.

Although this chapter intentionally avoids complex details of the \ac{cbfe} procedure and its
associated message passing-based inference due to their scope exceeding that of this thesis,
it primarily focuses on establishing notation and graphical intuition for the \ac{cbfe}
optimization procedure.
This preparation paves the way for exploring reactive message passing ideas in the next
chapter.
An interested reader can refer to \citep{yedidia_bethe_2001, yedidia_generalized_2002, yedidia_understanding_2001, yedidia_constructing_2005, senoz_variational_2021, senoz_thesis} for detailed explanation, history, and practicality of \ac{cbfe} minimization procedure.

The next chapter will combine previous research with the concept of reactive processing.
Reactivity will address significant questions such as how to respond to new observations and
perform continuous inference without interruption.
By adopting reactive processing, we aim to further enhance the practicality and adaptability
of Bayesian inference methodology.
