
\section{Inference by message passing on factor graphs}\label{chapter-02:section:message-passing}

The \acp{fg} are computational graphs and come with a graphical framework to execute Bayesian inference by sending messages between factor nodes.
To illustrate this process, let us consider a probabilistic model with hidden states $s_1$ and
$s_2$, along with an observation $y$: \begin{equation}
    p(y, s_1, s_2) = f_1(s_1)f_2(s_1, s_2)f_3(y, s_2),
  \end{equation} 
where $f_a(\cdot)$ denote non-negative factors. For instance, $f_1(s_1) = p(s_1)$, $f_2(s_1,s_2)=p(s_2|s_1)$ and $f_3(y, s_2)=p(y|s_2)$ would be a valid configuration.  

Suppose we are interested in the Bayesian posterior $p(s_2\vert \hat{y})$, which can be inferred through
\begin{equation}\label{eq:mp:inference_example} 
  \begin{aligned}
    p(s_2|\hat{y}) &\propto \int p(y, s_1,s_2) \delta(y-\hat{y}) \mathrm{d}s_1 \mathrm{d}y \\
    &= \int f_1(s_1)f_2(s_1, s_2)f_3(y, s_2)\delta(y-\hat{y}) \mathrm{d}s_1 \mathrm{d}y.
    \end{aligned}
\end{equation}

For large high-dimensional
generative models, the integral~\eqref{eq:mp:inference_example} quickly becomes intractable
due to the exponential explosion of the size of the latent space \citep{bishop_pattern_2006}.
However, the message passing interpretation of the inference provides a convenient solution to
this problem.
The factorization assumption~\eqref{eq:ffg:factorization_assumption} of $p$ allows the
application of the algebraic distributive law, which simplifies the computation
of~\eqref{eq:mp:inference_example} to a nested set of lower-dimensional integrals, as shown
in~\eqref{eq:mp:inference_example_factorized}.
This "trick" effectively reduces the exponential complexity of the computation to linear.
Inference in this model can be represented by a message passing scheme shown in
Figure~\ref{fig:mp:inference_example}, where each nested integral can be interpreted as a
message that flows along the edges between nodes
\begin{equation}
  \label{eq:mp:inference_example_factorized}
  p(s_2|\hat{y}) \propto \overbrace{\int \underbrace{f_1(s_1)}_{\mu_{12}}f_2(s_1, s_2) \,\mathrm{d}s_1}^{\mu_{23}} \cdot \underbrace{\int f_3(y, s_2)\overbrace{\delta(y - \hat{y})}^{\mu_{y3}}\mathrm{d}y }_{\mu_{22}}.
\end{equation}

\begin{figure}
  \centering
  \resizebox{0.9\textwidth}{!}{\begin{tikzpicture}
  \node[box] (p_1) {$f_1$};
  \node[box] (p_2) [right=of p_1, xshift=5mm] {$f_2$};
  \node[box] (p_3) [right=of p_2, xshift=15mm] {$f_3$};
  \node[clamped] (y) [right=of p_3, xshift=5mm] {};

  \path[line] (p_1) edge[-] node[pos=0.5, anchor=south]{$s_1$} (p_2);
  \path[line] (p_2) edge[-] node[pos=0.5, anchor=south]{$s_2$} (p_3);
  \path[line] (p_3) edge[-] node[pos=0.5, anchor=south]{$y$} (y);

  \node[densely dotted, fit=(p_1), draw, inner sep=2mm] (p_1_dashed) {};
  \node[densely dotted, fit=(p_1_dashed)(p_2), draw, inner ysep=2mm, inner xsep=2mm] (p_1_2_dashed) {};

  \draw[-{Stealth[black]}] ([xshift=1mm, yshift=4mm]p_1_dashed.south east) -- node[pos=0.5, anchor=north]{$\mu_{12}$} ([xshift=6mm, yshift=4mm]p_1_dashed.south east);
  \draw[-{Stealth[black]}] ([xshift=1mm, yshift=6mm]p_1_2_dashed.south east) -- node[pos=0.5, anchor=north]{$\mu_{23}$} ([xshift=6mm, yshift=6mm]p_1_2_dashed.south east);

  \node[densely dotted, fit=(y), draw, inner sep=4mm] (y_dashed) {};
  \node[densely dotted, fit=(y_dashed)(p_3), draw, inner ysep=2mm, inner xsep=2mm] (y_p_3_dashed) {};

  \draw[-{Stealth[black]}] ([xshift=-1mm, yshift=4mm]y_dashed.south west) -- node[pos=0.5, anchor=north]{$\mu_{y3}$} ([xshift=-6mm, yshift=4mm]y_dashed.south west);
  \draw[-{Stealth[black]}] ([xshift=-1mm, yshift=6mm]y_p_3_dashed.south west) -- node[pos=0.5, anchor=north]{$\mu_{22}$} ([xshift=-6mm, yshift=6mm]y_p_3_dashed.south west);
\end{tikzpicture}

}
  \caption{The message passing scheme for~\eqref{eq:mp:inference_example_factorized}.
    In this graphical representation, the factors $f_1$, $f_2$, and $f_3$ are represented by
    nodes, while the variables $s_1$, $s_2$, and $y$ are associated with edges.
    The small black node indicates that $y$ has been observed, and the posterior for $y$ is
    clamped to its observed value $\hat{y}$.
    Technically, one can interpret the small black node as an additional factor $\delta(y -
      \hat{y})$ that extends the original model.
    Each nested integral in~\eqref{eq:mp:inference_example_factorized} can be interpreted as a
    message that flows between nodes on the TFFG.
  }
  \label{fig:mp:inference_example}
\end{figure}

As a notational convention, as shown in Figure~\ref{fig:mp:notation}, we use the notation
$\mu_{ia}(s_i)$ to represent a message along edge $i$ towards a factor node $a$.
A message with index $i$ is always a function of $s_i$.
However, to reduce clutter and simplify the notation, we may sometimes use $\mu_{ia}$ instead
of $\mu_{ia}(s_i)$.
In this message passing interpretation, we also extend the generative model by incorporating
auxiliary "clamped" factors of the form $\delta(y - \hat{y})$, where $\delta(\cdot)$
represents either a Kronecker delta or a Dirac delta, depending on the context.
These additional factors effectively clamp the posterior for the observed variable $y$ to its
observed value $\hat{y}$.
Such clamped observed variables are depicted as small black nodes in \acp{tffg}.

\begin{figure}
  \centering
  \begin{subfigure}[t]{0.275\textwidth}
    \centering
    \resizebox{\textwidth}{!}{\begin{tikzpicture}
  \node[box] (a) {$f_a$};
  \node[box] (b) [right=of a, xshift=7.5mm] {$f_b$};

  \path[line] (a) edge[-] 
    node[pos=0.5, anchor=south]{$s_i$} 
    %node[pos=0.2, anchor=south]{$\substack{\mu_{ai}\\\rightarrow}$} 
    node[pos=0.3, anchor=north]{$\substack{\leftarrow\\\mu_{ia}}$} 
    %node[pos=0.8, anchor=south]{$\substack{\mu_{bi}\\\leftarrow}$} 
    node[pos=0.7, anchor=north]{$\substack{\rightarrow\\\mu_{ib}}$} 
  (b);
\end{tikzpicture}
}
    \caption{An example of flowing messages on around an edge-induced subgraph $\mathcal{G}(i) = (\mathcal{V}(i), i)$.}
    \label{fig:mp:g_i_msgs}
  \end{subfigure}
  \hfill
  \begin{subfigure}[t]{0.295\textwidth}
    \centering
    \resizebox{0.9\textwidth}{!}{
\begin{tikzpicture}[]
  %Node
  \node[box] (pa) {$f_a$};
  \node[istyle] (i) [right=of pa] {};
  \node[istyle] (d) [left=of pa, xshift=10pt] {$\dots$};
  \node[istyle] (j) [left=of pa, yshift=15pt] {};
  \node[istyle] (k) [left=of pa, yshift=-15pt] {};

  %Lines
  \path[line] (pa) edge[-]
  node[pos=0.9, anchor=south]{$s_i$}
  %node[pos=0.3, anchor=south]{$\substack{\mu_{ai}\\\rightarrow}$}
  node[pos=0.6, anchor=north]{$\substack{\leftarrow\\\mu_{ia}}$}
  (i);
  \path[line] (pa) edge[-]
  node[pos=0.9, anchor=south]{$s_j$}
  %node[pos=0.3, anchor=south, rotate=-30]{$\substack{\mu_{aj}\\\leftarrow}$}
  node[pos=0.4, anchor=south, rotate=-25]{$\substack{\mu_{ja}\\\rightarrow}$}
  (j);
  \path[line] (pa) edge[-]
  node[pos=0.9, anchor=north]{$s_k$}
  %node[pos=0.3, anchor=north, rotate=30]{$\substack{\leftarrow\\\mu_{ak}}$}
  node[pos=0.4, anchor=north, rotate=25]{$\substack{\rightarrow\\\mu_{ka}}$}
  (k);
\end{tikzpicture}

}
    \caption{An example of flowing messages around a node-induced subgraph $\mathcal{G}(a) = (a, \mathcal{E}(a))$.}
    \label{fig:mp:g_a_msgs}
  \end{subfigure}
  \hfill
  \begin{subfigure}[t]{0.335\textwidth}
    \centering
    \resizebox{\textwidth}{!}{
\begin{tikzpicture}[]
  %Node
  \node[box] (pa) {$f_a$};
  \node[box] (pb) [right=of pa, xshift=7.5mm] {$f_b$};
  \node[istyle] (d) [left=of pa, xshift=10pt] {$\dots$};
  \node[istyle] (j) [left=of pa, yshift=15pt] {};
  \node[istyle] (k) [left=of pa, yshift=-15pt] {};

  %Lines
  \path[line] (pa) edge[-] 
    node[pos=0.9, anchor=south]{$s_j$} 
    %node[pos=0.3, anchor=south, rotate=-30]{$\substack{\mu_{aj}\\\leftarrow}$}
    node[pos=0.4, anchor=south, rotate=-25]{$\substack{\mu_{ja}\\\rightarrow}$}
  (j);
  \path[line] (pa) edge[-] 
    node[pos=0.9, anchor=north]{$s_k$} 
    %node[pos=0.3, anchor=north, rotate=30]{$\substack{\leftarrow\\\mu_{ak}}$}
    node[pos=0.4, anchor=north, rotate=25]{$\substack{\rightarrow\\\mu_{ka}}$}
  (k);
  \path[line] (pa) edge[-] 
    node[pos=0.5, anchor=south]{$s_i$} 
    %node[pos=0.2, anchor=south]{$\substack{\mu_{ai}\\\rightarrow}$} 
    node[pos=0.3, anchor=north]{$\substack{\leftarrow\\\mu_{ia}}$} 
    %node[pos=0.8, anchor=south]{$\substack{\mu_{bi}\\\leftarrow}$} 
    node[pos=0.7, anchor=north]{$\substack{\rightarrow\\\mu_{ib}}$} 
  (pb);
\end{tikzpicture}
}
    \caption{An example of flowing messages around a local subgraph $\mathcal{G}(a, i) = (\mathcal{V}(i), \mathcal{E}(a))$.}
    \label{fig:mp:g_a_i_msgs}
  \end{subfigure}
  \caption{The notational convention for messages flowing on \acp{tffg}.
    Factor nodes are indexed by $a, b, c$ and edges are indexed by $i, j, k$.
    Neighboring edges of a node $a$ are defined as $\mathcal{E}(a)$.
    Neighboring nodes of an edge $i$ are defined as $\mathcal{V}(i)$.
    Note that nodes can have an arbitrary number of connected edges, which is indicated by the
    $\cdots$ symbol.
    A message along an edge $i$ towards a node $a$ is denoted as $\mu_{ia}$.
    Note that $\mu_{ia}$ is a function of $s_i$.
  }
  \label{fig:mp:notation}
\end{figure}

The resulting posterior $p(s_2\vert\hat{y})$ in~\eqref{eq:mp:inference_example_factorized}
is simply equal to the product of two (colliding) messages on the same edge $s_2$, divided by
the normalization constant \begin{equation} 
    p(s_2\vert\hat{y}) = \frac{\mu_{23}(s_2)\,\,\mu_{22}(s_2)}{\int \mu_{23}(s_2)\,\,\mu_{22}(s_2)\,\mathrm{d}s_2}.
\label{eq:mp:marginal_colliding}
\end{equation}

This procedure for
computing a posterior distribution defines the \ac{bp} message passing algorithm and, generally,
requires only the evaluation of low-dimensional integrals over local variables.
In general, \ac{bp} defines a message $\mu_{ib}(s_i)$ to be of
the following form \begin{equation}
    \label{eq:mp:bp_message} \mu_{ib}(s_i) = \int
    f_a(\bm{s}_a) \prod_{\substack{j \in \mathcal{E}(a)\\j \neq i}} \mu_{ja}(s_j)
    \,\,\mathrm{d}\bm{s}_{a\setminus i},
  \end{equation} where integration is replaced with a summation in the
case of discrete random variables. 
Note that a message is not necessarily a proper distribution w.r.t. to its argument.

\subsection{Variational inference as message passing}

As mentioned in Section~\ref{chapter-02:section:bethe-free-energy}, the \ac{cbfe} functional~\eqref{eq:bfe:marginalization}-\eqref{eq:bfe:factorized_bethe_simplified} already
serves as a variational objective for the \ac{bp} algorithm.
In certain situations, such as when all messages follow a Gaussian distribution and the
factors are linear, it becomes possible to use closed-form analytical solutions for the
messages~\eqref{eq:mp:bp_message}.
These closed-form formulas are commonly referred to as \textit{message update rules} or simply
\textit{rules}.
However, there is no guarantee that the message update rules~\eqref{eq:mp:bp_message} from the \ac{bp} algorithm can be analytically solved for generic factors with arbitrary connectivity.
Fortunately, not only \ac{bp} algorithm, but the entire \ac{cbfe} minimization procedure~\eqref{eq:bfe:bfe} can 
also be represented using message passing.
The rigorous and comprehensive derivation of this result is presented in~\citep{senoz_variational_2021, senoz_thesis}.
Here, we will solely describe the final results and provide the functional form of the \ac{cbfe}
messages.

As noted in Section~\ref{chapter-02:section:bethe-free-energy}, the \ac{cbfe} introduces additional
constraints on the family of variational distributions $\mathcal{Q}_B$.
Consider a factor $f_a$ and its associated variational distributions $q_a$ and $q_i$.
As a notational convention, we denote extra factorization
constraints~\eqref{eq:bfe:factorization_constraint} on the variational distributions as
rounded gray-shaded blocks (clusters) enclosing the corresponding edges, see
Figure~\ref{fig:mp:notation_qs}.
The \ac{cbfe} messages are derived as \begin{equation}
    \label{eq:mp:vi_message} \mu^{k+1}_{ib}(s_i) = \int f^{n,k}_a(\bm{s}^n_a) \prod_{\substack{j
        \in \mathcal{C}(a, i)\\j \neq i}} \mu^{k}_{ja}(s_j)\mathrm{d}s_j,\,\,
  \end{equation} where $\bm{s}^n_a = \mathcal{C}(a,
  i)$, $k$ is an iteration index of the optimization procedure, and $f^{n,k}_a$ is an auxiliary
factor, which is defined as \begin{equation}
  \label{eq:mp:auxiliary_p} f^{n,k}_a(\bm{s}^n_a) = \exp \Big( \int \Big\{ \prod_{\substack{m
      \in \mathcal{C}(a)\\m \neq n}} q^{m, k}_a(\bm{s}^m_a) \Big\} \log f_a(\bm{s}_a)
  \mathrm{d}\bm{s}_{a \setminus n} \Big).
\end{equation}

The variational distributions $q^{k+1}_i(s_i)$ around node $a$ are computed according to
\begin{equation}
    q^{k+1}_i(s_i) =
    \frac{\mu^{k}_{ia}(s_i)\,\mu^{k}_{ib}(s_i)}{\int \mu^{k}_{ia}(s_i)\,\mu^{k}_{ib}(s_i)
      \mathrm{d}s_i}
    \label{eq:mp:variational_q_i}
\end{equation} and the variational distributions $q^{n,k+1}_a(\bm{s}^n_a)$ are
computed according to \begin{equation} \label{eq:mp:variational_q_a} q^{n, k+1}_a(\bm{s}^n_a)
  = \frac{f^{n, k}_a(\bm{s}^n_a)\prod_{i \in n}\mu^{k}_{ia}(s_i)}{\iint
    f^{n,k}_a(\bm{s}^n_a)\prod_{i \in n}\mu^{k}_{ia}(s_i)\mathrm{d}\bm{s}^n_a}.
\end{equation}

Let us take the time to study this result in detail.
First, we now have an additional index $k$, which we refer to as the \textit{variational iteration} index.
This implies that we need to specify all $q^0_i(s_i)$ and $q^{n, 0}_a(\bm{s}^n_a)$ for the
initial iteration $k = 0$ of the \ac{cbfe} minimization procedure.
These variational distributions will be referred to as \textit{initial variational distributions}.
To reduce clutter, we will often omit the index $k$ and refer to variational messages and
auxiliary factors without the index.
Second, the equation~\eqref{eq:mp:vi_message} is suspiciously similar to the
equation~\eqref{eq:mp:bp_message}.
In fact, the equations differ only in the auxiliary factor $f^n_a$ and the iteration index
$k$.
This indicates that the \ac{cbfe} minimization procedure uses regular \ac{bp} messages on each iteration
within each shaded block (cluster) in Figure~\ref{fig:mp:notation_qs}, but uses a different
factor function.
This factor is computed by "fusing" the original factor together with the remaining clusters.
We show a visual representation of the \ac{cbfe} message passing updates in Figure~\ref{fig:mp:vmp_intuition}.

Although the messages for the \ac{cbfe} procedure may appear more complex, in most
situations, the auxiliary factors actually make it easier to derive the corresponding
messages.
As a special case, we may also consider the naive mean-field approximation
assumption~\eqref{eq:bfe:factorization_mean_field}, for which the
equations~\eqref{eq:mp:vi_message}-\eqref{eq:mp:auxiliary_p} reduces to
\begin{equation}\label{eq:mp:mf_message} \mu^{k+1}_{ia}(s_i) = \exp \Big( \int \Big\{
  \prod_{\substack{j \in \mathcal{E}(a)\\j \neq i}} q^k_j(s_j) \Big\} \log
  f_a(\bm{s}_a)\mathrm{d}\bm{s}_{a \setminus i} \Big).
\end{equation}
The message given by equation~\eqref{eq:mp:mf_message} offers significant computational
simplification compared to the more general \ac{cbfe} procedure.
This is achieved due to the factorization assumption, where each variable's distribution is
treated independently without considering dependencies between neighboring variables.
However, it is important to note that the mean-field approximation introduces a loss of
accuracy by neglecting correlations between variables, making it suitable for certain
scenarios where simplicity outweighs precision.

\begin{figure}
  \centering
  \begin{subfigure}[t]{0.310\textwidth}
    \centering
    \resizebox{\textwidth}{!}{
\begin{tikzpicture}[]
  %Node
  \node[box] (pa) {$f_a$};
  \node[istyle] (i) [right=of pa] {};
  \node[istyle] (d) [left=of pa] {$\dots$};
  \node[istyle] (j) [left=of pa, yshift=10pt] {};
  \node[istyle] (k) [left=of pa, yshift=-10pt] {};

  \node[qbox, inner sep=12pt, label={[label distance=-6pt]above right:{\scriptsize $q_a(\bm{s}_a)$}}] (si) {};

  %Lines
  \path[line] (pa) edge[-] node[pos=0.9, anchor=west]{$s_i$} node[pos=0.6, qbox, label={[label distance=-4pt]below:{\tiny $q_i(s_i)$}}](s_i){} (i);
  \path[line] (pa) edge[-] node[pos=0.9, anchor=east]{$s_j$} node[pos=0.5, qbox, label={[label distance=-4pt]above:{\tiny $q_j(s_j)$}}](s_j){} (j);
  \path[line] (pa) edge[-] node[pos=0.9, anchor=east]{$s_k$} node[pos=0.5, qbox, label={[label distance=-4pt]below:{\tiny $q_k(s_k)$}}](s_k){} (k);
\end{tikzpicture}
}
    \caption{Variational distributions with no extra factorization constraints around a node induced subgraph $\mathcal{G}(a)$.}
    \label{fig:mp:notation_qs_g_i_msgs}
  \end{subfigure}
  \hfill
  \begin{subfigure}[t]{0.310\textwidth}
    \centering
    \resizebox{\textwidth}{!}{
\begin{tikzpicture}[]
  %Node
  \node[box] (pa) {$f_a$};
  \node[istyle] (i) [right=of pa, yshift=0pt] {};
  \node[istyle] (d) [left=of pa, rotate=-20, yshift=12pt] {$\dots$};
  \node[istyle] (dd) [left=of pa, rotate=20, yshift=-12pt] {$\dots$};
  \node[istyle] (ddd) [right=of pa, rotate=-10, xshift=-15pt, yshift=-10pt] {$\dots$};
  \node[istyle] (j) [left=of pa, yshift=26pt, xshift=2pt] {};
  \node[istyle] (k) [left=of pa, yshift=-26pt, xshift=2pt] {};

  \node[qbox, inner sep=14pt, label={[label distance=-6pt]above left:{\scriptsize $q^m_a(\bm{s}^m_a)$}}] (s_m) [left=of pa, yshift=21pt, xshift=32pt] {};
  \node[qbox, inner sep=14pt, label={[label distance=-6pt]below left:{\scriptsize $q^l_a(\bm{s}^l_a)$}}] (s_l) [left=of pa, yshift=-21pt, xshift=32pt] {};
  \node[qbox, inner sep=14pt, label={[label distance=-3pt]above:{\scriptsize $q^n_a(\bm{s}^n_a)$}}] (s_n) [right=of pa, yshift=0pt, xshift=-29pt] {};

  %Lines
  \path[line] (pa) edge[-] node[pos=0.9, anchor=west]{$s_i$} node[pos=0.65, qbox, label={[label distance=-4pt]above:{\tiny $q_i(s_i)$}}](s_i){} (i);
  \path[line] (pa) edge[-] node[pos=0.9, anchor=east]{$s_j$} node[pos=0.5, qbox, label={[label distance=-4pt]above:{\tiny $q_j(s_j)$}}](s_j){} (j);
  \path[line] (pa) edge[-] node[pos=0.9, anchor=east]{$s_k$} node[pos=0.5, qbox, label={[label distance=-4pt]below:{\tiny $q_k(s_k)$}}](s_k){} (k);
\end{tikzpicture}
}
    \caption{Variational distributions with the structured mean-field factorization constraint $\mathcal{C}(a)$ around a node induced subgraph $\mathcal{G}(a)$.}
    \label{fig:mp:notation_qs_g_a_structured}
  \end{subfigure}
  \hfill
  \begin{subfigure}[t]{0.310\textwidth}
    \centering
    \resizebox{\textwidth}{!}{
\begin{tikzpicture}[]
  %Node
  \node[box] (pa) {$f_a$};
  \node[istyle] (i) [right=of pa] {};
  \node[istyle] (d) [left=of pa] {$\dots$};
  \node[istyle] (j) [left=of pa, yshift=10pt] {};
  \node[istyle] (k) [left=of pa, yshift=-10pt] {};

  %Lines
  \path[line] (pa) edge[-] node[pos=0.9, anchor=west]{$s_i$} node[pos=0.6, qbox, label={[label distance=-4pt]below:{\tiny $q_i(s_i)$}}](s_i){} (i);
  \path[line] (pa) edge[-] node[pos=0.9, anchor=east]{$s_j$} node[pos=0.5, qbox, label={[label distance=-4pt]above:{\tiny $q_j(s_j)$}}](s_j){} (j);
  \path[line] (pa) edge[-] node[pos=0.9, anchor=east]{$s_k$} node[pos=0.5, qbox, label={[label distance=-4pt]below:{\tiny $q_k(s_k)$}}](s_k){} (k);
\end{tikzpicture}}
    \caption{Variational distributions with the naive mean-field factorization constraint around a node induced subgraph $\mathcal{G}(a)$.}
    \label{fig:mp:notation_qs_g_a_mean_field}
  \end{subfigure}
  \caption{The notational convention for variational distributions on \acp{tffg}.
    Factor nodes are indexed by $a, b, c$, edges are indexed by $i, j, k$ and clusters are indexed
    by $n, m, l$.
    Neighboring edges of a node $a$ are defined as $\mathcal{E}(a)$.
    A node induced subgraph $\mathcal{G}(a)$ is defined as $(a, \mathcal{E}(a))$.
    Neighboring clusters of the variational distribution $q_a(\bm{s}_a)$ are defined as
    $\mathcal{C}(a)$.
    Factorized joint variational distributions $q^n_a(\bm{s}_a)$ and $q_i(s_i)$ are depicted by
    shaded circles.
    Individual variational distributions $q_i(s_i)$ enclose only a single edge $i$.
    A joint variational distribution (a cluster) $q^n_a(\bm{s}_a)$ encloses a collection of edges
    $i \in a$.
    Note that joint variational distributions $q^n_a(\bm{s}^n_a)$ can have an arbitrary number of
    enclosed edges, which is indicated by the $\cdots$ symbol.
  }
  \label{fig:mp:notation_qs}
\end{figure}

\begin{figure}
  \centering
  \begin{subfigure}[t]{0.45\textwidth}
    \centering
    \resizebox{\textwidth}{!}{\begin{tikzpicture}
  \node[box] (a) {$f_a$};
  \node[] (i) [right=of a, xshift=6mm, yshift=6mm] {};
  \node[] (j) [right=of a, xshift=6mm, yshift=-6mm] {};
  \node[] (d) [left=of a, rotate=-20, xshift=4mm, yshift=4mm] {$\dots$};
  \node[] (dd) [left=of a, rotate=20, xshift=4mm, yshift=-4mm] {$\dots$};
  \node[] (ddd) [right=of a] {$\dots$};
  \node[] (ji) [left=of a, yshift=23pt, xshift=2pt] {};
  \node[] (ki) [left=of a, yshift=-23pt, xshift=2pt] {};

  \node[qbox, inner sep=3.5mm, label={[label distance=-2.5mm]above right:{\tiny $q^m_a(\bm{s}^m_a)$}}] (s_m) [left=of a, yshift=5.5mm, xshift=8mm] {};
  \node[qbox, inner sep=3.5mm, label={[label distance=-2.5mm]below right:{\tiny $q^h_a(\bm{s}^h_a)$}}] (s_h) [left=of a, yshift=-5.5mm, xshift=8mm] {};

  \node[qbox, label={[label distance=-1mm]above:{\tiny $q^n_a(\bm{s}^n_a)$}}] (s_n) [fit=(ddd), xshift=3mm, inner sep=3mm]{};

  \node[box, inner sep=11pt, fill=black, densely dotted, fill opacity=0.025, label={above:{$f^n_a$}}] (a_tilde) [fit=(d)(dd)(ji)(ki)(a)] {};

  \path[line] (a) edge[-]
  node[pos=0.8, anchor=west, xshift=3mm]{$s_i$}
  node[pos=0.5, anchor=south, rotate=17]{$\substack{\mu_{ib}\\\rightarrow}$}
  (i);

  \path[line] (a) edge[-]
  node[pos=0.8, anchor=west, xshift=3mm]{$s_j$}
  node[pos=0.6, anchor=north, rotate=-17]{$\substack{\leftarrow \\ \mu_{ja}}$}
  (j);

  \path[line] (a) edge[-] (ji);
  \path[line] (a) edge[-] (ki);
\end{tikzpicture}

}
    \caption{The message passing scheme for \ac{cbfe} minimization with structured mean-field factorization assumption.
      Each cluster depicted as gray shaded circled area has arbitrary number of variables in it,
      which is indicated by the $\cdots$ symbol.
    }
    \label{fig:mp:vmp_intuition_structured}
  \end{subfigure}
  \hfill
  \begin{subfigure}[t]{0.42\textwidth}
    \centering
    \resizebox{\textwidth}{!}{
\begin{tikzpicture}
  \node[box] (a) {$f_a$};
  \node[] (i) [right=of a, xshift=6mm, yshift=0mm] {};
  \node[] (ji) [left=of a, yshift=23pt, xshift=2pt] {};
  \node[] (ki) [left=of a, yshift=-23pt, xshift=2pt] {};
  \node[] (d) [left=of a, xshift=25pt] {$\cdots$};

  \node[box, inner sep=11pt, fill=black, densely dotted, fill opacity=0.025, label={above:{$f^n_a$}}] (a_tilde) [fit=(ji)(ki)(a)] {};

  \path[line] (a) edge[-]
  node[pos=1.0, anchor=west]{$s_i$}
  node[pos=0.8, qbox, label={[label distance=-4pt]below:{\tiny $q_i(s_i)$}}](s_i){}
  node[pos=0.5, anchor=south, rotate=0]{$\substack{\mu_{ib}\\\rightarrow}$}
  (i);

  \path[line] (a) edge[-]
  node[pos=1.0, anchor=east]{$s_j$}
  node[pos=0.6, qbox, label={[label distance=-4pt]above:{\tiny $q_j(s_j)$}}](s_j){}
  (ji);
  \path[line] (a) edge[-]
  node[pos=1.0, anchor=east]{$s_k$}
  node[pos=0.6, qbox, label={[label distance=-4pt]below:{\tiny $q_k(s_k)$}}](s_k){}
  (ki);
\end{tikzpicture}


}
    \caption{The message passing scheme for \ac{cbfe} minimization with naive mean-field factorization assumption.
      Each cluster depicted as gray shaded circled area has only one variable in it.
    }
    \label{fig:mp:vmp_intuition_mean_field}
  \end{subfigure}
  \caption{The intuition behind message passing-based \ac{cbfe} minimization with factorization constraints~\eqref{eq:bfe:factorization_constraint} around a node $a$ with factor function $p_a(\bm{s}_a)$.
    The message passing updates are similar to BP message passing, but use a different auxiliary
    factor function to compute a message towards edge $s_i$.
    Clusters $\mathcal{C}(a)$ are shown as gray-shaded circled areas.
    The auxiliary factor function $p$ is computed by "fusing" all variational distributions
    $q^m_a(\bm{s}^m_a)$ (clusters), which do not include the edge $s_i$, according
    to~\eqref{eq:mp:auxiliary_p}.
    The procedure repeats for all clusters of variables around the node.
    The node and its associated clusters may have an arbitrary number of connected edges, which is
    indicated by the $\cdots$ symbol.
  }
  \label{fig:mp:vmp_intuition}
\end{figure}

% \subsection{Evaluating the Bethe Free Energy on factor graph}

% The decomposition of~\eqref{eq:bfe:bfe} shows that an iteration can compute the BFE over the
% nodes and edges of the graph locally.
% The will refer to the local BFE component of a node $a$ as \textit{factor bound free energy}
% $F[q_a, p_a]$ and to the local BFE component of an edge $i$ as \textit{variable bound entropy}
% $H[q_i]$.
% The algorithm for evaluating the Bethe Free Energy on factor graphs involves iterating over
% all nodes and edges in the graph and computing the associated "bound" components.
% Nuanced details of the evaluation of individual components for different types of factor nodes
% and the efficient implementation can be found in \citet{senoz_thesis}.
