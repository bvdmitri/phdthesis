\section{State space model of a car}\label{appendix:proofs:car_dynamics}

The dynamics of a moving car in 2d coordinates $(x_1, x_2)$ can be described by Newton's law
  \begin{equation}
    \label{eq:appendix:newtons_law} \bm{F}(t) = m \bm{a}(t),
  \end{equation}
  where
  $\bm{a}(t)$ is the acceleration, $m$ is the mass of the car, and $\bm{F}(t)$ is a vector of
  (unknown) forces acting on the car.
We can model the rate of change of $\bm{a}(t)$ as a white noise \begin{equation} \begin{split}
  \frac{\mathrm{d}^2 x_1}{\mathrm{d} t^2} &= r_1, \\ \frac{\mathrm{d}^2 x_2}{\mathrm{d} t^2} &=
  r_2.
\\
\end{split}
\end{equation}
We shall define new variables $\dot{x}_1 = \nicefrac{\mathrm{d} x_1}{\mathrm{d} t}$ and
  $\dot{x}_2 = \nicefrac{\mathrm{d} x_2}{\mathrm{d} t}$.
The model~\eqref{eq:appendix:newtons_law} can be then rewritten as \begin{equation}
  \frac{\mathrm{d}}{\mathrm{d} t}
  \begin{pmatrix}
    x_1 \\ x_2 \\ \dot{x}_1 \\ \dot{x}_2
  \end{pmatrix}
  = \underbrace{
    \begin{pmatrix}
      0 & 0 & 1 & 0 \\ 0 & 0 & 0 & 1 \\ 0 & 0 & 0 & 0
      \\ 0 & 0 & 0 & 0
    \end{pmatrix}
  }_{G}
  \begin{pmatrix}
    x_1 \\ x_2 \\ \dot{x}_1 \\ \dot{x}_2
  \end{pmatrix}
  + \underbrace{
    \begin{pmatrix}
      0 & 0 \\ 0 & 0 \\ 1 & 0 \\ 0 & 1
    \end{pmatrix}
  }_{H}
  \begin{pmatrix}
    r_1 \\ r_2
  \end{pmatrix}
  .
\end{equation}

This can be further rewritten as linear dynamical system
  \begin{equation}
    \label{eq:appendix:car_lds} \frac{\mathrm{d}}{\mathrm{d} t} = G s + H r
  \end{equation}
  where
  $s = (x_1, x_1, \dot{x_1}, \dot{x_2})$ is the state vector of the system and $r = (r_1, r_2)$
  is a white-noise process.
The equation~\ref{eq:appendix:car_lds} can be discretized in time with fixed $\Delta t$,
  leading to \begin{equation}
  \begin{pmatrix}
    x_{1, t} \\ x_{2, t} \\ \dot{x}_{1, t} \\
    \dot{x}_{1, t}
  \end{pmatrix}
  = \underbrace{
    \begin{pmatrix}
      1 & 0        & \Delta t & 0 \\ 0 & 1                            & 0
        & \Delta t                \\ 0 & 0 & 1 & 0 \\ 0 & 0 & 0 & 1
    \end{pmatrix}
  }_{A}
  \begin{pmatrix}
    x_{1, t - 1}
    \\ x_{2, t - 1} \\ \dot{x}_{1, t - 1} \\ \dot{x}_{2, t - 1}
  \end{pmatrix}
  +
  v_t,~~v_{t} \sim \mathcal{N}(0, \Sigma).
\end{equation}
This can be seen to be a (discrete) linear dynamic model of the form
  \begin{equation}
    s_t = A
    s_{t - 1} + v_t,
  \end{equation}
  where $s_t = (x_1, x_2, \dot{x}_1, \dot{x}_2)_t$ is the
  state of the system at time $t$.

Assuming that measurements are corrupted with a Gaussian noise, the measurement model can be
  written as
  \begin{equation}
    \begin{aligned}
      y_{1, t} & = x_{1, t} + \omega_1, \\ y_{2, t} & =
         x_{2, t} + \omega_2
    \end{aligned}
  \end{equation}
  or equivalently \begin{equation} y_t =
  \underbrace{
    \begin{pmatrix}
      1 & 0 & 0 & 0 \\ 0 & 1 & 0 & 0
    \end{pmatrix}
  }_{B} s_t +
  \omega_t,~~\omega_t \sim \mathcal{N}(0, \Omega).
\end{equation}
The dynamic and measurement models of the card form a linear Gaussian state space
  model~\eqref{eq:sim:lds} \begin{equation} \begin{aligned} s_t &= A s_{t - 1} + v_t,~~v_t
  \sim \mathcal{N}(0, \Sigma) \\ y_t &= B s_{t} + \omega_t,~~\omega_t \sim \mathcal{N}(0, \Omega).
\end{aligned}
\end{equation}
