\section{The cardinality assumption}\label{appendix:proofs:cardinality_a_i_2}

Here, we demonstrate that it is possible to introduce an additional assumption $d_i = 2$ in
equations (\ref{eq:bfe:factorized_bethe}-\ref{eq:bfe:marginalization}) without imposing any
restrictions on the class of functions defined in equation (\ref{eq:bfe:factorized_p}).

If there exists an $i$ such that $d_i = 1$, we can include an extra factor $f'(s_i) = 1$ to
our function.
This additional factor does not alter the original generative model but increases the
cardinality of $s_i$ by 1, effectively setting $d_i = 2$.

Similarly, if there exists an $i$ such that $d_i > 2$, we have the following probabilistic
model along with its corresponding factorization \begin{equation}
  p(s_i, \bm{s}_{\setminus i}) = f(\bm{s}_{\setminus i}) \prod_{k = 1}^{d_i} f_k(s_i, \bm{s}_{\setminus i}),
\end{equation}
We can introduce auxiliary variables $s_i'$ and $s_i''$ along with an auxiliary factor
$f_{=}(s_i, s_i', s_i'') = \delta(s_i - s_i')\delta(s_i - s_i'')$, where $\delta$ represents
either the Dirac or Kronecker delta function depending on the context.
With these additions, we can reformulate the probabilistic model as follows
\begin{equation}
  p(s_i, s_i', s_i'', \bm{s}_{\setminus i}) = 
    f(\bm{s}_{\setminus i}) 
    f_{=}(s_i, s_i', s_i'') 
    f_{1}(s_i', \bm{s}_{\setminus i})
    f_{2}(s_i'', \bm{s}_{\setminus i})
    \prod_{k = 3}^{d_i} f_k(s_i, \bm{s}_{\setminus i}).
\end{equation} In this reformulated model, the cardinality of the variable $s_i$
becomes $d_i - 1$.
We introduce two new auxiliary variables, $s_i'$ and $s_i''$, each with cardinalities $d_{i}'
  = 2$ and $d_{i}'' = 2$, respectively.
These models are equivalent because the probability distribution assigns zero probability to
cases where $s_i' \neq s_i$ or $s_i'' \neq s_i$.
\begin{equation}
  p(s_i, s_i', s_i'', \bm{s}_{\setminus i}) = 
    \begin{cases}
        p(s_i, \bm{s}_{\setminus i}),& \text{if } s_i = s_i' = s_i'' \\
        0,              & \text{otherwise}.
    \end{cases}
\end{equation}
As a result, we can express the original model specification in an equivalent form but with a
reduced cardinality for the selected variable $s_i$.
This process can be repeated until $d_i = 2$ is satisfied for all $i$.
