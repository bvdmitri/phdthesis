\section{State space model of a double pendulum}\label{appendix:proofs:double_pendulum_dynamics}

% https://diego.assencio.com/?index=1500c66ae7ab27bb0106467c68feebc6

It is convenient\footnote{Diego Assencio: The double pendulum - Lagrangian formulation~\url{https://diego.assencio.com/?index=1500c66ae7ab27bb0106467c68feebc6}} to describe a double pendulum system in terms of angles $\theta_1$ and
  $\theta_2$ between roods and their respective angular velocities $\dot{\theta}_1$ and
  $\dot{\theta}_2$, see Figure~\ref{fig:sim:double_pendulum_notation}.
In the Cartesian coordinate system the position of the bobs of the pendulums involved is then
  written as
  \begin{equation}
    \label{eq:appendix:double_pendulum_bobs}
    \begin{aligned}[c] x_1 &
                 = l_1 \sin(\theta_1) \\ x_2 & = l_1 \sin(\theta_1) + l_2\sin(\theta_2) \\
    \end{aligned}
    \qquad
    \begin{aligned}[c] y_1 & = -l_1 \cos(\theta_1) \\ y_2 & = -l_1\cos(\theta_1) -
                 l_2\cos(\theta_2)               \\
    \end{aligned}
  \end{equation}
  where $l_1$ and $l_2$ are length of the
  corresponding rods.
The velocities of the bobs are \begin{equation}
  \begin{aligned}[c] \dot{x}_1 & = l_1
                 \dot{\theta}_1\cos(\theta_1) \\ \dot{x}_2 & = l_1 \dot{\theta}_1\cos(\theta_1) +
                    l_2\dot{\theta}_2\cos{\theta_2}
  \end{aligned}
  \qquad \begin{aligned}[c] \dot{y}_1 & = l_1
  \dot{\theta}_1\sin(\theta_1) \\ \dot{y}_2 & = l_1 \dot{\theta}_1\sin(\theta_1) +
  l_2\dot{\theta}_2\sin{\theta_2}.
\end{aligned}
\end{equation}

The Lagrangian for the double pendulum is given by $L = E - P$, where $E$ and $P$ are the
  kinetic and potential energies of the system respectively.
The kinetic energy $E$ is given by \begin{equation}
  \label{eq:appendix:double_pendulum_kinetic_energy} \begin{aligned} T & =
  \frac{1}{2}m_1(\dot{x}_1^2 + \dot{y}_1^2) + \frac{1}{2}m_2(\dot{x}_2^2 + \dot{y}_2^2) \\ & =
  \frac{1}{2}m_1l_1^2\dot{\theta}_1^2 + \frac{1}{2}m_2\left[ l_1^2\dot{\theta}_1^2 +
    l_2^2\dot{\theta}_2^2 + 2 l_1 l_2 \dot{\theta}_1 \dot{\theta}_2 \cos(\theta_1 - \theta_2)
    \right].
\end{aligned}
\end{equation}
The potential energy is given by \begin{equation} \begin{aligned} P &= -m_1 g
  l_1\cos(\theta_1) - m_2 g \left( l_1 \cos(\theta_1) + l_2 \cos(\theta_2) \right) \\ &= -(m_1 +
  m_2)g l_1 \cos(\theta_1) - m_2 g l_2 \cos(\theta_2).
\end{aligned}
\end{equation}

The canonical momenta associated with the coordinates $\theta_1$ and $\theta_2$ can be
  obtained directly from $L$ \begin{equation} \begin{aligned} p_{\theta_1} &= \frac{\partial
    L}{\partial \dot{\theta}_1} = (m_1 + m_2)l_1^2 \dot{\theta}_1 + m_2 l_1 l_2 \dot{\theta}_2
  \cos(\theta_1 - \theta_2) \\ p_{\theta_2} &= \frac{\partial L}{\partial \dot{\theta}_2} = m_2
  l_2^2 \dot{\theta}_2 + m_2 l_1 l_2 \dot{\theta}_1 \cos(\theta_1 - \theta_2).
\end{aligned}
\end{equation}

The equations of motion of the system are the Euler-Lagrange equations:
  \begin{equation}\label{eq:appendix:double_pendulum_euler_lagrange}
  \frac{\mathrm{d}}{\mathrm{d}t} \left( \frac{\partial L}{\partial \dot{\theta_i}} \right) -
  \frac{\partial L}{\partial \theta_i} = 0 \Rightarrow \frac{\mathrm{d}p_{\theta_i}}{\mathrm{d}
  t} - \frac{\partial L}{\partial \theta_i} = 0 \qquad i = 1,2.
\end{equation}

After substituting
  \begin{equation}
    \begin{aligned}
      \frac{\mathrm{d} p_{\theta_1}}{\mathrm{d}
      t}                          & = (m_1 + m_2)l_1^2 \ddot{\theta}_1 + m_2 l_1 l_2 \ddot{\theta}_2 \cos(\theta_1 -
      \theta_2)                                                                                                      \\  & \qquad - m_2 l_1 l_2 \dot{\theta}_1 \dot{\theta}_2 \sin(\theta_1 - \theta_2) +
      m_2 l_1 l_2 \dot{\theta}_2^2 \sin(\theta_1 - \theta_2)                                                         \\ \frac{\mathrm{d}
      p_{\theta_2}}{\mathrm{d} t} & = m_2 l_2^2 \ddot{\theta}_2 + m_2 l_1 l_2
      \ddot{\theta}_1\cos(\theta_1 - \theta_2)                                                                       \\  & \qquad - m_2 l_1 l_2
         \dot{\theta}_1^2\sin(\theta_1 - \theta_2) + m_2 l_1 l_2 \dot{\theta}_1
      \dot{\theta}_2\sin(\theta_1 - \theta_2)                                                                        \\ \frac{\partial L}{\partial \theta_1} & = -m_2 l_1
         l_2 \dot{\theta}_1 \dot{\theta}_2 \sin(\theta_1 - \theta_2) - (m_1 + m_2) g l_1 \sin\theta_1
      \\ \frac{\partial L}{\partial \theta_2} & = m_2 l_1 l_2 \dot{\theta}_1 \dot{\theta}_2
         \sin(\theta_1 - \theta_2) - m_2 g l_2 \sin\theta_2
    \end{aligned}
  \end{equation}
  into the
  equation~\eqref{eq:appendix:double_pendulum_euler_lagrange} we get
  \begin{equation}\label{eq:appendix:double_pendulum_second_order} \begin{aligned} 0 & = (m_1 +
  m_2)l_1 \ddot{\theta}_1 + m_2 l_2 \ddot{\theta}_2 \cos(\theta_1 - \theta_2) \\ & \qquad + m_2
  l_2 \dot{\theta}_2^2\sin(\theta_1 - \theta_2) + (m_1 + m_2) g \sin\theta_1 \\ 0 & = l_2
  \ddot{\theta}_2 + l_1 \ddot{\theta}_1 \cos(\theta_1 - \theta_2) - l_1 \dot{\theta}_1^2
  \sin(\theta_1 - \theta_2) + g \sin \theta_2.
\end{aligned}
\end{equation}
The equations~\eqref{eq:appendix:double_pendulum_second_order} form a system of coupled
  second-order non-linear differential equations and can be rewritten as
  \begin{equation}
    \label{eq:appendix:double_pendulum_second_order_2}
    \begin{aligned}
      \ddot{\theta}_1 + \alpha_1
      \ddot{\theta}_2 & = \beta_1 \\ \ddot{\theta}_2 + \alpha_1 \ddot{\theta}_1 & = \beta_2,
    \end{aligned}
  \end{equation}
  where \begin{equation} \begin{aligned} \alpha_1 &:=
  \frac{l_2}{l_1} \left( \frac{m_2}{m_1 + m_2} \right)\cos(\theta_1 - \theta_2) \\ \alpha_2 &:=
  \frac{l_1}{l_2} \cos(\theta_1 - \theta_2) \\ \beta_1 &:= -\frac{l_2}{l_1} \left(
  \frac{m_2}{m_1 + m_2} \right) \dot{\theta}_2^2 \sin(\theta_1 - \theta_2) - \frac{g}{l_1}
  \sin\theta_1 \\ \beta_2 &:= \frac{l_1}{l_2} \dot{\theta}_1^2 \sin(\theta_1 - \theta_2) -
  \frac{g}{l_2}\sin\theta_2.
\end{aligned}
\end{equation}
The equations~\eqref{eq:appendix:double_pendulum_second_order_2} can be combined into a single
  equation
  \begin{equation}
    \label{eq:appendix:double_pendulum_operator_form} A
    \begin{pmatrix}
      \ddot{\theta}_1 \\ \ddot{\theta}_2
    \end{pmatrix}
    =
    \begin{pmatrix}
      1 & \alpha_1 \\ \alpha_2 &
         1
    \end{pmatrix}
    \begin{pmatrix}
      \ddot{\theta}_1 \\ \ddot{\theta}_2
    \end{pmatrix}
    =
    \begin{pmatrix}
      \beta_1 \\ \beta_2
    \end{pmatrix}
  \end{equation}
  where matrix $\mathrm{det}(A)
    = 1 - \alpha_1 \alpha_2 = 1 - (\nicefrac{m_2}{m_1 + m_2})\cos^2(\theta_1 - \theta_2) > 0$
  because $\nicefrac{m_2}{m_1 + m_2} < 1$ and $\cos^2(\theta_1 - \theta_2) \leq 1$ therefore
  \begin{equation} A^{-1} = \frac{1}{\mathrm{det}(A)}
  \begin{pmatrix}
    1 & -\alpha_1 \\ -\alpha_2
      & 1
  \end{pmatrix}
  = \frac{1}{1 - \alpha_1\alpha_2}
  \begin{pmatrix}
    1 & -\alpha_1 \\ -\alpha_2
      & 1
  \end{pmatrix}
  .
\end{equation}

Finally, we rewrite~\eqref{eq:appendix:double_pendulum_operator_form} into
  \begin{equation}
    \begin{pmatrix}
      \ddot{\theta}_1 \\ \ddot{\theta}_2
    \end{pmatrix}
    = A^{-1}
    \begin{pmatrix}
      \beta_1 \\ \beta_2
    \end{pmatrix}
    = \frac{1}{1 - \alpha_1 \alpha_2}
    \begin{pmatrix}
      \beta_1 -
      \alpha_1\beta_2 \\ -\alpha_2\beta_1 + \beta_2
    \end{pmatrix}
    ,
  \end{equation}
  which can be
  represented as a system of coupled first order differential equations
  \begin{equation}\label{eq:appendix:double_pendulum_final} \frac{\mathrm{d}}{\mathrm{d}t}
  \begin{pmatrix}
    \theta_1 \\ \theta_2 \\ \dot{\theta}_1 \\ \dot{\theta}_2
  \end{pmatrix}
  =
  \begin{pmatrix}
    \dot{\theta}_1     \\ \dot{\theta}_2 \\ \nicefrac{\beta_1 - \alpha_1 \beta_2}{1 -
    \alpha_1 \alpha_2} \\ \nicefrac{-\alpha_2 \beta_1 + \beta_2}{1 - \alpha_1 \alpha_2}
  \end{pmatrix}
  .
\end{equation}

Similarly to~\ref{appendix:proofs:car_dynamics}, by a suitable numerical integration scheme the~\eqref{eq:appendix:double_pendulum_final} can be discretized and converted to a (discrete) nonlinear dynamical system~\eqref{eq:sim:nlds} where $s_t = (\theta_1, \theta_2, \dot{\theta}_1,
    \dot{\theta}_2)_t$ is the state of the system at time $t$ and the state-transition function $f$ depends on the chosen numerical integration scheme (e.g. Runge-Kutta) \citep[Chapter~8]{hasselblatt_handbook_2002}. The measurement function $g$ can be any function. 


