\section{Conclusions}\label{chapter-03:section:conclusion}

In this chapter, we have presented the methods and implementation aspects of the \ac{rmp} framework, both for exact and approximate Bayesian inference, based on the minimization of the \ac{cbfe} functional.

The main contribution of this chapter is to explore the ideas of the reactive programming
paradigm and apply them to message passing-based Bayesian inference.
The resulting framework not only supports continual and adaptable inference but also attempts
to solve issues associated with fixed message passing schedules, including:

\begin{itemize}
  \item \textbf{Lazy computations}. A fixed global schedule
        does not support executing operations "lazily" and on-demand in different parts of the graph
        as soon as data arrive.
        However, the reactive programming paradigm is grounded on the ideas of lazy computations and
        performs all operations lazily and on demand.

  \item \textbf{Robust operability}.
        In many signal processing applications, robustness against missing data from a failed sensor
        is essential.
        A failing sensor would imply a model structure update that would require a temporary system
        reset to recompute an appropriate global message passing schedule.
        However, a reactive system does not need this reset since it does not rely on a message
        passing schedule that is tightly linked to the model structure.

  \item \textbf{Module adaptation}.
        Executing an optimal message passing schedule to support dynamic model adaptation is
        practically difficult.
        Real-time Bayesian inference often requires updating the model structure on the fly without
        interrupting the inference process as more data becomes available.
        The locality of the \ac{cbfe} procedure and the absence of the global message passing schedule in
        RMP makes it trivial to replace one part of the graph without stopping the inference in other
        parts.

  \item \textbf{Unpredictable or different update rates in multiple sensor data streams}.
        The reactive nature of the resulting procedure does not make any assumptions about the nature
        of the data generation process and naturally supports asynchronous streams from multiple
        sensor data streams with different update rates.

  \item \textbf{Which schedule is optimal?}.
        \Ac{rmp} does not expose any predefined schedule and simply reacts to data and sends messages as
        soon as possible.
        The actual order of message computations becomes data-dependent.
        This eliminates the need to fix a schedule a priori, considering that the world may change and
        is always to some degree unpredictable.

\end{itemize}

The tight connection of \ac{rmp} with the \ac{cbfe} procedure makes it possible to execute hybrid
algorithms with \ac{bp}, \ac{ep}, \ac{em}, and \ac{vmp} analytical message update rules, supporting local
factorization and form constraints on the variational family.

In the next chapter, we present our implementation of the \ac{rmp} framework in the form of a
software package for the Julia programming language called RxInfer.jl.
