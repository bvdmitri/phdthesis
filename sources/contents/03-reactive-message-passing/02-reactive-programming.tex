\section{Reactive programming}\label{chapter-03:section:reactive-programming}

This section introduces the fundamental concepts of the \acf{rp} paradigm,
including observables, subscriptions, actors, and operators.
\ac{rp} offers guidelines and ideas to simplify the handling of asynchronous streams of data and
events.
Understanding these concepts and their notation is essential for understanding the \ac{rmp}
framework.

Many modern programming languages support \ac{rp} through additional libraries and
packages\footnote{ReactiveX programming languages support
  \url{http://reactivex.io/languages.html}}.
For example, the JavaScript community has a popular RxJS library\footnote{JavaScript reactive
  extensions library \url{https://github.com/ReactiveX/rxjs}}, 
  and the Python community has its own RxPY library\footnote{Python reactive
  extensions library \url{https://github.com/ReactiveX/RxPY}}.

It is essential to note that there is no formal specification of \ac{rp} ideas, and naming
conventions and realizations may vary among different communities.
The objective of this section is not to provide a formal definition of \ac{rp}, but rather to offer
a straightforward description of the concepts that will be particularly useful for discussing
the \ac{rmp} framework later on.

\subsection{Observables}

The fundamental concept of the reactive programming paradigm is the use of
\textit{observables} to replace variables in programming languages.
Observables can be described as \textit{lazy push collections}, while conventional data
structures such as arrays, lists or dictionaries are considered \emph{pull} collections.
The term "pull" refers to the ability of users to directly request the state of these data
structures and "pull" their values.
In contrast, observables have a temporal aspect and may not have an immediate associated
value.
Instead, they \textit{push} (or \textit{emit}) their values over time in the future.
Observables do not allow direct inquiries about their state; instead, users can only
\textit{subscribe} to their \textit{updates} (or \textit{events}).
The term \textit{lazy} indicates that an observable typically does not produce any updates and
does not consume computing resources unless someone subscribes to its updates.
Observables do not impose any assumptions about the data generation process, update timings,
or durations, allowing flexibility in the time intervals between updates.
In general, observables can be seen as a generalization of regular pull collections.
For instance, a regular array can be treated as an observable that pushes all its data
simultaneously at the same time.

We represent an observable of a variable $x$ as $\obs{x}$, and a visual representation of the
observables in the context of reactive programming is shown in
Figure~\ref{fig:rmp:reactive_observable}.
Observables have the capability to emit various types of data, including integers, floats,
functions, or even other observables.
If an observable emits functions, like a probability density function, instead of simple
values, we use the notation $\obs{f}(x)$.
This notation indicates that the observable emits functions of $x$, not that it is a function
of $x$.

Observables can potentially produce an infinite number of updates, generating data lazily one
by one while utilizing a finite amount of computing resources.
For example, an observable might emit a number every second after a specific past event, or a sensor could deliver a speech signal as updates.
However, there is the possibility that an observable fails to generate an update and instead
sends an error.
An example of this would be a failing sensor that cannot no longer provide its data due to an
unexpected malfunction.
Observables can have a termination state.
After emitting all the desired updates or encountering an error, an observable might reach a
terminal state where it no longer produces any new data.

\begin{figure}
  \centering
  %\hspace{\fill}
  \begin{subfigure}[t]{.45\textwidth}
    \resizebox{\textwidth}{!}{
\begin{tikzpicture}[]
  %Nodes
  \node[] (left) {};
  \node[] (right) [right=of left, xshift=50mm] {};

  %Lines
  \path[line, densely dotted] (left) edge[-stealth] 
    node[pos=0.0, anchor=south]{$\obs{x}$} 
    node[pos=0.2, solid, roundbox, fill=white]{$1$} 
    node[pos=0.4, solid, roundbox, fill=white]{$2$} 
    node[pos=0.7, solid, roundbox, fill=white]{$3$} 
    node[pos=0.8, solid, roundbox, fill=white]{$4$} 
    node[pos=0.95]{$\vert$} 
  (right);
\end{tikzpicture}
}
    \caption{An observable emitting primitive (integer) values with an unspecified time interval.}
    \label{fig:rmp:reactive_observable_raw}
  \end{subfigure}
  \hspace{\fill}
  \begin{subfigure}[t]{.45\textwidth}
    \resizebox{\textwidth}{!}{
\begin{tikzpicture}[]
  %Nodes
  \node[] (left) {};
  \node[] (right) [right=of left, xshift=50mm] {};

  %Lines
  \path[line, densely dotted] (left) edge[-stealth] 
    node[pos=0.0, anchor=south]{$\obs{f}(x)$} 
    node[pos=0.2, solid, roundbox, fill=white]{$f_1$} 
    node[pos=0.4, solid, roundbox, fill=white]{$f_2$} 
    node[pos=0.7, solid, roundbox, fill=white]{$f_3$} 
    node[pos=0.8, solid, roundbox, fill=white]{$f_4$} 
    node[pos=0.95]{$\vert$} 
  (right);

\end{tikzpicture}
}
    \caption{An observable emitting functions of $x$ with an unspecified time interval.}
    \label{fig:rmp:reactive_observable_functions}
  \end{subfigure}
  %\hspace{\fill}
  \vspace{\fill}
  \\[3.5mm]
  \begin{subfigure}[t]{.45\textwidth}
    \resizebox{\textwidth}{!}{
\begin{tikzpicture}[]
  %Nodes
  \node[] (left) {};
  \node[] (right) [right=of left, xshift=50mm] {};

  %Lines
  \path[line, densely dotted] (left) edge[-stealth] 
    node[pos=0.0, anchor=south]{$\obs{x}$} 
    node[pos=0.15, solid, roundbox, fill=white]{$1$} 
    node[pos=0.30, solid, roundbox, fill=white]{$2$} 
    node[pos=0.45, solid, roundbox, fill=white]{$3$} 
    node[pos=0.60, solid, roundbox, fill=white]{$4$} 
    node[pos=0.75, solid, roundbox, fill=white]{$5$} 
    node[pos=0.90, solid, roundbox, fill=white]{\scriptsize $\cdot\!\cdot\!\cdot$}
  (right);

\end{tikzpicture}
}
    \caption{An observable emitting an infinite number of (integer) values with a specified time interval.}
    \label{fig:rmp:reactive_observable_infinite}
  \end{subfigure}
  \hspace{\fill}
  \begin{subfigure}[t]{.45\textwidth}
    \resizebox{\textwidth}{!}{
\begin{tikzpicture}[]
  %Nodes
  \node[] (left) {};
  \node[] (right) [right=of left, xshift=50mm] {};

  %Lines
  \path[line, densely dotted] (left) edge[-stealth] 
    node[pos=0.0, anchor=south]{$\obs{x}$} 
    node[pos=0.2, solid, roundbox, fill=white]{$1$} 
    node[pos=0.4, solid, roundbox, fill=white]{$2$} 
    node[pos=0.6, solid, roundbox, fill=white!80!red]{} 
  (right);

\end{tikzpicture}
}
    \caption{An observable that fails after a certain amount of time.}
    \label{fig:rmp:reactive_observable_failing}
  \end{subfigure}
  %\hspace{\fill}
  \caption{A visual representation of observable collections.
    An arrow represents a timeline.
    White circles denote updates at specific points on that timeline.
    Values inside white circles indicate the corresponding data of the updates.
    Red circles denote errors at specific points on that timeline.
    The bar at the end of the timeline indicates the termination event after which the observable
    stops sending new updates.
    In general, observables do not make any assumptions about when an update might happen, and the
    time duration between two updates may vary.
  }
  \label{fig:rmp:reactive_observable}
\end{figure}

\subsection{Actors}

An actor is a special computational unit that defines \textit{actions} for handling new
updates from an observable and specifies what to do in case of a termination event or an error.
Actors are often referred to as \textit{subscribers} (or \textit{listeners}). Actors are a central idea in the concept called the \textit{Actor Model} \citep{hewitt_actor_model}.
In the context of \ac{rp}, an actor can only listen for new updates and perform actions based on incoming updates, but, normally, they cannot manipulate the state of an observable.
However, actors are allowed to send or redirect updates from one observable to other actors in
the reactive system, or even create new observables from existing ones.
Using actors in the context of \ac{rp} offers several benefits and advantages:
\begin{itemize}
  \item \textbf{Modularity and encapsulation}. Actors provide a modular and
        encapsulated way of handling updates and actions.
        Each actor can be designed to perform a specific task or handle a particular type of update,
        making the overall system more organized and maintainable.
  \item \textbf{Concurrency and parallelism}.
        Actors can be designed to work independently and concurrently.
        This allows for efficient use of resources and can lead to improved performance and
        responsiveness in systems with multiple actors running in parallel.
  \item \textbf{Scalability}.
        Reactive systems based on actors can be easily scaled by adding more actors to handle increased workloads.
        This scalability is essential for handling large-scale real-time applications.
  \item \textbf{Event-driven behavior}.
        Actors are inherently event-driven, responding to updates from observables or messages from other actors.
        This event-driven behavior aligns well with the asynchronous nature of \ac{rp} and enables efficient handling of data streams and events.
  \item \textbf{Error isolation}.
        Actors can handle errors locally, without affecting the rest of the system.
        If one actor encounters an error, it can take the appropriate action without disrupting other actors.
  \item \textbf{Fault tolerance}.
        In a reactive system, actors can be designed to recover from errors or failures gracefully.
        The system can be designed to handle temporary failures and continue functioning without complete failure.
  \item \textbf{Reusability}.
        Actors can be reused in different parts of the system or even in entirely different systems,
        promoting code reusability and reducing the development effort.
  \item \textbf{Ease of testing}.
        Testing individual actors is generally easier compared to testing an entire monolithic system.
        The modularity and isolation of actors simplify unit testing, making it easier to ensure the correctness of each actor's behavior.
\end{itemize}

\subsection{Subscriptions}

In the context of \ac{rp}, actors utilize the \textit{subscription} mechanism to
initiate or terminate listening to updates from observables.
A visual representation of subscriptions in the reactive programming context is illustrated in
Figure~\ref{fig:rmp:reactive_subscription}.

Each subscription establishes an independent observable execution and consumes a finite amount
of computing resources.
Subscriptions play a critical role in the lazy evaluation of observables because they trigger
the actual execution of an observable.
By default, an observable remains inert, performing no computation, producing no data, and
consuming no computer resources if no actor has subscribed to it.
Consequently, it becomes vital to monitor and manage existing subscriptions within a reactive
system, promptly unsubscribing from observables when the corresponding data is no longer
required.

Using subscriptions in the context of reactive programming offers several benefits:

\begin{itemize}
  \item \textbf{Lazy evaluation}. 
        Subscriptions enable lazy evaluation of observables.
        An observable does not execute any computation, consume computing resources, or produce data until an actor subscribes to it.
  \item \textbf{Efficient resource usage}.
        By managing subscriptions properly, a reactive system can avoid unnecessary computations and
        memory usage. When an actor no longer needs updates from an observable, it can unsubscribe, freeing up
        resources that would otherwise be wasted on processing irrelevant data.
  \item \textbf{Dynamic reactivity}.
        Subscriptions enable dynamic reactivity to changes in the data or event streams.
        Actors can subscribe or unsubscribe based on specific conditions or events, adapting the
        system's behavior in response to real-time changes.
\end{itemize}

\begin{figure}
  \centering
  \resizebox{\textwidth}{!}{

\begin{tikzpicture}[]
  %Nodes
  \node[] (left) {};
  \node[] (right) [right=of left, xshift=80mm] {};

  %Lines
  \path[line, densely dotted] (left) edge[-stealth] 
    node[pos=0.0, anchor=south]{$\obs{x}$} 
    node[pos=0.2, solid, roundbox, fill=white](d1){$1$} 
    node[pos=0.3, solid, roundbox, fill=green!10!white, inner sep=0.3mm, minimum size=0mm](sstart){}
    node[pos=0.4, solid, roundbox, fill=white](d2){$2$} 
    node[pos=0.6, solid, roundbox, fill=green!10!white, inner sep=0.3mm, minimum size=0mm](send){}
    node[pos=0.7, solid, roundbox, fill=white](d3){$3$} 
    node[pos=0.8, solid, roundbox, fill=white](d4){$4$} 
    node[pos=0.95]{$\vert$} 
  (right);

  \node[] (sstartb) [below=of sstart, yshift=6mm, inner sep=0mm, minimum size=0mm] {};
  \node[] (sendb) [below=of send, yshift=6mm, inner sep=0mm, minimum size=0mm] {};

  \path[line, densely dotted] (sstart) edge[-] (sstartb);
  \path[line, densely dotted] (sstartb) edge[-] 
    node[pos=0.5, anchor=north](subscription){\small subscription}
  (sendb);
  \path[line, densely dotted] (sendb) edge[-] (send);

  \node[label={[xshift=2mm, yshift=-1mm]above left:{actor}}] (actor) [below=of d4, yshift=5mm, xshift=6mm] {\framebox{\small
    {\begin{varwidth}{\linewidth}
      what to do\\[-4mm]
      \begin{itemize}[leftmargin=5.5mm, itemsep=2pt,parsep=2pt]
        \setlength\itemsep{0mm}
        \item when data arrives
        \item when error happens
        \item when observable completes
    \end{itemize}\end{varwidth}}
  }};

      \path[line] (subscription) |- (actor);
\end{tikzpicture}
}
  \caption{The subscription happens at a specific point in time and allows actors to receive new values.
    In this example, an actor would receive only update \textcircled{\raisebox{-0.9pt}{2}}, since
    the subscription was executed after update \textcircled{\raisebox{-0.9pt}{1}} but was
    terminated before updates \textcircled{\raisebox{-0.9pt}{3}} and
    \textcircled{\raisebox{-0.9pt}{4}}.
  }
  \label{fig:rmp:reactive_subscription}
\end{figure}

%\subsection{Subject}

%A subject is a special type of actor that receives an update and simultaneously re-emits the
%same update to multiple actors.
%In other words, a subject utilizes a single subscription to some observable and replicates
%updates from that observable to multiple subscribers.
%This process of re-emitting all incoming updates is called \textit{multicasting}.
%One of the goals of a subject is to share the same observable execution and, therefore, save
%computer resources.
%A subject is effectively an actor and an observable at the same time.
%Subjects are particularly useful if some one wants to utilize a single subscription for
%multiple actors and thus save computer resources.

\subsection{Operators}

An operator is a function that takes one or more observables as its input and returns another
observable.
Operators represent simple building blocks of the \ac{rp} paradigm and are used to combine several
observables or transform and adjust their behavior.
A good analogy would be Lego pieces, which are very simple by themselves, but when
combined, they allow for the construction of complex, large, and robust systems.
It is worth noting that an operator is always a pure operation, which means that the input
observables remain unmodified.

Using operators in the context of reactive programming offers several benefits and advantages:
\begin{itemize}
  \item \textbf{Data transformation}. Operators enable the transformation of data emitted by observables, allowing           developers to manipulate, filter, aggregate, or modify data streams efficiently.
  \item \textbf{Composition}.
        Operators can be easily composed together, enabling developers to build sophisticated data flows by chaining multiple operators in a pipeline. This makes it convenient to express complex logic in a concise and readable manner.
  \item \textbf{Event synchronization}.
        Operators enable synchronization of multiple observable streams, facilitating coordination between different events and ensuring data consistency.
  \item \textbf{Performance optimization}.
        Some operators, such as buffering or debouncing, can help optimize performance by reducing the number of emitted events or batching them efficiently.
\end{itemize}

Next, we discuss some examples of operators that serve as essential building blocks of the
\ac{rmp} framework.

\subsubsection{The \texttt{map} operator}

The first essential operator is the \texttt{map} operator, which returns a new modified
observable.
It applies a given \textit{mapping function} $\mathcal{M}$ to each value emitted by a source
observable, see Figure~\ref{fig:rmp:reactive_operator_map}.
When we apply the \texttt{map} operator with a mapping function $\mathcal{M}$ to an
observable, we say that this observable is \textit{mapped} by the function $\mathcal{M}$.

%\dmitry{Do i need these listings?}
%\begin{figure}
%\begin{adjustbox}{minipage=\textwidth,margin=0pt \smallskipamount,center}
%\begin{jllisting}[caption={An example of the \texttt{map} operator application.
%We apply the \texttt{map} operator with $\mathcal{M}(x) = x^2$ to obtain a new observable that
%emits squared values from the original observable.
%The original observable remains unmodified.
%}, label={lst:map_example}, captionpos=b]
%# We create the mapped observable of squared values
%# of the original `source` observable
%source         = get_source()
%squared_source = map(source, x -> x^2)
%\end{jllisting}
%\end{adjustbox}
%\end{figure}

\begin{figure}
  \centering
  \resizebox{\textwidth}{!}{
\begin{tikzpicture}[]
  %Nodes
  \node[] (left) {};
  \node[] (right) [right=of left, xshift=50mm] {};

  %Lines
  \path[line, densely dotted] (left) edge[-stealth]
  node[pos=0.0, anchor=south]{$\obs{x}$}
  node[pos=0.2, solid, roundbox, fill=white](d1){$1$}
  node[pos=0.4, solid, roundbox, fill=white](d2){$2$}
  node[pos=0.7, solid, roundbox, fill=white](d3){$3$}
  node[pos=0.8, solid, roundbox, fill=white](d4){$4$}
  node[pos=0.95]{$\vert$}
  (right);

    %Nodes
  \node[] (cleft) [below=of left, yshift=8mm]{};
  \node[] (cright) [right=of cleft, xshift=50mm] {};

  %Lines
  \path[line, color=white] (cleft) edge[-stealth]
  node[pos=0.0, anchor=east]{}
  node[pos=0.2, inner sep=0mm, minimum size=0mm](c1){}
  node[pos=0.4, inner sep=0mm, minimum size=0mm](c2){}
  node[pos=0.7, inner sep=0mm, minimum size=0mm](c3){}
  node[pos=0.8, inner sep=0mm, minimum size=0mm](c4){}
  node[pos=0.95, inner sep=0mm, minimum size=0mm]{}
  (cright);

  %Nodes
  \node[] (mappedleft) [below=of cleft, yshift=8mm] {};
  \node[] (mappedright) [right=of mappedleft, xshift=50mm] {};

  %Lines
  \path[line, densely dotted, color=red!80!black] (mappedleft) edge[-stealth]
  node[pos=0.0, anchor=south]{$\obs{2x}$}
  node[pos=0.2, solid, roundbox, fill=white](m1){$2$}
  node[pos=0.4, solid, roundbox, fill=white](m4){$4$}
  node[pos=0.7, solid, roundbox, fill=white](m9){$6$}
  node[pos=0.8, solid, roundbox, fill=white](m16){$8$}
  node[pos=0.95]{$\vert$}
  (mappedright);

  \path[line, densely dotted, black] (d1) edge[-] (c1);
  \path[line, densely dotted, color=red!80!black] (c1) edge[-] (m1);
  \path[line, densely dotted, black] (d2) edge[-] (c2);
  \path[line, densely dotted, color=red!80!black] (c2) edge[-] (m4);
  \path[line, densely dotted, black] (d3) edge[-] (c3);
  \path[line, densely dotted, color=red!80!black] (c3) edge[-] (m9);
  \path[line, densely dotted, black] (d4) edge[-] (c4);
  \path[line, densely dotted, color=red!80!black] (c4) edge[-] (m16);

  \node[] (maplabeltext) [above=of left, xshift=30mm, yshift=-7mm, inner sep=0mm, minimum size=4mm, fill=white] {
  \scriptsize map($\obs{x}$, \,\circled{$x$}$\,\,\rightarrow\,\,$\darkredcircled{$2x$}\,\,)
  };
\end{tikzpicture}
}
  \caption{The \texttt{map} operator takes an observable and a mapping function $\mathcal{M}$ as arguments, creating a new observable that mirrors the original observable but with transformed values using the provided mapping function.
    The application of the \texttt{map} operator is pure, and the original observable remains
    unmodified.
    The new observable emits updates at the same points in time as the original observable.
    In this example, the original observable is a sequence of integer values, and the mapping
    function multiplies each integer by a factor of 2.
  }
  \label{fig:rmp:reactive_operator_map}
\end{figure}

\subsubsection{The \texttt{combineLatest} operator}

The second essential operator is the \texttt{combineLatest} operator.
It combines multiple observables to create a new observable whose values are calculated from
the latest values of each of the original input observables, see
Figure~\ref{fig:rmp:reactive_operator_combine_latest}.
The \texttt{combineLatest} operator has different update strategies that specify when the
resulting observable should emit new updates.
For example, one strategy could be to emit a new update only after all input observables have
been updated with a new value, see Figure~\ref{fig:rmp:reactive_operator_combine_latest_push_all}.
The second strategy could be to emit a new value any time any of the input observables has
been updated and to reuse the latest cached values for the remaining input observables, see
Figure~\ref{fig:rmp:reactive_operator_combine_latest_push_each}.
The first strategy is useful when we want to wait for all the combined observables to have new
data before performing some computation.
The second strategy is helpful when we want to perform calculations or operations on the
latest values from different observables as soon as any one of them updates.
The \texttt{combineLatest} operator stores a snapshot of only the latest values in all input
observables and does not store subsequent previous updates in case one of the inner
observables is delayed (unbounded buffering is possible with the \texttt{zip} operator
instead).

%\begin{figure*}[ht!]
%\begin{adjustbox}{minipage=\textwidth,margin=0pt \smallskipamount,center}
%\begin{jllisting}[caption={An example of the \texttt{combineLatest} operator application.
%We use the \texttt{combineLatest} operator to combine the latest values from two integer
%streams and additionally apply the \texttt{map} operator with $\mathcal{M}(x_1, x_2) = x_1^2 +
%x_2^2$.
%The resulting observable emits the sum of squared values of $x_1$ and $x_2$ as soon as both of
%them emit a new value.
%}, label={lst:combine_latest_example}, captionpos=b]
%# We assume that `source1` and `source2` are both observables of integers
%source1 = get_source1()
%source2 = get_source2()

%# We create a new observable by applying a combineLatest operator to it
%combined = combineLatest(source1, source2)

%# We can go further and apply a `map` operator to the combined observable
%# and create a stream of the sum of squares of the latest values
%# from `source1` and `source2`
%combined_sum = map(combined, (x1, x2) -> x1^2 + x2^2)
%\end{jllisting}
%\end{adjustbox}
%\end{figure*}

\begin{figure}
  \centering
  \hspace{\fill}%
  \begin{subfigure}[t]{.45\textwidth}
    \resizebox{\textwidth}{!}{
\begin{tikzpicture}[]
  %Nodes
  \node[] (tleft) {};
  \node[] (tright) [right=of tleft, xshift=50mm] {};
  \node[] (bleft) [below=of tleft, yshift=5mm] {};
  \node[] (bright) [right=of bleft, xshift=50mm] {};

  %Lines
  \path[line, densely dotted] (tleft) edge[-stealth]
  node[pos=0.0, anchor=south]{$\obs{x}$}
  node[pos=0.1, solid, roundbox, fill=white]{$1$}
  node[pos=0.8, solid, roundbox, fill=white]{$2$}
  %node[pos=0.95]{$\vert$}
  (tright);

  \path[line, densely dotted] (bleft) edge[-stealth]
  node[pos=0.0, anchor=south]{$\obs{y}$}
  node[pos=0.3, solid, roundbox, fill=white]{$3$}
  node[pos=0.5, solid, roundbox, fill=white]{$4$}
  (bright);

  \node[box, inner ysep=4.5mm, densely dotted, color=red!80!black, rounded corners=1mm] (clbbox) [fit=(tleft)(bright)] {};

  \node[node distance=1mm] (cllabel) [above=of clbbox] {combineLatest($\obs{x}$, $\obs{y}$)};
  %\node[node distance=1mm, xshift=3mm, color=red!80!black] (cllabel2) [left=of cllabel] {$(\obs{x}, \obs{y})$};

  \node[] (cleft) [below=of bleft] {};
  \node[] (cright) [right=of cleft, xshift=50mm] {};

  \path[line, densely dotted, color=red!80!black] (cleft) edge[-stealth]
  node[pos=0.0, anchor=south]{$(\obs{x}, \obs{y})$}
  node[pos=0.3, box, solid, rounded corners=3.5mm, inner xsep=2mm, fill=white]{$(1, 3)$}
  node[pos=0.8, box, solid, rounded corners=3.5mm, inner xsep=2mm, fill=white]{$(2, 4)$}
  %node[pos=0.95]{$\vert$}
  (cright);
\end{tikzpicture}
}
    \caption{The \texttt{combineLatest} operation that emits updates only after all input observables have been updated with a new value.}
    \label{fig:rmp:reactive_operator_combine_latest_push_all}
  \end{subfigure}
  \hspace{\fill}%
  \begin{subfigure}[t]{.45\textwidth}
    \resizebox{\textwidth}{!}{\input{contents/03-reactive-message-passing/figs/02-reactive_operator_combine_latest_push_each.tex}}
    \caption{The \texttt{combineLatest} operation that emits updates any time any of the input observables has been updated and reuses the latest values from the remaining input observables.}
    \label{fig:rmp:reactive_operator_combine_latest_push_each}
  \end{subfigure}
  \hspace{\fill}%
  \caption{
    The \texttt{combineLatest} operator combines two or more source observables into a single one and emits a combination of the latest values from all inner source observables.
    The update behavior depends on the chosen update strategy.
  }
  \label{fig:rmp:reactive_operator_combine_latest}
\end{figure}

The \ac{rp} paradigm includes many small and basic operators, such as
\texttt{filter}, \texttt{count}, and \texttt{start\_with}.
However, the \texttt{map} and \texttt{combineLatest} operators are the foundation of the \ac{rmp}
framework and will be particularly useful for further discussions.

