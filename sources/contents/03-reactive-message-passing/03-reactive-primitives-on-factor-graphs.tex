\section{Reactive primitives for message passing on factor graphs}\label{chapter-03:section:reactive-primitives-on-factor-graphs}

This section presents a reformulation of classic message passing-based inference using the
\ac{rp} paradigm.
We introduce new reactive primitives specifically designed for message passing on factor graphs, which is
compatible with the \ac{cbfe} minimization procedure.
In this approach, we consider all variables of interest as observables that can change in
value over time.
This includes messages $\mu_{ia}(s_i)$, variational distributions $q_i(s_i)$ for all
variables, joint variational distributions $q_a(\bm{s}_a)$ for all factors, and auxiliary
factors $p^n_a(\bm{s}^n_a)$ in the corresponding \ac{tffg} of a probabilistic generative model.
In this context, we treat edges and nodes as special types of reactive primitives that respond
to updates in the observables from their edge- and node-induced subgraph respectively.
We leverage the \texttt{combineLatest} operator to accumulate these updates and the
\texttt{map} operator to compute the corresponding integrals.
The subscription mechanism allows us to start listening for changes in variational
distributions.
This approach results in a reactive system that efficiently resolves message passing updates,
automatically reacts to flowing messages, and updates itself lazily whenever new data arrives.
Such a reactive paradigm enables a flexible and dynamic approach to message passing-based
inference on factor graphs and performs continual and reactive \ac{cbfe} minimization.

\subsection{Reactive edges and variational distributions}

Within the context of \ac{tffg}, each edge corresponds to a single variable $s_i$.
In the \ac{rmp} framework, the primary function of an edge is to react to the
colliding messages of the connected nodes and to emit the corresponding variational
distribution $q(s_i)$.
To achieve this, we define the observable of a variational distribution $\obs{q}(s_i)$ as a
combination of two adjacent message observables from connected nodes, using the
\texttt{combineLatest} operator, and their corresponding normalized product, obtained with the
\texttt{map} operator.
This combined usage of the \texttt{combineLatest} and \texttt{map} operators is referred to as
the \textit{default computational pipeline}.

A visual representation of this process is illustrated in Figure~\ref{fig:rmp:reactive_edge}.
This reactive approach to handling edges efficiently manages message interactions, allowing
for the dynamic computation of variational distributions in the \ac{rmp} framework.

\begin{figure}
  \centering
  \resizebox{\textwidth}{!}{

\begin{tikzpicture}[]
  %Nodes
  \node[box] (a) {$f_a$};
  \node[box] (b) [right of=a, xshift=35mm] {$f_b$};

  \node[] (ai1) [left of=a, yshift=5mm] {};
  \node[] (ai2) [left of=a, yshift=-5mm] {};
  \node[] (aid) [left of=a, xshift=3mm] {$\cdots$};

  \path[line] (ai1) edge[-] (a);
  \path[line] (ai2) edge[-] (a);

  \node[] (bi1) [right of=b, yshift=5mm] {};
  \node[] (bi2) [right of=b, yshift=-5mm] {};
  \node[] (bid) [right of=b, xshift=-3mm] {$\cdots$};

  \path[line] (bi1) edge[-] (b);
  \path[line] (bi2) edge[-] (b);

  \path[line] (a) edge[-] 
  node[pos=0.1, yshift=1mm, anchor=south](mleft){\tiny $\obs{\mu}_{ai}$}
  node[pos=0.5, anchor=north, yshift=-1mm](si){$s_i$}
  % node[pos=0.5, qbox, label={[label distance=-2mm]above left:{\tiny $q(s_i)$}}](qsi){}
  node[pos=0.9, yshift=1mm, anchor=south](mright){\tiny $\obs{\mu}_{bi}$}
  (b);

  \node[box, yshift=0.8mm, node distance=6mm, inner sep=1mm, minimum size=0mm, minimum width=16mm] (combine) [above of=si] {\tiny \texttt{combineLatest}};
  \node[box, node distance=5.5mm, inner sep=1mm, minimum size=0mm, minimum width=16mm] (map) [above of=combine] {\tiny \texttt{map}};

  \path[line, densely dotted] (mleft) edge[-stealth] (combine);
  \path[line, densely dotted] (mright) edge[-stealth] (combine);
  \path[line, densely dotted] (combine) edge[-stealth] (map);

  %\node[box, densely dotted, inner xsep=2mm] (qsibbox) [fit=(combine)(map)] {};

  \node[node distance=5mm, xshift=-7mm] (qsi) [above of=map] {\tiny $\obs{q}_i(s_i)$};
  \node[node distance=6mm] (qsiempty) [above of=map] {};

  \path[line, densely dotted] (map) edge[-stealth] (qsiempty);

\end{tikzpicture}
}
  \caption{A visual representation of observable computation for a variational distribution $q_i(s_i)$ of a variable $s_i$ and its corresponding edge with the default computational pipeline.
    The pipeline reacts to changes in two colliding messages using the \texttt{combineLatest}
    operator and computes an updated variational distribution with the \texttt{map} operator,
    according to~\eqref{eq:mp:variational_q_i}.
  }
  \label{fig:rmp:reactive_edge}
\end{figure}

\subsection{Reactive nodes and outbound messages}

In the context of \ac{tffg}, a factor node $a$ consists of a factor function $f_a(\bm{s}_a)$ and a
set of connected edges, where each edge refers to a variable in a probabilistic model.
The \ac{cbfe} minimization procedure also defines a set of constraints for the variational distributions
$q^n_a(\bm{s}^n_a)$ (clusters) and $q_i(s_i)$.

The reactivity in the node in the \ac{rmp} framework is divided into two parts.
The first part accumulates updates from clusters with the \texttt{combineLatest} operator and
forms observables for the auxiliary factors $f^n_a(\bm{s}^n_a)$ with the \texttt{map}
operator, according to~\eqref{eq:mp:auxiliary_p}.
The second part accumulates updates from the observables of auxiliary factors
$\obs{f}^n_a(\bm{s}^n_a)$ and the observables of associated messages from the same cluster
with the \texttt{combineLatest} operator, and computes a message with the \texttt{map}
operator, according to~\eqref{eq:mp:vi_message}.
As with the edge, we refer to the combination of the \texttt{combineLatest} and \texttt{map}
operators as the default computational pipeline.
A visual representation of the corresponding procedure is shown in
Figure~\ref{fig:rmp:reactive_node}.

\begin{figure}
  \centering
  \begin{subfigure}[t]{0.95\textwidth}
    \centering
    \resizebox{0.95\textwidth}{!}{
\begin{tikzpicture}
  \node[box, minimum size=28mm, fill=white] (a) {};
  \node[] (i) [right=of a, xshift=0mm, yshift=6mm] {};
  \node[] (j) [right=of a, xshift=0mm, yshift=-6mm] {};
  \node[] (d) [left=of a, rotate=-20, xshift=4mm, yshift=4mm] {$\dots$};
  \node[] (dd) [left=of a, rotate=20, xshift=4mm, yshift=-4mm] {$\dots$};
  \node[opacity=0.15] (ddd) [right=of a] {$\dots$};
  \node[] (ji) [left=of a, yshift=23pt, xshift=2pt] {};
  \node[] (ki) [left=of a, yshift=-23pt, xshift=2pt] {};

  \node[qbox, inner sep=3.5mm] (s_m) [left=of a, yshift=5.5mm, xshift=8mm] {};
  \node[qbox, inner sep=3.5mm] (s_h) [left=of a, yshift=-5.5mm, xshift=8mm] {};

  \node[qbox, color=white!80!black, label={[label distance=-1mm, opacity=0.15]above:{\tiny $q^n_a(\bm{s}^n_a)$}}] (s_n) [fit=(ddd), xshift=-3mm, inner sep=3mm]{};

  \path[line, opacity=0.15] (a) edge[-]
  node[pos=0.8, anchor=west, xshift=3mm, opacity=0.15]{$s_i$}
  % node[pos=0.5, anchor=south, rotate=17]{$\substack{\mu_{ai}\\\rightarrow}$}
  (i);

  \path[line, opacity=0.15] (a) edge[-]
  node[pos=0.8, anchor=west, xshift=3mm, opacity=0.15]{$s_j$}
  % node[pos=0.6, anchor=north, rotate=-17]{$\substack{\leftarrow \\ \mu_{ja}}$}
  (j);

  \path[line] (a) edge[-] (ji);
  \path[line] (a) edge[-] (ki);

  \node[node distance=0mm] (s_m_label) [above=of s_m, xshift=-1mm] {\tiny $\obs{q}^m_a(\bm{s}^m_a)$};
  \node[node distance=0mm] (s_h_label) [below=of s_h, xshift=-1mm] {\tiny $\obs{q}^l_a(\bm{s}^l_a)$};

  \node[box, inner ysep=1mm, minimum size=0mm, minimum width=22mm] (combine) [yshift=-9mm] {\scriptsize \texttt{combineLatest}};
  \node[box, inner ysep=1mm, minimum size=0mm, minimum width=22mm] (map) [yshift=-2mm] {\scriptsize \texttt{map}};

  \draw[densely dotted, -stealth] (s_m_label.east) -- node {} ++(1.5mm, 0mm) |- ([yshift=1mm]combine.west);
  \draw[densely dotted, -stealth] (s_h_label.east) -- node {} ++(1.5mm, 0mm) |- ([yshift=-1mm]combine.west);
  \draw[densely dotted, -stealth] (combine) -- (map);

  \node[node distance=5mm] (mapto) [above=of map] {};
  \node[node distance=2.5mm] (plabel) [above=of map, xshift=-7mm] {\scriptsize $\obs{f}^n_a(\bm{s}^n_a)$};
  \draw[densely dotted, -stealth] (map) -- (mapto);

  %\node[box, densely dotted, inner sep=2mm] (pipelinebbox) [fit=(combine)(map)] {};
\end{tikzpicture}


}
    \caption{A visual representation of an observable computation for an auxiliary function $f^n_a(\bm{s}^n_a)$ for a node $a$ and its associated clusters $\mathcal{C}(a)$.
      The pipeline reacts to changes in the clusters with the \texttt{combineLatest} operator and
      computes a new auxiliary function with the \texttt{map} operator, according
      to~\eqref{eq:mp:auxiliary_p}.
    }
    \label{fig:rmp:reactive_node_auxiliary_p_stage}
  \end{subfigure}
  \hfill
  \begin{subfigure}[t]{\textwidth}
    \centering
    \resizebox{0.95\textwidth}{!}{
\begin{tikzpicture}
  \node[box, minimum size=28mm, fill=white] (a) {};
  \node[] (i) [right=of a, xshift=0mm, yshift=6mm] {};
  \node[] (j) [right=of a, xshift=0mm, yshift=-6mm] {};
  \node[opacity=0.15] (d) [left=of a, rotate=-20, xshift=4mm, yshift=4mm] {$\dots$};
  \node[opacity=0.15] (dd) [left=of a, rotate=20, xshift=4mm, yshift=-4mm] {$\dots$};
  \node[] (ddd) [right=of a] {$\dots$};
  \node[] (ji) [left=of a, yshift=23pt, xshift=2pt] {};
  \node[] (ki) [left=of a, yshift=-23pt, xshift=2pt] {};

  \node[qbox, color=white!80!black, inner sep=3.5mm] (s_m) [left=of a, yshift=5.5mm, xshift=8mm] {};
  \node[qbox, color=white!80!black, inner sep=3.5mm] (s_h) [left=of a, yshift=-5.5mm, xshift=8mm] {};

  \node[qbox, label={[label distance=-1mm]above:{\tiny $q^n_a(\bm{s}^n_a)$}}] (s_n) [fit=(ddd), xshift=-3mm, inner sep=3mm]{};

  \path[line] (a) edge[-]
  node[pos=0.8, anchor=west, xshift=3mm]{$s_i$}
  % node[pos=0.5, anchor=south, rotate=17]{$\substack{\mu_{ai}\\\rightarrow}$}
  (i);

  \path[line] (a) edge[-]
  node[pos=0.8, anchor=west, xshift=3mm]{$s_j$}
  node[pos=0.6, anchor=north, rotate=-17](mja){$\substack{\leftarrow \\ \obs{\mu}_{ja}}$}
  (j);

  \path[line, opacity=0.15] (a) edge[-] (ji);
  \path[line, opacity=0.15] (a) edge[-] (ki);

  \node[node distance=0mm, opacity=0.15] (s_m_label) [above=of s_m, xshift=-1mm] {\tiny $\obs{q}^m_a(\bm{s}^m_a)$};
  \node[node distance=0mm, opacity=0.15] (s_h_label) [below=of s_h, xshift=-1mm] {\tiny $\obs{q}^l_a(\bm{s}^l_a)$};

  \node[box, inner ysep=1mm, minimum size=0mm, minimum width=22mm] (combine) [yshift=0mm] {\scriptsize \texttt{combineLatest}};
  \node[box, inner ysep=1mm, minimum size=0mm, minimum width=22mm] (map) [yshift=6.5mm] {\scriptsize \texttt{map}};

  \draw[densely dotted, -stealth] (mja.west) -| ([xshift=3mm]combine.south);

  \draw[densely dotted, -stealth] (combine) -- (map);
  \node[node distance=10mm] (mapto) [right=of map] {};
  \path[line, densely dotted] (map) edge[-stealth]
  node[pos=0.75, anchor=south]{\scriptsize $\obs{\mu}_{ib}$}
  (mapto);

  %\node[box, densely dotted, inner sep=2mm] (pipelinebbox) [fit=(combine)(map)] {};

  \node[node distance=1mm] (auxp) [below=of combine, xshift=-7.5mm] {\scriptsize $\obs{f}^n_a(\bm{s}^n_a)$};

  \draw[densely dotted, -stealth] (auxp.east) -| (combine);
\end{tikzpicture}


}
    \caption{A visual representation of an observable computation for an outbound message $\mu_{ib}(s_i)$.
      The pipeline reacts to changes in the associated auxiliary function $f^n_a(\bm{s}^n_a)$ and
      inbound messages from the cluster $\mathcal{C}(a, i)$ with the \texttt{combineLatest} operator
      and computes outbound messages with the \texttt{map} operator, according
      to~\eqref{eq:mp:vi_message}.
    }
    \label{fig:rmp:reactive_node_message_stage}
  \end{subfigure}
  \caption{A visual representation of the default computational pipeline for computing an outbound variational message $\mu_{ib}(s_i)$ divided into two stages.
    The first stage reactively computes the auxiliary function $f^n_a(\bm{s}^n_a)$, and the second
    stage reactively computes the actual message.
  }
  \label{fig:rmp:reactive_node}
\end{figure}

Figure~\ref{fig:rmp:reactive_node} visualizes the generic procedure for the computation of the outbound message, 
but the actual list of arguments to the \texttt{combineLatest} operator in both
stages depends on the local factorization constraints $\mathcal{C}(a)$ around a node $a$.

Consider a simple example of the \ac{cbfe} minimization procedure without additional local
factorization constraints.
In this case, as discussed before, the optimization procedure must be equivalent to
\ac{bp}, where messages are computed according to~\eqref{eq:mp:bp_message}.
Consequently, the first stage of the pipeline is not active because there is no need to compute an
auxiliary function.
On the other hand, with the naive mean-field local factorization constraint, the message is
computed according to equation~\eqref{eq:mp:mf_message}, which depends only on the variational
distributions $q_i(s_i)$.
In this case, the second stage of the pipeline is not active, as the resulting message is
simply equal to the computed auxiliary function.

Various combinations of input arguments for the \texttt{combineLatest} operator
lead to different local constraints, and may represent different message
passing algorithms with distinct accuracy and speed characteristics.
The main idea is to support as wide a range of inference algorithms as possible while
providing granular control over the speed versus accuracy trade-off.

\subsection{Reactive joint variational distributions}

The procedure that computes a local variational factorized distribution $q^n_a(\bm{s}^n_a)$
for a cluster $n$ around a node $a$ in \ac{tffg} is similar to the previous cases and involves the
use of the \texttt{combineLatest} and \texttt{map} operators in its default computational
pipeline.
The default computational pipeline accumulates changes in the auxiliary factor
$f^n_a(\bm{s}^n_a)$ and in the messages $\mu_{ia}(s_i)$ for each $i \in C(a)$ with the
\texttt{combineLatest} operator and computes the variational distribution with the
\texttt{map} operator, as shown in equation~\eqref{eq:mp:variational_q_a}.
Figure~\ref{fig:rmp:reactive_cluster} provides a visual representation of this corresponding
procedure.

\begin{figure}
  \centering
  \resizebox{0.8\textwidth}{!}{\begin{tikzpicture}
  \node[box, minimum size=10mm, fill=white] (a) {};
  \node[] (pna) {\tiny $\obs{f}^n_a(\bm{s}^n_a)$};
  \node[] (i) [right=of a, xshift=0mm, yshift=6mm] {};
  \node[] (j) [right=of a, xshift=0mm, yshift=-6mm] {};
  \node[node distance=0.5mm] (ddd) [right=of a, yshift=1mm] {\scriptsize $\dots$};
  \node[] (ji) [left=of a, yshift=17pt, xshift=10pt] {};
  \node[] (ki) [left=of a, yshift=-17pt, xshift=10pt] {};
  \node[opacity=0.15] (d) [left=of a, rotate=-20, xshift=8mm, yshift=4mm] {$\dots$};
  \node[opacity=0.15] (dd) [left=of a, rotate=20, xshift=8mm, yshift=-4mm] {$\dots$};

  \node[qbox, color=white!80!black, inner sep=2.5mm] (s_m) [left=of a, yshift=4.0mm, xshift=10mm] {};
  \node[qbox, color=white!80!black, inner sep=2.5mm] (s_h) [left=of a, yshift=-4.0mm, xshift=10mm] {};

  \node[qbox, node distance=4mm, inner sep=8.5mm] (s_n) [right=of a]{};

  \path[line] (a) edge[-]
  node[pos=1.0, anchor=south]{$s_i$}
  node[pos=0.4, anchor=south, rotate=20](mia){$\substack{\obs{\mu}_{ia}\\\leftarrow}$}
  (i);

  \path[line] (a) edge[-]
  node[pos=1.0, anchor=north]{$s_j$}
  node[pos=0.4, anchor=north, rotate=-20](mja){$\substack{\leftarrow \\ \obs{\mu}_{ja}}$}
  (j);

  \path[line, opacity=0.15] (a) edge[-] (ji);
  \path[line, opacity=0.15] (a) edge[-] (ki);

  \node[box, minimum size=0mm, inner ysep=1mm, node distance=9mm, minimum width=16mm] (combine) [right=of a, yshift=-2.5mm]{\tiny \texttt{combineLatest}};
  \node[box, minimum size=0mm, inner ysep=1mm, node distance=2mm, minimum width=16mm] (map) [above=of combine]{\tiny \texttt{map}};

  \draw[densely dotted, -stealth] (mia.east) -- node {} ++(0.5mm, 0mm) |- ([yshift=1.5mm]combine);
  \draw[densely dotted, -stealth] (mja.east) -- node {} ++(0.5mm, 0mm) |- ([yshift=-1.5mm]combine);
  \draw[densely dotted, -stealth] ([xshift=-1mm]pna.east) -- node {} ++(4.5mm, 0mm) |- ([yshift=0mm]combine);
  \path[line, densely dotted] (combine) edge[-stealth] (map);

  \node[node distance=3mm] (qna) [above=of map] {};
  \node[node distance=0mm] (qnalabel) [above=of map, xshift=5mm] {\tiny $\obs{q}^n_a(\bm{s}^n_a)$};

  \path[line, densely dotted] (map) edge[-stealth] (qna);

\end{tikzpicture}

}
  \caption{
    A visual representation of the computation for the local variational distribution observable for a cluster $n$ around a node $a$ is depicted in Figure~\ref{fig:rmp:reactive_cluster}.
    The pipeline involves using the \texttt{combineLatest} operator to accumulate changes in the
    corresponding auxiliary function $f^n_a(\bm{s}^n_a)$ and messages $\mu_{ia}(s_i)$ along the
    edges that belong to the cluster $\mathcal{C}(a)$.
    Subsequently, the \texttt{map} operator is used to compute the variational distribution
    $q^n_a(\bm{s}^n_a)$, according to~\eqref{eq:mp:variational_q_a}.
  }
  \label{fig:rmp:reactive_cluster}
\end{figure}

\subsection{Reactive computation of Bethe Free Energy}

The message passing-based interpretation of the \ac{cbfe} procedure guarantees the minimization of
the \ac{cbfe} functional \citep{senoz_thesis}.
However, in certain situations, it becomes essential to compute the actual value of the \ac{cbfe}
functional based on current posterior estimates.

The default computational pipeline for the \ac{cbfe} observable on an arbitrary \ac{tffg} with three nodes and two edges is illustrated in Figure~\ref{fig:rmp:reactive_bethe_free_energy}.
The \ac{cbfe} value is computed according to~\eqref{eq:bfe:bfe}, decomposed into a set of local
computations for each node $a$ and each edge $i$.
First, for each node $a$, the pipeline forms an observable $\obs{F}[q_a]$ that reacts
to changes in the joint variational distribution $\obs{q}_a(\bm{s}_a)$ and computes the
factor-bound local Free Energy term using the \texttt{map} operator.
Second, for each edge $i$, the pipeline forms an observable $\obs{U}(s_i)$ that reacts to
changes in the variational distributions $\obs{q}_i(s_i)$ and computes the variable-bound
local entropy term with the \texttt{map} operator.
Finally, the procedure accumulates changes in the local Free Energy term observables
$\obs{F}[q_a]$ and the local entropy term observables $\obs{U}(s_i)$ using the
\texttt{combineLatest} operator and computes their sum with the \texttt{map} operator.

\begin{figure}
  \centering
  \resizebox{\textwidth}{!}{\begin{tikzpicture}[]
  \node[box] (a) {$f_a$};
  \node[box] (b) [right=of a, xshift=10mm] {$f_b$};
  \node[box] (c) [right=of b, xshift=10mm] {$f_c$};

  %\node[node distance=4mm] (lai) [left=of a, yshift=3mm] {};
  %\node[node distance=1mm] (lad) [left=of a] {$\cdots$};
  %\node[node distance=4mm] (laj) [left=of a, yshift=-3mm] {};

  %\draw[line] (a) edge[-] (lai);
  %\draw[line] (a) edge[-] (laj);

  %\node[node distance=4mm] (lci) [right=of c, yshift=3mm] {};
  %\node[node distance=1mm] (lcd) [right=of c] {$\cdots$};
  %\node[node distance=4mm] (lcj) [right=of c, yshift=-3mm] {};

  %\draw[line] (c) edge[-] (lci);
  %\draw[line] (c) edge[-] (lcj);

  \draw[line] (a) edge[-]
  node[qbox, pos=0.5]{}
  node[pos=0.5, anchor=south, yshift=1mm](qi){\scriptsize $\obs{q}_i(s_i)$}
  (b);

  \draw[line] (b) edge[-]
  node[qbox, pos=0.5]{}
  node[pos=0.5, anchor=south, yshift=1mm](qj){\scriptsize $\obs{q}_j(s_j)$}
  (c);

  \node[box, node distance=0mm, minimum size=0mm, inner ysep=1mm] (qimap) [above=of qi, yshift=2.5mm] {\tiny \texttt{map}};
  \node[box, node distance=0mm, minimum size=0mm, inner ysep=1mm] (qjmap) [above=of qj, yshift=2.5mm] {\tiny \texttt{map}};

  \node[node distance=0mm] (qimaplabel) [right=of qimap] {\scriptsize $\obs{H}[q_i]$};
  \node[node distance=0mm] (qjmaplabel) [right=of qjmap] {\scriptsize $\obs{H}[q_j]$};

  \draw[line, densely dotted] (qi) edge[-stealth] (qimap);
  \draw[line, densely dotted] (qj) edge[-stealth] (qjmap);

  \node[qbox] (qa) [fit=(a)] {};
  \node[node distance=0mm] (qal) [below=of qa, yshift=2mm] {\tiny $\obs{q}_a(\bm{s}_a)$};
  \node[qbox] (qb) [fit=(b)] {};
  \node[node distance=0mm] (qbl) [below=of qb, yshift=2mm] {\tiny $\obs{q}_b(\bm{s}_b)$};
  \node[qbox] (qc) [fit=(c)] {};
  \node[node distance=0mm] (qcl) [below=of qc, yshift=2mm] {\tiny $\obs{q}_c(\bm{s}_c)$};

  \node[box, node distance=0mm, minimum size=0mm, inner ysep=1mm] (qalmap) [below=of qal, yshift=-2.5mm] {\tiny \texttt{map}};
  \node[box, node distance=0mm, minimum size=0mm, inner ysep=1mm] (qblmap) [below=of qbl, yshift=-2.5mm] {\tiny \texttt{map}};
  \node[box, node distance=0mm, minimum size=0mm, inner ysep=1mm] (qclmap) [below=of qcl, yshift=-2.5mm] {\tiny \texttt{map}};

  \node[node distance=0mm] (qalmaplabel) [right=of qalmap] {\scriptsize $\obs{F}[q_a]$};
  \node[node distance=0mm] (qblmaplabel) [right=of qblmap] {\scriptsize $\obs{F}[q_b]$};
  \node[node distance=0mm] (qclmaplabel) [right=of qclmap] {\scriptsize $\obs{F}[q_c]$};

  \draw[line, densely dotted] (qal) edge[-stealth] (qalmap);
  \draw[line, densely dotted] (qbl) edge[-stealth] (qblmap);
  \draw[line, densely dotted] (qcl) edge[-stealth] (qclmap);

  \node[box, minimum size=0mm, inner ysep=1mm, node distance=5mm] (finalcombine) [right=of c] {\tiny \texttt{combineLatest}};
  \node[box, minimum size=0mm, inner ysep=1mm, node distance=4mm] (finalmap) [right=of finalcombine] {\tiny \texttt{map}};
  \node[minimum size=0mm, inner ysep=1mm, node distance=6mm] (finalmaplabel) [right=of finalmap] {};

  \node[box, minimum size=0mm, inner ysep=1mm, node distance=4mm] (topmap) [above=of finalcombine] {\tiny \texttt{map}};
  \node[box, minimum size=0mm, inner ysep=1mm, node distance=4mm] (bottommap) [below=of finalcombine] {\tiny \texttt{map}};

  \node[node distance=0mm] (topmaplabel) [right=of topmap] {\scriptsize $\obs{H}[q]$};
  \node[node distance=0mm] (bottommaplabel) [right=of bottommap] {\scriptsize $\obs{F}[q]$};

  \node[box, minimum size=0mm, inner ysep=1mm, node distance=5mm] (topcombine) [above=of topmap] {\tiny \texttt{combineLatest}};
  \node[box, minimum size=0mm, inner ysep=1mm, node distance=9mm] (bottomcombine) [below=of bottommap] {\tiny \texttt{combineLatest}};

  \draw[line, densely dotted] (finalcombine) edge[-stealth] (finalmap);
  \draw[line, densely dotted] (finalmap) edge[-stealth]
  node[pos=1.0, anchor=south]{\scriptsize $\obs{F}_B[q]$}
  (finalmaplabel);

  \draw[line, densely dotted] (topmap) edge[-stealth] (finalcombine);
  \draw[line, densely dotted] (bottommap) edge[-stealth] (finalcombine);

  \draw[line, densely dotted] (topcombine) edge[-stealth] (topmap);
  \draw[line, densely dotted] (bottomcombine) edge[-stealth] (bottommap);

  \draw[line, densely dotted, -stealth] (qimap) |- (topcombine.north west);
  \draw[line, densely dotted, -stealth] (qjmap) |- (topcombine.south west);

  \draw[line, densely dotted, -stealth] (qalmap) |- (bottomcombine.south west);
  \draw[line, densely dotted, -stealth] (qblmap) |- (bottomcombine.west);
  \draw[line, densely dotted, -stealth] (qclmap) |- (bottomcombine.north west);
\end{tikzpicture}

}
  \caption{
    A visual representation of the \ac{bfe} observable computational pipeline for an arbitrary \ac{tffg} with nodes $a$, $b$, and $c$, along with edges $i$ and $j$.
    The resulting observable reacts to changes in local Free Energy term observables for each node
    and changes in entropy term observables for each edge.
    These changes are accumulated using the \texttt{combineLatest} operator, and their sum is
    computed with the \texttt{map} operator, according to~\eqref{eq:bfe:bfe}.
  }
  \label{fig:rmp:reactive_bethe_free_energy}
\end{figure}

