\section{Reactive inference}\label{chapter-03:section:reactive-continual-inference}

\subsection{Lazy and eager computations}

The procedures outlined in the previous sections establish observables and reactive primitives
for all essential concepts of the message passing-based \ac{cbfe} optimization procedure.
These include messages $\obs{\mu}_{ia}(s_i)$, variational distributions $\obs{q}_i(s_i)$ and
$\obs{q}_a(\bm{s}_a)$, and the \ac{vfe} functional $\obs{F}_B[q]$.
The interpretation of Bayesian inference based on reactive message passing does not
incorporate an explicit scheduling procedure.
Instead, it allows the system to react to changes in associated observables.
Consequently, the inference remains idle during intervals without new observations or when
some messages in certain parts of the model's graph are not yet available.
Notably, the \ac{rmp} framework does not rely on any specific order of computations, as certain
messages may take longer to compute, particularly in the presence of extensive approximation
methods.

In the \ac{rmp} formulation, all operations are performed lazily, only when strictly required.
The \texttt{combineLatest} operators serve as intermediate "guards," ensuring that the system
does not progress further until all necessary information has been updated and accumulated.
However, it is also possible to modify the underlying strategy for the \texttt{combineLatest}
operator and compute messages eagerly, as soon as even a single argument to the
\texttt{combineLatest} operator changes, as discussed in
Section~\ref{chapter-03:section:reactive-programming}.
This flexibility in the proposed architecture allows for various trade-offs between lazy and
eager computations, catering to different needs.

\subsection{Continual inference}

The defined observables may, in some cases, depend on their own updates, particularly if the
model's graph contains structural loops.
However, the \ac{rp} framework naturally handles such scenarios, as all
observables are defined solely in terms of the \texttt{combineLatest} and \texttt{map}
operators.

The \ac{rmp} framework is designed to support infinite data stream processing at its core, without
assuming any specifics about the underlying nature of the update-generating process for
observables.
In fact, this flexibility allows the \ac{rmp} framework to effortlessly perform continual and
reactive inference on infinite data streams.
By embracing the potential for loops in the observables' dependencies, the \ac{rmp} framework
efficiently manages and processes continuous data streams, making it well-suited for handling
real-time updates and infinite data scenarios with minimal additional effort.

Here we present an example of such an application in the context of continual inference in
state-space probabilistic models.
Graphical probabilistic state-space models are frequently used in signal processing
applications, where prior knowledge of the \textit{current state} is combined with a
\textit{state transition} function to predict or update the subsequent state of a system.
In many cases, it is assumed that the next state depends solely on the previous state and not
on states that occurred far in the past.
This property is known as the \textit{Markov property}, and models exhibiting this behavior
are referred to as \textit{Markov models}.
In Markov models, states are often associated with a time label $t$, and it is assumed that
the state evolves over time.

In traditional Markov models, we typically construct a graph and perform
filtering using a message-passing approach, where messages flow exclusively in the forward
direction.
However, with \ac{rmp}, we have the flexibility to create only a single
section of the graph, where the prior distributions react to the newly computed posteriors and update themselves automatically. Such a system does not need manual intervention and simply updates itself in the presence of new observations.
Furthermore, this system effectively creates an infinite reaction chain that may be interpreted as a factor graph
with messages flowing only in a forward-time direction.
We refer to such an architecture as an \textit{infinite factor graph}.
This architecture supports data streams with potentially an infinite number of sequential
observations and performs Bayesian inference as soon as the data arrive.
By carefully choosing strategies for the \texttt{combineLatest} operator (see
Section~\ref{chapter-03:section:reactive-programming}) we can avoid an infinite messaging
loop, and we will show an example of such an inference application in
Section~\ref{chapter-05:section:hierarchical-filter}.

\begin{figure}
  \centering
  \resizebox{0.85\textwidth}{!}{\begin{tikzpicture}
  \node[box] (a) {$f_a$};
  \node[smallbox] (b) [right=of a] {$=$};
  \node[box, node distance=2mm] (c) [below=of b] {$f_b$};

  \path[line] (a) edge[-] (b);
  \path[line] (b) edge[-] (c);

  \node[] (tm1) [left=of a] {};
  \node[] (t) [right=of b] {};
  \node[clamped, label={below:{$y$}}, node distance=2mm] (y) [below=of c] {};

  \path[line] (tm1) edge[-] 
  node[pos=0.1, anchor=south](mbtm1){\tiny $\obs{\mu}_{t-1}$}
  (a);
  \path[line, densely dotted] (mbtm1) edge[-stealth] ([xshift=8mm]mbtm1);

  \path[line] (b) edge[-] 
  node[pos=0.5, anchor=south](mbt){\scriptsize $\obs{\mu}_{t}$}
  (t);
  \path[line, densely dotted] (mbt) edge[-stealth] ([xshift=8mm]mbt);
  \draw[densely dotted, -stealth] (mbt.east) -- node {} ++(1.5mm, 0mm) -- node {} ++(0mm, 4.5mm) -| ([yshift=-2mm]mbtm1.north);

  \path[line] (c) edge[-] 
  %node[pos=0.7, anchor=west](myc){\scriptsize $\obs{\mu}_{yc}$}
  (y);
  %\path[line, densely dotted] (myc) edge[-stealth] ([yshift=6mm]myc);
\end{tikzpicture}
}
  \caption{
    An example of an infinite factor graph in the \ac{rmp} framework for an arbitrary Markov model with state transition factor $f_a$ and observational factor $f_b$.
    The graph contains only a single time section of the Markov model.
    The small black square indicates the observed variable.
    The dotted lines signify that updates from the posterior for the next time section are used
    directly to update the corresponding priors for the next observation, at which point the loop
    repeats itself.
    The reactive inference computes messages reactively as soon as a new observation $y$ is
    available and updates priors automatically.
  }
  \label{fig:rmp:reactive_inference}
\end{figure}

We provide a visual representation of this procedure in
Figure~\ref{fig:rmp:reactive_inference}.
The system automatically reacts to new observations $y$ and computes messages as soon as a new
data point becomes available.
The prior adapts to changes in the posterior and updates itself accordingly.
During periods between new observations, the system remains idle, conserving computer
resources.
In \ac{cbfe} minimization, it is possible to perform additional variational iterations
during these idle times to improve the approximation of~\eqref{eq:mp:variational_q_a}
and~\eqref{eq:mp:variational_q_i}.

\subsection{Robustness}

The proposed architecture utilizes the \texttt{combineLatest} operator to accumulate updates
from different observables.
However, it is possible that one of the arguments fails to deliver its data, for example, due
to a failing sensor that no longer generates any data.
The strength of the \ac{rmp} framework lies in its ability to handle such scenarios without
disrupting the entire process.
The \texttt{combineLatest} operator does not make any assumptions about the underlying
update-generating process in the observables, allowing easy replacement of a failed observable
with a backup.

To handle failures in observables, the \ac{rp} framework provides a set of operators,
such as \texttt{onErrorResumeNext}.
This operator manages both the main data observable and a collection of backup data
observables.
If the main observable terminates or fails to deliver data, the operator immediately
subscribes to one of its backups, ensuring a smooth continuation of the process.
Figure~\ref{fig:rmp:reactive_robustness} showcases a visualization of how the
\texttt{onErrorResumeNext} operator can be employed to address robustness issues effectively.
By employing this pattern at strategic points within the architecture, we can ensure a
resilient and continual inference process, even in the presence of failing observables.

\begin{figure}
  \centering
  \resizebox{\textwidth}{!}{\begin{tikzpicture}
  \node[box] (a) {$f_a$};

  \node[] (j) [left=of a, yshift=5mm, xshift=3mm] {};
  \node[opacity=0.25] (d) [left=of a, xshift=9mm] {$\cdots$};
  \node[] (k) [left=of a, yshift=-5mm, xshift=3mm] {};

  \draw[line, opacity=0.25] (a) edge[-] node[pos=1.0, anchor=east]{$s_j$} (j);
  \draw[line, opacity=0.25] (a) edge[-] node[pos=1.0, anchor=east]{$s_k$} (k);

  \node[] (i) [right=of a] {};

  \draw[line] (a) edge[-] 
  node[pos=0.75, anchor=south](mia){}
  (i);

  \node[node distance=4mm] (mial) [left=of mia] {};
  %\draw[line, densely dotted] (mia) edge[-stealth] (mial);

  \node[box, node distance=4mm, minimum size=0mm, inner ysep=1mm] (operator) [right=of mia] {\tiny \texttt{onErrorResumeNext}};
  \draw[line, densely dotted] (operator) edge[-stealth] 
  node[pos=0.3, anchor=south](mia){$\substack{\obs{\mu}_{ia}}$}
  (mial);

  \node[clamped, node distance=5mm, label={right:{\scriptsize sensor}}] (fs) [right=of operator, yshift=5mm] {};
  \node[clamped, node distance=5mm, label={right:{\scriptsize backup sensor $1$}}] (b1s) [right=of operator, yshift=0mm] {};
  \node[clamped, node distance=5mm, label={right:{\scriptsize backup sensor $2$}}] (b2s) [right=of operator, yshift=-5mm] {};

  \path[line] (fs) -| (operator);
  \path[line] (b1s) -- (operator);
  \path[line] (b2s) -| (operator);
\end{tikzpicture}
}
  \caption{
    Robust $\mu_{ia}$ message computation for a node $a$ and an edge $i$ from an observation coming from a sensor.
    The \texttt{onErrorResumeNext} operator combines multiple sources of data as observables but
    subscribes to and sends data from one of the observables.
    If the current observable terminates or fails to deliver its data, the operator immediately
    subscribes to one of its backups, ensuring robust data delivery.
  }
  \label{fig:rmp:reactive_robustness}
\end{figure}

\subsection{Adaptivity}

In addition to robustness, we may also want to replace one part of the graph with another,
depending on the situation.
This could involve using a more accurate sub-model to represent the underlying process or,
conversely, opting for a simpler model that provides a rougher approximation but requires
fewer computational resources, thus making them available for other tasks.
In either case, making such ad-hoc adjustments in one of the arguments of the
\texttt{combineLatest} operator with a new observable is straightforward.
This new observable can be generated by a different process to suit the changing needs.
The \ac{rp} framework offers a set of operators to switch between different
sources of events on demand, and one such operator is called \texttt{switchMap}.
With \texttt{switchMap}, various sources of data are handled, but events are redirected only
from the desired source.
This setup is akin to \texttt{onErrorResumeNext}, with the key distinction being that the desired source of events can be changed manually depending on the situation.
Figure~\ref{fig:rmp:reactive_adaptability} provides a visualization of this adaptive setting.

\begin{figure}
  \centering
  \resizebox{\textwidth}{!}{
\begin{tikzpicture}
  \node[box] (a) {$f_a$};

  \node[] (j) [left=of a, yshift=5mm, xshift=3mm] {};
  \node[opacity=0.25] (d) [left=of a, xshift=9mm] {$\cdots$};
  \node[] (k) [left=of a, yshift=-5mm, xshift=3mm] {};

  \draw[line, opacity=0.25] (a) edge[-] node[pos=1.0, anchor=east]{$s_j$} (j);
  \draw[line, opacity=0.25] (a) edge[-] node[pos=1.0, anchor=east]{$s_k$} (k);

  \node[] (i) [right=of a] {};

  \draw[line] (a) edge[-] 
  node[pos=0.75, anchor=south](mia){}
  (i);

  \node[node distance=4mm] (mial) [left=of mia] {};
  %\draw[line, densely dotted] (mia) edge[-stealth] (mial);

  \node[box, node distance=4mm, minimum size=0mm, inner ysep=1mm] (operator) [right=of mia] {\tiny \texttt{switchMap}};
  \draw[line, densely dotted] (operator) edge[-stealth] 
  node[pos=0.3, anchor=south](mia){$\substack{\obs{\mu}_{ia}}$}
  (mial);

  \node[box, minimum size=1mm, node distance=6mm] (tm) [right=of operator, yshift=3mm] {};
  \node[box, minimum size=1mm, node distance=1mm] (tm1) [right=of tm] {}; \draw[line] (tm) edge[-] (tm1);
  \node[box, minimum size=1mm, node distance=1mm] (tm2) [above=of tm] {}; \draw[line] (tm) edge[-] (tm2);
  \node[box, minimum size=1mm, node distance=1mm] (tm3) [above=of tm1] {}; \draw[line] (tm1) edge[-] (tm3);
  \node[box, minimum size=1mm, node distance=1mm] (tm4) [left=of tm2] {}; \draw[line] (tm2) edge[-] (tm4);
  \node[box, minimum size=1mm, node distance=1mm] (tm5) [right=of tm1] {}; \draw[line] (tm5) edge[-] (tm1);
  \node[box, label={above:{\scriptsize complex model}}] (tbm) [fit=(tm)(tm1)(tm2)(tm3)(tm4)(tm5)] {};
  
  \node[box, minimum size=1mm, node distance=6mm] (bm) [right=of operator, yshift=-3mm] {};
  \node[box, minimum size=1mm, node distance=1mm] (bm1) [right=of bm] {}; \draw[line] (bm) edge[-] (bm1);
  \node[box, minimum size=1mm, node distance=1mm] (bm2) [below=of bm] {}; \draw[line] (bm) edge[-] (bm2);
  \node[box, label={below:{\scriptsize simple model}}] (bbm) [fit=(bm)(bm1)(bm2)] {};

  \path[line] (tm) -| ([xshift=3mm]operator);
  \path[line] (bm) -| ([xshift=3mm]operator);

  \node[node distance=5mm] (switch) [above=of operator] {};
  \draw[line, densely dotted] (switch) edge[-stealth] 
  node[pos=0.1, anchor=east]{\tiny $\obs{m}$} 
  (operator);
\end{tikzpicture}
}
  \caption{
    Adaptive $\mu_{ia}$ message computation for a node $a$ and an edge $i$ from different models.
    This could involve using a more accurate sub-model to represent the underlying process or,
    conversely, opting for a simpler model that provides a rougher approximation but requires
    fewer computational resources, thus making them available for other tasks.
    The \texttt{switchMap} operator combines multiple sources of messages as observables from
    different models but subscribes to and sends data from one of the observables.
    The variable $m$ defines the current sub-model of interest.
    The \texttt{switchMap} operator then switches between two models based on the latest event
    from the $m$ observable.
  }
  \label{fig:rmp:reactive_adaptability}
\end{figure}

\subsection{The default computational pipeline adjustments}

The default computational pipelines in the proposed architecture can be adjusted using custom
operators in various scenarios.
Custom operators offer the advantage of applying additional computational steps locally,
catering to different edges and factor nodes' specific needs.
This flexibility allows users to choose local approximation techniques that align with
specific requirements, such as prioritizing speed over accuracy or vice versa.
By incorporating custom operators into the default message computational pipeline, the
corresponding inference algorithm can be modified to achieve better performance for specific
custom models \citep{zhang_unifying_2021, akbayrak_reparameterization_2019}.

For instance, as discussed in Section~\ref{chapter-02:section:bethe-free-energy}, the
variational family $\mathcal{Q}_{B}$ might impose additional functional form constraints on
variational distributions $q_i(s_i)$.
Functional form constraints can be useful when analytical solutions
for~\eqref{eq:mp:variational_q_i} are not available in closed form.
As an example, consider the point-mass form constraint, which approximates the variational
distribution with its mode, implicitly leading to the \ac{em} algorithm \citep{senoz_variational_2021}.
To apply the point-mass form constraint, we can adjust the default
procedure~\eqref{eq:mp:variational_q_i} in the \texttt{map} operator of the default
computational pipeline for an edge $i$ as follows: \begin{equation}
  \label{eq:rmp:pointmass_constraint} \begin{split} q_i(s_i) &= \delta(s_i - \hat{s}_i), \\
  \hat{s}_i &= \arg\max_{s_i} \mu_{ia}(s_i) \mu_{ib}(s_i).
\end{split}
\end{equation}
Equation ~\eqref{eq:rmp:pointmass_constraint} is computationally simpler as it does not require
knowing the normalization constant as in~\eqref{eq:mp:variational_q_i}.
However, the resulting posterior does not retain uncertainty over the variable.
However, it is essential to note that any extra approximation step will inevitably affect the
inference results, and their performance will be model dependent
\citep{akbayrak_extended_2021}.
The key idea here is that the \ac{rmp} framework allows easy and straightforward modifications to
specific needs and applications that may trade off performance and accuracy.

Similarly to the pipeline for the reactive edge definition, it is entirely valid to adjust the
reactive operators in the default computational pipeline for the reactive node definition.
For example, a custom operator might apply an extra approximation method in cases where
computing an exact analytical message update rule is not feasible, see
Figure~\ref{fig:rmp:reactive_node_pipeline_N}.
Other examples of custom operators may include logging or debugging messages, see
Figure~\ref{fig:rmp:reactive_node_pipeline_logging}.

Moreover, the \texttt{combineLatest} operator can specify when to react to new messages updates, either when all messages have been updated or when any of the messages have been updated.
The framework does not specify the optimal strategy, as it may vary depending on the
application.

\begin{figure}
  \centering
  \begin{subfigure}[t]{0.45\textwidth}
    \centering
    \resizebox{\textwidth}{!}{\begin{tikzpicture}[]
  \node[box, minimum size=30mm] (a) {};
  \node[] (b) [right=of a, xshift=5mm] {};

  \node[box, inner ysep=1mm, minimum size=0mm, minimum width=22mm] (combine) [yshift=-6mm] {\scriptsize \texttt{combineLatest}};
  \node[box, inner ysep=1mm, minimum size=0mm, minimum width=22mm] (map) [yshift=0mm] {\scriptsize \texttt{map}};
  \node[box, inner ysep=1mm, minimum size=0mm, minimum width=22mm] (approximate) [yshift=6mm] {\scriptsize \texttt{approximate}};

  \path[line, densely dotted] (combine) edge[-stealth] (map);
  \path[line, densely dotted] (map) edge[-stealth] (approximate);

  \node[] (inputs) [left=of combine] {};
  \node[] (msg) [right=of approximate, xshift=3mm] {};

  \node[box, densely dotted, inner sep=2mm] (pipelinebbox) [fit=(combine)(map)(approximate)] {};

  \path[line] (a) edge[-]
  node[pos=1.0, anchor=west]{$s_i$}
  (b);

  \path[line, densely dotted] (approximate) edge[-stealth]
  node[pos=0.7, anchor=south]{$\mu_{ib}$}
  (msg);

  \path[line, densely dotted] (inputs) edge[-stealth] (combine);
\end{tikzpicture}


}
    \caption{An example of the default computational pipeline adjustment with an extra approximation step after both the \texttt{combineLatest} and the \texttt{map} operators.
    }
    \label{fig:rmp:reactive_node_pipeline_N}
  \end{subfigure}
  \hfill
  \begin{subfigure}[t]{0.45\textwidth}
    \centering
    \resizebox{\textwidth}{!}{
\begin{tikzpicture}[]
  \node[box, minimum size=30mm] (a) {};
  \node[] (b) [right=of a, xshift=5mm] {};

  \node[box, inner ysep=1mm, minimum size=0mm, minimum width=22mm] (combine) [yshift=-6mm] {\scriptsize \texttt{combineLatest}};
  \node[box, inner ysep=1mm, minimum size=0mm, minimum width=22mm] (logging) [yshift=0mm] {\scriptsize \texttt{log in a file}};
  \node[box, inner ysep=1mm, minimum size=0mm, minimum width=22mm] (map) [yshift=6mm] {\scriptsize \texttt{map}};

  \path[line, densely dotted] (combine) edge[-stealth] (logging);
  \path[line, densely dotted] (logging) edge[-stealth] (map);

  \node[] (inputs) [left=of combine] {};
  \node[] (msg) [right=of map, xshift=3mm] {};

  \node[box, densely dotted, inner sep=2mm] (pipelinebbox) [fit=(combine)(map)(logging)] {};

  \path[line] (a) edge[-]
  node[pos=1.0, anchor=west]{$s_i$}
  (b);

  \path[line, densely dotted] (map) edge[-stealth]
  node[pos=0.7, anchor=south]{$\mu_{ib}$}
  (msg);

  \path[line, densely dotted] (inputs) edge[-stealth] (combine);
\end{tikzpicture}


}
    \caption{An example of the default computational pipeline adjustment with an extra logging step for easier debugging after the \texttt{combineLatest} operator but before the \texttt{map} operator.
    }
    \label{fig:rmp:reactive_node_pipeline_logging}
  \end{subfigure}
  \caption{A visualization of the diverse adjustments to the default computational pipeline of a reactive factor node.
    Additional custom reactive operators can be incorporated at any stage of the pipeline to suit
    specific scenarios, such as prioritizing speed over accuracy or enabling extra debugging
    information during the inference process.
  }
  \label{fig:rmp:reactive_node_pipeline}
\end{figure}

