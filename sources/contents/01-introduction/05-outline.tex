\section{Outline of this dissertation}\label{chapter-04:section:outline}

Chapter~\ref{chapter-02} reviews \ac{vi} through \ac{vmp} and 
addresses \ref{question:scalability} (scalability) by discussing the \ac{cbfe} minimization framework.
The \ac{cbfe} framework unites many well-known \ac{vi} algorithms under one elegant
mathematical framework and provides a straightforward way to balance computational load
and accuracy of inference.
The chapter also introduces factor graphs in detail as a convenient representation of sparse,
large, and hierarchical probabilistic models and describes \ac{vmp} as a
way to perform efficient and scalable \ac{cbfe} optimization on factor graphs.

Chapter~\ref{chapter-03} addresses \ref{question:reactivity} (real-time processing) and proposes the idea of \ac{rmp}.
\ac{rmp} is a reactive programming-based implementation of the \ac{vmp} algorithm.
It describes how nodes and edges should react to local changes and essentially implements the
continual \ac{cbfe} minimization procedure in response to new observations in real time.
The absence of a global message passing schedule addresses \ref{question:robustness} (robustness).
The distinct nodes and edges in the factor graph are isolated and can be adapted individually
without interrupting or affecting the whole inference procedure.

Chapter~\ref{chapter-04} addresses \ref{question:user-experience} (user experience) and presents our own
implementation of the proposed architecture, which we call RxInfer.
We demonstrate a user-friendly specification of probabilistic models and inference constraints.
The resulting Bayesian inference procedure can be automatically derived from a probabilistic
language that closely resembles the actual system equations in mathematical form.

Chapter~\ref{chapter-05} addresses \ref{question:utility} (utility) and shows the practical
applicability of the proposed architecture.
The resulting architecture, particularly our own implementation, has been battle tested in
large, sophisticated probabilistic models and is capable of running various
well-known Bayesian inference algorithms.
Moreover, it is straightforward to combine different Bayesian inference algorithms under one
framework and develop custom, novel, and efficient Bayesian inference algorithms for new problems.

Chapter~\ref{chapter-06} concludes the dissertation, summarizes the contributions, reflects on
potential drawbacks of the proposed architecture, and provides ideas for possible future
research directions.

% \bdv{I love this chapter now. Great job!}
% \dmitry{erg bedankt}
