\section{Research questions}\label{chapter-01:section:questions}

%\subsection{Neuroscientist walks into an engineering bar}

%We now return to motivate the problem and discuss the challenges.
%We now return to the motivation of the problem.

The central research issue in this thesis is how to transfer ideas from the neuroscience community and \ac{fep} in particular to the design of useful engineering systems.
How can we design an intelligent system that works with a similar computational scheme (\ac{vfe} minimization) as the brain, and what are its essential components? If we embrace the principles of \ac{fep} and \ac{aif}, then we must also embrace Bayesian inference as the selected approach to reasoning and updating beliefs in the presence of new information.
However, what are the key ingredients for the realization of \ac{aif} agents through (approximate) Bayesian inference?
Once again, we draw inspiration from our understanding of the functioning of the brain.

The brain has dozens of billions of neurons \citep{herculano-houzel_human_2009}, therefore,
inference in such a large model must be \textbf{efficient} and \textbf{scalable}.
On the other hand, the brain loses thousands of neurons per day \citep{barrett_optimal_2016},
so inference in the brain must be \textbf{robust} and stay operational even under local failures.
The brain reacts in real-time, how else could we learn to ride a bike\footnote{tenzij je fietsen in je genen hebt zitten}? 
Therefore, inference in \ac{aif} agents must be \textbf{fast} and
\textbf{continual}.
The brain interacts in an environment that is constantly changing, consequently, inference
must \textbf{adapt} to changes \textbf{without interruption}.
The brain is lazy in the sense that it does not focus on (i.e. process) everything at the same time with the same attention.
Rather, the brain prioritizes important issues and updates beliefs with varying power budgets, varying precisions, and varying time scales.
%As a consequence, the inference process in the brain is \textbf{lazy} and infers only when necessary.

\begin{figure}
  \centering
  \resizebox{1.0\textwidth}{!}{\begin{tikzpicture}[node distance=1mm]
  %Nodes
  \node[box, minimum width=22mm] (realtime) {Real-time};
  \node[box, minimum width=22mm] (scalable) [right=of realtime] {Scalable};
  \node[box, minimum width=22mm] (robust) [right=of scalable] {Robust};
  \node[box, minimum width=22mm] (adaptive) [right=of robust] {Adaptive};
  
  \node[box, minimum width=22mm] (lazy) [below=of realtime] {Lazy};
  \node[box, minimum width=22mm] (versatile) [right=of lazy] {Versatile};
  \node[box, minimum width=22mm] (lowpower) [right=of versatile] {Low-power};
  \node[box, minimum width=22mm] (easytouse) [right=of lowpower] {Easy-to-use};

  \node[box, densely dotted] (bayes) [fit=(realtime)(easytouse)] {};
  \node[] (bayeslabel) [above=of bayes] {$\text{Approximate Bayesian inference}^*$};
  \node[box, inner xsep=5mm, inner ysep=3mm, rounded corners, minimum width=100mm] (bayesbox) [fit=(bayes)(bayeslabel)] {};
  
  \node[box, inner xsep=5mm, inner ysep=1mm, rounded corners, node distance=3mm, minimum width=80mm] (vfe) [above=of bayesbox] {
    Variational Free Energy minimization
  };

  \node[box, inner xsep=5mm, inner ysep=1mm, rounded corners, node distance=3mm, minimum width=80mm] (fep) [above=of vfe] {
    Free Energy Principle \& Active Inference
  };

  \node[box, inner xsep=5mm, inner ysep=1mm, rounded corners, node distance=3mm, minimum width=80mm] (nai) [above=of fep] {
    Natural Intelligent Behaviour
  };

  \draw[] (nai) edge[-stealth] (fep);
  \draw[] (fep) edge[-stealth] (vfe);
  \draw[] (vfe) edge[-stealth] (bayesbox);
  
  % \node[] (feplabel) [above=of fep] {Variational Free Energy minimization};

  % \node[box, inner xsep=5mm, inner ysep=3mm, rounded corners] (aif) [fit=(fep)(feplabel)] {};
  % \node[] (aiflabel) [above=of aif] {Free Energy Principle \& Active Inference};

  % \node[box, rounded corners, inner xsep=5mm, inner ysep=3mm, label={Natural Intelligent Behaviour}] (nai) [fit=(aif)(aiflabel)] {};

\end{tikzpicture}}
  \caption{
    Inspired by the powerful consequences of nature's principle of minimizing the \ac{vfe}
functional in the least amount of time, in this dissertation we are motivated by the challenge of
developing useful software tool for realizing synthetic \ac{aif} agents in which all processing follows 
from efficient \ac{vfe} minimization implemented as approximate, real-time, scalable and robust Bayesian inference. 
$\mathbin{^*\!~}$ In this dissertation we specifically focus on approximate Bayesian inference as a foundation for \ac{aif}.
  }
\end{figure}

Can we design a Bayesian inference framework that implements all of these properties? Specifically, in this thesis we aim to answer the following main question:

\newcommand{\mainquestion}{What is a suitable
  implementation for real-time, efficient, and robust Bayesian inference in a large-scale model for streaming data?
}

\begin{rqbox}
  \mainquestion
\end{rqbox}

Answering this question would be essential to design intelligent synthetic \ac{aif} agents that operate
continuously, act reactively, and learn autonomously.
It is worth noting that such an implementation would be useful not only for \ac{aif} agents but for
any application where real-time Bayesian inference is required, including audio processing,
self-driving vehicles, weather forecasting, extended reality video processing, and others.
Even though we take our motivation and inspiration from \ac{fep} and \ac{aif}, we are aware that progress on real-time, scalable  Bayesian inference may lead to an impact beyond the domain of synthetic \ac{aif} agents.
Therefore, our objective is to design a comprehensive Bayesian inference implementation for many related applications.
In particular, the work is driven by the following concrete research questions.

\newcommand{\scalabilityquestion}{
  How can Bayesian inference be efficiently realized in large multidimensional probabilistic
  generative models?
}

\begin{questions}
  \item \textbf{Scalability}. \scalabilityquestion
  \label{question:scalability}
\end{questions}

The world is a complex environment and any adequate probabilistic generative model of such an
environment would likely include thousands of variables.
The architecture for inference should scale comfortably to support large probabilistic models
that may involve both continuous and discrete states, as well as complex relationships between
these variables.

\newcommand{\reactivityquestion}{
  How can Bayesian inference be guaranteed to process streaming data in real-time?
}

\begin{questions}[resume] 
  \item \textbf{Real-time processing}. \reactivityquestion
  \label{question:reactivity}
\end{questions}

Bayesian inference in \ac{aif} agents in real-time applications should operate without "resetting" or "ending".
It should continuously incorporate new observations from streaming data, ensuring that
previously inferred information is not ignored.
This real-time processing of streaming data is essential for implementing an action-perception
control loop, where timely inference is crucial.
By linking the notion of real-time processing with streaming data, we must acknowledge the
dynamic nature of the real world and the need for continual Bayesian inference updates.

\newcommand{\robustnessquestion}{
  How can Bayesian inference continue without interruption under structural model adaptation?
}

\begin{questions}[resume]
  \item \textbf{Robustness}. \robustnessquestion
  \label{question:robustness}
\end{questions}

A real-world environment is prone to change and the existing structure of a probabilistic model may become suboptimal. 
As an intelligent agent interacts with its environment and collects new data, it seeks to update its probabilistic model structure to better capture the underlying patterns and dynamics of the observed phenomena.
There are situations, however, where changes in the probabilistic model occur unintentionally.
For instance, sudden failure or malfunction of a model component may be driven by a collision with another object. 
In these circumstances, it is crucial for the inference process to maintain operational continuity as much as possible.
The system should be designed to handle structural changes in the model.
This requires the ability to recover from failures, repair or update the model when necessary, and seamlessly continue the inference process without interruptions.

\newcommand{\userexperiencequstion}{
  How can we implement such a framework in a user-friendly manner?
}

\begin{questions}[resume]
  \item \textbf{User Experience}. \userexperiencequstion
  \label{question:user-experience}
\end{questions}

Consider popular machine learning frameworks such as \texttt{TensorFlow} or \texttt{PyTorch},
which have greatly simplified the implementation of Deep Learning (DL) models and made
them accessible to a wider audience.
These frameworks provide user-friendly interfaces, abstracting away many technical
complexities and allowing users to focus on model design and experimentation.
In the context of real-time Bayesian inference for autonomous intelligent agents, it is
crucial to present the implementation in a similarly user-friendly manner.
The focus on user experience is essential to reduce the required level of competence, thus allowing
a wider range of individuals to participate in the design and implementation of these
sophisticated systems.

\newcommand{\utilityquestion}{
  How can we support a wide range of probabilistic models and inference constraints?
}

\begin{questions}[resume]
  \item \textbf{Utility}. \utilityquestion
  \label{question:utility}
\end{questions}

Although it is possible to implement fast and robust real-time Bayesian inference for specific
models and specific applications, our objective is to create a universal approach that can universally accommodate
a wide range of probabilistic models and inference constraints.
By addressing this broader scope, we aim to develop a versatile framework that provides
practical utility and extends its applicability beyond specific use cases.

