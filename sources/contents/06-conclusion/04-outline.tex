\section{Final thoughts}\label{chapter-06:section:outline}


The importance of having the right tool to test research ideas is often underestimated.
Today researchers use computer software to explore their ideas and communicate their results.
We do not write matrix-vector multiplication routines from scratch and use libraries like
\texttt{BLAS} or \texttt{numpy}. 
We do not draw plots by hand, but use \texttt{ggplot} or \texttt{matplotlib}.
Easy access to the appropriate tools is essential for modern research.
However, when the right tool does not exist, we have no choice but to create it and share it with others. 
Implementing scientific software itself is a complex task that comes with additional
challenges that are not typically encountered in traditional software development.
Scientific software often undergoes active research, resulting in rapidly changing
requirements and design decisions.
In addition, it is particularly challenging to design software that addresses novel research ideas, which have never been explored before or are not well understood.
The requirements may change as research progresses and new insights emerge.
The software designer must invest time and effort to gain a deep understanding of the
domain to develop the appropriate software architecture and algorithms.
Furthermore, there may be no well-established methodologies or best practices to follow.
This means that the software designer must navigate uncharted territory and make informed
design decisions without existing frameworks or guidelines on which to rely.
In the end, the software should be flexible to accommodate different needs and ideas while being 
user-friendly and efficient.
The purpose of the present dissertation was to provide a powerful and flexible tool that helps others find solutions to related questions in a reproducible, user-friendly, performant, and enjoyable way.
Although the resulting tool may not be the perfect one, it appears to serve its purpose while a better tool is
being developed. 
