\section{Constraints specification}\label{chapter-04:section:constraints-specification}

\subsection{The \texttt{@constraints} language}

The RxInfer framework uses a special language to specify constraints for the variational
family of distributions $\mathcal{Q}_B$.
This language is implemented as a macro called \jlinl{@constraints}.
The \jlinl{@constraints} macro parses a textual description of the variational family of
distributions and generates an internal data structure used during the inference process.
An example of constraints specification for the variational family of distributions
$\mathcal{Q}_B$ is shown in Listing~\ref{lst:rxi:constraints_example}.
\begin{figure*}[h!]
  \begin{adjustbox}{minipage=\textwidth,margin=0pt \smallskipamount,center}
    \jlinputlisting[label={lst:rxi:constraints_example}, caption={An example of constraints specification for the variational family of distributions $\mathcal{Q}_B$.
          In this example, we assume that the model has three random variables: $x$, $y$, and $z$.
          The constraints impose an additional factorization constraint on the variational distribution $q(x, y, z)$, requiring it to be of the form $q(x, y)q(z)$.
          Additionally, the constraints specify that the functional form of the variational distribution $q(x)$ must follow the Normal distribution.
          In Julia, the \jlinl{::} operator is used to specify the type of an argument in a function definition.
        },captionpos=b]{contents/04-rxinfer/code/03-constraints.jl}
  \end{adjustbox}
\end{figure*}

In this example, the model includes three random variables: $x$, $y$, and $z$.
The constraints impose an additional factorization constraint on the variational distribution
$q(x, y, z)$, requiring it to be in the form of $q(x, y)q(z)$.
Furthermore, the constraints specify that the functional form of the variational distribution
$q(x)$ must follow the Normal distribution.
In Julia, the \jlinl{::} operator is used to specify the type of an argument in a function
definition.

\subsection{The naive mean-field constraint}

A shorthand notation can be used to specify factorization constraints for a sequence of random
variables $\bm{s}$.
For example, the following constraints, as shown in Listing~\ref{lst:rxi:constraints_mf},
specify the naive mean-field factorization constraint~\eqref{eq:bfe:factorization_mean_field}
for a sequence of random variables $\bm{s}$.
\begin{figure*}[h!]
  \begin{adjustbox}{minipage=\textwidth,margin=0pt \smallskipamount,center}
    \jlinputlisting[label={lst:rxi:constraints_mf}, caption={A shorthand notation for the factorization constraint over a sequence of random variables $\bm{s}$.
        },captionpos=b]{contents/04-rxinfer/code/03-constraints-mf.jl}
  \end{adjustbox}
\end{figure*}

\subsection{Conditional constraints}

The \jlinl{@constraints} macro can be defined as a function with additional arguments that
allow conditional specification of constraints based on input values.
Consider the example provided in Listing~\ref{lst:rxi:constraints_function}, which conditionally
specifies the naive mean-field factorization
constraint~\eqref{eq:bfe:factorization_mean_field} for a sequence of random variables $\bm{s}$
and functional form constraints based on the input values of its arguments.
\begin{figure*}[h!]
  \begin{adjustbox}{minipage=\textwidth,margin=0pt \smallskipamount,center}
    \jlinputlisting[label={lst:rxi:constraints_function}, caption={An example of conditional constraints construction based on the input values of the arguments in the constraints specification function.
          In this example, the constraint function conditionally specifies the naive mean-field
          factorization constraint~\eqref{eq:bfe:factorization_mean_field} for a sequence of random
          variables $\bm{s}$ and functional form constraints based on the input values of its arguments.
        },captionpos=b]{contents/04-rxinfer/code/03-constraints-function.jl}
  \end{adjustbox}
\end{figure*}

