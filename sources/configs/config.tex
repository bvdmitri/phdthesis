% Version 1.0 by Joos Buijs, Eindhoven, August 2014
% Use at your own risk.
% Attribution is appreciated

% Personal packages
\usepackage[b5paper, total={5in, 7in}, left=0.9in, top=1.5in]{geometry}
\usepackage{soul}
\usepackage{epigraph}
\usepackage{booktabs}
\usepackage{nicefrac}
\usepackage{amsthm}
\usepackage{thmtools}
\declaretheorem{theorem}
\declaretheoremstyle[%
  spaceabove=-6pt,%
  spacebelow=7pt,%
  headfont=\normalfont\itshape,%
  postheadspace=1em,%
  qed=\qedsymbol%
]{mystyle}
\declaretheorem[name={Proof},style=mystyle,unnumbered,
]{prf}

%
% The style file for Joos' PhD thesis
% Contains settings, macro definitions and package inclusions
%

% Loosly based on the classicthesis config file (but then stripped down since documentclass book was way better)

% ****************************************************************************************************
% Custom commands
% ****************************************************************************************************
% \usepackage{pgfplots}
\usetikzlibrary{intersections}
\usepackage{bm}
\usetikzlibrary{positioning}
% \usepgfplotslibrary{external} 
% \tikzexternalize
\usepgfplotslibrary{fillbetween}
\usepgfplotslibrary{patchplots}
\pgfplotsset{compat=1.5}
\usetikzlibrary{pgfplots.groupplots}
\usetikzlibrary{shapes.geometric,backgrounds}

\usepackage{tikz}
\usepackage{xifthen}
\usepackage{tkz-euclide}

\usetikzlibrary{external}
% \tikzexternalize[shell escape=-enable-write18] %- See more at: http://www.howtotex.com/tips-tricks/
% \tikzexternalize[prefix=figures,shell escape=-enable-write18]
\pgfsetlayers{background,main}  % set the order of the layers (main is the standard layer)
\newcommand*\circled[1]{\tikz[baseline=(char.base)]{
            \node[shape=circle,draw,inner sep=1pt, text width=3mm, align=center] (char) {#1};}}
\newcommand*\darkcircled[1]{\tikz[baseline=(char.base)]{
            \node[shape=circle,draw,inner sep=1pt, text width=3mm, align=center, fill=darkgrey, text=white, font=\bfseries] (char) {#1};}}
\newcommand*\darkredcircled[1]{\tikz[baseline=(char.base)]{
            \node[shape=circle,draw,inner sep=1pt, text width=3mm, align=center, color=red!80!black] (char) {#1};}}
\newcommand*\smallcircled[1]{\tikz[baseline=(char.base)]{
            \node[shape=circle, draw=black, text=black,draw,inner sep=0pt,minimum width=3.5mm] (char) {#1};}}
\newcommand*\bwcircled[1]{\tikz[baseline=(char.base)]{
            \node[shape=circle,double,draw,inner sep=0pt, text width=3mm, align=center] (char) {#1};}}
            
\newcommand*\blacksmallcircled[1]{\tikz[baseline=(char.base)]{
\node[shape=circle,fill=darkgrey, text=white, draw,inner sep=0pt,minimum width=3.5mm] (char) {#1};}}
            
\newcommand*\redsmallcircled[1]{\tikz[baseline=(char.base)]{
            \node[shape=circle, draw=red, text=red,draw,inner sep=0pt,minimum width=3.5mm] (char) {#1};}}
            
\newcommand*\reddarksmallcircled[1]{\tikz[baseline=(char.base)]{
           \node[shape=circle, draw=red, text=white,,fill=red, draw,inner sep=0pt,minimum width=3.5mm] (char) {#1};}}

\usetikzlibrary{calc, arrows, arrows.meta, fit, positioning, patterns, decorations.pathreplacing, shapes, angles}
\tikzstyle{dash} = [dashed, -latex,>=latex]
\tikzstyle{line} = [draw, -latex,>=latex]
\tikzstyle{smallbox} = [draw, minimum size=5.0mm]
\tikzstyle{box} = [draw, minimum size=7.0mm]
\tikzstyle{bigbox} = [draw, minimum size=10.0mm]
\tikzstyle{hugebox} = [draw, minimum size=13.0mm]
\tikzstyle{rectangle} = [draw, minimum width=10.0mm, minimum height=20.0mm]
\tikzstyle{switch} = [trapezium, trapezium angle=120, draw, rotate=90,  inner ysep=5pt, outer sep=5pt,
minimum height=7mm, minimum width=7mm]
\tikzstyle{roundbox} = [draw, circle, inner sep=0pt, minimum size=5mm]
\tikzstyle{clamped} = [draw, fill=black, minimum size=0.15cm]
\tikzstyle{msgcircle} = [shape=circle, draw, inner sep=0pt, minimum size=4mm, fill=white, font=\scriptsize]
\tikzstyle{darkmsgcircle} = [shape=circle, draw, inner sep=0pt, minimum size=4mm, fill=darkgrey, text=white, font=\scriptsize]
\tikzstyle{redmsgcircle} = [shape=circle, draw=red, inner sep=0pt, minimum size=4mm, text=red, font=\scriptsize]
\tikzstyle{reddarkmsgcircle} = [shape=circle, draw=red, inner sep=0pt, minimum size=4mm, fill=red, text=white, font=\scriptsize]
\tikzstyle{msgdoublecircle} = [shape=circle, double, double distance=1.5pt, draw, inner sep=0pt, minimum size=5mm, fill=white]
\tikzstyle{darkmsgdoublecircle} = [shape=circle, double, double distance=1.5pt, draw, inner sep=0pt, minimum size=5mm, fill=darkgrey, text=white, font=\bfseries]
\tikzstyle{qbox} = [draw, circle, black!20!white, thick, densely dotted, minimum size=2mm, inner sep=3pt, outer sep=3pt, fill=darkgrey, fill opacity=0.1, font=\tiny, text=black]

% Plate notation argument order: x-left, x-right, y-top, y-bottom
\newcommand{\brackets}[4]{
	\draw[dashed, rounded corners=0.3cm, line width = 1pt] ($({#1},{#3})+(-0.3,0)$) -- ($({#1},{#3})+(-0.3,0.3)$) -- ($({#2},{#3})+(0.3,0.3)$) -- ($({#2},{#3})+(0.3,0)$);
	\draw[rounded corners=0.3cm, line width = 1pt] ($({#1},{#4})+(-0.3,0)$) -- ($({#1},{#4})+(-0.3,-0.3)$) -- ($({#2},{#4})+(0.3,-0.3)$) -- ($({#2},{#4})+(0.3,0)$);}

\newcommand{\parentheses}[4]{
      \draw[rounded corners=0.3cm, line width = 0.75pt] ($({#1},{#3})+(0,0.3)$) -- ($({#1},{#3})+(-0.3,0)$) -- ($({#1},{#4})+(-0.3,0)$) -- ($({#1},{#4})+(0,-0.3)$);
      \draw[rounded corners=0.3cm, line width = 0.75pt] ($({#2},{#3})+(0,0.3)$) -- ($({#2},{#3})+(0.3,0)$) -- ($({#2},{#4})+(0.3,0)$) -- ($({#2},{#4})+(0,-0.3)$);}

% Message notation argument order:
%     Circle position relative to arrow [up, down, left, right];
%     Arrow direction [up, down, left, right];
%     Start node coordinate;
%     End node coordinate
%     Position along edge;
%     Message text;
\newcommand{\msg}[6]{
      % Circle left arrow down
      \ifthenelse{\isin{#1}{left} \AND \isin{#2}{down}}{
            \coordinate (anchor) at ($({#3})!{#5}!({#4})$);
            \node[xshift=-6.0mm] at (anchor) {#6};
            \node[xshift=-1.0mm] at (anchor) {$\downarrow$};
      }{}
      % Circle right arrow down
      \ifthenelse{\isin{#1}{right} \AND \isin{#2}{down}}{
            \coordinate (anchor) at ($({#3})!{#5}!({#4})$);
            \node[xshift=6.0mm] at (anchor) {#6};
            \node[xshift=1.0mm] at (anchor) {$\downarrow$};
      }{}

      % Circle down arrow right
      \ifthenelse{\isin{#1}{down} \AND \isin{#2}{right}}{
            \coordinate (anchor) at ($({#3})!{#5}!({#4})$);
            \node[ yshift=-4.0mm] at (anchor) {#6};
            \node[yshift=-1.0mm] at (anchor) {$\rightarrow$};
      }{}
      % Circle up arrow right
      \ifthenelse{\isin{#1}{up} \AND \isin{#2}{right}}{
            \coordinate (anchor) at ($({#3})!{#5}!({#4})$);
            \node[ yshift=4.0mm] at (anchor) {#6};
            \node[yshift=1.0mm] at (anchor) {$\rightarrow$};
      }{}

      % Circle down arrow left
      \ifthenelse{\isin{#1}{down} \AND \isin{#2}{left}}{
            \coordinate (anchor) at ($({#3})!{#5}!({#4})$);
            \node[ yshift=-4.0mm] at (anchor) {#6};
            \node[yshift=-1.0mm] at (anchor) {$\leftarrow$};
      }{}
      % Circle up arrow left
      \ifthenelse{\isin{#1}{up} \AND \isin{#2}{left}}{
            \coordinate (anchor) at ($({#3})!{#5}!({#4})$);
            \node[ yshift=4.0mm] at (anchor) {#6};
            \node[yshift=1.0mm] at (anchor) {$\leftarrow$};
      }{}

      % Circle left arrow down
      \ifthenelse{\isin{#1}{left} \AND \isin{#2}{up}}{
            \coordinate (anchor) at ($({#3})!{#5}!({#4})$);
            \node[ xshift=-6.0mm] at (anchor) {#6};
            \node[xshift=-1.0mm] at (anchor) {$\uparrow$};
      }{}
      % Circle right arrow down
      \ifthenelse{\isin{#1}{right} \AND \isin{#2}{up}}{
            \coordinate (anchor) at ($({#3})!{#5}!({#4})$);
            \node[ xshift=6.0mm] at (anchor) {#6};
            \node[xshift=1.0mm] at (anchor) {$\uparrow$};
      }{}
}

% Dark messages
\newcommand{\darkmsg}[6]{
      % Circle left arrow down
      \ifthenelse{\isin{#1}{left} \AND \isin{#2}{down}}{
            \coordinate (anchor) at ($({#3})!{#5}!({#4})$);
            \node[darkmsgcircle, xshift=-5.5mm] at (anchor) {#6};
            \node[xshift=-1.5mm] at (anchor) {$\downarrow$};
      }{}
      % Circle right arrow down
      \ifthenelse{\isin{#1}{right} \AND \isin{#2}{down}}{
            \coordinate (anchor) at ($({#3})!{#5}!({#4})$);
            \node[darkmsgcircle, xshift=5.5mm] at (anchor) {#6};
            \node[xshift=1.5mm] at (anchor) {$\downarrow$};
      }{}

      % Circle down arrow right
      \ifthenelse{\isin{#1}{down} \AND \isin{#2}{right}}{
            \coordinate (anchor) at ($({#3})!{#5}!({#4})$);
            \node[darkmsgcircle, yshift=-6.0mm] at (anchor) {#6};
            \node[yshift=-2.0mm] at (anchor) {$\rightarrow$};
      }{}
      % Circle up arrow right
      \ifthenelse{\isin{#1}{up} \AND \isin{#2}{right}}{
            \coordinate (anchor) at ($({#3})!{#5}!({#4})$);
            \node[darkmsgcircle, yshift=6.0mm] at (anchor) {#6};
            \node[yshift=2.0mm] at (anchor) {$\rightarrow$};
      }{}

      % Circle down arrow left
      \ifthenelse{\isin{#1}{down} \AND \isin{#2}{left}}{
            \coordinate (anchor) at ($({#3})!{#5}!({#4})$);
            \node[darkmsgcircle, yshift=-6.0mm] at (anchor) {#6};
            \node[yshift=-2.0mm] at (anchor) {$\leftarrow$};
      }{}
      % Circle up arrow left
      \ifthenelse{\isin{#1}{up} \AND \isin{#2}{left}}{
            \coordinate (anchor) at ($({#3})!{#5}!({#4})$);
            \node[darkmsgcircle, yshift=6.0mm] at (anchor) {#6};
            \node[yshift=2.0mm] at (anchor) {$\leftarrow$};
      }{}

      % Circle left arrow down
      \ifthenelse{\isin{#1}{left} \AND \isin{#2}{up}}{
            \coordinate (anchor) at ($({#3})!{#5}!({#4})$);
            \node[darkmsgcircle, xshift=-5.5mm] at (anchor) {#6};
            \node[xshift=-1.5mm] at (anchor) {$\uparrow$};
      }{}
      % Circle right arrow down
      \ifthenelse{\isin{#1}{right} \AND \isin{#2}{up}}{
            \coordinate (anchor) at ($({#3})!{#5}!({#4})$);
            \node[darkmsgcircle, xshift=5.5mm] at (anchor) {#6};
            \node[xshift=1.5mm] at (anchor) {$\uparrow$};
      }{}
}

\newcommand{\bwmsg}[6]{
      % doublecircle left arrow down
      \ifthenelse{\isin{#1}{left} \AND \isin{#2}{down}}{
            \coordinate (anchor) at ($({#3})!{#5}!({#4})$);
            \node[msgdoublecircle, xshift=-5.5mm] at (anchor) {#6};
            \node[xshift=-1.5mm] at (anchor) {$\downarrow$};
      }{}
      % doublecircle right arrow down
      \ifthenelse{\isin{#1}{right} \AND \isin{#2}{down}}{
            \coordinate (anchor) at ($({#3})!{#5}!({#4})$);
            \node[msgdoublecircle, xshift=5.5mm] at (anchor) {#6};
            \node[xshift=1.5mm] at (anchor) {$\downarrow$};
      }{}

      % doublecircle down arrow right
      \ifthenelse{\isin{#1}{down} \AND \isin{#2}{right}}{
            \coordinate (anchor) at ($({#3})!{#5}!({#4})$);
            \node[msgdoublecircle, yshift=-6.0mm] at (anchor) {#6};
            \node[yshift=-2.0mm] at (anchor) {$\rightarrow$};
      }{}
      % doublecircle up arrow right
      \ifthenelse{\isin{#1}{up} \AND \isin{#2}{right}}{
            \coordinate (anchor) at ($({#3})!{#5}!({#4})$);
            \node[msgdoublecircle, yshift=6.0mm] at (anchor) {#6};
            \node[yshift=2.0mm] at (anchor) {$\rightarrow$};
      }{}

      % doublecircle down arrow left
      \ifthenelse{\isin{#1}{down} \AND \isin{#2}{left}}{
            \coordinate (anchor) at ($({#3})!{#5}!({#4})$);
            \node[msgdoublecircle, yshift=-6.0mm] at (anchor) {#6};
            \node[yshift=-2.0mm] at (anchor) {$\leftarrow$};
      }{}
      % doublecircle up arrow left
      \ifthenelse{\isin{#1}{up} \AND \isin{#2}{left}}{
            \coordinate (anchor) at ($({#3})!{#5}!({#4})$);
            \node[msgdoublecircle, yshift=6.0mm] at (anchor) {#6};
            \node[yshift=2.0mm] at (anchor) {$\leftarrow$};
      }{}

      % doublecircle left arrow down
      \ifthenelse{\isin{#1}{left} \AND \isin{#2}{up}}{
            \coordinate (anchor) at ($({#3})!{#5}!({#4})$);
            \node[msgdoublecircle, xshift=-5.5mm] at (anchor) {#6};
            \node[xshift=-1.5mm] at (anchor) {$\uparrow$};
      }{}
      % doublecircle right arrow down
      \ifthenelse{\isin{#1}{right} \AND \isin{#2}{up}}{
            \coordinate (anchor) at ($({#3})!{#5}!({#4})$);
            \node[msgdoublecircle, xshift=5.5mm] at (anchor) {#6};
            \node[xshift=1.5mm] at (anchor) {$\uparrow$};
      }{}
}

% Dark messages
\newcommand{\bwdarkmsg}[6]{
      % doublecircle left arrow down
      \ifthenelse{\isin{#1}{left} \AND \isin{#2}{down}}{
            \coordinate (anchor) at ($({#3})!{#5}!({#4})$);
            \node[darkmsgdoublecircle, xshift=-5.5mm] at (anchor) {#6};
            \node[xshift=-1.5mm] at (anchor) {$\downarrow$};
      }{}
      % doublecircle right arrow down
      \ifthenelse{\isin{#1}{right} \AND \isin{#2}{down}}{
            \coordinate (anchor) at ($({#3})!{#5}!({#4})$);
            \node[darkmsgdoublecircle, xshift=5.5mm] at (anchor) {#6};
            \node[xshift=1.5mm] at (anchor) {$\downarrow$};
      }{}

      % doublecircle down arrow right
      \ifthenelse{\isin{#1}{down} \AND \isin{#2}{right}}{
            \coordinate (anchor) at ($({#3})!{#5}!({#4})$);
            \node[darkmsgdoublecircle, yshift=-6.0mm] at (anchor) {#6};
            \node[yshift=-2.0mm] at (anchor) {$\rightarrow$};
      }{}
      % doublecircle up arrow right
      \ifthenelse{\isin{#1}{up} \AND \isin{#2}{right}}{
            \coordinate (anchor) at ($({#3})!{#5}!({#4})$);
            \node[darkmsgdoublecircle, yshift=6.0mm] at (anchor) {#6};
            \node[yshift=2.0mm] at (anchor) {$\rightarrow$};
      }{}

      % doublecircle down arrow left
      \ifthenelse{\isin{#1}{down} \AND \isin{#2}{left}}{
            \coordinate (anchor) at ($({#3})!{#5}!({#4})$);
            \node[darkmsgdoublecircle, yshift=-6.0mm] at (anchor) {#6};
            \node[yshift=-2.0mm] at (anchor) {$\leftarrow$};
      }{}
      % doublecircle up arrow left
      \ifthenelse{\isin{#1}{up} \AND \isin{#2}{left}}{
            \coordinate (anchor) at ($({#3})!{#5}!({#4})$);
            \node[darkmsgdoublecircle, yshift=6.0mm] at (anchor) {#6};
            \node[yshift=2.0mm] at (anchor) {$\leftarrow$};
      }{}

      % doublecircle left arrow down
      \ifthenelse{\isin{#1}{left} \AND \isin{#2}{up}}{
            \coordinate (anchor) at ($({#3})!{#5}!({#4})$);
            \node[darkmsgdoublecircle, xshift=-5.5mm] at (anchor) {#6};
            \node[xshift=-1.5mm] at (anchor) {$\uparrow$};
      }{}
      % doublecircle right arrow down
      \ifthenelse{\isin{#1}{right} \AND \isin{#2}{up}}{
            \coordinate (anchor) at ($({#3})!{#5}!({#4})$);
            \node[darkmsgdoublecircle, xshift=5.5mm] at (anchor) {#6};
            \node[xshift=1.5mm] at (anchor) {$\uparrow$};
      }{}
}

\newcommand{\redbackmsg}[6]{
      % Circle left arrow down
      \ifthenelse{\isin{#1}{left} \AND \isin{#2}{down}}{
            \coordinate (anchor) at ($({#3})!{#5}!({#4})$);
            \node[reddarkmsgcircle, xshift=-5mm] at (anchor) {#6};
            \node[xshift=-1.5mm] at (anchor) {$\downarrow$};
      }{}
      % Circle right arrow down
      \ifthenelse{\isin{#1}{right} \AND \isin{#2}{down}}{
            \coordinate (anchor) at ($({#3})!{#5}!({#4})$);
            \node[reddarkmsgcircle, xshift=5mm] at (anchor) {#6};
            \node[xshift=1.5mm] at (anchor) {$\downarrow$};
      }{}

      % Circle down arrow right
      \ifthenelse{\isin{#1}{down} \AND \isin{#2}{right}}{
            \coordinate (anchor) at ($({#3})!{#5}!({#4})$);
            \node[reddarkmsgcircle, yshift=-5.0mm] at (anchor) {#6};
            \node[yshift=-2.0mm] at (anchor) {$\rightarrow$};
      }{}
      % Circle up arrow right
      \ifthenelse{\isin{#1}{up} \AND \isin{#2}{right}}{
            \coordinate (anchor) at ($({#3})!{#5}!({#4})$);
            \node[reddarkmsgcircle, yshift=5.0mm] at (anchor) {#6};
            \node[yshift=2.0mm] at (anchor) {$\rightarrow$};
      }{}

      % Circle down arrow left
      \ifthenelse{\isin{#1}{down} \AND \isin{#2}{left}}{
            \coordinate (anchor) at ($({#3})!{#5}!({#4})$);
            \node[reddarkmsgcircle, yshift=-5.0mm] at (anchor) {#6};
            \node[yshift=-2.0mm] at (anchor) {$\leftarrow$};
      }{}
      % Circle up arrow left
      \ifthenelse{\isin{#1}{up} \AND \isin{#2}{left}}{
            \coordinate (anchor) at ($({#3})!{#5}!({#4})$);
            \node[reddarkmsgcircle, yshift=5.0mm] at (anchor) {#6};
            \node[yshift=2.0mm] at (anchor) {$\leftarrow$};
      }{}

      % Circle left arrow down
      \ifthenelse{\isin{#1}{left} \AND \isin{#2}{up}}{
            \coordinate (anchor) at ($({#3})!{#5}!({#4})$);
            \node[reddarkmsgcircle, xshift=-5.0mm] at (anchor) {#6};
            \node[xshift=-1.5mm] at (anchor) {$\uparrow$};
      }{}
      % Circle right arrow down
      \ifthenelse{\isin{#1}{right} \AND \isin{#2}{up}}{
            \coordinate (anchor) at ($({#3})!{#5}!({#4})$);
            \node[reddarkmsgcircle, xshift=5.0mm] at (anchor) {#6};
            \node[xshift=1.5mm] at (anchor) {$\uparrow$};
      }{}
}

\newcommand{\redmsg}[6]{
      % Circle left arrow down
      \ifthenelse{\isin{#1}{left} \AND \isin{#2}{down}}{
            \coordinate (anchor) at ($({#3})!{#5}!({#4})$);
            \node[redmsgcircle, xshift=-5mm] at (anchor) {#6};
            \node[xshift=-1.5mm] at (anchor) {$\downarrow$};
      }{}
      % Circle right arrow down
      \ifthenelse{\isin{#1}{right} \AND \isin{#2}{down}}{
            \coordinate (anchor) at ($({#3})!{#5}!({#4})$);
            \node[redmsgcircle, xshift=5mm] at (anchor) {#6};
            \node[xshift=1.5mm] at (anchor) {$\downarrow$};
      }{}

      % Circle down arrow right
      \ifthenelse{\isin{#1}{down} \AND \isin{#2}{right}}{
            \coordinate (anchor) at ($({#3})!{#5}!({#4})$);
            \node[redmsgcircle, yshift=-5.0mm] at (anchor) {#6};
            \node[yshift=-2.0mm] at (anchor) {$\rightarrow$};
      }{}
      % Circle up arrow right
      \ifthenelse{\isin{#1}{up} \AND \isin{#2}{right}}{
            \coordinate (anchor) at ($({#3})!{#5}!({#4})$);
            \node[redmsgcircle, yshift=5.0mm] at (anchor) {#6};
            \node[yshift=2.0mm] at (anchor) {$\rightarrow$};
      }{}

      % Circle down arrow left
      \ifthenelse{\isin{#1}{down} \AND \isin{#2}{left}}{
            \coordinate (anchor) at ($({#3})!{#5}!({#4})$);
            \node[redmsgcircle, yshift=-5.0mm] at (anchor) {#6};
            \node[yshift=-2.0mm] at (anchor) {$\leftarrow$};
      }{}
      % Circle up arrow left
      \ifthenelse{\isin{#1}{up} \AND \isin{#2}{left}}{
            \coordinate (anchor) at ($({#3})!{#5}!({#4})$);
            \node[redmsgcircle, yshift=5.0mm] at (anchor) {#6};
            \node[yshift=2.0mm] at (anchor) {$\leftarrow$};
      }{}

      % Circle left arrow down
      \ifthenelse{\isin{#1}{left} \AND \isin{#2}{up}}{
            \coordinate (anchor) at ($({#3})!{#5}!({#4})$);
            \node[redmsgcircle, xshift=-5.0mm] at (anchor) {#6};
            \node[xshift=-1.5mm] at (anchor) {$\uparrow$};
      }{}
      % Circle right arrow down
      \ifthenelse{\isin{#1}{right} \AND \isin{#2}{up}}{
            \coordinate (anchor) at ($({#3})!{#5}!({#4})$);
            \node[redmsgcircle, xshift=5.0mm] at (anchor) {#6};
            \node[xshift=1.5mm] at (anchor) {$\uparrow$};
      }{}
}

\newcommand{\msgcircle}[6]{
      % Circle left arrow down
      \ifthenelse{\isin{#1}{left} \AND \isin{#2}{down}}{
            \coordinate (anchor) at ($({#3})!{#5}!({#4})$);
            \node[msgcircle,xshift=-5.0mm] at (anchor) {#6};
            \node[xshift=-1.5mm] at (anchor) {$\downarrow$};
      }{}
      % Circle right arrow down
      \ifthenelse{\isin{#1}{right} \AND \isin{#2}{down}}{
            \coordinate (anchor) at ($({#3})!{#5}!({#4})$);
            \node[msgcircle,xshift=5.0mm] at (anchor) {#6};
            \node[xshift=1.5mm] at (anchor) {$\downarrow$};
      }{}

      % Circle down arrow right
      \ifthenelse{\isin{#1}{down} \AND \isin{#2}{right}}{
            \coordinate (anchor) at ($({#3})!{#5}!({#4})$);
            \node[msgcircle, yshift=-5.0mm] at (anchor) {#6};
            \node[yshift=-2.0mm] at (anchor) {$\rightarrow$};
      }{}
      % Circle up arrow right
      \ifthenelse{\isin{#1}{up} \AND \isin{#2}{right}}{
            \coordinate (anchor) at ($({#3})!{#5}!({#4})$);
            \node[msgcircle, yshift=5.0mm] at (anchor) {#6};
            \node[yshift=2.0mm] at (anchor) {$\rightarrow$};
      }{}

      % Circle down arrow left
      \ifthenelse{\isin{#1}{down} \AND \isin{#2}{left}}{
            \coordinate (anchor) at ($({#3})!{#5}!({#4})$);
            \node[msgcircle, yshift=-5.0mm] at (anchor) {#6};
            \node[yshift=-2.0mm] at (anchor) {$\leftarrow$};
      }{}
      % Circle up arrow left
      \ifthenelse{\isin{#1}{up} \AND \isin{#2}{left}}{
            \coordinate (anchor) at ($({#3})!{#5}!({#4})$);
            \node[msgcircle, yshift=5.0mm] at (anchor) {#6};
            \node[yshift=2.0mm] at (anchor) {$\leftarrow$};
      }{}

      % Circle left arrow down
      \ifthenelse{\isin{#1}{left} \AND \isin{#2}{up}}{
            \coordinate (anchor) at ($({#3})!{#5}!({#4})$);
            \node[msgcircle, xshift=-5.0mm] at (anchor) {#6};
            \node[xshift=-1.5mm] at (anchor) {$\uparrow$};
      }{}
      % Circle right arrow down
      \ifthenelse{\isin{#1}{right} \AND \isin{#2}{up}}{
            \coordinate (anchor) at ($({#3})!{#5}!({#4})$);
            \node[msgcircle, xshift=5.0mm] at (anchor) {#6};
            \node[xshift=1.5mm] at (anchor) {$\uparrow$};
      }{}
}

% nodes style 
\tikzset{mainstyle/.style={fill=white, draw=black, shape=rectangle, align=center}}
\tikzset{istyle/.style={mainstyle, draw=white, minimum size=0mm, inner sep=0pt, text width=0mm}}
\tikzset{dstyle/.style={mainstyle, minimum size=5mm, inner sep=0pt, text width=4mm}}
\tikzset{sstyle/.style={mainstyle, minimum size=8mm, inner sep=0pt, text width=5mm}}
\tikzset{ostyle/.style={fill=darkgrey, draw=black, shape=rectangle, minimum size=0.2cm, inner sep=0pt, text width=2mm}}

% main nodes
\tikzstyle{observation}=[ostyle]
\tikzstyle{deterministic}=[dstyle]
\tikzstyle{stochastic}=[sstyle]

% auxiliary nodes
\tikzstyle{filter}=[mainstyle, minimum width=1cm, minimum height=0.5cm]
\tikzstyle{selector}=[fill=white, draw=black, shape=trapezium, rotate=180, minimum width=1cm, minimum height=0.5cm]

% auxiliary fonts
\newcommand{\tikzxmark}{%
\tikz[scale=0.23] {
    \draw[line width=0.7,line cap=round] (0,0) to [bend left=6] (1,1);
    \draw[line width=0.7,line cap=round] (0.2,0.95) to [bend right=3] (0.8,0.05);
}}
\newcommand{\tikzcmark}{%
\tikz[scale=0.23] {
    \draw[line width=0.7,line cap=round] (0.25,0) to [bend left=10] (1,1);
    \draw[line width=0.8,line cap=round] (0,0.35) to [bend right=1] (0.23,0);
}}

% plots preamble

\pgfplotsset{%
    layers/standard/.define layer set={%
        background,axis background,axis grid,axis ticks,axis lines,axis tick labels,pre main,main,axis descriptions,axis foreground%
    }{
        grid style={/pgfplots/on layer=axis grid},%
        tick style={/pgfplots/on layer=axis ticks},%
        axis line style={/pgfplots/on layer=axis lines},%
        label style={/pgfplots/on layer=axis descriptions},%
        legend style={/pgfplots/on layer=axis descriptions},%
        title style={/pgfplots/on layer=axis descriptions},%
        colorbar style={/pgfplots/on layer=axis descriptions},%
        ticklabel style={/pgfplots/on layer=axis tick labels},%
        axis background@ style={/pgfplots/on layer=axis background},%
        3d box foreground style={/pgfplots/on layer=axis foreground},%
    },
}


\newcommand{\phd}{Ph.D.\xspace}

%Personal names etc.
\newcommand{\myTerm}{myTerm} %Example of a term you use often. Prevent typos, use macros!

%Normal text abbreviations
\newcommand{\ie}{i.e.\xspace}
\newcommand{\Ie}{I.e.\xspace}
\newcommand{\eg}{e.g.\xspace}
\newcommand{\Eg}{E.g.\xspace}
\newcommand{\vs}{v.s.\xspace}
\newcommand{\etc}{etc.\xspace}
%the following used to work but is broken somehow...
%\newcommand*{\etc}{%
%    \@ifnextchar{.}%
%        {etc}%
%        {etc.\@\xspace}%
%}
\newcommand{\etal}{et al.\xspace}
\newcommand{\cf}{cf.\xspace}

%EXTEND when/if necessary

\newcommand{\bd}[1]{
  \begin{definition}
    (#1)\\}
    \newcommand{\ed}{
  \end{definition}
}

%This one needs/wants/has to be first
\usepackage[dvipsnames,table]{xcolor} % [dvipsnames]
\usepackage[topthumbmargin=55mm,width=12mm,eventxtindent=5mm,oddtxtexdent=3mm]{thumbs}
\usepackage{pgfplots}
\usetikzlibrary{intersections}
\usepackage{bm}
\usetikzlibrary{positioning}
% \usepgfplotslibrary{external} 
% \tikzexternalize
\usepgfplotslibrary{fillbetween}
\usepgfplotslibrary{patchplots}
\pgfplotsset{compat=1.5}
\usetikzlibrary{pgfplots.groupplots}
\usetikzlibrary{shapes.geometric,backgrounds}

\usepackage{tikz}
\usepackage{xifthen}
\usepackage{tkz-euclide}

\usetikzlibrary{external}
% \tikzexternalize[shell escape=-enable-write18] %- See more at: http://www.howtotex.com/tips-tricks/
% \tikzexternalize[prefix=figures,shell escape=-enable-write18]
\pgfsetlayers{background,main}  % set the order of the layers (main is the standard layer)
\newcommand*\circled[1]{\tikz[baseline=(char.base)]{
            \node[shape=circle,draw,inner sep=1pt, text width=3mm, align=center] (char) {#1};}}
\newcommand*\darkcircled[1]{\tikz[baseline=(char.base)]{
            \node[shape=circle,draw,inner sep=1pt, text width=3mm, align=center, fill=darkgrey, text=white, font=\bfseries] (char) {#1};}}
\newcommand*\darkredcircled[1]{\tikz[baseline=(char.base)]{
            \node[shape=circle,draw,inner sep=1pt, text width=3mm, align=center, color=red!80!black] (char) {#1};}}
\newcommand*\smallcircled[1]{\tikz[baseline=(char.base)]{
            \node[shape=circle, draw=black, text=black,draw,inner sep=0pt,minimum width=3.5mm] (char) {#1};}}
\newcommand*\bwcircled[1]{\tikz[baseline=(char.base)]{
            \node[shape=circle,double,draw,inner sep=0pt, text width=3mm, align=center] (char) {#1};}}
            
\newcommand*\blacksmallcircled[1]{\tikz[baseline=(char.base)]{
\node[shape=circle,fill=darkgrey, text=white, draw,inner sep=0pt,minimum width=3.5mm] (char) {#1};}}
            
\newcommand*\redsmallcircled[1]{\tikz[baseline=(char.base)]{
            \node[shape=circle, draw=red, text=red,draw,inner sep=0pt,minimum width=3.5mm] (char) {#1};}}
            
\newcommand*\reddarksmallcircled[1]{\tikz[baseline=(char.base)]{
           \node[shape=circle, draw=red, text=white,,fill=red, draw,inner sep=0pt,minimum width=3.5mm] (char) {#1};}}

\usetikzlibrary{calc, arrows, arrows.meta, fit, positioning, patterns, decorations.pathreplacing, shapes, angles}
\tikzstyle{dash} = [dashed, -latex,>=latex]
\tikzstyle{line} = [draw, -latex,>=latex]
\tikzstyle{smallbox} = [draw, minimum size=5.0mm]
\tikzstyle{box} = [draw, minimum size=7.0mm]
\tikzstyle{bigbox} = [draw, minimum size=10.0mm]
\tikzstyle{hugebox} = [draw, minimum size=13.0mm]
\tikzstyle{rectangle} = [draw, minimum width=10.0mm, minimum height=20.0mm]
\tikzstyle{switch} = [trapezium, trapezium angle=120, draw, rotate=90,  inner ysep=5pt, outer sep=5pt,
minimum height=7mm, minimum width=7mm]
\tikzstyle{roundbox} = [draw, circle, inner sep=0pt, minimum size=5mm]
\tikzstyle{clamped} = [draw, fill=black, minimum size=0.15cm]
\tikzstyle{msgcircle} = [shape=circle, draw, inner sep=0pt, minimum size=4mm, fill=white, font=\scriptsize]
\tikzstyle{darkmsgcircle} = [shape=circle, draw, inner sep=0pt, minimum size=4mm, fill=darkgrey, text=white, font=\scriptsize]
\tikzstyle{redmsgcircle} = [shape=circle, draw=red, inner sep=0pt, minimum size=4mm, text=red, font=\scriptsize]
\tikzstyle{reddarkmsgcircle} = [shape=circle, draw=red, inner sep=0pt, minimum size=4mm, fill=red, text=white, font=\scriptsize]
\tikzstyle{msgdoublecircle} = [shape=circle, double, double distance=1.5pt, draw, inner sep=0pt, minimum size=5mm, fill=white]
\tikzstyle{darkmsgdoublecircle} = [shape=circle, double, double distance=1.5pt, draw, inner sep=0pt, minimum size=5mm, fill=darkgrey, text=white, font=\bfseries]
\tikzstyle{qbox} = [draw, circle, black!20!white, thick, densely dotted, minimum size=2mm, inner sep=3pt, outer sep=3pt, fill=darkgrey, fill opacity=0.1, font=\tiny, text=black]

% Plate notation argument order: x-left, x-right, y-top, y-bottom
\newcommand{\brackets}[4]{
	\draw[dashed, rounded corners=0.3cm, line width = 1pt] ($({#1},{#3})+(-0.3,0)$) -- ($({#1},{#3})+(-0.3,0.3)$) -- ($({#2},{#3})+(0.3,0.3)$) -- ($({#2},{#3})+(0.3,0)$);
	\draw[rounded corners=0.3cm, line width = 1pt] ($({#1},{#4})+(-0.3,0)$) -- ($({#1},{#4})+(-0.3,-0.3)$) -- ($({#2},{#4})+(0.3,-0.3)$) -- ($({#2},{#4})+(0.3,0)$);}

\newcommand{\parentheses}[4]{
      \draw[rounded corners=0.3cm, line width = 0.75pt] ($({#1},{#3})+(0,0.3)$) -- ($({#1},{#3})+(-0.3,0)$) -- ($({#1},{#4})+(-0.3,0)$) -- ($({#1},{#4})+(0,-0.3)$);
      \draw[rounded corners=0.3cm, line width = 0.75pt] ($({#2},{#3})+(0,0.3)$) -- ($({#2},{#3})+(0.3,0)$) -- ($({#2},{#4})+(0.3,0)$) -- ($({#2},{#4})+(0,-0.3)$);}

% Message notation argument order:
%     Circle position relative to arrow [up, down, left, right];
%     Arrow direction [up, down, left, right];
%     Start node coordinate;
%     End node coordinate
%     Position along edge;
%     Message text;
\newcommand{\msg}[6]{
      % Circle left arrow down
      \ifthenelse{\isin{#1}{left} \AND \isin{#2}{down}}{
            \coordinate (anchor) at ($({#3})!{#5}!({#4})$);
            \node[xshift=-6.0mm] at (anchor) {#6};
            \node[xshift=-1.0mm] at (anchor) {$\downarrow$};
      }{}
      % Circle right arrow down
      \ifthenelse{\isin{#1}{right} \AND \isin{#2}{down}}{
            \coordinate (anchor) at ($({#3})!{#5}!({#4})$);
            \node[xshift=6.0mm] at (anchor) {#6};
            \node[xshift=1.0mm] at (anchor) {$\downarrow$};
      }{}

      % Circle down arrow right
      \ifthenelse{\isin{#1}{down} \AND \isin{#2}{right}}{
            \coordinate (anchor) at ($({#3})!{#5}!({#4})$);
            \node[ yshift=-4.0mm] at (anchor) {#6};
            \node[yshift=-1.0mm] at (anchor) {$\rightarrow$};
      }{}
      % Circle up arrow right
      \ifthenelse{\isin{#1}{up} \AND \isin{#2}{right}}{
            \coordinate (anchor) at ($({#3})!{#5}!({#4})$);
            \node[ yshift=4.0mm] at (anchor) {#6};
            \node[yshift=1.0mm] at (anchor) {$\rightarrow$};
      }{}

      % Circle down arrow left
      \ifthenelse{\isin{#1}{down} \AND \isin{#2}{left}}{
            \coordinate (anchor) at ($({#3})!{#5}!({#4})$);
            \node[ yshift=-4.0mm] at (anchor) {#6};
            \node[yshift=-1.0mm] at (anchor) {$\leftarrow$};
      }{}
      % Circle up arrow left
      \ifthenelse{\isin{#1}{up} \AND \isin{#2}{left}}{
            \coordinate (anchor) at ($({#3})!{#5}!({#4})$);
            \node[ yshift=4.0mm] at (anchor) {#6};
            \node[yshift=1.0mm] at (anchor) {$\leftarrow$};
      }{}

      % Circle left arrow down
      \ifthenelse{\isin{#1}{left} \AND \isin{#2}{up}}{
            \coordinate (anchor) at ($({#3})!{#5}!({#4})$);
            \node[ xshift=-6.0mm] at (anchor) {#6};
            \node[xshift=-1.0mm] at (anchor) {$\uparrow$};
      }{}
      % Circle right arrow down
      \ifthenelse{\isin{#1}{right} \AND \isin{#2}{up}}{
            \coordinate (anchor) at ($({#3})!{#5}!({#4})$);
            \node[ xshift=6.0mm] at (anchor) {#6};
            \node[xshift=1.0mm] at (anchor) {$\uparrow$};
      }{}
}

% Dark messages
\newcommand{\darkmsg}[6]{
      % Circle left arrow down
      \ifthenelse{\isin{#1}{left} \AND \isin{#2}{down}}{
            \coordinate (anchor) at ($({#3})!{#5}!({#4})$);
            \node[darkmsgcircle, xshift=-5.5mm] at (anchor) {#6};
            \node[xshift=-1.5mm] at (anchor) {$\downarrow$};
      }{}
      % Circle right arrow down
      \ifthenelse{\isin{#1}{right} \AND \isin{#2}{down}}{
            \coordinate (anchor) at ($({#3})!{#5}!({#4})$);
            \node[darkmsgcircle, xshift=5.5mm] at (anchor) {#6};
            \node[xshift=1.5mm] at (anchor) {$\downarrow$};
      }{}

      % Circle down arrow right
      \ifthenelse{\isin{#1}{down} \AND \isin{#2}{right}}{
            \coordinate (anchor) at ($({#3})!{#5}!({#4})$);
            \node[darkmsgcircle, yshift=-6.0mm] at (anchor) {#6};
            \node[yshift=-2.0mm] at (anchor) {$\rightarrow$};
      }{}
      % Circle up arrow right
      \ifthenelse{\isin{#1}{up} \AND \isin{#2}{right}}{
            \coordinate (anchor) at ($({#3})!{#5}!({#4})$);
            \node[darkmsgcircle, yshift=6.0mm] at (anchor) {#6};
            \node[yshift=2.0mm] at (anchor) {$\rightarrow$};
      }{}

      % Circle down arrow left
      \ifthenelse{\isin{#1}{down} \AND \isin{#2}{left}}{
            \coordinate (anchor) at ($({#3})!{#5}!({#4})$);
            \node[darkmsgcircle, yshift=-6.0mm] at (anchor) {#6};
            \node[yshift=-2.0mm] at (anchor) {$\leftarrow$};
      }{}
      % Circle up arrow left
      \ifthenelse{\isin{#1}{up} \AND \isin{#2}{left}}{
            \coordinate (anchor) at ($({#3})!{#5}!({#4})$);
            \node[darkmsgcircle, yshift=6.0mm] at (anchor) {#6};
            \node[yshift=2.0mm] at (anchor) {$\leftarrow$};
      }{}

      % Circle left arrow down
      \ifthenelse{\isin{#1}{left} \AND \isin{#2}{up}}{
            \coordinate (anchor) at ($({#3})!{#5}!({#4})$);
            \node[darkmsgcircle, xshift=-5.5mm] at (anchor) {#6};
            \node[xshift=-1.5mm] at (anchor) {$\uparrow$};
      }{}
      % Circle right arrow down
      \ifthenelse{\isin{#1}{right} \AND \isin{#2}{up}}{
            \coordinate (anchor) at ($({#3})!{#5}!({#4})$);
            \node[darkmsgcircle, xshift=5.5mm] at (anchor) {#6};
            \node[xshift=1.5mm] at (anchor) {$\uparrow$};
      }{}
}

\newcommand{\bwmsg}[6]{
      % doublecircle left arrow down
      \ifthenelse{\isin{#1}{left} \AND \isin{#2}{down}}{
            \coordinate (anchor) at ($({#3})!{#5}!({#4})$);
            \node[msgdoublecircle, xshift=-5.5mm] at (anchor) {#6};
            \node[xshift=-1.5mm] at (anchor) {$\downarrow$};
      }{}
      % doublecircle right arrow down
      \ifthenelse{\isin{#1}{right} \AND \isin{#2}{down}}{
            \coordinate (anchor) at ($({#3})!{#5}!({#4})$);
            \node[msgdoublecircle, xshift=5.5mm] at (anchor) {#6};
            \node[xshift=1.5mm] at (anchor) {$\downarrow$};
      }{}

      % doublecircle down arrow right
      \ifthenelse{\isin{#1}{down} \AND \isin{#2}{right}}{
            \coordinate (anchor) at ($({#3})!{#5}!({#4})$);
            \node[msgdoublecircle, yshift=-6.0mm] at (anchor) {#6};
            \node[yshift=-2.0mm] at (anchor) {$\rightarrow$};
      }{}
      % doublecircle up arrow right
      \ifthenelse{\isin{#1}{up} \AND \isin{#2}{right}}{
            \coordinate (anchor) at ($({#3})!{#5}!({#4})$);
            \node[msgdoublecircle, yshift=6.0mm] at (anchor) {#6};
            \node[yshift=2.0mm] at (anchor) {$\rightarrow$};
      }{}

      % doublecircle down arrow left
      \ifthenelse{\isin{#1}{down} \AND \isin{#2}{left}}{
            \coordinate (anchor) at ($({#3})!{#5}!({#4})$);
            \node[msgdoublecircle, yshift=-6.0mm] at (anchor) {#6};
            \node[yshift=-2.0mm] at (anchor) {$\leftarrow$};
      }{}
      % doublecircle up arrow left
      \ifthenelse{\isin{#1}{up} \AND \isin{#2}{left}}{
            \coordinate (anchor) at ($({#3})!{#5}!({#4})$);
            \node[msgdoublecircle, yshift=6.0mm] at (anchor) {#6};
            \node[yshift=2.0mm] at (anchor) {$\leftarrow$};
      }{}

      % doublecircle left arrow down
      \ifthenelse{\isin{#1}{left} \AND \isin{#2}{up}}{
            \coordinate (anchor) at ($({#3})!{#5}!({#4})$);
            \node[msgdoublecircle, xshift=-5.5mm] at (anchor) {#6};
            \node[xshift=-1.5mm] at (anchor) {$\uparrow$};
      }{}
      % doublecircle right arrow down
      \ifthenelse{\isin{#1}{right} \AND \isin{#2}{up}}{
            \coordinate (anchor) at ($({#3})!{#5}!({#4})$);
            \node[msgdoublecircle, xshift=5.5mm] at (anchor) {#6};
            \node[xshift=1.5mm] at (anchor) {$\uparrow$};
      }{}
}

% Dark messages
\newcommand{\bwdarkmsg}[6]{
      % doublecircle left arrow down
      \ifthenelse{\isin{#1}{left} \AND \isin{#2}{down}}{
            \coordinate (anchor) at ($({#3})!{#5}!({#4})$);
            \node[darkmsgdoublecircle, xshift=-5.5mm] at (anchor) {#6};
            \node[xshift=-1.5mm] at (anchor) {$\downarrow$};
      }{}
      % doublecircle right arrow down
      \ifthenelse{\isin{#1}{right} \AND \isin{#2}{down}}{
            \coordinate (anchor) at ($({#3})!{#5}!({#4})$);
            \node[darkmsgdoublecircle, xshift=5.5mm] at (anchor) {#6};
            \node[xshift=1.5mm] at (anchor) {$\downarrow$};
      }{}

      % doublecircle down arrow right
      \ifthenelse{\isin{#1}{down} \AND \isin{#2}{right}}{
            \coordinate (anchor) at ($({#3})!{#5}!({#4})$);
            \node[darkmsgdoublecircle, yshift=-6.0mm] at (anchor) {#6};
            \node[yshift=-2.0mm] at (anchor) {$\rightarrow$};
      }{}
      % doublecircle up arrow right
      \ifthenelse{\isin{#1}{up} \AND \isin{#2}{right}}{
            \coordinate (anchor) at ($({#3})!{#5}!({#4})$);
            \node[darkmsgdoublecircle, yshift=6.0mm] at (anchor) {#6};
            \node[yshift=2.0mm] at (anchor) {$\rightarrow$};
      }{}

      % doublecircle down arrow left
      \ifthenelse{\isin{#1}{down} \AND \isin{#2}{left}}{
            \coordinate (anchor) at ($({#3})!{#5}!({#4})$);
            \node[darkmsgdoublecircle, yshift=-6.0mm] at (anchor) {#6};
            \node[yshift=-2.0mm] at (anchor) {$\leftarrow$};
      }{}
      % doublecircle up arrow left
      \ifthenelse{\isin{#1}{up} \AND \isin{#2}{left}}{
            \coordinate (anchor) at ($({#3})!{#5}!({#4})$);
            \node[darkmsgdoublecircle, yshift=6.0mm] at (anchor) {#6};
            \node[yshift=2.0mm] at (anchor) {$\leftarrow$};
      }{}

      % doublecircle left arrow down
      \ifthenelse{\isin{#1}{left} \AND \isin{#2}{up}}{
            \coordinate (anchor) at ($({#3})!{#5}!({#4})$);
            \node[darkmsgdoublecircle, xshift=-5.5mm] at (anchor) {#6};
            \node[xshift=-1.5mm] at (anchor) {$\uparrow$};
      }{}
      % doublecircle right arrow down
      \ifthenelse{\isin{#1}{right} \AND \isin{#2}{up}}{
            \coordinate (anchor) at ($({#3})!{#5}!({#4})$);
            \node[darkmsgdoublecircle, xshift=5.5mm] at (anchor) {#6};
            \node[xshift=1.5mm] at (anchor) {$\uparrow$};
      }{}
}

\newcommand{\redbackmsg}[6]{
      % Circle left arrow down
      \ifthenelse{\isin{#1}{left} \AND \isin{#2}{down}}{
            \coordinate (anchor) at ($({#3})!{#5}!({#4})$);
            \node[reddarkmsgcircle, xshift=-5mm] at (anchor) {#6};
            \node[xshift=-1.5mm] at (anchor) {$\downarrow$};
      }{}
      % Circle right arrow down
      \ifthenelse{\isin{#1}{right} \AND \isin{#2}{down}}{
            \coordinate (anchor) at ($({#3})!{#5}!({#4})$);
            \node[reddarkmsgcircle, xshift=5mm] at (anchor) {#6};
            \node[xshift=1.5mm] at (anchor) {$\downarrow$};
      }{}

      % Circle down arrow right
      \ifthenelse{\isin{#1}{down} \AND \isin{#2}{right}}{
            \coordinate (anchor) at ($({#3})!{#5}!({#4})$);
            \node[reddarkmsgcircle, yshift=-5.0mm] at (anchor) {#6};
            \node[yshift=-2.0mm] at (anchor) {$\rightarrow$};
      }{}
      % Circle up arrow right
      \ifthenelse{\isin{#1}{up} \AND \isin{#2}{right}}{
            \coordinate (anchor) at ($({#3})!{#5}!({#4})$);
            \node[reddarkmsgcircle, yshift=5.0mm] at (anchor) {#6};
            \node[yshift=2.0mm] at (anchor) {$\rightarrow$};
      }{}

      % Circle down arrow left
      \ifthenelse{\isin{#1}{down} \AND \isin{#2}{left}}{
            \coordinate (anchor) at ($({#3})!{#5}!({#4})$);
            \node[reddarkmsgcircle, yshift=-5.0mm] at (anchor) {#6};
            \node[yshift=-2.0mm] at (anchor) {$\leftarrow$};
      }{}
      % Circle up arrow left
      \ifthenelse{\isin{#1}{up} \AND \isin{#2}{left}}{
            \coordinate (anchor) at ($({#3})!{#5}!({#4})$);
            \node[reddarkmsgcircle, yshift=5.0mm] at (anchor) {#6};
            \node[yshift=2.0mm] at (anchor) {$\leftarrow$};
      }{}

      % Circle left arrow down
      \ifthenelse{\isin{#1}{left} \AND \isin{#2}{up}}{
            \coordinate (anchor) at ($({#3})!{#5}!({#4})$);
            \node[reddarkmsgcircle, xshift=-5.0mm] at (anchor) {#6};
            \node[xshift=-1.5mm] at (anchor) {$\uparrow$};
      }{}
      % Circle right arrow down
      \ifthenelse{\isin{#1}{right} \AND \isin{#2}{up}}{
            \coordinate (anchor) at ($({#3})!{#5}!({#4})$);
            \node[reddarkmsgcircle, xshift=5.0mm] at (anchor) {#6};
            \node[xshift=1.5mm] at (anchor) {$\uparrow$};
      }{}
}

\newcommand{\redmsg}[6]{
      % Circle left arrow down
      \ifthenelse{\isin{#1}{left} \AND \isin{#2}{down}}{
            \coordinate (anchor) at ($({#3})!{#5}!({#4})$);
            \node[redmsgcircle, xshift=-5mm] at (anchor) {#6};
            \node[xshift=-1.5mm] at (anchor) {$\downarrow$};
      }{}
      % Circle right arrow down
      \ifthenelse{\isin{#1}{right} \AND \isin{#2}{down}}{
            \coordinate (anchor) at ($({#3})!{#5}!({#4})$);
            \node[redmsgcircle, xshift=5mm] at (anchor) {#6};
            \node[xshift=1.5mm] at (anchor) {$\downarrow$};
      }{}

      % Circle down arrow right
      \ifthenelse{\isin{#1}{down} \AND \isin{#2}{right}}{
            \coordinate (anchor) at ($({#3})!{#5}!({#4})$);
            \node[redmsgcircle, yshift=-5.0mm] at (anchor) {#6};
            \node[yshift=-2.0mm] at (anchor) {$\rightarrow$};
      }{}
      % Circle up arrow right
      \ifthenelse{\isin{#1}{up} \AND \isin{#2}{right}}{
            \coordinate (anchor) at ($({#3})!{#5}!({#4})$);
            \node[redmsgcircle, yshift=5.0mm] at (anchor) {#6};
            \node[yshift=2.0mm] at (anchor) {$\rightarrow$};
      }{}

      % Circle down arrow left
      \ifthenelse{\isin{#1}{down} \AND \isin{#2}{left}}{
            \coordinate (anchor) at ($({#3})!{#5}!({#4})$);
            \node[redmsgcircle, yshift=-5.0mm] at (anchor) {#6};
            \node[yshift=-2.0mm] at (anchor) {$\leftarrow$};
      }{}
      % Circle up arrow left
      \ifthenelse{\isin{#1}{up} \AND \isin{#2}{left}}{
            \coordinate (anchor) at ($({#3})!{#5}!({#4})$);
            \node[redmsgcircle, yshift=5.0mm] at (anchor) {#6};
            \node[yshift=2.0mm] at (anchor) {$\leftarrow$};
      }{}

      % Circle left arrow down
      \ifthenelse{\isin{#1}{left} \AND \isin{#2}{up}}{
            \coordinate (anchor) at ($({#3})!{#5}!({#4})$);
            \node[redmsgcircle, xshift=-5.0mm] at (anchor) {#6};
            \node[xshift=-1.5mm] at (anchor) {$\uparrow$};
      }{}
      % Circle right arrow down
      \ifthenelse{\isin{#1}{right} \AND \isin{#2}{up}}{
            \coordinate (anchor) at ($({#3})!{#5}!({#4})$);
            \node[redmsgcircle, xshift=5.0mm] at (anchor) {#6};
            \node[xshift=1.5mm] at (anchor) {$\uparrow$};
      }{}
}

\newcommand{\msgcircle}[6]{
      % Circle left arrow down
      \ifthenelse{\isin{#1}{left} \AND \isin{#2}{down}}{
            \coordinate (anchor) at ($({#3})!{#5}!({#4})$);
            \node[msgcircle,xshift=-5.0mm] at (anchor) {#6};
            \node[xshift=-1.5mm] at (anchor) {$\downarrow$};
      }{}
      % Circle right arrow down
      \ifthenelse{\isin{#1}{right} \AND \isin{#2}{down}}{
            \coordinate (anchor) at ($({#3})!{#5}!({#4})$);
            \node[msgcircle,xshift=5.0mm] at (anchor) {#6};
            \node[xshift=1.5mm] at (anchor) {$\downarrow$};
      }{}

      % Circle down arrow right
      \ifthenelse{\isin{#1}{down} \AND \isin{#2}{right}}{
            \coordinate (anchor) at ($({#3})!{#5}!({#4})$);
            \node[msgcircle, yshift=-5.0mm] at (anchor) {#6};
            \node[yshift=-2.0mm] at (anchor) {$\rightarrow$};
      }{}
      % Circle up arrow right
      \ifthenelse{\isin{#1}{up} \AND \isin{#2}{right}}{
            \coordinate (anchor) at ($({#3})!{#5}!({#4})$);
            \node[msgcircle, yshift=5.0mm] at (anchor) {#6};
            \node[yshift=2.0mm] at (anchor) {$\rightarrow$};
      }{}

      % Circle down arrow left
      \ifthenelse{\isin{#1}{down} \AND \isin{#2}{left}}{
            \coordinate (anchor) at ($({#3})!{#5}!({#4})$);
            \node[msgcircle, yshift=-5.0mm] at (anchor) {#6};
            \node[yshift=-2.0mm] at (anchor) {$\leftarrow$};
      }{}
      % Circle up arrow left
      \ifthenelse{\isin{#1}{up} \AND \isin{#2}{left}}{
            \coordinate (anchor) at ($({#3})!{#5}!({#4})$);
            \node[msgcircle, yshift=5.0mm] at (anchor) {#6};
            \node[yshift=2.0mm] at (anchor) {$\leftarrow$};
      }{}

      % Circle left arrow down
      \ifthenelse{\isin{#1}{left} \AND \isin{#2}{up}}{
            \coordinate (anchor) at ($({#3})!{#5}!({#4})$);
            \node[msgcircle, xshift=-5.0mm] at (anchor) {#6};
            \node[xshift=-1.5mm] at (anchor) {$\uparrow$};
      }{}
      % Circle right arrow down
      \ifthenelse{\isin{#1}{right} \AND \isin{#2}{up}}{
            \coordinate (anchor) at ($({#3})!{#5}!({#4})$);
            \node[msgcircle, xshift=5.0mm] at (anchor) {#6};
            \node[xshift=1.5mm] at (anchor) {$\uparrow$};
      }{}
}

% nodes style 
\tikzset{mainstyle/.style={fill=white, draw=black, shape=rectangle, align=center}}
\tikzset{istyle/.style={mainstyle, draw=white, minimum size=0mm, inner sep=0pt, text width=0mm}}
\tikzset{dstyle/.style={mainstyle, minimum size=5mm, inner sep=0pt, text width=4mm}}
\tikzset{sstyle/.style={mainstyle, minimum size=8mm, inner sep=0pt, text width=5mm}}
\tikzset{ostyle/.style={fill=darkgrey, draw=black, shape=rectangle, minimum size=0.2cm, inner sep=0pt, text width=2mm}}

% main nodes
\tikzstyle{observation}=[ostyle]
\tikzstyle{deterministic}=[dstyle]
\tikzstyle{stochastic}=[sstyle]

% auxiliary nodes
\tikzstyle{filter}=[mainstyle, minimum width=1cm, minimum height=0.5cm]
\tikzstyle{selector}=[fill=white, draw=black, shape=trapezium, rotate=180, minimum width=1cm, minimum height=0.5cm]

% auxiliary fonts
\newcommand{\tikzxmark}{%
\tikz[scale=0.23] {
    \draw[line width=0.7,line cap=round] (0,0) to [bend left=6] (1,1);
    \draw[line width=0.7,line cap=round] (0.2,0.95) to [bend right=3] (0.8,0.05);
}}
\newcommand{\tikzcmark}{%
\tikz[scale=0.23] {
    \draw[line width=0.7,line cap=round] (0.25,0) to [bend left=10] (1,1);
    \draw[line width=0.8,line cap=round] (0,0.35) to [bend right=1] (0.23,0);
}}

% plots preamble

\pgfplotsset{%
    layers/standard/.define layer set={%
        background,axis background,axis grid,axis ticks,axis lines,axis tick labels,pre main,main,axis descriptions,axis foreground%
    }{
        grid style={/pgfplots/on layer=axis grid},%
        tick style={/pgfplots/on layer=axis ticks},%
        axis line style={/pgfplots/on layer=axis lines},%
        label style={/pgfplots/on layer=axis descriptions},%
        legend style={/pgfplots/on layer=axis descriptions},%
        title style={/pgfplots/on layer=axis descriptions},%
        colorbar style={/pgfplots/on layer=axis descriptions},%
        ticklabel style={/pgfplots/on layer=axis tick labels},%
        axis background@ style={/pgfplots/on layer=axis background},%
        3d box foreground style={/pgfplots/on layer=axis foreground},%
    },
}

% Preamble for often used mathematical expressions

\usepackage{amsfonts,amssymb}
\renewcommand{\d}[1]{\operatorname{d}\!{#1}}
\renewcommand{\exp}[1]{\operatorname{exp}\!\left({#1}\right)}
\renewcommand{\log}[1]{\operatorname{log}\!\left({#1}\right)}
\newcommand{\lognb}[1]{\operatorname{log}{#1}}
\renewcommand{\ln}[1]{\operatorname{ln}\!\left({#1}\right)}
\newcommand{\tr}[1]{\operatorname{tr}\!\left({#1}\right)}
\newcommand{\argmax}[2]{\underset{#1}{\operatorname{argmax}}\!\left({#2}\right)}
\newcommand{\argmaxnb}[2]{\underset{#1}{\operatorname{argmax}}\,{#2}}
\newcommand{\argmin}[2]{\underset{#1}{\operatorname{argmin}}\!\left({#2}\right)}
\newcommand{\T}{\operatorname{T}}
\newcommand{\model}{\text{m}}
\newcommand{\identity}{\mathrm{I}}
\newcommand{\eu}{\bm{e}_{1}}
\newcommand{\const}{\text{const}}
\newcommand{\scal}[1]{\operatorname{#1}}
\newcommand{\vect}[1]{\boldsymbol{#1}}
\newcommand{\matr}[1]{\mathbf{#1}}
\newcommand{\rvar}[1]{\mathit{#1}} % random variable
\newcommand{\rvect}[1]{\mathit{\boldsymbol{#1}}} % random vector
\newcommand{\rmatr}[1]{\mathit{\boldsymbol{#1}}} % random matrix
\newcommand{\mean}[1]{\overline{#1}}
\newcommand{\cov}[1]{\mathbf{S}_{#1}}
\newcommand{\covs}[1]{S_{#1}} % covariance for 1d
\newcommand{\E}[1]{\mathbb{E}\!\left[{#1}\right]}
\newcommand{\U}[1]{\mathbb{U}\!\left[{#1}\right]}
\newcommand{\Ux}[2]{\mathbb{U}_{#1}\!\left[{#2}\right]}
\renewcommand{\H}[1]{\mathbb{H}\!\left[{#1}\right]}
\newcommand{\Hx}[2]{\mathbb{H}_{#1}\!\left[{#2}\right]}
\newcommand{\F}[1]{\mathbb{F}\!\left[{#1}\right]}
\newcommand{\Fx}[2]{\operatorname{F}_{#1}\!\left[{#2}\right]}
\newcommand{\Ex}[2]{\mathbb{E}_{#1}\!\left[{#2}\right]}
\newcommand{\Exm}[2]{\underset{#1}{\mathbb{E}}\!\left[{#2}\right]}
\newcommand{\Etx}[2]{\tilde{\mathbb{E}}_{#1}\!\left[{#2}\right]}
\newcommand{\Var}[1]{\mathbb{V}\!\operatorname{ar}\!\left[{#1}\right]}
\newcommand{\Cov}[1]{\mathbb{C}\!\operatorname{ov}\!\left[{#1}\right]}
\newcommand{\Covx}[2]{\mathbb{C}\!\operatorname{ov}_{#1}\!\left[{#2}\right]}
\newcommand{\N}[1]{\mathcal{N}\!\left({#1}\right)}
\newcommand{\W}[1]{\mathcal{W}\!\left({#1}\right)}
\newcommand{\logN}[1]{\operatorname{log}\!\mathcal{N}\!\left({#1}\right)}
\newcommand{\Gam}[1]{\mathcal{G}am\!\left({#1}\right)}
\newcommand{\Exp}[1]{\mathcal{E}xp\!\left({#1}\right)}
\newcommand{\Ig}[1]{\mathcal{I}nv\mathcal{G}am\!\left({#1}\right)}
\newcommand{\Uni}[1]{\mathcal{U}\!\left({#1}\right)}
\newcommand{\Ber}[1]{\mathcal{B}er\!\left({#1}\right)}
\newcommand{\Cat}[1]{\mathcal{C}at\!\left({#1}\right)}
\newcommand{\Bet}[1]{\mathcal{B}eta\!\left({#1}\right)}
\newcommand{\Dir}[1]{\mathcal{D}ir\!\left({#1}\right)}
\newcommand{\logb}[2]{\operatorname{log}_{#1}\!\left({#2}\right)}

\newcommand{\HL}{\mathrm{L}}
\newcommand{\BF}{\mathrm{BF}}
\newcommand{\pluscirc}{\circ\mkern-11.45mu +}

\newcommand\yestag{\addtocounter{equation}{1}\tag{\theequation}}

\newcommand{\gmm}[1][]{$\Gamma$MM}
\newcommand{\gmms}[1][]{$\Gamma$MMs}
\newcommand{\gm}[1][]{$\Gamma$M}

\definecolor{darkred}{rgb}{0.5,0,0}
\definecolor{darkgreen}{rgb}{0,0.5,0}
\definecolor{darkblue}{rgb}{0,0,0.5}
\definecolor{halfgray}{gray}{0.55} % chapter numbers will be semi transparent .5 .55 .6 .0
\definecolor{webgreen}{rgb}{0,.5,0}
\definecolor{webbrown}{rgb}{.6,0,0}
\definecolor{LightGray}{gray}{0.95}
\definecolor{beige}{gray}{0.92}
\definecolor{darkgrey}{gray}{0.25}
\definecolor{lightgrey}{gray}{0.94}

% ********************************************************************
% Setup, finetuning, and useful commands
% ********************************************************************
\newcounter{dummy} % necessary for correct hyperlinks (to index, bib, etc.)
\newcounter{q_counter} % counter for research questions
\newlength{\abcd} % for ab..z string length calculation
\providecommand{\mLyX}{L\kern-.1667em\lower.25em\hbox{Y}\kern-.125emX\@}
% ****************************************************************************************************

%---------------------------------------%
% Default fig and table positioning     %
%---------------------------------------%

% From http://tex.stackexchange.com/questions/140568/how-to-set-default-positioning-of-figure-table-document-wide (BY MOI :D)
% and http://tex.stackexchange.com/questions/8351/what-do-makeatletter-and-makeatother-do
\makeatletter
\providecommand*\setfloatlocations[2]{\@namedef{fps@#1}{#2}}
\makeatother
\setfloatlocations{figure}{hbtp}
\setfloatlocations{table}{htbp}
%\setfloatlocations{subfigure}{t} %does not seem to work the same as adding [t] to begin{subfigure}

%
% Footnote command that has no marker or counter increase
% FROM http://en.wikibooks.org/wiki/LaTeX/Footnotes_and_Margin_Notes
%
\makeatletter
\def\blfootnote{\xdef\@thefnmark{}\@footnotetext}
\makeatother

% ****************************************************************************************************
% 3. Loading some handy packages (alphabetically mainly)
% ****************************************************************************************************

%might be disabled:
\listfiles %include list of used files in .log

% \usepackage[fleqn]{amsmath}   % math environments and more by the AMS
\usepackage{amsmath}
\newtheorem{definition}{Definition}[chapter]  %We use mainly definitions and number them by chapter

\usepackage{algorithm,algorithmic}
% \usepackage{algpseudocode}   % yet another package for pseudo code

\usepackage{amssymb}        % f.i. the circular arrows,
\usepackage{acro}           % for acronyms

\DeclareAcronym{ai}{short=AI,long=Artificial Intelligence}
\DeclareAcronym{gpu}{short=GPU,long=Graphical Processing Unit}
\DeclareAcronym{fep}{short=FEP,long=Free Energy Principle}
\DeclareAcronym{vfe}{short=VFE,long=Variational Free Energy}
\DeclareAcronym{aif}{short=AIF,long=Active Inference}
\DeclareAcronym{mcmc}{short=MCMC,long=Markov Chain Monte Carlo}
\DeclareAcronym{nuts}{short=NUTS,long=No U-Turn Sampling}
\DeclareAcronym{hmc}{short=HMC,long=Hamiltonian Monte Carlo}
\DeclareAcronym{pg}{short=PG,long=Particle Gibbs sampler}
\DeclareAcronym{bbvi}{short=BBVI,long=Black Box Variational Inference}
\DeclareAcronym{advi}{short=ADVI,long=Automatic Differentiation Variational Inference}
\DeclareAcronym{ad}{short=AD,long=Automatic Differentiation}
\DeclareAcronym{vi}{short=VI,long=Variational Inference}
\DeclareAcronym{elbo}{short=ELBO,long=Evidence Lower Bound}
\DeclareAcronym{gm}{short=GM,long=Graphical Model}
\DeclareAcronym{fg}{short=FG,long=Factor Graph}
\DeclareAcronym{vmp}{short=VMP,long=Variational Message Passing}
\DeclareAcronym{rp}{short=RP,long=Reactive Programming}
\DeclareAcronym{rmp}{short=RMP,long=Reactive Message Passing}
\DeclareAcronym{ppl}{short=PPL,long=Probabilistic Programming Language}
\DeclareAcronym{cbfe}{short=CBFE,long=Constrained Bethe Free Energy}
\DeclareAcronym{kl}{short=KL,long=Kullback–Leibler divergence}
\DeclareAcronym{bfe}{short=BFE,long=Bethe Free Energy}
\DeclareAcronym{bp}{short=BP,long=Belief Propagation}
\DeclareAcronym{lp}{short=LP,long=Laplace Propagation}
\DeclareAcronym{ep}{short=EP,long=Expectation Propagation}
\DeclareAcronym{em}{short=EM,long=Expectation Maximization}
\DeclareAcronym{tffg}{short=TFFG,long=Terminated Forney-style Factor Graph}
\DeclareAcronym{dsl}{short=DSL,long=Domain Specific Language}
\DeclareAcronym{api}{short=API,long=Application Programming Interface}
\DeclareAcronym{gcv}{short=GCV,long=Gaussian-with-Controlled-Variance}
\DeclareAcronym{hgf}{short=HGF,long=Hierarchical Gaussian Filter}
\DeclareAcronym{ae}{short=PE,long=Prediction Error}
\DeclareAcronym{lds}{short=LDS,long=Linear Dynamical System}
\DeclareAcronym{nlds}{short=NLDS,long=NonLinear Dynamical System}
\DeclareAcronym{rts}{short=RTS,long=Rauch-Tung-Striebel}
\DeclareAcronym{hnds}{short=HNDS, long=Hierarchical Nonlinear Dynamical System}
\DeclareAcronym{cpu}{short=CPU, long=Central Processing Unit}

\usepackage[T2A]{fontenc}% T2A for cyrillics (http://tex.stackexchange.com/questions/664/why-should-i-use-usepackaget1fontenc)
\usepackage[russian,dutch,english]{babel} %we need Dutch too, but main text is english so load english last
\usepackage{booktabs}       % for better rules in tables
\usepackage{bm}             % for bold math (eg numbers), still looks kinda ugly
% \usepackage{bibentry}       % Inline bibliography entries
% \nobibliography*
\usepackage{natbib}
\usepackage[pagebackref=true]{hyperref}
\renewcommand*\backref[1]{\ifx#1\relax \else (Cited on #1) \fi}
%I did not manage to get the bibtex package to work, but PLEASE do try it yourself. It is way easier to customize things!!!
% \usepackage[
%     backend=bibtex,
% %    backend=biber,
%     backref=true,
%     sortcites=true,
% %   style=numeric-comp,
% %    style=authoryear-icomp,
% %    sortlocale=de_DE,
% %    natbib=true,
% %    url=false,
% %    doi=true,
% %    eprint=false
% ]{biblatex}
%\addbibresource{biblatex-examples.bib}

\usepackage{tcolorbox}      % For fancy boxes around some text

\usepackage{calc}           % Calculations, mainly coordinates in TikZ

\usepackage{caption}        % for better captions...
\captionsetup{format=hang,font=small}

\newcommand{\chapquote}[2]{
  \begin{quotation}
    \textit{#1}
  \end{quotation}
  \begin{flushright}
    -- #2
  \end{flushright}
  \vspace{10pt}}
\newcommand{\chapbib}[1]{
  \begin{quote}
    \textit{#1}
  \end{quote}
  \vskip 10pt}

%disabled in favor of biblatex package
\usepackage{cite}           % for neat citations (\eg sorted and abbreviated to 1-20)
\usepackage{comment}        % add a comment environment

\usepackage{datetime}   % time access (mainly for page footer)
\yyyymmdddate
\usepackage{doi}            % provides the \doi{} command that makes DOIs into hyperlinks (not needed if biblatex is enabled I guess)
\usepackage{dsfont}         % For natural numbers symbol etc.

\usepackage{emptypage}      % Empty pages between chapters!!!!

%Fancy headers!!!
\usepackage{fancyhdr}
\usepackage[Lenny]{fncychap} %We choose lenny, refer to package doc for more styles
%\usepackage[Bjornstrup]{fncychap}
\pagestyle{fancy}
%For more setup see below

% \usepackage{fourier}       %For better math font style (tip by Jack van Wijk)

\usepackage{graphicx}       % Graphics (only text is booooring)
% \usepackage{sidecap}      % Use floatrow instead
\graphicspath{{./Figures/}} % Which we load from a subdir
\usepackage[xindy,toc]{glossaries}  % sorted glossary entries

\usepackage{hhline}         % for more specific lines in tables
\usepackage{hyphenat}
\hyphenation{re-cor-ded re-quire-ments di-men-sio-nal}    % These words were not hyphenated correctly automatically but are now! Extend as necessary

%\usepackage[latin9]{inputenc}
\usepackage[utf8]{inputenc}  % To prevent '' etc. to appear. 'latin9' is default (I guess) and 'utf8' fails on some 'no-break space' characters which are hard to find...

% \usepackage{lastpage}       % To be able to say how many pages we have
\usepackage{longtable}      % For multi-page tables

\usepackage{makeidx}        % For creating an index which every proper book should have %NOTE: should be loaded before hyperref
%\usepackage{showidx}        % Show index keys in the margin (is not on good terms with hyperref so do NOT enable)
% \usepackage{mathtools}      % mathtools builds on and extends amsmath package, used to break long equations over multiple lines in ch. 6
%\usepackage{microtype}      % shrink/size font for nicer flow and less hypenation http://tex.stackexchange.com/a/586/27955
% \usepackage{minted}
\usepackage{mparhack}       % get marginpar right (known bug)
\usepackage{multirow}       % multiple rows and columns in tables

\usepackage{newverbs}
% \newverbcommand{\bverb}
%     {\begin{lrbox}{\verbbox}}
%     {\end{lrbox}\colorbox{LightGray}{\box\verbbox}}

\usepackage{paralist}       % for inline lists
\usepackage{pdflscape}      % land scape pages
\usepackage{pdfpages}
\usepackage{pifont}         % For the \ding command (used for \tick, \cross, etc.)
\usepackage{placeins}       % Introduces the \FloatBarrier command, but can also be set to never let floats cross section boundaries
\usepackage[draft]{prelim2e}       %Mark prelimenary versions by adding [draft]
\renewcommand{\PrelimWords}{\relax}
\renewcommand{\PrelimText}{\footnotesize[\,\today\ at \currenttime\ -- \pageref{LastPage} pages -- \myVersion\,]} %Adding a compile date/time to each page footer is handy, believe me
%\renewcommand{\PrelimText}{\footnotesize[\,\today\ at \thistime\ -- \myVersion\,]}
%\renewcommand{\PrelimText}{\footnotesize[\,\myVersion\,]}
\renewcommand{\PrelimText}{} %Clears the footer all together

%nocites
% \usepackage[nocites,norefs]{refcheck}       % Be notified of unused labels (does not seem to work with autoref :S) and bibliography references, see http://ctan.org/pkg/refcheck, use nocites to disable bibtex keys in biblio
\usepackage{rotating}       % For rotating in tables from Excel (using Excel plug-in)
\usepackage{accents}
\usepackage{standalone}     % To make the individual TikZ files also compilable :)
%\usepackage{floatrow}       % For aligning subfigures and their captions

%FROM http://tex.stackexchange.com/questions/6850/table-and-figure-side-by-side-with-independent-captions
% Table float box with bottom caption, box width adjusted to content
%\newfloatcommand{capbtabbox}{table}[][\FBwidth]

\usepackage{tabularx}       % for vertical alignment in tables
\usepackage{diagbox}        % for diag cells
\newcolumntype{C}[1]{>{\centering\let\newline\\\arraybackslash\hspace{0pt}}m{#1}} % for equal width centered cells
\usepackage[nottoc]{tocbibind}      % We want the bibliography in the TOC please!
\usepackage{subcaption}
\usepackage{gensymb}

%todonotes
\def\graffito\emph{\marginpar}
%\newcommand{\todo}[1]{\marginpar{\emph{\footnotesize#1}}} %Preferred because this actually works (todonotes broke after mergin classicthesis-config with my.sty) and is actually neater and less intrusive/ugly
\usepackage[colorinlistoftodos,backgroundcolor=white,linecolor=gray,disable]{todonotes}
% \usepackage[colorinlistoftodos,backgroundcolor=white,linecolor=gray]{todonotes}  %More advanced todos, default todo has no color
%And different predefined todos (change as you see fit)
%\newcommand{\RW}[1][]{\todo[color=green!40]{RW: #1}}
\newcommand{\atProf}[1]{\todo[color=blue!40]{@W: #1}}
\newcommand{\status}[1]{\todo[color=purple!40]{STATUS: #1}}

\newcommand{\should}[1]{\todo[color=red!40]{#1}}  % 'must do' todos
\newcommand{\could}[1]{\todo[color=orange!40]{#1}}   % easy improvement todos
\newcommand{\would}[1]{\todo[color=yellow!40]{#1}}   % suggestions
\newcommand{\may}[1]{\todo{#1}}     % maybe...

\newcommand{\feedback}[1]{\todo[color=black!40]{#1}} %Feedback of PhD committee

\newcommand{\forFinal}[1]{\todo[color=brown!40]{#1}} %Action points before final version is created

%tododnotes makes tikz pics which are externalized which we do not want:
\makeatletter
\renewcommand{\todo}[2][]{\tikzexternaldisable\@todo[noline,#1]{#2}\tikzexternalenable}
\makeatother

%
%%%%%TO TURN TODO NOTES ON/OFF ::::
%
%\renewcommand{\todo}[2][]{}

\usepackage{wasysym}        % For the correct sized circle inside the BPMN gateways...

\usepackage{xspace}         %To add space to command definitions

%And some packages need to be last...
% \usepackage{fixltx2e} % fixes some LaTeX stuff
\usepackage{textcomp} % fix warning with missing font shapes

%
% MORE DETAILED CONFIGURATION BELOW (IN SEMI-RANDOM ORDER)
%

% ****************************************************************************************************
% Setup code listings
% ****************************************************************************************************
\usepackage{listings}
\usepackage[theme=grayscale]{jlcode}

% float env for listings
\newfloat{lstfloat}{htbp}{lop}
\floatname{lstfloat}{Listing}
\floatstyle{plain}
\def\lstfloatautorefname{Listing}

% ****************************************************************************************************
% PDFLaTeX, hyperreferences and citation backreferences
% ****************************************************************************************************
% PDFLaTeX
% ocgcolorlinks : ensures that colored links are printed as black (when supported by PDF viewer!) (from http://tex.stackexchange.com/questions/4425/is-there-a-way-to-have-coloured-hyperref-hyperlinks-in-the-pdf-but-have-them-pr)
% NOTE: ocgcolorlinks does not seem to work for me, so made all links black!
\PassOptionsToPackage{hyphens}{url}
\PassOptionsToPackage{hidelinks,pdftex,hyperfootnotes=false,pdfpagelabels,ocgcolorlinks}{hyperref}
\RequirePackage{hyperref}

% ********************************************************************
% Setup the style of the backrefs from the bibliography
% ********************************************************************

% disable when using biblatex!!!
% \newcommand{\backrefnotcitedstring}{\relax}%(Not cited.)
% \newcommand{\backrefcitedsinglestring}[1]{(Cited on page~#1.)}
% \newcommand{\backrefcitedmultistring}[1]{(Cited on pages~#1.)}
% 		\PassOptionsToPackage{hyperpageref}{backref}
% \usepackage[hyperpageref]{backref} % to be loaded after hyperref package
%   \renewcommand{\backreftwosep}{ and~} % separate 2 pages
%   \renewcommand{\backreflastsep}{, and~} % separate last of longer list
%   \renewcommand*{\backref}[1]{}  % disable standard
%   \renewcommand*{\backrefalt}[4]{% detailed backref
%       \ifcase #1 %
%          \backrefnotcitedstring%
%       \or%
%          \backrefcitedsinglestring{#2}%
%       \else%
%          \backrefcitedmultistring{#2}%
%       \fi}%

% ********************************************************************
% Hyperreferences
% ********************************************************************
\hypersetup{
  colorlinks=false,
  %    colorlinks=true,
  linktocpage=true, pdfstartpage=3, pdfstartview=FitV,%
  % uncomment the following line if you want to have black links (\eg, for printing)
  %colorlinks=false, linktocpage=false, pdfborder={0 0 0}, pdfstartpage=3, pdfstartview=FitV,%
  linktoc=all, %make whole line in TOC clickable
  breaklinks=true, pdfpagemode=UseNone, pageanchor=true, pdfpagemode=UseOutlines,%
  plainpages=false, bookmarksnumbered, bookmarksopen=true, bookmarksopenlevel=1,%
  hypertexnames=true, pdfhighlight=/O,%nesting=true,%frenchlinks,%
  urlcolor=webbrown, linkcolor=RoyalBlue, citecolor=webgreen, %pagecolor=RoyalBlue,%
  %urlcolor=Black, linkcolor=Black, citecolor=Black, %pagecolor=Black,%
  pdftitle={\myTitle},%
  pdfauthor={\textcopyright \myName, \myUni, \myFaculty},%
  pdfsubject={},%
  pdfkeywords={\myKeywords},%
  pdfcreator={pdfLaTeX},%
  pdfproducer={LaTeX with hyperref}}
\hypersetup{hidelinks}
% \hypersetup{pdfauthor={Name}}

%Correct autoref names
\makeatletter
\addto\extrasenglish{
  \renewcommand*{\figureautorefname}{Figure}%
  \renewcommand*{\tableautorefname}{Table}%
  \renewcommand*{\partautorefname}{Part}%
  \renewcommand*{\chapterautorefname}{Chapter}%
  \renewcommand*{\sectionautorefname}{Section}%
  \renewcommand*{\subsectionautorefname}{Section}%
  \renewcommand*{\subsubsectionautorefname}{Section}%
  \providecommand{\subfigureautorefname}{\figureautorefname}%
  \providecommand{\definitionautorefname}{Definition} % It's that easy!: http://tex.stackexchange.com/questions/46258/how-to-get-correct-autoref-for-theorems
}
\makeatother

%
% Fancy Headers!!!
%

%Lesli Lamports LaTeX book style headers, from fancyhdr manual page 13
%\fancyheadoffset[LE,RO]{\marginparsep+\marginparwidth}
\fancyheadoffset[LE,RO]{0.5in}
\renewcommand{\chaptermark}[1]{\markboth{#1}{}}
\renewcommand{\sectionmark}[1]{\markright{\thesection\ #1}}
\fancyhf{}
\fancyhead[LE,RO]{\bfseries\thepage}
\fancyhead[LO]{\bfseries\rightmark}
\fancyhead[RE]{\bfseries\leftmark}
\fancypagestyle{plain}{%
  \fancyhead{} % get rid of headers
  \renewcommand{\headrulewidth}{0pt} % and the line
}
%
\makeatletter

\ChNumVar{\fontsize{36}{80}\usefont{OT1}{pag}{m}{n}\selectfont}
\renewcommand{\DOCH}{%
  %Horizontal line left top
  \settowidth{\px}{\CNV\FmN{\@chapapp}}
  \addtolength{\px}{2pt}

  %Horizontal line top
  \setlength{\py}{0pt}
  \setlength{\mylen}{0pt}
  \setlength{\abovedisplayskip}{2.5pt}

  \setlength{\belowdisplayskip}{2.5pt}
  \setlength{\textfloatsep}{2.5pt}
  \setlength{\parskip}{0pt}

  \settowidth{\pxx}{\CNoV\thechapter}
  \addtolength{\pxx}{-1pt}
  \par
  \parbox[b]{\textwidth}{%
    \raggedright%
    \color{gray}
    \CNoV\FmN{\@chapapp}
    \hskip1pt%
    \thechapter%
    \hskip1pt%
    \color{black}
    \rule{\RW}{\pyy}\par\nobreak%
    \vskip -\baselineskip%
    \vskip -\pyy%
    \hskip \mylen%
    \vskip \pyy}%
  \vskip 20\p@}

\renewcommand{\DOTI}[1]{%
  \raggedright
  \CTV\FmTi{#1}\par\nobreak
  \vskip 25\p@}

\renewcommand{\DOTIS}[1]{%
  \raggedright
  \CTV\FmTi{#1}\par\nobreak
  \vskip 25\p@}

\makeatother

%Set font style to 'charter'
\renewcommand{\rmdefault}{bch}
\renewcommand{\bfdefault}{b}

\makeglossaries

\usepackage{svg}
