\chapter{Biography}

Dmitry Bagaev was born on December 22nd, 1995, in Nizhnekamsk, Tatarstan Republic, Russian
Federation.
At the age of 15, he was admitted to a specialized high school in Moscow called the Advanced
Educational Scientific Center (AESC), which was founded by Andrey Kolmogorov.
% AESC consistently ranked among the top 1 to 3 best schools in Russia.
It was at this school that Dmitry's passion for mathematics and physics was nurtured and
inspired by excellent teachers.

After completing his studies at AESC, Dmitry embarked on his bachelor's degree at the Faculty
of Computational Mathematics and Cybernetics at Moscow State University.
During his undergraduate years, Dmitry also began working as a junior software developer at
the Institute of Biorganic Chemistry (IBCH) in Moscow, under the supervision of Mikhail
Shugay.
This experience ignited his passion for programming and software development.
Interestingly, the work in the unrelated field of biorganic chemistry and immunology later
became an essential part of Dmitry's dissertation, as it involved significant work with
reactive programming.
Dmitry continues to maintain his previous research efforts and software developed at the IBCH.

Upon completing his bachelor's degree, Dmitry pursued Master's degree at the Faculty of
Computational Mathematics at Moscow State University, this time under the guidance of Igor
Konshin at the Institute of Numerical Mathematics (INM).
During his Master's program, Dmitry engaged in mathematical modeling research on
super-computers involving tens of thousands of CPUs.
He also had the opportunity to intern at ExxonMobil HQ in Texas, USA.
This experience not only enhanced Dmitry's communication and presentation skills but also
sparked his interest in the world of low-level code optimization.
Much like his experience at IBCH, the fascination and expertise in low-level code optimization
became an essential part of his current dissertation.

At the end of his Master study, a friend of Dmitry casually invited him for a trip in Europe
during some unrelated conversation.
Dmitry hesitated but accepted the invitation.
One of the countries in the list was the Netherlands.
During this trip, Dmitry fell in love with the country.
Seizing perfect timing, Dmitry decided to explore open Ph.D.
positions in the Netherlands (he did not submit a single Ph.D. application elsewhere).
Despite having no prior experience with Bayesian inference or statistical methods, he was
surprised and delighted when Bert de Vries, the professor of BIASlab at TU/e, invited him for
an interview.
This opportunity arose just a few days after Dmitry's application had been rejected by the
University of Delft, where he was a close second candidate.

Although Dmitry had little hope of being successful, he saw the interview as an excellent
opportunity to revisit the Netherlands and asked Bert de Vries if he could visit the BIASlab
in person.
The visit only strengthened Dmitry's determination to dive into a field of unknowns (pun
intended).
Since then, Dmitry has been conducting extensive research with a group of amazing and
passionate people at BIASlab.
His research focused on the application of reactive programming ideas in the context of
large-scale efficient Bayesian inference, which has been documented in the present
dissertation.

Beyond research, he finds joy in sporting activities like skydiving, snowboarding, wakeboarding, skiing,
trampoline jumping, and gym.
His passion for creativity extends to drumming and video editing.
He also enjoys reading and embraces the Dutch cycling culture, exploring its scenic routes.

% \pagestyle{empty}